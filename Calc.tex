\documentclass[a4paper, 14pt]{article}
\usepackage[margin=1in]{geometry}
\usepackage{amsfonts, 
			amsmath, 
			amssymb,
			amsthm,
			pgfplots,
			tikz,
			graphicx,
			caption,
			float,
			physics
			}
\usepackage[none]{hyphenat}
\usepackage{fancyhdr} %create a custom header and footer
\usepackage[utf8]{inputenc}
\usepackage[english, main=ukrainian]{babel}
\usepgfplotslibrary{fillbetween}
\usepackage[unicode]{hyperref}
\usetikzlibrary{spy}
\usepackage{ifthen}
\usepackage{pdfpages}
\usepackage{enumitem}

\fancyhead{}
\fancyfoot{}
\parindent 0ex
\DeclareMathOperator*\uplim{\overline{lim}}
\DeclareMathOperator*\downlim{\underline{lim}}
\def\huge{\displaystyle}


\def\qed{$\blacksquare$}

\def\rightproof{$\boxed{\Rightarrow}$ }
\def\leftproof{$\boxed{\Leftarrow}$ }

\def\noProof{\\ \textit{Без доведення.}}

\newtheoremstyle{theoremdd}% name of the style to be used
  {\topsep}% measure of space to leave above the theorem. E.g.: 3pt
  {\topsep}% measure of space to leave below the theorem. E.g.: 3pt
  {\normalfont}% name of font to use in the body of the theorem
  {0pt}% measure of space to indent
  {\bfseries}% name of head font
  {}% punctuation between head and body
  { }% space after theorem head; " " = normal interword space
  {\thmname{#1}\thmnumber{ #2}\textnormal{\thmnote{ \textbf{#3}\\}}}

\theoremstyle{theoremdd}
\newtheorem{theorem}{Theorem}[subsection]
  
\theoremstyle{theoremdd}
\newtheorem{definition}[theorem]{Definition}

\theoremstyle{theoremdd}
\newtheorem{samedef}[theorem]{Definition}

\theoremstyle{theoremdd}
\newtheorem{example}[theorem]{Example}

\theoremstyle{theoremdd}
\newtheorem{proposition}[theorem]{Proposition}

\theoremstyle{theoremdd}
\newtheorem{remark}[theorem]{Remark}

\theoremstyle{theoremdd}
\newtheorem{lemma}[theorem]{Lemma}

\theoremstyle{theoremdd}
\newtheorem{corollary}[theorem]{Corollary}

\makeatletter
\renewenvironment{proof}[1][Proof.\\]{\par
\pushQED{\hfill \qed}%
\normalfont \topsep6\p@\@plus6\p@\relax
\trivlist
\item\relax
{\bfseries
#1\@addpunct{.}}\hspace\labelsep\ignorespaces
}{%
\popQED\endtrivlist\@endpefalse
}
\makeatother

\newenvironment{pfMI}{\vspace*{-3mm} \textbf{\\ Proof MI. \\}}{\hfill $\blacksquare$}
\newenvironment{pfNoTh}{\textbf{Proof. \\}}{$\blacksquare$}


\begin{document}
\includepdf{preview.jpg}
\tableofcontents
\newpage

%\iffalse %IMPORTANT COMMENT
	\section*{Необхідні тули для розвитку матана}
	\subsection*{Шкільні речі та трошки про те, як розвивати множину дійсних чисел}
	Вже з такими числами було більш-менш ознайомлено в школі. Починалось все з натуральних чисел
	$$ \mathbb{N} = \{1,2,3,\dots\} $$
	Далі пішли цілі  числа \\
	$$ \mathbb{Z} = \{\dots,-3,-2,-1,0,1,2,3,\dots\} $$
	Саме в цілих числах ми змогли визначити вже операцію $+$, але цього недостатньо. \\
	Потім раціональні числа. \\
	$$ \mathbb{Q} = \left\{ \dfrac{m}{n} | m \in \mathbb{Z}, n \in \mathbb{N} \right\} $$
	А тут вже ми змогли визначити операцію $\cdot$, і цього теж мало.\\
	Настав саме час дослідити поле дійсних чисел - $\mathbb{R}$.\\
	Одна з головних мотивацій зробити - це прямокутний трикутник зі сторонами $1$.
	\begin{figure}[H]
	\centering
	\begin{tikzpicture}
	\draw[thick] (0,0)--(2,0) node at (-0.3,1) {$1$};
	\draw[thick] (0,0)--(0,2) node at (1,-0.3) {$1$};
	\draw[thick] (2,0)--(0,2) node at (1.2,1.2) {$x$};
	\end{tikzpicture}
	\end{figure}
	За теоремою Піфагора, ми вже знаємо, що
	$x^2 = 1^2 + 1^2 \implies x^2 = 2$. І от тут виникли проблеми:
	\begin{proposition}
	$\not\exists  x \in \mathbb{Q}: \hspace{0.3cm} x^2 = 2$. Або інакше кажучи, $\sqrt{2} \not\in \mathbb{Q}$.
	\end{proposition}
	
	\begin{proof}
	!Припустимо, що все ж таки $\exists x \in \mathbb{Q}$, тобто $x= \dfrac{m}{n}, m \in \mathbb{Z}, n \in \mathbb{N}$, нескоротима дріб, для якого\\
	$x^2 = 2 \implies \dfrac{m^2}{n^2} = 2 \implies m^2 = 2n^2$.\\
	Оскільки $2n^2$ - це парне число, то $m^2$ - також парне, а тому $m$ - парне, тоді таке число представимо у вигляді $m = 2k, k \in \mathbb{Z}$.\\
	$4k^2 = 2n^2 \implies 2k^2 = n^2$\\
	Оскільки $2k^2$ - це парне число, то $n^2$ - також парне, а тому $n$ - парне, тоді таке число представимо у вигляді $n = 2l, l \in \mathbb{Z}$.\\
	Проте $m,n$ одночасно не можуть бути парними, оскільки ми отримаємо скоротиму дріб, а за умовою ми не брали таких. Суперечність!\\
	Отже, наше припущення було невірним.
	\end{proof}
	
	\begin{theorem}
	Множина раціональних чисел $\mathbb{Q}$ - зліченна.
	\end{theorem}

	\begin{proof}
	Нагадаю собі, що зліченною називають таку множину, де кожному числу ставиться відповідний номер. Коротше, ми можемо їх нумерувати. Вважаю, що доведення буде не самий rigorous, але тим не менш, підтвердити цей факт можна.\\
	Раціональні числа запишемо в такому порядку:
	\begin{figure}[H]
	\centering
	\begin{tikzpicture}[scale = 0.8, every node/.style={scale=0.8}]
	\foreach \y in {1,2,...,4} {
		\foreach \x in {1,...,4} {
			\node at ({2*\x},4-\y) {$\dfrac{-\x}{\y}$};
			\node at ({2*\x-1},4-\y) {$\dfrac{\x}{\y}$};
			}
			\node at (0, 4-\y) {$\dfrac{0}{\y}$};
		}
	\foreach \x in {0,1,...,8}
		\node at (\x,-1) {$\vdots$};
	\foreach \y in {0,...,3}
		\node at (9,\y) {$\dots$};
	\node at (9,-1) {$\ddots$};
	\end{tikzpicture}
	\end{figure}
	А тепер ми будемо проходитись по раціональних числам такою змійкою:
	\begin{figure}[H]
	\centering
	\begin{tikzpicture}[scale = 0.8, every node/.style={scale=0.8}]
		\draw[red, thick] (0,3)--(1,3)--(1,2)--(0,2)--(0,1)--(2,1)--(2,3)--(3,3)--(3,0)--(0,0)--(0,-1)--(4,-1)--(4,3)--(5,3)--(5,-1);
	\foreach \y in {1,2,...,4} {
		\foreach \x in {1,...,4} {
			\node at ({2*\x},4-\y) {$\dfrac{-\x}{\y}$};
			\node at ({2*\x-1},4-\y) {$\dfrac{\x}{\y}$};
			}
			\node at (0, 4-\y) {$\dfrac{0}{\y}$};
		}
	\foreach \x in {0,1,...,8}
		\node at (\x,-1) {$\vdots$};
	\foreach \y in {0,...,3}
		\node at (9,\y) {$\dots$};
	\node at (9,-1) {$\ddots$};
	\end{tikzpicture}
	\end{figure}
	І поки ми проходитимемо змійкою, ми будемо ігнорувати такі дроби, що повторюють числа. Наприклад, ми перетнули $\dfrac{1}{1}$, тоді $\dfrac{2}{2}, \dfrac{3}{3},\dots$ ігноруватимемо.\\
	Кожне число ми нумеруємо. Тобто дійсно, ми отримали, що $\mathbb{Q}$ - зліченна множина.
	\end{proof}		
	
	Саме ці два твердження є головною мотивацією розвивати нову множину. В грубому сенсі, це все означає, що множина $\mathbb{Q}$ - неповна множина, тобто на числовій прямій є \lq\lq дірки\rq\rq. І саме $\mathbb{R}$ прибирає ці самі  \lq\lq дірки\rq\rq . \\
	Є декілька варіантів, як конструювати цю множину. Я не буду цього робити, бо це вже інша історія. Ми же побудуємо множину дійсних чисел на основі певних аксіом.
	
	\subsection*{Модуль числа та ціла частина числа}
	\begin{definition}
	\textbf{Модулем числа} $a$ називають таку функцію:
	\begin{align*}
	|a| = \begin{cases} a, & a \geq 0 \\ -a, & a < 0 \end{cases}
	\end{align*}
	Число $|a|$ описує відстань між т. $0$ та числом $a$ на числовій прямій
	\end{definition}
	
	\begin{proposition}
	Справедливі такі властивості:\\
	1. $|ab| = |a| |b|$\\
	2. $|a|^2 = a^2$\\
	3. $|a| \leq c \iff -c \leq a \leq c$, якщо $c \geq 0$\\
	4. $-|a| \leq a \leq |a|$\\
	\textit{Доведення зрозуміле. Та й було вже таке в школі.}
	\end{proposition}
	
	\begin{definition}
	\textbf{Цілою частиною числа} $x$ називають найближче менше ціле число.\\
	Позначення: $[x]$
	\end{definition}
	
	\begin{example}
	$[2.5] = 2$, $[-\pi] = -4$, $[2022] = 2022$
	\end{example}
	Скоро ми дізнаємось, що ця штука завжди визначена, тобто можна від кожного числа брати цілу частину.
	
	\subsection*{Принцип математичної індукції}
	\begin{definition}
	Числова множина $E$ називається \textbf{індуктивною}, якщо
	\begin{align*}
		\forall x \in E: x+1 \in E
	\end{align*}
	\end{definition}
	
	\begin{theorem}
	Множина натуральних чисел $\mathbb{N}$ - мінімальна індуктивна множина, що містить $1$.
	\end{theorem}
	
	\begin{remark}
	Математично кажучи про множину  $\mathbb{N}$:\\
	$\forall E$ - індуктивна: $1 \in E \Rightarrow \mathbb{N} \subset E$.\\
	\end{remark}
	
	\begin{proof}
	1) Те, що $\mathbb{N}$ індуктивна, зрозуміло, тому що $\forall k \in \mathbb{N}: k+1 \in \mathbb{N}$.\\
	2) Оскільки $1 \in E$ і, більш того, вона є індуктивною, то $2 \in E, 3 \in E, \dots, k \in E$.\\
	А тому $\forall k \in \mathbb{N} \Rightarrow k \in E$.\\
	Таким чином, $\mathbb{N} \subset E$.
	\end{proof}
	
	\begin{corollary}[Принцип мат. індукції]
	Розглянемо числову множину $E = \{n \in \mathbb{N}: P(n)\}$. Тут $P(n)$ - це деяка умова.\\
	Тоді якщо $1 \in E$ та індуктивна, то $E = \mathbb{N}$.
	\end{corollary}
	
	\begin{proof}
	За умовою наслідка, маємо, що $E \subset \mathbb{N}$.\\
	Оскільки $1 \in E$ та індуктивна, то за попередньою теоремою, $\mathbb{N} \subset E$. Отже, $E = \mathbb{N}$.
	\end{proof}
	
	Про що цей наслідок: ми хочемо стверджитись, що $P(n)$ виконується при будь-яких $n \in \mathbb{N}$. Для цього треба зробити три кроки:\\
	\textbf{1. База індукції}\\
	Перевіряємо, що $P(1)$ виконується.
	\bigskip \\
	\textbf{2. Крок індукції}\\
	Вважаємо, що $P(n)$ - виконано. Показуємо, що $P(n+1)$ виконується.\\
	Двома кроками доводимо, що наша множина $E$ - індуктивна, що містить одиницю. Отже, МІ доведено, а тому $P(n)$ виконується завжди.
	
	\begin{example}
	Довести, що $1 + 2 + \dots + n = \dfrac{n(n+1)}{2}$.\\
	Тут множина $E = \left\{n \in \mathbb{N}: 1 + 2 + \dots + n = \dfrac{n(n+1)}{2} \right\}$.\\
	1. База індукції\\
	$1 \in E \Rightarrow 1 = \dfrac{1(1+1)}{2} = 1$
	\bigskip \\
	2. Крок індукції\\
	Нехай $k \in E$, тобто 
	$1 + 2 + \dots + k = \dfrac{k(k+1)}{2}$\\
	Доведемо, що $k+1 \in E$\\
	$1+ 2 + \dots + k + (k+1) = \dfrac{k(k+1)}{2} + k = \dfrac{k(k+1)+2k}{2} = \dfrac{(k+1)(k+2)}{2}$\\
	Отже, $k+1 \in E$.
	А значить, $E = \mathbb{N}$, тобто наша формула виконується $\forall n \in \mathbb{N}$. МІ доведено.
	\end{example}
	
	\begin{example}
	Довести, що $\forall n \in \mathbb{N} \setminus \{1\}: 2^n \geq n$.\\
	Тут множина $E = \{n \in \mathbb{N} \setminus \{1\}: 2^n \geq n \}$.\\
	1. База індукції\\
	$2 \in E \Rightarrow 2^2 \geq 2$
	\bigskip \\
	2. Крок індукції\\
	Нехай $k \in E$, тобто $2^k \geq k$.\\
	Доведемо, що $k+1 \in E$. Маємо\\
	$2^{k+1} = 2 \cdot 2^k \geq 2k = k + k > k+1$\\
	Отже, $k+1 \in E$, тобто $E = \mathbb{N} \setminus \{1\}$, тобто наше твердження виконується $\forall n \neq 1$. МІ доведено.
	\end{example}
	
	\subsection*{Основні нерівності}
	\begin{theorem}[Нерівність Бернуллі]
	Для всіх $x > -1$ виконано $(1+x)^n \geq 1+nx, \hspace{0.2cm} \forall n \geq 1$
	\end{theorem}
	
	\begin{pfMI}
	1. База індукції: при $n=1$: $(1+x)^1 \geq 1+1\cdot x$. Нерівність виконується.\\
	2. Крок індукції: нехай для фіксованого $n$ дана нерівність виконується. Доведемо для значення $n+1$.\\
	$(1+x)^{n+1}=(1+x)(1+x)^n \geq (1+x)(1+nx)=1+(n+1)x+nx^2 \geq 1+(n+1)x$\\
	Отже, така нерівність справедлива $\forall n \geq 1$. МІ доведено.
	\end{pfMI}
	
	\begin{theorem}[Нерівність Коші]
	Для всіх $a_1,\dots,a_n > 0$ виконано
	$\dfrac{a_1+\cdots+a_n}{n} \geq \sqrt[n]{a_1 \cdots a_n}, \hspace{0.2cm} \forall n \geq 1$
	\end{theorem}
	
	\begin{proof}
	Тимчасове перепозначення: $\huge A_n = \frac{a_1+\cdots+a_n}{n}, \hspace{1cm} G_n = \sqrt[n]{a_1 \cdots a_n}$.\\
	Зрозуміло, що $\huge \frac{A_n}{A_{n-1}} > 0 \Rightarrow \frac{A_n}{A_{n-1}}-1>-1$. \\ Тоді за нерівністю Бернуллі,
	$\huge \left(1+ \left(\frac{A_n}{A_{n-1}} -1 \right) \right)^n \geq 1 + n \cdot \left(\frac{A_n}{A_{n-1}} -1 \right)$
	$\Rightarrow \huge \frac{(A_n)^n}{(A_{n-1})^n} \geq \frac{a_n}{A_{n-1}}$\\
	$\Rightarrow \huge (A_n)^n \geq a_n (A_{n-1})^{n-1}$, $\forall n \geq 1$ \\ Тоді $(A_n)^n \geq a_n (A_{n-1})^{n-1} \geq \cdots \geq a_n a_{n-1} \cdots a_1$. \\ Отже, $A_n \geq G_n$, що й хотіли довести.
	\end{proof}
	
	\begin{theorem}[Нерівність трикутника]
	Для будь-яких чисел $x,y$ виконано $|x+y| \leq |x| + |y|$
	\end{theorem}
	
	\begin{proof}
	Із властивостей модуля, маємо, що\\
	$-|x| \leq x \leq x \hspace{1cm} -|y| \leq y \leq |y|$.\\
	Складемо ці нерівності - отримаємо:\\
	$-(|x|+|y|) \leq x+y \leq |x|+|y| \Rightarrow |x+y| \leq |x| + |y|$
	\end{proof}
	
	\begin{corollary}
	$|x-y| \leq |x| + |y|$\\
	\textit{Вказівка: $|x-y| = |x+(-y)|$.}
	\end{corollary}
	
	\begin{corollary}
	$||x|-|y|| \leq |x-y|$\\
	\textit{Вказівка: $|x| = |x-y+y|$, так само $|y| = |y-x+x|$.}
	\end{corollary}
	\newpage
	%\fi %IMPORTANT COMMENT
	
    
    %\iffalse %IMPORTANT COMMENT
	\section{Множина дійсних чисел}
	\subsection{Аксіоматика множини дійсних чисел, принцип Дедекінда}
	\textbf{Множину дійсних чисел} позначають за $\mathbb{R}$. Визначимо її так, щоб ми малі ті самі операції додавання, множення та відношення порядку як раніше:\\
	\textbf{- додавання}\\
	$a+b=b+a$ - комутативність;\\
	$(a+b)+c=a+(b+c)$ - асоціативність;\\
	$\exists 0 \in\mathbb{R}: a+0=a$ - існування нейтрального елементу;\\
	$\exists (-a) \in\mathbb{R}: a+(-a)=0$ - існування оберненого елементу.
	\bigskip \\
	
	\textbf{- множення}\\
	$a \cdot b=b \cdot a$ - комутативність;\\
	$(a \cdot b) \cdot c=a \cdot (b \cdot c)$ - асоціативність;\\
	$\exists 1 \in\mathbb{R}: a \cdot 1=a$ - існування нейтрального елементу;\\
	$\huge \exists \left(\frac{1}{a}\right) \in\mathbb{R} \setminus \{0\}: a \cdot \frac{1}{a}=1$ - існування оберненого елементу;\\
	$(a+b) \cdot c = a \cdot c + b \cdot c$ - дистрибутивність.
	\bigskip \\
	
	\textbf{- відношення порядку}\\
	Якщо $a \leq b$ та $b \leq a$, то $a = b$\\
	Якщо $a \leq b$ та $b \leq c$, то $a \leq c$\\
	Якщо $a \leq b$, то $a+c \leq b+c$\\
	Якщо $a \leq b$ та $c>0$, то $ac \leq bc$
	\bigskip \\	
	
	На відміну від раціональних чисел, в дійсних числах виникая нова аксіома.
	\bigskip \\
	\textbf{Аксіома неперервності. }\\
	Нехай є дві множини $A,B \subset \mathbb{R}$. Відомо, що $\forall a \in A$, $\forall b \in B: a \leq b$. Тоді $\exists c \in \mathbb{R}: a \leq c \leq b$
	\bigskip \\
	Тепер завдяки цьому, ми прибираємо \lq\lq дірки\rq\rq \text{} з числової прямої.
	\bigskip \\
	Надалі також ми будемо іноді користуватись множиною дійсних чисел, до якої ми додамо \lq\lq точки\rq\rq \text{} $-\infty$ та $+\infty$.
	$$\bar{\mathbb{R}} = \mathbb{R} \cup \{-\infty, +\infty\}$$
	Такі спеціальні значення, що $\forall x \in \mathbb{R}: -\infty < x < +\infty$.\\
	Також ми можемо продовжити визначення операцій:\\
	$x + (+\infty) = +\infty$, якщо $x \neq -\infty$ \hspace{1cm} $x + (-\infty) = -\infty$, якщо $x \neq +\infty$

\subsection{Точкові межі}
\begin{definition}
Задано множини $A,B \subset \mathbb{R}$.\\
	Множина $A$ називається \textbf{обмеженою зверху}, якщо
	\begin{align*}
	\exists c \in \mathbb{R}: \forall a \in A: a \leq c
	\end{align*}
	Множина $B$ називається \textbf{обмеженою знизу}, якщо
	\begin{align*}
	\exists d \in \mathbb{R}: \forall b \in B: b \geq d
	\end{align*}
\end{definition}
	Множину всіх чисел, що обмежують множину зверху, позначу за $UpA$, тобто
	\begin{align*}
	UpA = \{c \in \mathbb{R}: \forall a \in A: a \leq c \}
	\end{align*}
	Множину всіх чисел, що обмежують множину зверху, позначу за $DownB$, тобто
	\begin{align*}
	DownB = \{d \in \mathbb{R}: \forall b \in B: b \geq d \}
	\end{align*}
	
	\begin{example}
	Задано множину $A = \{1-2^{-n} | n \in \mathbb{N}\} = \left\{\dfrac{1}{2}, \dfrac{3}{4}, \dfrac{7}{8}, \dots \right\}$.\\
	Є обмеженою зверху числом $2 \in \mathbb{R}$, тобто $\forall a \in A: a < 2$.\\
	Є обмеженою знизу числом $0 \in \mathbb{R}$, тобто $\forall a \in A: a > 0$.
	\begin{figure}[H]
	\centering
	\begin{tikzpicture}[scale = 4]
	\draw[thick,->] (-0.1,0)--(2.1,0);
	%\draw[red] (1,-1pt)--(1,1pt) node at (1,-0.1) {$1$};
	\draw[red] ({1/2},-1pt)--({1/2},1pt) node at ({1/2},-0.12) {$\dfrac{1}{2}$};;
	\draw[red] ({3/4},-1pt)--({3/4},1pt) node at ({3/4}, -0.12) {$\dfrac{3}{4}$};
	\draw[red] ({7/8},-0.75pt)--({7/8},0.75pt) node at ({7/8}, -0.12) {$\dfrac{7}{8}$};
	\draw[red] ({15/16},-0.75pt)--({15/16},0.75pt);
	\draw[red] ({31/32},-0.5pt)--({31/32},0.5pt) node[black, anchor = south, scale=0.8]{$\dots$};
	\draw[red] ({63/64},-0.5pt)--({63/64},0.5pt);
	\draw[red] ({127/128},-0.25pt)--({127/128},0.25pt);
	\draw[red] ({255/256},-0.25pt)--({255/256},0.25pt);
	
	\draw[blue] (2,-1.5pt)--(2,1.5pt) node at (2,-0.1) {$2$};
	\draw[blue] (0,-1.5pt)--(0,1.5pt) node at (0,-0.1) {$0$};
	\end{tikzpicture}
	\end{figure}
	\end{example}
	Судячи з малюнку, ми розуміємо, що ми сильно грубо обмежили множину зверху та знизу. Ми хочемо більш точну межу. Для цього допоможе нам пару фактів.
	
	\begin{proposition}
	Якщо $c \in UpA$ та $c_1 > c$, то $c_1 \in UpA$.
	Якщо $d \in DownB$ та $d_1 < d$, то $d_1 \in DownB$.\\
	\textit{Обидва твердження випливають з визначення множин.}
	\end{proposition}
	
	\begin{remark}
	Множина $UpA$ обмежена знизу, а множина $DownB$ обмежена зверху.\\
	\textit{Випливає з означень обмеженості.}
	\end{remark}
	
	\begin{proposition}
	Для множини $UpA$ існує мінімальний елемент, а для множини $DownB$ існує максимальний елемент. Причому, вони єдині.
	\end{proposition}
	\begin{proof}
	Маємо множину $A$ та множину $UpA$ - всі числа, що обмежують зверху множину $A$.\\
	Тобто $\forall a \in A: \forall c \in UpA: a \leq c$. За аксіомою відокремленості, $\exists c' \in \mathbb{R}: a \leq c' \leq c \Rightarrow c' \in UpA$\\
	$\forall c \in UpA: c' \leq c \Rightarrow c' = \min UpA$\\
	Доведемо єдиність.\\
	!Припустимо, що $\exists c'' = \min UpA$.
	Але це автоматично не є можливо, оскільки якщо $c'' > c'$, то $c''$ не є більше мінімальним елементом, а якщо $c'' < c'$, то вже $c'$ не є мінімальним елементом. Суперечність!
	\bigskip \\
	Для $DownB$ доведення аналогічне.
	\end{proof}
	
	\begin{definition} Задано множини $A,B \subset \mathbb{R}$.\\
	\textbf{Точковою верхньою межею} називають таке число:
	\begin{align*}
	\sup A = \min UpA
	\end{align*}
	\textbf{Точковою нижньою межею} називають таке число:
	\begin{align*}
	\inf B = \max DownB
	\end{align*}
	\end{definition}
	
	\begin{theorem}[Критерій супремуму/інфімуму]
	$c' = \sup A \iff \begin{cases} 
	 \forall a \in A: a \leq c' \\
	 \forall \varepsilon > 0: \exists a_{\varepsilon} \in A: a_{\varepsilon} > c' - \varepsilon
	\end{cases}$ \hspace{0.1cm}
	$d' = \inf B \iff \begin{cases} 
	 \forall b \in B: b \geq d'\\
	 \forall \varepsilon > 0: \exists b_{\varepsilon} \in B: b_{\varepsilon} < d' + \varepsilon
	\end{cases}$
	\begin{figure}[H]
	\centering
	\begin{tikzpicture}
	\draw[thick, ->] (0,0)--(5,0) node at (1,0.5) {$A$};
	\draw[thick] (4,-3pt)--(4,3pt) node at (4.8,0.5) {$c' = \sup A$};
	\draw node at (3.5,0) {$($};
	\draw node at (3.5, -0.5) {$c' - \varepsilon$};
	\draw[thick, red] (3.5,0)--(4,0);
	\end{tikzpicture}
	\qquad
	\begin{tikzpicture}
	\draw[thick, ->] (0,0)--(5,0) node at (3.5,0.5) {$B$};
	\draw[thick] (2,-3pt)--(2,3pt) node at (1.5,0.5) {$d' = \inf B$};
	\draw node at (2.5,0) {$)$};
	\draw node at (2.8, -0.5) {$d' + \varepsilon$};
	\draw[thick, red] (2,0)--(2.5,0);
	\end{tikzpicture}
	\caption*{Другий пункт кожного критерію звучить так. Якщо я цей супрему зменшу на певну величину, то це не буде супремумом, а значить, знайдеться певний елемент, що буде його перевищувати. Аналогічно з інфімумом.}
	\end{figure}
	\end{theorem}
	
	\begin{proof}
	\rightproof Дано: $c' = \sup A$.\\
	Тоді автоматично $c' \in UpA$, тобто $\forall a \in A: a \leq c'$.\\
	Оскільки це мінімальне значення, то $\forall \varepsilon > 0: c' - \varepsilon \notin UpA \implies \exists a_{\varepsilon} \in A: a_{\varepsilon} > c' - \varepsilon$\\
	(остання умова - це заперечення означення обмеженості зверху).
	\bigskip \\
	\leftproof Дано: система з двох умов.\\
	З другої умови випливає, що не лише $c' \in UpA$, а ще й $c' = \min UpA = \sup A$.
	\bigskip \\
	Доведення інфімуму є аналогічним.
	\end{proof}
	
	\begin{example}
	Повернімось до множини $A = \{1-2^{-n} | n \in \mathbb{N}\} = \left\{\dfrac{1}{2}, \dfrac{3}{4}, \dfrac{7}{8}, \dots \right\}$.\\
	Доведемо, що $\sup A = 1$.\\
	Дійсно, $\forall a \in A: a = 1 - \dfrac{1}{2^n} < 1$.\\
	Залишилось довести, що $\forall \varepsilon > 0: \exists a_{\varepsilon}: a_{\varepsilon} > 1 - \varepsilon$.\\
	Або $\exists n: 1 - 2^{-n} > 1 -\varepsilon$.\\
	Або $1 - \dfrac{1}{2^n} > 1 - \dfrac{1}{n} > 1 - \varepsilon \Rightarrow \dfrac{1}{n} < \varepsilon \Rightarrow n > \dfrac{1}{\varepsilon}$.\\
	Можна обрати такий номер $n$, щоб $n > \dfrac{1}{\varepsilon}$, і тоді елемент з цим номером задовільнятиме умові.
	\bigskip \\
	Аналогічно доводиться, що $\inf A = \dfrac{1}{2}$.
	\begin{figure}[H]
	\centering
	\begin{tikzpicture}[scale = 4]
	\draw[thick,->] (0.1,0)--(1.1,0);
	%\draw[red] (1,-1pt)--(1,1pt) node at (1,-0.1) {$1$};
	\draw[red] ({1/2},-1pt)--({1/2},1pt) node at ({1/2},-0.2) {$\dfrac{1}{2}$};;
	\draw[blue] ({1/2},-2.5pt)--({1/2},-1pt);
	\draw[blue] ({1/2},1pt)--({1/2},2.5pt);
	\draw[red] ({3/4},-1pt)--({3/4},1pt) node at ({3/4}, -0.12) {$\dfrac{3}{4}$};
	\draw[red] ({7/8},-0.75pt)--({7/8},0.75pt) node at ({7/8}, -0.12) {$\dfrac{7}{8}$};
	\draw[red] ({15/16},-0.75pt)--({15/16},0.75pt);
	\draw[red] ({31/32},-0.5pt)--({31/32},0.5pt) node[black, anchor = south, scale=0.8]{$\dots$};
	\draw[red] ({63/64},-0.5pt)--({63/64},0.5pt);
	\draw[red] ({127/128},-0.25pt)--({127/128},0.25pt);
	\draw[red] ({255/256},-0.25pt)--({255/256},0.25pt);
	
	\draw[blue] (1,-2.5pt)--(1,2.5pt) node at (1,-0.2) {$1$};
	\end{tikzpicture}
	\end{figure}
	\end{example}
	
	\begin{remark}
	В цьому прикладі, до речі, $\inf A = \min A$, оскільки сам інфімум міститься на множині $A$. Водночас $\sup A \neq \max A$, тому що цей елемент не знаходиться на множині $A$.
	\end{remark}
	
	\begin{definition}
	Множина $F \subset \mathbb{R}$ називається \textbf{обмеженою}, якщо або вона є обмеженою зверху та знизу одночасно.
	\end{definition}
	
	\begin{remark}
	Означення того, що $F$ - \textbf{обмежена}, можна переписати в більш зручному вигляді:
	\begin{align*}
	\exists p>0: \forall f \in F: |f| \leq p
	\end{align*}
	\end{remark}
	Якщо $A$ не є обмеженою зверху, то вважаємо $\sup A = +\infty$.\\
	Якщо $B$ не є обмеженою знизу, то вважаємо $\inf B = -\infty$.\\
	
	\subsection{Принцип Архімеда та її наслідки}
	\begin{theorem}
	Множина натуральних чисел $\mathbb{N}$ не є обмеженою зверху.\\
	Математично кажучи, $\forall a \in \mathbb{R}: \exists n \in \mathbb{N}: n > a$.
	\end{theorem}
	
	\begin{proof}
	!Припустимо, що все ж таки обмежена зверху, тобто $\forall n \in \mathbb{N}: n \leq a$.\\
	Встановимо $\huge\sup \mathbb{N} = u$. За критерієм, зокрема для $\varepsilon = 1: \exists m \in \mathbb{N}: m > u-1 \Rightarrow u < m+1$\\
	Проте маємо, що натуральне число $m+1$ перевищує супремуму. Суперечність!
	\end{proof}
	
	\begin{corollary}
	Множина цілих чисел $\mathbb{Z}$ взагалі не обмежена
	\end{corollary}
	
	\begin{proof}
	Зафіксуємо два числа $a,-a \in \mathbb{R}$. Тоді за попередньою теоремою,\\
	$\exists n \in \mathbb{N} \Rightarrow n \in \mathbb{Z}: n > a$.\\
	$\exists m \in \mathbb{N}: m > (-a) \Rightarrow -m < a$, тут вже $-m \in \mathbb{Z}$.
	\end{proof}
	
	\begin{theorem}[Принцип Архімеда]
	$\forall x \in \mathbb{R}: \forall y > 0: \exists ! k \in \mathbb{Z}: (k-1)y \leq x < ky$
	\begin{figure}[H]
	\begin{tikzpicture}
	\draw[thick] (0,0)--(4.5,0) node[anchor = south] {$x = 4.5$};
	\draw[thick,red] (0,-0.6)--(1,-0.6) node[anchor = south] {$y = 1$};
	\draw[thick,red] (0,-0.6-0.05)--(0,-0.6+0.05);
	\draw[thick,red] (1,-0.6-0.05)--(1,-0.6+0.05);
	\end{tikzpicture}
	\qquad
	\begin{tikzpicture}
	\draw[thick] (0,0)--(4.5,0);
	\foreach \i in {0,1,2,3,4} {
		\draw[thick, red] (\i,0.2)--(\i+1,0.2);
		\draw[thick, red] (\i,0.15)--(\i,0.25);
		}
	\draw[thick, red] (5,0.15)--(5,0.25);
	\end{tikzpicture}
	\caption*{Принцип Архімеда каже, що знійдеться така кількість червоних відрізків, яку можна відкласти на чорну лінію, щоб довжина була менше лінії, а при додаванні наступного відрізка довжина буде більше лінії. Якщо число від'ємне, то можна вважати, що ми йдемо в інший напрямок.}
	\end{figure}
	\end{theorem}
	
	\begin{proof}
	Нехай маємо якийсь $x \in \mathbb{R}$, а також $y > 0$.\\
	Задамо множину $S = \left\{l \in \mathbb{Z} :  x < ly \right\}$ - множина всіх цілих чисел, щоб чорна лінія $x$ була менше за довжиною ніж сума червоних відрізків $y$ з кількістю $l$.\\
	Перепишемо інакше: $S = \left\{l \in \mathbb{Z} :  l > \dfrac{x}{y} \right\}$.\\
	Множина $S$ - обмежена знизу; не порожня, тому що зверху не є обмеженою. Отже, можемо мати $\inf S = m$ (поки не знаємо, що це якесь ціле число). \\
	За критерієм, $\exists k \in S \implies k \in \mathbb{Z}: m \leq k < m+1$. А тому $k = \min S$.\\
	Таким чином, $k \in S$, отримали, що $k > \dfrac{x}{y} \implies x < ky$\\
	Також тоді маємо, що $k-1 \not \in S$, тоді $k-1 \leq \dfrac{x}{y} \implies x \geq (k-1)y$.\\
	Остаточно: $(k-1)y \leq x < ky$.
	\end{proof}
	
	\begin{remark}
	Саме завдяки принципу Архімеда, ми можемо гарантувати коректність визначення цілої частини числа.\\
	Якщо $x \in \mathbb{R}$ та встановимо $y = 1$, то тоді $\exists !k \in \mathbb{Z}: k \leq x < k+1$. І тоді $k = [x]$.
	\end{remark}
	
	\begin{corollary}
	$\forall \varepsilon > 0: \exists n \in \mathbb{N}: \dfrac{1}{n} < \varepsilon$
	\end{corollary}
	
	\begin{proof}
	Встановимо $x = 1$, $y = \varepsilon$. Тоді за принципом Архімеда, $(n-1)\varepsilon \leq 1 < n\varepsilon \implies \dfrac{1}{n} < \varepsilon$.
	\end{proof}
	
	\begin{corollary}
	Задано таке число $a \geq 0$, для якого $\forall \varepsilon > 0: a < \varepsilon$. Тоді $a = 0$.
	\end{corollary}
	
	\begin{proof}
	!Припустимо, що $a \neq 0$, тобто $a > 0$. Тоді звідси $\dfrac{1}{a} > 0$\\
	Тоді за щойно отриманим наслідком, $\exists n: \dfrac{1}{n} < a$.\\
	Проте ми також маємо, що для $a < \dfrac{1}{n} = \varepsilon$. Суперечність!
	\end{proof}
	
	\begin{corollary}
	Задано такі два числа $a,b \in \mathbb{R}$, що $a < b$.\\
	Тоді в інтервалі $(a,b)$ знайдеться принаймні одне раціональне число.\\
	Математично кажучи, $\exists q \in \mathbb{Q}: a < q < b$.
	\end{corollary}
	
	\begin{proof}
	Оскільки $a<b$, то звідси $b-a>0$. Тоді $\exists n: \dfrac{1}{n} < b-a$.\\
	Визначимо $q = \dfrac{[na]+1}{n}$. Перевіримо, що таке раціональне число дійсно лежить в $(a,b)$.\\
	$q = \dfrac{\textcolor{red}{[na]}+1}{n} > \dfrac{\textcolor{red}{na-1}+1}{n} = a$\\
	$q = \dfrac{\textcolor{red}{[na]}+1}{n} < \dfrac{\textcolor{red}{na}+1}{n}= a + \dfrac{1}{n} < a + b - a = b$\\
	Отже, дійсно, $\exists q \in (a,b)$.
\end{proof}

	\begin{corollary}
	Задано такі два числа $a,b \in \mathbb{R}$, що $a < b$.\\
	Тоді в інтервалі $(a,b)$ знайдеться принаймні одне ірраціональне число.\\
	Математично кажучи, $\exists x \in \mathbb{R} \setminus \mathbb{Q}: a < q < b$
	\end{corollary}
	
	\begin{proof}
	Оскільки $a < b$, то звідси $\dfrac{a}{\sqrt{2}} < \dfrac{b}{\sqrt{2}}$. За попереднім наслідком, $\exists q \in \mathbb{Q}: \dfrac{a}{\sqrt{2}} < q < \dfrac{b}{\sqrt{2}}$.\\
	Тоді якщо $x = q \sqrt{2}$, то звідси $a < x < b$. А число $x \in \mathbb{R} \setminus \mathbb{Q}$, бо квадратний корінь є ірраціональним.
	\end{proof}
	
	\subsection{Відкриті, замкнені множини}
	\begin{definition}
	$\varepsilon$\textbf{-околом} точки $x$ будемо називати таку множину:
	\begin{align*}
	U_{\varepsilon}(x) = \{a \in \mathbb{R}: |x-a| < \varepsilon \} \overset{\text{або}}{=} (x-\varepsilon,x+\varepsilon)
	\end{align*}
	\end{definition}
	
	\begin{figure}[H]
	\centering
		\begin{tikzpicture}
		\draw[thick,->] (0,0)--(5,0);
		\fill[black] (2.5,0) circle (1pt) node [anchor = north] {$x$};
		\node[black] at (1,0) {$($};
		\node[black] at (4,0) {$)$};
		\node at (1,0) [anchor = north] {$x-\varepsilon$};
		\node at (4,0) [anchor = north] {$x+\varepsilon$};
		\end{tikzpicture}
	\end{figure}
	\textbf{Проколеним} $\varepsilon$\textbf{-околом} точки $x$ будемо називати таку множину:
	\begin{align*}
	\overset{\circ}{U}_{\varepsilon}(x) = U_{\varepsilon}(x) \setminus \{x\}
	\end{align*}
\begin{definition}
Задамо множину $A \subset \mathbb{R}$ та елемент $a \in A$.\\
Точку $a$ називають \textbf{внутрішньою}, якщо
\begin{align*}
\exists \varepsilon > 0: U_{\varepsilon}(a) \subset A
\end{align*}
А множина $A$ називається \textbf{відкритою}, якщо кожна її точка - внутрішня.
\end{definition}

\begin{example}
Розглянемо множини: $(a,b), [a,b], (a,+\infty), [a,+\infty), \emptyset, \mathbb{R}$.\\
$(a,b)$ - відкрита, оскільки $\forall x \in (a,b): \exists \varepsilon = \min\{|x-a|,|x-b|\}:  U_{\varepsilon}(x) \subset (a,b)$.\\
Тобто звідси кожна точка $x$ - внутрішня точка.
\bigskip \\
$[a,b]$ - НЕ відкрита.\\
!Припустимо, що $a$ - внутрішня точка, тоді $\exists \varepsilon > 0: (a-\varepsilon, a+\varepsilon) \subset [a,b]$, проте $a-\dfrac{\varepsilon}{2} \in (a - \varepsilon, a + \varepsilon)$ і водночас $a-\dfrac{\varepsilon}{2} \not \in [a,b]$, тому т. $a$ не може бути внутрішньою. Суперечність! \\
Аналогічні міркування для $b$. Решта - внутрішні, задавши той самий $\varepsilon$, як попередього разу.
\bigskip \\
$(a,+\infty)$ - відкрита, тому що $\forall x: \exists \varepsilon = |x-a|$.
\bigskip \\
$[a,+\infty)$ - НЕ відкрита через т. $a$: не є внутрішньою. Міркування аналогічні. Решта - внутрішні з тим самим $\varepsilon$.
\bigskip \\
$\emptyset$ - відкрита. Оскільки порожня множина не містить точок, ми не зможемо знайти точку в порожній множині, яка НЕ Є внутрішньою, щоб зруйнувати означення.
\bigskip \\
$\mathbb{R}$ - відкрита.
\bigskip \\
\end{example}

\begin{proposition}
Якщо $\{A_{\lambda}\}$ - сім'я зліченних відкритих підмножин, то $\huge \bigcup_{\lambda} A_{\lambda}$ - відкрита.
\end{proposition}

\begin{proof}
Візьмемо довільну точку $a \in \huge \bigcup_{\lambda} A_{\lambda} \Rightarrow$ принаймні одному з сімей множин $a \in A_{\lambda}$.\\
Така множина є відкритою, а тому $a$ - внутрішня точка.\\
Із нашого ланцюга отримаємо: $\forall a \in \huge \bigcup_{\lambda} A_{\lambda} \Rightarrow a - $ внутрішня. Тобто $\huge \bigcup_{\lambda} A_{\lambda}$ - відкрита.
\end{proof}

\begin{example}
Маємо $A = (1,2) \cup (4,16) \cup (32, 64)$. Попередньо ми знаємо, що будь-який інтервал є відкритою множиною. Тому їхнє об'єднання, тобто $A$, буде відкритою множиною.
\end{example}

\begin{definition}
Задамо множину $A \subset \mathbb{R}$ та елемент $a \in \mathbb{R}$.\\
Точку $a$ називають \textbf{граничною} множини $A$, якщо
\begin{align*}
\forall \varepsilon > 0: \exists x \in A: x \in \overset{\circ}{U}_{\varepsilon}(a)
\end{align*}
А множина $A$ називається \textbf{замкненою}, якщо вона містить всі граничні точки.
\end{definition}
Поки приклад наводити не буду, оскільки таким означенням не завжди зручно перевіряти на замкненість певну множину. Тож потрібне інше означення.
\begin{proposition}
$a$ - гранична точка $A \iff$ $\forall \varepsilon > 0: A \cap U_{\varepsilon}(a)$ - нескінченна множина.
\end{proposition}

\begin{proof}
\rightproof Дано: $a$ - гранична точка $A$.\\
!Припустимо, що $\exists \varepsilon^* > 0: A \cap (a-\varepsilon^*,a+\varepsilon^*)$ - скінченна, тобто\\
$x_1,\dots,x_n \in A \cap (a-\varepsilon^*,a+\varepsilon^*) \Rightarrow \begin{cases} |x_1-a| < \varepsilon^* \\ \vdots \\ |x_n-a| < \varepsilon^* \end{cases}$.\\
Оскільки $a$ - гранична т. $A$, то задамо $\varepsilon = \huge\min_{i = \overline{1,n}} |x_i-a|$. Тоді $\exists x \in A: x \neq a: x \in (a-\varepsilon,a+\varepsilon)$.\\
Проте це - неправда, оскільки ми отримали окіл ще менше, а при перетині ми не знайдемо жодної точки $x \neq a$. Суперечність!
\bigskip \\

\leftproof Дано: $\forall \varepsilon > 0: A \cap U_{\varepsilon}(a)$ - нескінченна множина.\\
Тоді $\forall \varepsilon > 0: \exists x \in A \cap U_{\varepsilon}(a)$. Зокрема $\exists x = a - \dfrac{\varepsilon}{2} \in A: x \neq a: |x-a| < \varepsilon$.\\
Отже, $a$ - гранична т. $A$.
\end{proof}

\begin{example}
Розглянемо множини: $(a,b), [a,b], (a,+\infty), [a,+\infty), \emptyset, \mathbb{R}$.\\
$(a,b)$ - НЕ замкнена.\\
Розглянемо т. $a$. Вона є граничною для множини $(a,b)$, оскільки\\
$\forall \varepsilon > 0: \exists x \in (a,b): x = a - \dfrac{\varepsilon}{2}: |x-a| < \varepsilon$\\
Для точки $b$ аналогічні міркування. Але множина $(a,b)$ не містить граничну т. $a,b$.
\bigskip \\

$[a,b]$ - замкнена, тому що $\forall x \in [a,b]: \forall \varepsilon > 0: [a,b] \cap (x-\varepsilon,x+\varepsilon) = \left[
\begin{gathered}
\left[a,x+\varepsilon \right) \\
\left(x-\varepsilon, x + \varepsilon \right) \\
\left(x - \varepsilon, b\right] \\
\left[a,b\right]
\end{gathered}
 \right.$ - всі вони нескінченні множини.
\bigskip \\
$(a,+\infty)$ - НЕ замкнена, тому що точка $a$ - гранична для $(a,+\infty)$, але множині не належить.
\bigskip \\
$[a,+\infty)$ - замкнена (аналогічно).
\bigskip \\
$\emptyset$ - замкнена: вона містить всі свої граничні точки, яких просто нема.
\bigskip \\
$\mathbb{R}$ - замкнена.
\end{example}

\begin{proposition}
$A$ - відкрита множина $\iff \overline{A}$ - замкнена множина
\end{proposition}

\begin{proof}
\rightproof Дано: $A$ - відкрита множина.\\
!Припустімо, що $\overline{A}$ - НЕ замкнена множина, тобто вона містить НЕ всі свої граничні точки, тобто $\exists a' \in A$, яка буде граничною для $\overline{A}$\\
Оскільки $a' \in A$, то вона є внутрішньою, тобто $\exists \varepsilon > 0: (a'-\varepsilon,a'+\varepsilon) \subset A \Rightarrow (a'-\varepsilon,a'+\varepsilon) \cap \overline{A} = \emptyset$. Суперечність! Бо тут, навпаки, не має виконуватись рівність.
\bigskip \\
\leftproof Дано: $\overline{A}$ - замкнена множина.\\
!Припустімо, що $A$ - НЕ відкрита множина, тобто $\exists a \in A$, яка НЕ є внутрішньою, тобто\\
$\forall \varepsilon > 0: U_{\varepsilon}(a) \not\subset A \Rightarrow U_{\varepsilon}(a) \cap \overline{A} \neq \emptyset$, тобто $a$ - гранична точка $\overline{A}$.\\
Оскільки $\overline{A}$ - замкнена, то вона містить всі свої граничні точки, проте $a \not\in \overline{A}$. Суперечність!
\end{proof}

\begin{remark}
Факт: єдині множини, які є одночасно відкритими та замкненими, - це $\emptyset, \mathbb{R}$.
\end{remark}

\begin{proposition}
Якщо $\{A_{\lambda}\}$ - сім'я замкнених підмножин, то $\huge \bigcap_{\lambda} A_{\lambda}$ - замкнена.\\
\textit{Випливає із} \textbf{Prp. 1.4.4.}, \textbf{Prp. 1.4.9.} \textit{та правила де Моргана.}
\end{proposition}
	
	\subsection{Основні твердження аналізу}
	\begin{theorem}[Лема Кантора про вкладені відрізки]
	Задано відрізки таким чином: $\forall n \geq 1: [a_n, b_n] \supset [a_{n+1}, b_{n+1}]$. Тоді:\\
	1) $\exists c \in \mathbb{R}: \forall n \geq 1: c \in [a_n,b_n]$\\
	2) Якщо додатково $\forall \varepsilon > 0: \exists N \in \mathbb{N}: b_N - a_N < \varepsilon$, то тоді така точка - єдина.
	\begin{figure}[H]
	\centering
	\begin{tikzpicture}
	\draw[thick, ->] (0,0)--(6,0);
	\draw node at (0.5,0) {$[$}; \draw node at (5.5,0) {$]$};
	\draw node at (1.2,0) {$[$}; \draw node at (4.5,0) {$]$};
	\draw node at (2.8,0) {$[$}; \draw node at (3.5,0) {$]$};
	
	\draw node at (0.5,-0.5) {$a_1$}; \draw node at (5.5,-0.5) {$b_1$};
	\draw node at (1.2,-0.5) {$a_2$}; \draw node at (4.5,-0.5) {$b_2$};
	\draw node at (2.8,-0.5) {$a_3$}; \draw node at (3.5,-0.5) {$b_3$};
	
	\node at (3.2,0.25) {$\dots$};
	\end{tikzpicture}
	\end{figure}
	\end{theorem}
	
	\begin{proof}
	1) Із умови випливає, що $\forall n,m \in \mathbb{N}:$\\
	$a_1 \leq a_2 \leq \dots \leq a_n \leq \dots < \dots \leq b_n \leq \dots \leq b_2 \leq b_1$\\
	Отже, $\forall n,m \in \mathbb{N}: a_n \leq b_m$.\\
	Розглянемо множини $A = \{a_1,\dots,a_n\}, B = \{b_1, \dots, b_m\}$.
	Тоді за принципом Дедекінда, $\exists c \in \mathbb{R}: \forall n,m \in \mathbb{N}: a_n \leq c \leq b_m$.
	Таким чином, $\forall n \geq 1: c \in [a_n,b_n]$.
	\bigskip \\
	2) Розглянемо окремо, коли $\forall \varepsilon > 0: \exists N: b_N - a_N < \varepsilon$.\\
	!Припустимо, що $\exists c' \in \mathbb{R}: \forall n \geq 1: c' \in [a_n,b_n]$, але $c \neq c'$\\
	Задамо $\varepsilon = |c' - c| > 0$.
	Тоді $\exists N: b_N - a_N < \varepsilon$\\
	Але $c,c' \in [a_N,b_N]$, тому $\varepsilon = |c'-c| < a_n-b_n < \varepsilon$. Суперечність!\\
	Отже, така точка - єдина.
	\end{proof}
	
	\begin{theorem}[Лема Больцано-Вейєрштрасса]
	Задано множину $A$ - обмежена множина з нескінченною кількістю елементів. Тоді вона містить принаймні одну граничну точку.
	\end{theorem}
	
	\begin{proof}
	Оскільки $A$ - обмежена, то
	$\begin{cases} \exists a \in \mathbb{R}: \forall x \in A: x \geq a \\
	  \exists b \in \mathbb{R}: \forall x \in A: x \leq b \end{cases}$\\
	Тобто маємо множину $[a,b] \supset A$.\\
	Розіб'ємо множину $[a,b]$ навпіл: $\left[a, \dfrac{a+b}{2}\right]$ та $\left[\dfrac{a+b}{2},b \right]$.\\
	Оскільки $A$ має нескінченну кількість чисел, то принаймні одна з множин $\left[a, \dfrac{a+b}{2}\right] \cap A$ або $\left[\dfrac{a+b}{2}, b\right] \cap A$ - нескінченна множина. Ту половину позначимо за множину $[a_1,b_1]$ (якщо обидва нескінченні, то вибір довільний). Тоді $A \cap [a_1,b_1]$ - нескінченна множина.\\
	Розіб'ємо множину $[a_1,b_1]$ навпіл: $\left[a_1, \dfrac{a_1+b_1}{2}\right]$ та $\left[\dfrac{a_1+b_1}{2},b_1 \right]$.\\
	І за аналогічними міркуваннями одна з множин нескінченна, позначу за $[a_2,b_2]$. Тоді $A \cap [a_2,b_2]$ - нескінченна множина.\\
	Розіб'ємо множину $[a_2,b_2]$ навпіл: $\left[a_2, \dfrac{a_2+b_2}{2}\right]$ та $\left[\dfrac{a_2+b_2}{2},b_2 \right]$.\\
	$\vdots$\\
	В результаті матимемо вкладені відрізки: $[a,b] \supset [a_1,b_1] \supset [a_2,b_2] \supset \dots$\\
Причому, $\forall n \geq 1: b_n - a_n = \dfrac{b-a}{2^n}$.\\
	Зафіксуємо $\varepsilon > 0$ та перевіримо, чи існує $N$, що $b_N - a_N < \varepsilon$.\\
	Маємо: $b_N - a_N = \dfrac{b-a}{2^N} < \dfrac{b-a}{N} < \varepsilon \implies N > \dfrac{b-a}{\varepsilon}$\\
	Отже, маємо $N = \left[ \dfrac{b-a}{\varepsilon} \right]+1$, для якого нерівність $b_N-a_N < \varepsilon$ виконано. Тоді за лемою Кантора, $\exists! c \in \mathbb{R}: \forall n \geq 1: c \in [a_n,b_n]$.
	\bigskip \\
	А далі покажемо, що $c$ - дійсно гранична точка множини $A$.\\
	Зафіксуємо $\varepsilon > 0$. Знайдемо, чи існує $N$, щоб $b_N - a_N = \dfrac{b-a}{2^N} < \dfrac{\varepsilon}{2} \implies \dots \implies N > \dfrac{2(b-a)}{\varepsilon}$\\
	Тоді $[a_N,b_N] \subset (c-\varepsilon, c+\varepsilon)$, оскільки $c-a_N \leq \dfrac{\varepsilon}{2}$ та $b_N -c \leq \dfrac{\varepsilon}{2}$. І це все виконується $\forall \varepsilon > 0$.\\
	Таким чином, $A \cap (c-\varepsilon, c+\varepsilon) \supset A \cap [a_n,b_n]$ - нескінченна множина. Отже, $c$ - гранична точка $A$.
	\end{proof}
	
	\begin{definition}
	Задано множину $A \subset \mathbb{R}$.\\
	Система множин $\{U_{\alpha}\}$ називається \textbf{покриттям} множини $A$, якщо
	\begin{align*}
	A \subset \bigcup_{\alpha} U_{\alpha}
	\end{align*}
	\end{definition}
	
	\begin{example}
	Відрізок $[2,3]$ може мати покриття $\{\textcolor{red}{(1,2.5)}, \textcolor{blue}{[2.1,2.8)}, \textcolor{green}{[2.5,3]}\}$.
	\begin{figure}[H]
	\centering
	\begin{tikzpicture}
	\draw[thick,->] (0,0)--(4,0);
	\node at (2,0) {$[$};
	\node at (3,0) {$]$};
	\node at (2,-0.5) {$2$};
	\node at (3,-0.5) {$3$};
	
	\draw[red] (1,0.1)--(2.5,0.1);
	\draw[blue] (2.1,0.1)--(2.8,0.1);
	\draw[green] (2.5,0.1)--(3,0.1);
	\end{tikzpicture}
	\end{figure}
	\end{example}
	
	\begin{theorem}[Лема Гейне-Бореля]
	Будь-який відрізок можна покрити скінченною кількістю інтервалів.
	\end{theorem}
	
	\begin{proof}
	Задано відрізок $[a,b]$. Треба довести, що є такий набір інтервалів $U_k, k = \overline{1,n}$, де їхня кількість - скінченна.\\
	!Припустимо, що $[a,b]$ покривається лише нескінченною кількістю інтервалів. \\ 
	Ідея доведення є майже аналогічним з лемою Больцано-Вейєрштраса. Ми ділимо відрізок навпіл. Після ділення ми обираємо той відрізок, який покривається нескінченною кількістю інтервалів. Із обраним відрізком робимо те саме.\\
	Матимемо знову вкладені відрізки $[a,b] \supset [a_1,b_1] \supset [a_2,b_2] \supset \dots$. Причому, $b_n - a_n = \dfrac{b-a}{2^n}$. Ми вже доводили, що $\exists! c \in \mathbb{R}: \forall n \geq 1: c \in [a_n,b_n]$.\\
	Оскільки $c \in [a_1,b_1]$, то тоді знайдеться інтервал $U = (\alpha,\beta) \ni c$ - один із інтервалів покриття.\\
	Нехай задамо $\varepsilon = \min \{ c-\alpha, \beta-c \}$. Тоді ми можемо завжди знайти номер $N$, щоб $b_N - a_N = \dfrac{b-a}{2^N} < \varepsilon$ (аналогічна процедура).
	\begin{figure}[H]
	\centering
	\begin{tikzpicture}
	\draw[thick] (0,0)--(4.5,0);
	\node at (1,0) {$($}; \node at (1,-0.5) {$\alpha$};
	\node at (4,0) {$)$}; \node at (4,-0.5) {$\beta$};
	\fill (2,0) circle (1pt) node [anchor = south] {$c$};
	\node[scale = 0.8] at (1.85,0) {$[$}; \node[scale = 0.8] at (1.85,-0.25) {$a_N$};
	\node[scale = 0.8] at (2.3,0) {$]$}; \node[scale = 0.8] at (2.3,-0.25) {$b_N$};
	\end{tikzpicture}
	\end{figure}
	Звідси випливає, що $[a_N,b_N] \subset (\alpha, \beta)$. Тобто відрізок покривається одним інтервалом. Проте ми казали, що це неможливо. Суперечність!
	\end{proof}
		
	\begin{theorem}
	Множина дійсних чисел $\mathbb{R}$ - незліченна.
	\end{theorem}
	
	\begin{proof}
	Для початку перевіримо, що відрізок $I = [0,1]$ - незліченна множина.\\
	!Припустимо, що $I = \{x_1,x_2,x_3,\dots\}$, тобто зліченна множина.\\ Розіб'ємо $I$ на три (не обов'язково рівні) частини. Тоді принаймні в одному з розбитті не потрапить число $x_1$. Саме число $x_1$ може бути або в одному з трьох відрізків, або навіть одночасно в двох. Саме тому ми ділимо на три частини. Тому позначимо той відрізок, що не має $x_1$ як відрізок $I_1$\\
	Розіб'ємо $I_1$ на три частини. Аналогічно, знайдеться відрізок, де не буде числа $x_2$. Позначимо цей відрізок $I_2$.\\
	Розіб'ємо $I_2$ на три частини. І знову, є відрізок $I_3$, куди не потрапило число $x_3$.\\
	$\vdots$\\
	Отримали систему вкладених відрізков $I \subset I_1 \subset I_2 \subset I_3 \subset \cdots$
	\begin{figure}[H]
	\centering
	\begin{tikzpicture}[scale = 4]
	\draw[thick, ->] (-0.5,0)--(1.5,0);
	\node at (0,0) {$[$}; \node at (0,-0.1) {$0$};
	\node at (1,0) {$]$}; \node at (1,-0.1) {$1$};
	\draw ({1/3},-1pt)--({1/3},1pt);
	\draw ({2/3},-1pt)--({2/3},1pt);
	\fill (0.7,0) circle (0.25pt) node at (0.75,-0.1) {$x_1$};
	\fill (0.85,0) circle (0.25pt) node at (0.9,-0.1) {$x_2$};
	\draw ({1/9},-0.75pt)--({1/9},0.75pt);
	\draw ({2/9},-0.75pt)--({2/9},0.75pt);
	\draw ({7/27},-0.5pt)--({7/27},0.5pt);
	\draw ({8/27},-0.5pt)--({8/27},0.5pt);
	\fill ({85/270},0) circle (0.25pt) node at ({1/3},-0.1) {$x_3$};
	\draw ({22/81},-0.25pt)--({22/81},0.25pt);
	\draw ({23/81},-0.25pt)--({23/81},0.25pt);
	\end{tikzpicture}
	\end{figure}
	Тоді за лемою Кантора, знайдемо якусь т. $c$, яка належить будь-якому відрізку.\\
	$c \in I_1 \implies c \neq x_1$, $c \in I_2 \implies c \neq x_2$, $c \in I_3 \implies c \neq x_3$, $\dots$\\
	Можна зробити висновок, що $\forall n \geq 1: c \neq x_n$, а тому точка $c$ не має нумерації. Суперечність!\\
	Отже, $[0,1]$ - незліченна множина, а тому тим паче $[0,1] \subset \mathbb{R}$ - незліченна множина.
	\end{proof}
	\newpage
	\hspace{2cm}
	\newpage
	%\fi %IMPORTANT COMMENT
	
	%\iffalse %IMPORTANT COMMENT
	\section{Границі числової послідовності}
	\subsection{Основні означення}
	\begin{definition}
	Неформально \textbf{числовою послідовністю} називають якийсь набір чисел  \\ $\{a_n, n \geq 1\}$.\\
	А формально називають це відображенням $a: \mathbb{N} \to \mathbb{R}$, де $a(n) = a_n$. Кожному номеру послідовності зіставляється певне число.
	\end{definition}
	
	\begin{example}
	Розглянемо декілька прикладів, де послідовності по-різному задаються:\\
	1. $\{a_n, n \geq 1\} = \{\underset{a_1}{1},\underset{a_2}{2},\underset{a_3}{-1},\underset{a_4}{3},\underset{a_5}{0},\dots\}$ - якась рандомна послідовність;\\
	2. $\{b_n, n \geq 1\}$, де $b_n = \dfrac{1}{n}$ - послідовність, що задається формулою;\\
	3. $\{c_n, n \geq 1\}$, де $c_1 = 1$, а також $c_{n+1} = \dfrac{c_n}{n+1}$ - рекурсивна послідовність.
	\end{example}
	
	\begin{remark}
	Я надалі буду користуватись формальним означенням послідовності. А це означає, що кожний член послідовності буде визначений. Тобто не буде казусів як в числової послідовності $\{a_n, n \geq 1\}, a_n = \dfrac{1}{n-2022}$, де 2022-й елемент не є визначеним.
	\end{remark}
	
	\begin{definition} Задано послідовність $\{a_n, n \geq 1\}$.
	Число $a$ називається \textbf{границею числової послідовності}, якщо
	\begin{align*}
	\forall \varepsilon > 0: \exists N(\varepsilon) \in \mathbb{N}: \forall n \geq N: |a_n - a| < \varepsilon
	\end{align*}
	Позначення: $\displaystyle \lim_{n \to \infty} a_n = a$ або $a_n \overset{n \to \infty}{\longrightarrow} a$.\\
	Якщо в послідовності існує чисельна границя, тобто $a \in \mathbb{R}$, то така послідовність називається \textbf{збіжною}. В інакшому випадку - \textbf{розбіжною}.
	\end{definition}
	
	\begin{theorem}
	Для збіжної послідовності існує єдина границя.
	\end{theorem}
	
	\begin{proof}
	!Припустимо, що задано збіжну числову послідовність $\{a_n, n \geq 1\}$, для якої існують дві границі:\\
	$\displaystyle \lim_{n \to \infty} a_n = a_1, \lim_{n \to \infty} a_n = a_2$.\\
	Врахуємо, що $a_1<a_2$. Для $a_1>a_2$ міркування є аналогічними.\\
	Оскільки границі існують, ми можемо задати $\displaystyle \varepsilon = \frac{a_2-a_1}{3}$. Тоді:\\
	$\displaystyle \exists N_1: \forall n \geq N_1: |a_n-a_1|< \frac{a_2-a_1}{3} \Rightarrow a_n < a_1 + \frac{a_2-a_1}{3}$\\
	$\displaystyle \exists N_2: \forall n \geq N_2: |a_n-a_2|< \frac{a_2-a_1}{3} \Rightarrow a_n > a_2 - \frac{a_2-a_1}{3}$\\
	Аби обидві нерівності працювали одночасно, ми зафіксуємо новий $N= \max\{N_1,N_2\}$. Тоді:\\
	$\displaystyle \forall n \geq N: a_n < \frac{a_1+(a_1+a_2)}{3} < \frac{a_2+(a_1+b_2)}{3}<a_n$. Суперечність! \\ 
	Отже, обидва різних ліміти не можуть існувати одночасно.
	\end{proof}
	
	\begin{example}
	Доведемо за означенням, що $\displaystyle\lim_{n \to \infty} \frac{1}{n} = 0$.\\
	Задано довільне $\varepsilon > 0$. Необхідно знайти $\displaystyle N: \forall n \geq N: \left|\frac{1}{n}-0 \right|<\varepsilon$.\\
	$\huge \abs{\frac{1}{n} - 0} < \varepsilon \iff \displaystyle \frac{1}{n}<\varepsilon \iff n > \frac{1}{\varepsilon}$\\
	Зафіксуймо $\displaystyle N = \left[\frac{1}{\varepsilon} \right] + 1$. Можна, до речі, сказати, що існують $N$, які задовільняють нерівності нижче, тобто не вказувати конкретне значення. Тоді маємо:\\
	$\forall \varepsilon > 0: \exists N = \huge \left[\frac{1}{\varepsilon} \right] + 1: \forall n \geq N: n > \frac{1}{\varepsilon} \implies \abs{\frac{1}{n} - 0} < \varepsilon$\\
	Отже, означення виконується, тому $\displaystyle\lim_{n \to \infty} \frac{1}{n} = 0$.
	\begin{figure}[H]
	\centering
	\resizebox{0.8\textwidth}{!} {
	\begin{tikzpicture}
	\draw[thick, ->] (-2,0)--(17,0) node[anchor = north west] {$a_n = \dfrac{1}{n}$};
	\foreach \i [evaluate=\i as \x using int(16/ \i)] in {16,8,4,2,1}
		\filldraw (\i,0) circle (1pt) node[anchor = north] {$\dfrac{1}{\x}$};
	\draw (0,-1pt)--(0,1pt) node[anchor = north] {$0$};
	\node[red] at (1.6,0) {$)$};
	\node[red] at (-1.6,0) {$($};
	\node[anchor = south, red] at (1.6,0) {$0+\varepsilon$};
	\node[anchor = south, red] at (-1.6,0) {$0-\varepsilon$};
	\end{tikzpicture}
}
	\caption*{Тут на малюнку я обрав $\varepsilon = 0.1$. Також я не всі елементи послідовності позначив. Тоді починаючи з $n=11$ (або з $12$, $13$,...), всі решта члени не покидатимуть червоні дужки. Якщо члени не будуть покидати ці лінії для будь-якого заданого $\varepsilon$, то тоді границя існує.}
	\end{figure}
	\end{example}

	\begin{example}
	Доведемо за означенням, що $\displaystyle\lim_{n \to \infty} \sqrt[n]{n}=1$.\\
	Знову задамо довільне $\varepsilon > 0$. Знову необхідно знайти $\displaystyle N: \forall n \geq N: \left|\sqrt[n]{n}-1  \right|<\varepsilon \iff \sqrt[n]{n}<1+\varepsilon$.\\
	Використовуючи нерівність Коші, ми отримаємо таку оцінку:\\
	$\displaystyle \sqrt[n]{n}= \sqrt[n]{\sqrt{n}\cdot\sqrt{n}\cdot 1 \cdots 1} \leq \frac{\sqrt{n}+\sqrt{n}+1+\cdots+1}{n} = \frac{2\sqrt{n}+n-2}{n} = \frac{2}{\sqrt{n}}+1-\frac{2}{n}<\frac{2}{\sqrt{n}}+1$. Тоді:\\
	$\displaystyle \sqrt[n]{n} < \frac{2}{\sqrt{n}} + 1 < 1 + \varepsilon \iff \frac{2}{\sqrt{n}} < \varepsilon \iff n > \frac{4}{\varepsilon^2}$\\
	Тепер зафіксуємо $\displaystyle N = \left[\frac{4}{\varepsilon^2} \right] + 2021$. Тоді $\forall n \geq N$ всі нерівності виконуються, зокрема $\left|\sqrt[n]{n}-1  \right|<\varepsilon$.\\
	Остаточно: $\displaystyle\lim_{n \to \infty} \sqrt[n]{n}=1$.
	\end{example}
	
	\begin{example}
	Доведемо, що не існує $\displaystyle \lim_{n \to \infty} (-1)^n$.\\
	Запишемо заперечення до означення збіжної границі:\\
	$\exists \varepsilon^* > 0: \forall N: \exists n \geq N: |a_n - a| \geq \varepsilon^*$.\\
	Встановимо $\varepsilon^* = |1+a|$. Тоді $\forall N: \exists n = 2N+1 : |(-1)^n - a|= |-1-a| = |1+a| \geq \varepsilon$.\\
	Отже, ми порушили означення. Тоді дійсно, маємо розбіжну послідовність.
\begin{figure}[H]
\centering
\resizebox{0.5\textwidth}{!} {
\begin{tikzpicture}

\draw[thick, ->] (-2,0)--(2,0) node[anchor = north west] {$a_n = (-1)^n$};
\filldraw (1,0) circle (1pt) node[anchor = north] {$1$};
\filldraw (-1,0) circle (1pt) node[anchor = north] {$-1$};
\node[anchor = south, red] at (1+0.8,0) {$1+\varepsilon$};
\node[anchor = south, red] at (1-0.8,0) {$1-\varepsilon$};
\node[red] at (1+0.5,0) {$)$};
\node[red] at (1-0.5,0) {$($};
\end{tikzpicture}
}
\caption*{Тут на малюнку я встановил границю $a=1$. Лише для деяких $\varepsilon$ всі члени потраплятимуть всередину. Однак, скажімо, не для $\varepsilon = 0.5$ як на малюнку - ось чому ліміт не може бути рівним $1$. І так для кожного $a$.}
\end{figure}
	\end{example}

	\begin{definition}
	Послідовність $\{a_n, n \geq 1\}$ називається \textbf{обмеженою}, якщо
	\begin{align*}
	\exists C>0: \forall n \geq 1: |a_n|\leq C
	\end{align*}
	\end{definition}
	
	\begin{theorem}
	Будь-яка збіжна послідовність є обмеженою.
	\end{theorem}
	
	\begin{proof}
	Нехай задано збіжну послідовність $\{a_n, n \geq 1\}$, тобто для неї\\ $\displaystyle \exists \lim_{n \to \infty} a_n = a \overset{\textrm{def.}}{\iff}\forall \varepsilon > 0: \exists N(\varepsilon): \forall n \geq N: |a_n-a| < \varepsilon$.\\
	Оскільки ліміт існує, то задамо $\varepsilon = 1$. Тоді: $\forall n \geq N: \abs{a_n-a}<1$
	Спробуємо оцінити вираз $|a_n|$ для нашого бажаного:\\
	$|a_n| = |a_n - a + a| \leq |a_n-a|+|a| < 1 + |a|$. Це виконується $\forall n \geq N$. Інакше кажучи, всі числа, починаючи з $N$, є обмеженими.\\
	Покладемо $C=\max\{|a_1|,|a_2|,\cdots, |a_{N-1}|, 1+|a|\}$. Тоді отримаємо, що $\forall n\geq1: |a_n|\leq C$ \\ 
	Отже, числова послідовність - обмежена.
	\end{proof}
	
	\begin{remark}
	Обернене твердеження не є вірним. Підтверджується \textbf{Ex. 2.1.7.}
	\end{remark}
	
	\begin{definition}
	Послідовність $\{a_n, n \geq 1\}$ \textbf{має границю} $\infty$, якщо: 			\begin{align*}
	\forall E>0: \exists N(E) \in \mathbb{N}: \forall n \geq N: |a_n|>E
	\end{align*}
	Якщо $+\infty$, то $a_n > E$. \hspace{0.5cm} Якщо $-\infty$, то $-a_n > E$.
	\end{definition}
	
	\begin{example}
	Доведемо за означенням, що $\displaystyle\lim_{n \to \infty} 2^n = +\infty$.\\
	Задано довільне $E>0$. Необхідно знайти $N: \forall n \geq N: 2^n>E$.\\
	Доводили, що $2^n \geq n$. Вимагатимемо тепер, щоб $n > E$.\\
	Фіксуймо $N=\left[ E \right] + 2$. Тоді $\forall n \geq N: n > E$, а тим паче $2^n > n > E$\\
	Тому $\displaystyle\lim_{n \to \infty} 2^n = +\infty$.
	\begin{figure}[H]
\centering
\resizebox{0.9\textwidth}{!} {
\begin{tikzpicture}
\draw[thick, ->] (-2,0)--(17,0) node[anchor = north west] {$a_n = 2^n$};
\foreach \i in {16,8,4,2}
	\filldraw (\i,0) circle (1pt) node[anchor = north] {$\i$};
\node[red] at (6,0) {$|$};
\node[anchor = south, red] at (6,0) {$E$};
\end{tikzpicture}
}
\caption*{Тут на малюнку $E = 6$. Тоді починаючи з $n=3$ (або з $4$, $5$,...), всі решта члени будуть правіше за червону лінію.}
\end{figure}
	\end{example}

\begin{example}
Доведемо, що $\huge \lim_{n \to \infty} (-1)^n 2^n = \infty$.\\
Задано довільне $E > 0$. Необхідно знайти $N: \forall n \geq N: |(-1)^n 2^n| = 2^n > E$.
Але це ми вже доводили зверху. Важливо тут те, що не можна визначитись, чи $+\infty$, чи $-\infty$ через знакочередованість.
\end{example}
	
	
	\subsection{Нескінченно малі/великі послідовності}
	\begin{definition}
	Задано послідовність $\{a_n, n \geq 1\}$.\\
	Вона називається \textbf{нескінченно малою (н.м.)}, якщо $\huge\lim_{n \to \infty} a_n = 0$.\\
	Вона називається \textbf{нескінченно великою (н.м.)}, якщо $\huge\lim_{n \to \infty} a_n = \infty$.\\
	\end{definition}
	
	\begin{example}
	Зокрема $a_n = \dfrac{1}{n}$ є нескінченно малою, а $a_n = 2^n$ є нескінченно великою, виходячи з минулих прикладів.
	\end{example}
	
	\begin{theorem}[Арифметика послідовностей н.м. та н.в.]
	Задано п'ять різні послідовності: \\
	$\{a_n\}$ - н.м. \hspace{0.5cm} $\{b_n\}$ - н.м. \hspace{0.5cm} $\{c_n\}$ - обмежена \hspace{0.5cm} $\{d_n\}$ - н.в. \\ $\{p_n\}$ - послідовність, що віддалена від нуля, тобто $\exists \delta>0: \forall n\geq 1: |p_n|\geq \delta$\\
	Тоді маємо нові шість послідовності: \\ $\{a_n+b_n\}$ - н.м. \hspace{0.5cm} $\forall C \in \mathbb{R}: \{C a_n\}$ - н.м. \hspace{0.5cm} $\{c_n \cdot a_n\}$ - н.м. \\ $\left\{ \dfrac{1}{a_n} \right\}$ - н.в. \hspace{0.5cm} $\left\{ \dfrac{1}{d_n} \right\}$ - н.м. \hspace{0.5cm} $\{p_n \cdot d_n \}$ - н.в.
	\end{theorem}
	
	\begin{proof}
	1) $\displaystyle \lim_{n \to \infty} a_n = 0, \lim_{n \to \infty} b_n = 0 \overset{\textrm{def.}}{\iff}$\\
	$\forall \varepsilon > 0: \exists N_1(\varepsilon): \forall n \geq N_1: |a_n-0| = |a_n| < \dfrac{\varepsilon}{2}$\\
	$\forall \varepsilon > 0: \exists N_2(\varepsilon): \forall n \geq N_2: |b_n-0| = |b_n| < \dfrac{\varepsilon}{2}$\\
	Нехай існує $N=\max\{N_1,N_2\}$. Тоді $\forall n \geq N: |a_n+b_n - 0| = |a_n+b_n| \leq |a_n|+|b_n| < \varepsilon$.\\
	Отже, $\{a_n+b_n, n \geq 1\}$ - н.м.
	\bigskip \\
	3) $\displaystyle \lim_{n \to \infty} a_n = 0 \overset{\textrm{def.}}{\iff}$ $\forall \varepsilon > 0: \exists N(\varepsilon): \forall n \geq N: |a_n-0| = |a_n| < \dfrac{\varepsilon}{M}$\\
	де $M>0$ - таке число, що $\forall n \geq 1: |c_n| \leq M$ - означення обмеженності.\\
	Тоді $\forall n \geq N: |a_n \cdot c_n - 0| = |a_n \cdot c_n| = |a_n| \cdot |c_n| < \varepsilon$.\\
	Отже, $\{a_n \cdot c_n, n \geq 1 \}$ - н.м.
	\bigskip \\
	4) $\displaystyle \lim_{n \to \infty} a_n = 0 \overset{\textrm{def.}}{\iff}$ $\forall \varepsilon > 0: \exists N(\varepsilon): \forall n \geq N: |a_n-0| = |a_n| < \varepsilon$\\
	Зафіксуємо $\displaystyle \varepsilon = \frac{1}{E}$ для всіх $E>0$. Тоді $\displaystyle \exists N(E): \forall n \geq N: |a_n|<\frac{1}{E} \iff \abs{\frac{1}{a_n}} > E$.\\
	Отже, $\left\{ \dfrac{1}{a_n}, n \geq 1 \right\}$ - н.в.
	\bigskip \\
	2), 6) доводиться як 3). 5) доводиться аналогічно як 4)
	\end{proof}
	
	\begin{theorem}[Про характеризацію збіжної послідовності]
	Задано послідовність $\{a_n, n \geq 1\}$.\\
	Послідовність $\{a_n, n \geq 1\}$ - збіжна $\iff$ існує $\{\alpha_n, n \geq 1\}$ - така н.м. послідовність, що $a_n = a+\alpha_n$.
	\end{theorem}
	
	\begin{proof}
	\rightproof Дано: $\{a_n, n \geq 1\}$ - збіжна, тобто\\
	$\forall \varepsilon > 0: \exists N: \forall n \geq N: |a_n-a| < \varepsilon$.\\
	Позначимо $a_n-a=\alpha_n$. Тоді $a_n=a+\alpha_n$ та послідовність $\{\alpha_n, n \geq 1\}$ - н.м., оскільки \\ $|\alpha_n - 0| = |\alpha_n| = |a_n - a| < \varepsilon$.
	\bigskip \\
	\leftproof Дано: $\{\alpha_n, n \geq 1\}$ - н.м., де $a_n = a + \alpha_n$. Тоді\\
	$\forall \varepsilon > 0: \exists N: \forall n \geq N: |\alpha_n| < \varepsilon \implies |a_n - a| < \varepsilon$\\
	Отже, $\{a_n, n \geq 1\}$ - збіжна.
	\end{proof}
	
	\begin{theorem}[Арифметика границь]
	Задані $\{a_n, n \geq 1\}$, $\{b_n, n \geq 1\}$ - збіжні \hspace{0.5cm} та $\exists \huge \lim_{n \to \infty} a_n = a, \exists \huge \lim_{n \to \infty} b_n = b$. Тоді:\\
	\begin{tabular}{lll}
	1) $\{a_n+b_n, n \geq 1\}$ - збіжна та & $\displaystyle \exists \lim_{n \to \infty} (a_n+b_n) = \lim_{n \to \infty} a_n+\lim_{n \to \infty} b_n$;\\
	2) $\forall C \in \mathbb{R}: \{C \cdot a_n, n \geq 1\}$ - збіжна та & $\displaystyle \exists \lim_{n \to \infty} C \cdot a_n = C \cdot \lim_{n \to \infty} a_n$;\\
	3) $\{a_n \cdot b_n, n \geq 1\}$ - збіжна та & $\displaystyle \exists \lim_{n \to \infty} (a_n \cdot b_n) = \lim_{n \to \infty} a_n \cdot \lim_{n \to \infty} b_n$;\\
	4) $\left\{\dfrac{a_n}{b_n}, n \geq 1 \right\}$ - збіжна при $b_n \neq 0, b \neq 0$ та & $\displaystyle \exists \lim_{n \to \infty} \frac{a_n}{b_n} = \frac{\displaystyle \lim_{n \to \infty} a_n}{\displaystyle \lim_{n \to \infty} b_n}$.
	\end{tabular}
	
	\end{theorem}
	
	\begin{proof}
	Обидві послідовності збіжні за умовою. Тоді за попередньою теоремою, $a_n = a + \alpha_n$ та $b_n=b+\beta_n$, де $\{\alpha_n\}, \{\beta_n\}$ - н.м. послідовності. Тоді:\\
	1) $a_n+b_n=a+\alpha_n+b+\beta_n=(a+b)+(\alpha_n+\beta_n)$, причому $\{\alpha_n + \beta_n\}$ - н.м. \\ Отже, послідовність $\{a_n+b_n, n \geq 1\}$ - збіжна та має границю $\displaystyle \lim_{n \to \infty} (a_n+b_n) = a+b = \lim_{n \to \infty} a_n+\lim_{n \to \infty} b_n$.
	\bigskip \\
	2) Це зрозуміло.
	\bigskip \\
	3) $a_n b_n - ab = (a+\alpha_n)(b+\beta_n) - ab = \alpha_n b + \alpha_n \beta_n + a \beta_n = \gamma_n$, причому послідовність \\ $\{\gamma_n = \alpha_n b + \alpha_n \beta_n + a \beta_n \}$ - н.м. \\
	Отже, послідовність $\{a_n b_n, n \geq 1\}$ - збіжна та має границю $\displaystyle \lim_{n \to \infty} (a_n \cdot b_n) = ab = \lim_{n \to \infty} a_n \cdot \lim_{n \to \infty} b_n$.
	\bigskip \\
	4) В принципі, це є наслідком 3), якщо представити послідовність $\dfrac{a_n}{b_n} = a_n \cdot \dfrac{1}{b_n}$.\\
	Треба лишень довести, що $\dfrac{1}{b_n} \to \dfrac{1}{b}, n \to \infty$.\\
	Відомо, що $b_n \to b \iff \forall \varepsilon > 0: \exists N': \forall n \geq N: |b_n-b| < \varepsilon$.\\
	Зафіксую $\varepsilon = \dfrac{|b|}{2}$, тоді $\exists N'': \forall n \geq N'':  |b| = |b - b_n + b_n| \leq |b - b_n| + |b_n| < \dfrac{|b|}{2} + |b_n| \implies |b_n| > \dfrac{|b|}{2}$.\\
	Я хочу одночасно $|b_n| > \dfrac{|b|}{2}$ та $|b_n - b| < \varepsilon$, тож нехай $N = \max \{N', N'' \}$. Це вже $N = N(\varepsilon)$, тоді\\
	$\forall n \geq N: \abs{\dfrac{1}{b_n} - \dfrac{1}{b}} = \dfrac{|b_n -b|}{|b_n| |b|} < \dfrac{\varepsilon}{ \dfrac{|b|}{2} |b|} = \dfrac{2}{|b|^2} \varepsilon$.\\
	Таким чином, можна твердити, що $\dfrac{1}{b_n} \to \dfrac{1}{b}, n \to \infty \implies \dfrac{a_n}{b_n} \to \dfrac{a}{b}$, тобто $\left\{ \dfrac{a_n}{b_n}, n \geq 1 \right\}$ - збіжна.
	\end{proof}
	
	\begin{example}
	Знайти границю $\huge \lim_{n \to \infty} \dfrac{(-1)^n + \dfrac{1}{n}}{\dfrac{1}{n^2}- (-1)^n}$.\\
	Як робити неправильно: $\huge \lim_{n \to \infty} \dfrac{(-1)^n + \frac{1}{n}}{\dfrac{1}{n^2}- (-1)^n} = \dfrac{\huge \lim_{n \to \infty} \left( (-1)^n + \frac{1}{n} \right) }{\huge \lim_{n \to \infty} \left( \frac{1}{n^2} - (-1)^n \right)} = \dots$\\
	Проблема тут полягає в тому, що $(-1)^n + \dfrac{1}{n}$ та $\dfrac{1}{n^2}-(-1)^n$ - це розбіжні послідовності. Тому я не можу використати арифметику границі в частках.\\
	Як робити правильно: $\huge \lim_{n \to \infty} \dfrac{(-1)^n + \frac{1}{n}}{\dfrac{1}{n^2}- (-1)^n} = \lim_{n \to \infty} \dfrac{1 + \dfrac{1}{(-1)^n n}}{\dfrac{1}{(-1)^n n^2} - 1} \boxed{=} \dfrac{\huge\lim_{n \to \infty} \left( 1 + \dfrac{1}{(-1)^n n} \right)}{\huge\lim_{n \to \infty} \left( \dfrac{1}{(-1)^n n^2} - 1 \right)} \boxed{\boxed{=}}
	\\ \boxed{\boxed{=}} \dfrac{\huge\lim_{n \to \infty} 1 + \lim_{n \to \infty} \dfrac{1}{(-1)^n n}}{\huge\lim_{n \to \infty} \dfrac{1}{(-1)^n n^2} - \lim_{n \to \infty} 1 } = \dfrac{1+0}{0-1} = -1$\\
	Рівність $\boxed{\boxed{=}}$ коректна: оскільки кожна послідовність чисельника - збіжна, то їхня сума теж збіжна. Знаменник аналогічно. Тоді рівність $\boxed{=}$ теж коректна: через збіжність, маємо, що частка збіжна.
	\end{example}
	
	\subsection{Нерівності в границях}
	\begin{theorem}[Граничний перехід в нерівності]
	Задано дві збіжні числові послідовності $\{a_n, n \geq 1\}$, $\{b_n, n \geq 1\}$ таким чином, що $\exists N': \forall n \geq N': a_n \leq b_n$. Тоді $\displaystyle \lim_{n \to \infty} a_n \leq \lim_{n \to \infty} b_n$.
	\end{theorem}
	
	\begin{proof}
	Задано дві збіжні послідовності, для яких $\displaystyle \lim_{n \to \infty} a_n =a, \lim_{n \to \infty} b_n = b$.\\
	!Припустимо, що $a>b$ та розглянемо $\displaystyle \varepsilon = \frac{a-b}{2}$. Тоді за означенням границі,\\
	$\exists N_1: \forall n \geq N_1:|a_n-a|<\varepsilon \Rightarrow a_n>a-\varepsilon$\\
	$\exists N_2: \forall n \geq N_2:|b_n-b|<\varepsilon \Rightarrow b_n<b+\varepsilon$.\\
	Задамо $N=\max \{N_1, N_2\}$. Тоді  $\displaystyle b_n < b+\varepsilon = b + \frac{a-b}{2}=\frac{a+b}{2}=a-\frac{a-b}{2} = a-\varepsilon<a_n \\ \implies b_n<a_n$. Суперечність!
	\end{proof}
	
	\begin{corollary}
	Задано збіжну числову послідовность $\{b_n, n \geq 1\}$ таким чином, що $\exists N': \forall n \geq N': a \leq b_n$, де $a \in \mathbb{R}$. Тоді $\displaystyle a \leq \lim_{n \to \infty} b_n$.\\
	\textit{Вказівка: розглянути послідовність $\{a_n = a, n \geq 1\}$ - так звана стаціонарна послідовність.}
	\end{corollary}
	
	\begin{remark}
	Для нерівності $\geq$ аналогічно все. А також ця теорема спрацьовує для $<$ або $>$, проте нерівність з границями залишається нестрогою.\\
	Наприклад, $a_n = \dfrac{1}{n^2}, b_n = \dfrac{1}{n}$. Зрозуміло, що $a_n < b_n$. Але звідси $\huge\lim_{n \to \infty} a_n = \huge\lim_{n \to \infty} b_n = 0$.
	\end{remark}
	
	\begin{theorem}[Теорема про 3 послідовності]
	Задані три послідовності: $\{a_n, n \geq 1\}$,$\{b_n, n \geq 1\}$,$\{c_n, n \geq 1\}$ таким чином, що $\displaystyle \lim_{n \to \infty} a_n = \displaystyle \lim_{n \to \infty} b_n = a$. Більш того, $\exists N': \forall n \geq N': a_n \leq c_n \leq b_n$.
	Тоді $\exists \displaystyle \lim_{n \to \infty} c_n = a$
	\end{theorem}
	
	\begin{proof}
	$\displaystyle \lim_{n \to \infty} a_n = a \overset{\textrm{def.}}{\iff}$ $\forall \varepsilon > 0: \exists N_1(\varepsilon): \forall n \geq N_1: |a_n-a| < \varepsilon$ $\Rightarrow a_n > a - \varepsilon$\\
	$\displaystyle \lim_{n \to \infty} b_n = a \overset{\textrm{def.}}{\iff}$ $\forall \varepsilon > 0: \exists N_2(\varepsilon): \forall n \geq N_2: |b_n-a| < \varepsilon$ $\Rightarrow b_n<a+\varepsilon$\\
	Зафіксуємо $N=\max\{N_1, N_2, N'\}$. Тоді $\forall n \geq N:$\\
	$a-\varepsilon< a_n \leq c_n \leq b_n < a+\varepsilon \Rightarrow |c_n - a|<\varepsilon$.\\
	Отже, $\displaystyle \lim_{n \to \infty} c_n = a$.
	\end{proof}
	
	\begin{example}
	Знайти границю $\displaystyle \lim_{n \to \infty} \sqrt[n]{2^n+7^n}$.\\
	Можна отримати наступну оцінку:\\
	$\sqrt[n]{7^n} \leq \sqrt[n]{2^n+7^n} \leq \sqrt[n]{n\cdot 7^n}$\\
	Ця нерівність виконується завжди, починаючи з якогось номера $n$. Рахуємо ліміти з обох сторін:\\
	$\displaystyle \lim_{n \to \infty} \sqrt[n]{7^n} = 7$ \hspace{1cm}
	$\displaystyle \lim_{n \to \infty} \sqrt[n]{n \cdot 7^n} = 7 \lim_{n \to \infty} \sqrt[n]{n} = 7$\\
	Тому з цього випливає, що $\displaystyle \lim_{n \to \infty} \sqrt[n]{2^n+7^n} = 7$.
	\end{example}
	
	\subsection{Монотонні послідовності}
	\begin{definition}
	Послідовність $\{a_n, n \geq 1\}$ називається: \\
	\begin{tabular}{ll}
	\textbf{строго монотонно зростаючою}, якщо & $\forall n \geq 1: a_{n+1} > a_n$; \\
	\textbf{монотонно не спадною}, якщо & $\forall n \geq 1: a_{n+1} \geq a_n$; \\
	\textbf{строго монотонно спадною}, якщо & $\forall n \geq 1: a_{n+1} < a_n$; \\
	\textbf{монотонно не зростаючою}, якщо & $\forall n \geq 1: a_{n+1} \leq a_n$.
	\end{tabular}
	\end{definition}
	
	\begin{example}
	Дослідимо послідовність $\{a_n = \sqrt{n}, n \geq 1 \}$ на монотонність.\\
	$\displaystyle a_{n+1} - a_n = \sqrt{n+1} - \sqrt{n} = \frac{n+1-n}{\sqrt{n+1} + \sqrt{n}} = \frac{1}{\sqrt{n+1} + \sqrt{n}} > 0$\\
	$\Rightarrow a_{n+1}>a_n$, тобто дана послідовність зростає.
	\end{example}
	
	\begin{theorem}[Теорема Вейєрштрасса]
	Будь-яка обмежена зверху/знизу та монотонно неспадна/незростаюча, починаючи з деякого номеру, послідовність є збіжною.
	\end{theorem}
	\begin{proof}
	Нехай задано послідовність $\{a_n, n \geq 1\}$, яка є обмеженю зверху та монотонно неспадної.\\
	Оскільки вона монотонна, а ще - обмежена, то $\exists \huge\sup_{n \geq 1}\{a_n\} = a < +\infty$.\\
	За критерієм sup: \\
	$\forall n \geq 1: a_n \leq a$;\\
	$\forall \varepsilon > 0: \exists N(\varepsilon): a_{N} > a - \varepsilon$.\\
	Отримаємо наступний ланцюг нерівностей: $\forall n \geq N:$.\\
	$a-\varepsilon < a_N \leq a_n \leq a < a + \varepsilon \Rightarrow |a_n-a|<\varepsilon$.\\
	Отже, $\displaystyle \exists \lim_{n \to \infty} a_n = \sup_{n \geq 1}\{a_n\}$.
	\bigskip \\
	Для інших випадків монотонності все аналогічно.
	\end{proof}
	
	\begin{example}
	Довести, що для послідовності $\left\{ a_n = \dfrac{2000^n}{n!}, n \geq 1 \right\}$ існує границя та обчислити її.\\
	Перевіримо на монотонність:\\
	$\displaystyle \frac{a_{n+1}}{a_n} = \frac{2000^{n+1} n!}{(n+1)! 2000^n} = \frac{2000}{n+1}$\\
	Отримаємо, що $a_{n+1} < a_n$ принаймні $\forall n \geq 2000$. Тоді за Вейєрштрассом, $\exists \huge \lim_{n \to \infty} a_n = a$.\\
	Тоді також $\huge \lim_{n \to \infty} a_{n+1} = a$.\\
	Отже, $\huge \lim_{n \to \infty} a_{n+1} = \lim_{n \to \infty} \dfrac{2000}{n+1} a_n = \lim_{n \to \infty} \dfrac{2000}{n+1} \lim_{n \to \infty} a_n = 0$.
	\end{example}
	
	\begin{remark}
	Такими самими міркуваннями можна довести, що $\displaystyle \frac{n^k}{b^n}, \frac{b^n}{n!}, \frac{n!}{n^n} \to 0$, якщо $n \to \infty$.
	\end{remark}
	
	\begin{example}
	Дізнатись, який вираз більший при надто великих $n$: \hspace{0.2cm}
	$2^n$ або $n^{1000}$.\\
	Відомо, що $\huge \lim_{n \to \infty} \dfrac{n^{1000}}{2^n} = 0$. Якщо зафіксую $\varepsilon = 1$, то $\exists N: \forall n \geq N: \dfrac{n^{1000}}{2^n} < 1$\\
	Значить, $2^n > n^{1000}$ для дуже великих $n$.
	\end{example}
	
	\subsection{Число $e$}
	Розглянемо послідовність $\left\{ a_n = \left( 1+\dfrac{1}{n} \right)^n, n \geq 1 \right\}$. Спробуємо для неї знайти границю.\\
	1. Покажемо, що вона є монотонно зростаючою.\\
	$\displaystyle \frac{a_{n+1}}{a_n} = \frac{\displaystyle \left(1+\frac{1}{n+1} \right)^{n+1}}{\displaystyle \left(1+\frac{1}{n} \right)^n} = \left(1 + \frac{1}{n+1} \right) \left( \frac{\displaystyle 1+\frac{1}{n+1}}{\displaystyle 1 + \frac{1}{n}} \right)^n = \frac{n+2}{n+1} \cdot \left( \frac{n(n+2)}{(n+1)^2} \right)^n = \\ = \frac{n+2}{n+1} \cdot \left(1 - \frac{1}{(n+1)^2} \right)^n = \frac{\displaystyle \frac{n+2}{n+1}}{\displaystyle 1-\frac{1}{(n+1)^2}} \cdot \left( 1 - \frac{1}{(n+1)^2} \right)^{n+1} = \frac{n+2}{n+1} \cdot \frac{(n+1)^2}{n^2+2n} \cdot \left( 1 - \frac{1}{(n+1)^2} \right)^{n+1} \boxed{\geq}$\\
	Тут ми маємо права на третю дужку використати нерівність Бернуллі, оскільки $\displaystyle - \frac{1}{(n+1)^2} > -1$.\\
	$\displaystyle \boxed{\geq} \frac{n+2}{n+1} \cdot \frac{(n+1)^2}{n^2+2n} \cdot \left(1 - \frac{n+1}{(n+1)^2} \right) = \frac{n+1}{n} \left(1-\frac{1}{n+1} \right) = 1$\\
	Коротше, $\displaystyle \frac{a_{n+1}}{a_n} \geq 1 \Rightarrow a_{n+1} \geq a_n$. Тобто наша послідовність монотонно зростає.
	\bigskip \\
	
	2. Доведемо, що вона є обмеженою зверху.\\
	Для цього треба розглянути $\left\{ b_n = \left( 1 + \dfrac{1}{n} \right)^{n+1} \right\}$ і довести, що:\\
	а) $\forall n \geq 1: a_n < b_n$;\\
	b) вона є монотонно спадною.
	\bigskip \\
	a) Перший пункт зрозумілий, оскільки $\displaystyle \left(1+\frac{1}{n} \right)^n < \left(1+\frac{1}{n} \right)^{n+1}$ через однакову основу степені, що є більше одинички.\\
	b) А це розпишу:\\
	$\displaystyle \frac{b_{n-1}}{b_n} = \frac{\displaystyle \left(1+\frac{1}{n-1} \right)^{n}}{\displaystyle \left(1+\frac{1}{n} \right)^{n+1}} = \frac{1}{\displaystyle \left(1+\frac{1}{n}\right)} \cdot \left(\frac{n^2}{n^2-1}\right)^n = \frac{n}{n+1} \cdot \left(1+\frac{1}{n^2-1} \right)^n \boxed{\geq}$\\
	За аналогічними причинами я можу користатися нерівностю Бернуллі для другої дужки.\\
	$\displaystyle \boxed{\geq} \frac{n}{n+1} \left(1+\frac{n}{n^2-1}\right) = \frac{n}{n+1} + \frac{n^2}{(n+1)(n^2-1)} = \frac{n^3+n^2-n}{n^3+n^2-n-1} > 1$\\
	Коротше, $\displaystyle \frac{b_{n-1}}{b_n} > 1 \Rightarrow b_n < b_{n-1}$. Тобто ця послідовність монотонно спадає.\\
	В результаті всього можемо отримати наступну обмеженність:\\
	$2=a_1 \leq a_2 \leq \cdots \leq a_n < b_n \leq \cdots \leq b_2 \leq b_1 = 4$.\\
	А це означає, що для послідовності $\left\{a_n = \left(1 + \dfrac{1}{n} \right)^n, n \geq 1 \right\}$ існує границя:
	$$\huge \lim_{n \to \infty}\left(1+\frac{1}{n} \right)^n = e \approx 2.71...$$
	До речі, для $\left\{b_n = \left( 1 + \dfrac{1}{n} \right)^{n+1}, n \geq 1 \right\}$ така сама границя, бо \\ $\huge\lim_{n \to \infty} \left( 1 + \dfrac{1}{n} \right)^{n+1} = \lim_{n \to \infty} \left( 1 + \dfrac{1}{n} \right)^n \lim_{n \to \infty} \left(1 + \dfrac{1}{n} \right) = e \cdot 1 = e$.
	\bigskip \\	
	А тепер, оскільки $\{a_n \}$ зростає, а $\{b_n \}$ спадає та обидва обмежені, то $\forall n \geq 1:$\\
	$a_n<e<b_n$\\
	$\displaystyle \left(1+\frac{1}{n} \right)^n < e < \left(1+\frac{1}{n} \right)^{n+1}$\\
	Зробимо нове позначення: $\log_{e} a =\ln a$. Тоді:\\
	$\displaystyle n \ln \left(1+\frac{1}{n} \right) < 1 < (n+1) \ln \left(1+\frac{1}{n} \right)$\\
	В результаті ми можемо отримати одну оцінку:
	 $$\frac{1}{1+n} < \ln (1+\frac{1}{n}) < \frac{1}{n}$$
	
	
	\subsection{Підпослідовності}
	\begin{definition}
	\textbf{Послідовністю натуральних чисел} називають строго зростаючу послідовність $\{n_k, k \geq 1\} \subset \mathbb{N}$.
	\end{definition}
	
	\begin{definition}
	Задано послідовність $\{a_n, n \geq 1\}$ та послідовність натуральних чисел $\{n_k, k \geq 1\}$\\
	Послідовність $\{a_{n_k}, k \geq 1\}$ називається \textbf{підпослідовністю}.
	\end{definition}
	
	\begin{example} Маємо послідовність натуральних чисел
	$\{n_k = 2k, k \geq 1 \} = \{2,4,6,8,\dots \} \subset \mathbb{N}$.
	До речі, що цікаво, $n_k = 2k > k$, це згодом доведемо\\
	Маємо послідовність $\{a_n = (-1)^n, n \geq 1\}$. Тоді якщо використати нашу послідовність натуральних чисел, отримаємо підпослідовність $\{a_{n_k} = a_{2k} = (-1)^{2k} = 1, k \geq 1\}$
	\end{example}
	
	\begin{lemma}
	Задано послідовність натуральних чисел $\{n_k, k \geq 1\}$. Для всіх $k \geq 1$ виконано $n_k \geq k$
	\end{lemma}
	
	\begin{pfMI}
	База індукції: $k = 1$, тоді або $n_1 = 1$, або $n_1 > 1$ - все чудово\\
	Крок індукції: нехай $n_m \geq m$ виконано для $k=m$. Доведемо для $k=m+1$\\
	Якщо $n_m = m$, то автоматично $n_{m+1} \geq m+1$\\
	Якщо $n_m > m$, то тоді $n_m \geq m+1$. Оскільки строго зростає послідовність, то $n_{m+1} > n_m \geq m+1$\\
	Отже, $\forall k \geq 1: n_k \geq k$. МІ доведено
	\end{pfMI}
	
	\begin{proposition}
	Якщо для послідовності $\{a_n, n \geq 1\}:$ $\displaystyle \exists \lim_{n \to \infty} a_n = a$, то для кожної підпослідовності $\huge \{a_{n_k}, k \geq 1\}:\exists \lim_{k \to \infty} a_{n_k} = a$.
	\end{proposition}
	
	\begin{proof}
	$\displaystyle \exists \lim_{n \to \infty} a_n = a \iff \forall \varepsilon > 0: \exists N(\varepsilon): \forall n \geq N: |a_n-a| < \varepsilon$\\
	Візьмемо підпослідовність $\{a_{n_k}, k \geq 1\}$. Оскільки послідовність $\{n_k, k \geq 1\}$ - строга зростаюча послідовність натуральних чисел, то $\exists \huge \lim_{k \to \infty} n_k = +\infty$.\\
	Тоді для $E = N(\varepsilon): \exists K(\varepsilon): \forall k \geq K: n_k > N$\\
	Зокрема оскільки $n_k > N$, то одразу $|a_{n_k} - a| < \varepsilon \implies \displaystyle \lim_{k \to \infty} a_{n_k} = a$.
	\end{proof}
	
	\begin{proposition}
	Якщо для кожної підпослідовність $\{a_{n_k}, k \geq 1\}: \exists \huge\lim_{k \to \infty} a_{n_k} = a$, то для послідовності $\{a_n, n \geq 1\}: \exists \huge\lim_{n \to \infty} a_n = a$
	\end{proposition}
	
	\begin{proof}
	!Припустимо, що $\huge\lim_{n \to \infty} a_n \neq a$, тобто $\exists \varepsilon^* > 0: \forall N: \exists n \geq N: |a_n-a| \geq \varepsilon^*$\\
	При $N = 1$ маємо, що $\exists n = n_1 \geq 1: |a_{n_1} - a| \geq \varepsilon^*$\\
	При $N = n_1$ маємо, що $\exists n = n_2 > n_1: |a_{n_2} - a| \geq \varepsilon^*$\\
	$\vdots$\\
	При $N = n_k$ маємо, що $\exists n = n_k > \dots > n_2 > n_1: |a_{n_k} - a| \geq \varepsilon^*$\\
	Тобто $\forall k \geq 1: |a_{n_k} - a| \geq \varepsilon^*$. А це означає, що $\huge\lim_{k \to \infty} a_{n_k} \neq a$ для побудованої підпослідовності. Суперечність!
	\end{proof}
	
	\begin{theorem}[Теорема Больцано-Вейєрштрасса]
	Для будь-якої обмеженої послідовності існує збіжна підпослідовність.
	\end{theorem}
	
	\begin{proof}
	Розглянемо послідовність $\{a_n, n \geq 1\}$. Існують 2 випадки за кількістю елементів:\\
	1. Послідовність - скінченна (як в \textbf{Ex. 2.6.3.}). Тоді одне із значень послідовності буде прийматись нескінченну кількість разів. Отримаємо стаціонарну підпослідовність, яка є збіжною.
	\bigskip \\
	2. Послідовність - нескінченна (як в \textbf{Ex. 2.6.8.}). Нехай $A$ - множина всіх можливих значень послідовності. Оскільки вона є обмеженою, то за лемою Больцано-Вейерштрасса, в неї існує гранична точка $b_* \iff \forall \varepsilon > 0:A \cap (b_*-\varepsilon, b_*+\varepsilon)$ - нескінченна множина.\\
	Розглянемо $\varepsilon = \displaystyle \frac{1}{k}$.\\
	$\displaystyle k = 1: A \cap(b_*-1, b_*+1) \rotatebox[origin=c]{180}{$\in$}a_{n_1}$\\
	$\displaystyle k = 2: A \cap(b_*-\frac{1}{2}, b_*+\frac{1}{2}) \rotatebox[origin=c]{180}{$\in$}a_{n_2}$, вимагаємо $n_2>n_1$.\\
	$\vdots$\\
	Побудовали підпослідовність $\{a_{n_k}, k \geq 1\}$ таким чином, що $\displaystyle b_*-\frac{1}{k} < a_{n_k} < b_*+\frac{1}{k}$.\\
	А далі спрямуємо $k$ до нескінченності. В результаті чого отримаємо:\\
	$\displaystyle \underset{\rotatebox[origin=c]{-45}{$\rightarrow$}}{b_*-\frac{1}{k}} < \underset{\underset{\displaystyle b_*}{\rotatebox[origin=c]{90}{$\leftarrow$}}}{a_{n_k}} < \underset{\rotatebox[origin=c]{45}{$\leftarrow$}}{b_*+\frac{1}{k}}, \hspace{0.2cm} k \to \infty$\\
	Тоді за теоремою про 2 поліцая, $\displaystyle \exists \lim_{k \to \infty} a_{n_k} = b_*$.
	\end{proof}
	
	\begin{corollary}
	Для обмеженої послідовності множина часткових границь - непорожня. Таку множину позначу за $X$.
	\end{corollary}
		
	\begin{theorem} Для будь-якої необмеженої послідовності існує н.в. підпослідовність.
	\end{theorem}
	
	\begin{proof}
	Задано $\{a_n, n \geq 1\}$ - необмежена зверху $\implies \forall C > 0: \exists n \geq 1: a_n > C$.\\
	Нехай $C = 1$. Тоді $\exists n = n_1 \geq 1: a_{n_1} > 1$\\
	Нехай $C = 2$. Тоді $\exists n = n_2 > n_1: a_{n_2} > 2$\\
	$\vdots$\\
	Нехай $C = k$. Тоді $\exists n = n_k > \dots > n_2 > n_1: a_{n_k} > k$\\
	Отже, маємо підпослідовність $\{a_{n_k}, k \geq 1\}$, де $\forall k \geq 1: a_{n_k} > k \implies 0 <\dfrac{1}{a_{n_k}} < \dfrac{1}{k}$.\\
	Якщо $k \to 0$, то $\dfrac{1}{a_{n_k}} \to 0$. Отже, $a_{n_k} \to +\infty$.\bigskip \\
	Для необмежено знизу аналогічно.
	\end{proof}
	
	\begin{definition} Задано послідовність $\{a_n, n \geq 1\}$.\\
	\textbf{Верхньою границею} називають число:
	\begin{align*}
	\displaystyle \uplim_{n \to \infty} a_n \overset{\textrm{або}}{=} \limsup_{n \to \infty} a_n = \sup X
	\end{align*}
	\textbf{Нижньою границею} називають число:
	\begin{align*}
	\displaystyle \downlim_{n \to \infty} a_n \overset{\textrm{або}}{=} \liminf_{n \to \infty} a_n = \inf X
	\end{align*}
	\end{definition}
	
	\begin{example}
	Знайдемо часткові границі для послідовнонсті $\{a_n, n \geq 1\}$, де $a_n = (-1)^{n-1} \left(2 + \dfrac{3}{n} \right)$\\
	Якщо $n = 2k-1$, то маємо підпослідовність $\left\{a_{n_k} = 2 + \dfrac{3}{2k-1} \right\}$\\
	$\huge\lim_{k \to \infty} \left( 2 + \dfrac{3}{2k-1} \right) = 2$\\
	Якщо $n = 2k$, то маємо підпослідовність $\left\{a_{n_k} = -2 - \dfrac{3}{2k}\right\}$\\
	$\huge\lim_{k \to \infty} \left(-2 - \dfrac{3}{2k} \right) = -2$\\
	Але це не всі можливі підпослідовності. Я можу, наприклад, перші 10 членів взяти з непарним номером, а решта - з парним номером, яка буде прямувати до вже існуючих часткових границь\\
	Постає питання, чи є ще інші часткові границі. Інтуітивно, ні. \\
	Множина часткових границь: $X = \{-2, 2\}$. Тоді за означенням верхньої та нижньої границі, \\ $\huge \uplim_{n \to \infty} a_n = 2, \huge \downlim_{n \to \infty} a_n = -2$.\\
	Зауважимо одразу, що $\huge \sup_{n \geq 1} \{a_n\} = 5$ та $\huge \inf_{n \geq 1} \{a_n\} = -3.5$.
	\begin{figure}[H]
	\centering
	\begin{tikzpicture}[scale = 1.5]
	\centering
	\foreach \i in {1,2,...,24} {
	\fill[red] ({2 + (3)/(2*\i-1)},0) circle (1pt);
	\fill[red] ({-2 - (3)/(2*\i)},0) circle (1pt);
	}
	\draw[thick, ->] (-4,0)--(5.5,0);
	\node at (5,-0.2) {$5$};
	\node at (2,-0.2) {$2$};
	\node at (-2-0.1,-0.2) {$-2$};
	\node at (-3.5-0.1,-0.2) {$-3.5$};
	\end{tikzpicture}
	\end{figure}
	\end{example}
	
	\begin{example}
	Є ще така послідовність $\left\{ 0, 1, \dfrac{1}{2}, \dfrac{1}{4}, \dfrac{3}{4}, \dfrac{1}{8}, \dfrac{3}{8}, \dfrac{5}{8}, \dfrac{7}{8}, \dots \right\}$.\\
	У неї множина часткових границь задається так: $X = [0,1]$.
	\end{example}
	
	\begin{remark}
	Якщо послідовність $\{a_n, n \geq 1\}$ не є обменежою:\\
	- зверху, то $\displaystyle \uplim_{n \to \infty} = +\infty$;\\
	- знизу, то $\displaystyle \downlim_{n \to \infty} = -\infty$.
	\end{remark}
	
	Як бачимо, що будь-яка обмежена послідовність має максимум та мінімум. І справді, наступна теорема це підтверджує:
	
	\begin{theorem}
	Будь-яка обмежена послідовність має верхню/нижню границю.
	\end{theorem}
	
	\begin{remark}
	Інакше кажучи, $\huge \uplim_{n \to \infty} a_n \in X, \hspace{1cm} \downlim_{n \to \infty} a_n \in X$ для послідовності $\{a_n, n \geq 1\}$.
	\end{remark}
	
	\begin{proof}
	Позначимо $x_* = \inf X$. Оскільки $X$ - множина часткових границь, то\\
	$\forall \varepsilon > 0: \exists x_{\varepsilon} \in X: x_* \leq x_{\varepsilon} < x_* + \dfrac{\varepsilon}{2}$.\\
	Оскільки $x_\varepsilon \in X$, то тоді це - часткова границя для послідовності $\{a_n, n \geq 1\}$. Тому за Больцано-Вейерштрасса, $\exists \{a_{n_m}^{(\varepsilon)}, m \geq 1\}: \huge \lim_{m \to \infty} a_{n_m}^{(\varepsilon)} = x_{\varepsilon} \implies \exists M(\varepsilon): \forall m \geq M: |a_{n_m}^{(\varepsilon)} - x_{\varepsilon}| < \varepsilon$\\
	$\Rightarrow |a_{n_m}^{(\varepsilon)} - x_*| = |a_{n_m}^{(\varepsilon)} - x_{\varepsilon} + x_{\varepsilon} - x_*| \leq |a_{n_m}^{(\varepsilon)} - x_{\varepsilon}| + |x_{\varepsilon} - x_*| < \dfrac{\varepsilon}{2} + \dfrac{\varepsilon}{2} = \varepsilon$.\\
	При $\varepsilon = 1$ маємо: $|a_{n_{M(1)}}^{(1)} - x_*| < 1$\\
	При $\varepsilon = \dfrac{1}{2}$ маємо: $|a_{n_{M(\frac{1}{2})}}^{(\frac{1}{2})} - x_*| < \dfrac{1}{2}$\\
	$\vdots$\\
	А тепер розглянемо підпослідовність $\{a_{n_k}, k \geq 1\}$, таку, що $a_{n_k} = a_{n_{M(\frac{1}{k})}}^{(\frac{1}{k})}$.\\
	За побудовою, $|a_{n_k} - x_*| < \dfrac{1}{k} \implies$ $\displaystyle \underset{\rotatebox[origin=c]{-45}{$\rightarrow$}}{x_*-\frac{1}{k}} < \underset{\underset{\displaystyle x_*}{\rotatebox[origin=c]{90}{$\leftarrow$}}}{a_{n_k}} < \underset{\rotatebox[origin=c]{45}{$\leftarrow$}}{x_*+\frac{1}{k}}, \hspace{0.2cm} k \to \infty$\\
	Таким чином, для $\{a_{n_k}, k \geq 1\}$ існує $\huge \lim_{k \to \infty} a_{n_k} = x_* = \downlim_{n \to \infty} a_n$.
	\bigskip \\
	Для точної верхньої границі аналогічно.
	\end{proof}
	
	\begin{theorem}
	Задано $\{a_n, n \geq 1\}$ - обмежена та $L^* \in \mathbb{R}$. Наступні твердження еквівалентні:\\
	1) $L^* = \huge \uplim_{n \to \infty} a_n$.\\
	2) $\forall \varepsilon > 0:$ проміжок $(L^*+\varepsilon, + \infty)$ містить скінченну кількість елементів та проміжок $(L^*-\varepsilon, + \infty)$ містить нескінченну кількість елементів.\\
	3) Нехай задано послідовність $\{b_m, m \geq 1\}$, де $b_m = \huge \sup_{n \geq m} \{a_n\}$. Тоді $\exists \huge \lim_{m \to \infty} b_m = L^*$.
	\end{theorem}
	
	\begin{proof}
	$\boxed{1) \Rightarrow 2)}$ Дано: умова 1)\\
	Тоді $L^* = \huge \sup X$. За попередньою теоремою, $L^* \in X$, тож існує $\{a_{n_k}, k \geq 1\}$, для якої \\ $\huge \lim_{k \to \infty} a_{n_k} = L^* \implies \forall \varepsilon > 0: \exists K: \forall k \geq K: L^*-\varepsilon < a_{n_k} < L^*+\varepsilon$.\\
	Звідси ми вже маємо, що на проміжку $(L^*-\varepsilon, +\infty)$ маємо нескінченну кількість елементів.\\
	!А далі припустимо, що $\exists \varepsilon^* > 0:$ проміжок $(L^*+\varepsilon^*, + \infty)$ має НЕскінченну кількість елементів.\\
	Оскільки $\{a_n, n \geq 1\}$ - обмежена, то за теоремою Больцано-Вейєрштрасса, маємо підпослідовність $\{a_{n_m}, m \geq 1\}$ таку, що $a_{n_m} > L^* + \varepsilon^*$.\\
	Тоді звідси, за граничним переходом нерівності, $\huge \lim_{m \to \infty} a_{n_m} = L^{**} \geq L^* + \varepsilon^*$. \\ 
	Тобто $L^{**} > L^*$, але $L^*$ - верхня границя. Суперечність!\\
	Висновок: $\forall \varepsilon > 0:$ проміжок $(L^*+\varepsilon, + \infty)$ має скінченну кількість елементів.
	\bigskip \\
	
	$\boxed{2) \Rightarrow 3)}$ Дано: умова 2)\\
	Для початку розглянемо $\{b_m, m \geq 1\}$ та покажемо, що в неї дійсно є границя.\\
	$b_{m+1} \leq b_m$, тобто $\huge \sup_{n \geq m+1} \{a_n\} \leq \sup_{n \geq m} \{a_n\}$. Думаю, зрозуміло.\\
	Також оскільки $\{a_n, n \geq 1\}$ - обмежена, то $\{b_m, m \geq 1 \}$ теж обмежена. Тоді за Вейєрштрассом, $\exists \huge \lim_{m \to \infty} b_m = \inf_{m \geq 1} \{b_m\}$.\\
	Оскільки $(L^*+\dfrac{\varepsilon}{2}, + \infty)$ має скінченну кількість елементів, то $\exists M: \forall m \geq M: x_m \leq L^* + \dfrac{\varepsilon}{2}$.\\
	Тобто знайдеться номер останнього члену, після якого всі інші члени будуть вже за межами інтервалу.\\
	Тоді $\forall m \geq M: b_m < L^* + \varepsilon$.\\
	Також оскільки $(L^*-\varepsilon, + \infty)$ має нескінченну кількість елементів, то $b_m > L^* - \varepsilon$, $\forall m \geq 1$.\\
	Остаточно, $\forall \varepsilon > 0: \exists M: \forall m \geq M: |b_m - L^*| < \varepsilon$. Отже, $\huge \lim_{m \to \infty} b_m = L^*$.
	\bigskip \\
	
	$\boxed{3) \Rightarrow 1)}$ Дано: умова 3)\\
	Ідея доведення: $L^* = \sup X$ означає, що:\\
	- яку б я підпослідовність з частковою границею я б не взяв, всі вони будуть не перевищувати числа $L^*$\\
	- знайдемо деяку підпослідовність, де члени будуть перебільшувати $L^*$, зменшений на деяке число\\
	Візьмемо деяку підпослідовність $\{a_{n_k}, k \geq 1\}$. Пам'ятаємо про $n_k \geq k$, тоді маємо нерівність \\ $a_{n_k} \leq b_{n_k} \leq b_k \implies \huge \lim_{k \to \infty} a_{n_k} \leq L^*$.\\
	$\huge \lim_{m \to \infty} b_m = L^* \implies \forall \varepsilon > 0: \exists M: \forall m \geq M: L^*-\varepsilon < b_m < L^*+\varepsilon$\\
	Для номера $M$ виконано $b_M > L^*-\varepsilon$. Але не всі $a_n$, де $n \geq M$, можуть виконувати нерівність.\\
	Отже, виділимо підпослідовність $\{a_{n_k}^{\varepsilon}, k \geq 1\}$, для яких
	$a_{n_k}^{\varepsilon} > L^* - \varepsilon$.\\
	А тоді $a_{\varepsilon} > L^* - \varepsilon$.
	Таким чином, ми отримали:\\
	$\begin{gathered}
	\forall a \in A: a \leq L^* \\
	\forall \varepsilon > 0: \exists a_{\varepsilon}: a_{\varepsilon} > L^* - \varepsilon
	\end{gathered} \implies L^* = \huge \sup X = \uplim_{n \to \infty} a_n$.
	\end{proof}
	
	\subsection{Фундаментальна послідовність}
	\begin{definition}
	Послідовність $\{a_n, n \geq 1\}$ називається \textbf{фундаментальною}, якщо
	\begin{align*}
	\forall \varepsilon > 0: \exists N \in \mathbb{N}: \forall n,m \geq N: |a_n - a_m| < \varepsilon
	\end{align*}
	\end{definition}
	
	\begin{theorem}[Критерій Коші]
	Послідовність $\{a_n, n \geq 1\}$ є збіжною $\iff$ вона є фундаментальною.
	\end{theorem}
	
	\begin{proof}
	\rightproof Дано: $\{a_n,n \geq 1\}$ - збіжна, тобто: $\forall \varepsilon >0: \exists N: $\\
	$\huge \forall n \geq N: |a_n - a| < \frac{\varepsilon}{2}$\\
	$\huge \forall m \geq N: |a_m - a| < \frac{\varepsilon}{2}$\\
	А тоді отримаємо $|a_n - a_m| = |a_n - a + a - a_m| \leq |a_n - a| + |a_m - a| < \varepsilon$.\\
	Отже, послідовність є фундаментальною.
	\bigskip \\
	
	\leftproof Дано: $\{a_n, n \geq 1\}$ - фундаментальна, тобто $\forall \varepsilon > 0: \exists N \in \mathbb{N}: \forall n,m \geq N: |a_n - a_m| < \varepsilon$.\\
	І. Доведемо, що вона є обмеженою\\
	Для $\varepsilon = 1: \exists N: \forall n \geq N, m = N: |a_n - a_N| < 1$\\
	$\implies |a_n| = |a_n - a_N + a_N| \leq |a_n - a_N| + |a_N| < 1 + |a_N|$.\\
	Задамо $C = \max\{|a_1|, \dots, |a_{N-1}|, |1|+|a_N|\}$. Тоді $\forall n \geq 1: |a_n| \leq C$, тобто обмежена.
	\bigskip \\
	II. Доведемо її збіжність\\
	Оскільки наша послідовність обмежена, виділимо збіжну підпослідовність $\{a_{n_k}, k \geq 1\}$, \\ $\huge \lim_{n \to \infty} a_{n_k} = a \implies$
	$\huge \forall \varepsilon > 0: \exists K: \forall k \geq K \geq n_K: |a_{k} - a| < \frac{\varepsilon}{2}$.\\
	Покладемо $m = n_K$. Тоді:\\
	$|a_n - a| = |a_n - a_{n_K} + a_{n_K} - a| \leq |a_n - a_{n_K}| + |a_{n_K} - a| < \varepsilon$.\\
	Тобто $\huge \exists \lim_{n \to \infty} a_n = a$.
	\end{proof}
	
	\begin{remark}
	Означення фундаментальної послідовності можна записати й таким чином:
	\begin{align*}
	\forall \varepsilon > 0: \exists N \in \mathbb{N}: \forall n \geq N: \forall p \geq 1: |a_{n+p} - a_n| < \varepsilon
	\end{align*}
	\end{remark}
	
	\begin{example}
	Розглянемо послідовність $\{a_n, n \geq 1\}$, де $a_n = \dfrac{\sin 1}{1^2} + \dfrac{\sin 2}{2^2} + \dots + \dfrac{\sin n}{n^2}$\\
	Доведемо її фундаментальність за означенням.\\
	$|x_{n+p} - x_n| \leq \dfrac{1}{(n+1)^2} + \dots + \dfrac{1}{(n+p)^2} \leq \dfrac{1}{n(n+1)} + \dots + \dfrac{1}{(n+p-1)(n+p)} = \\ = \dfrac{1}{n} - \dfrac{1}{n-1} + \dots + \dfrac{1}{n+p-1} - \dfrac{1}{n+p} = \dfrac{1}{n} - \dfrac{1}{n+p} \leq \dfrac{1}{n} < \varepsilon \implies n > \dfrac{1}{\varepsilon}$\\
	Встановимо $N = \left[ \dfrac{1}{\varepsilon} \right] + 1$. Тоді $\forall n \geq N: \forall p \geq 1: |x_{n+p} - x_n| < \varepsilon$.\\
	Отже, наша послідовність - фундаментальна.
	\end{example}
	\newpage
	
	%\fi %IMPORTANT COMMENT
	
	%\iffalse %IMPORTANT COMMENT
	\section{Границі функції}
	Залишу для початку загублену теорему, яка нам знадобиться надалі.
	\begin{theorem}
	Задано множину $A \subset \mathbb{R}$.\\
	$a$ - гранична точка $A$ $\iff \exists$ $\{a_n, n \geq 1\}$ $\subset A: \huge \lim_{n \to \infty} a_n = a$, причому $\forall n \geq 1: a_n \neq a$.
	\end{theorem}
	
	\begin{proof}
	\rightproof Дано: $a$ - гранична т. $A$, тоді $\forall \varepsilon > 0: (a-\varepsilon, a + \varepsilon) \cap A$ - нескінченна множина.\\
	$\varepsilon = 1: \exists a_1 \in (a-1,a+1) \cap A$\\
	$\varepsilon = \dfrac{1}{2}: \exists a_2 \in (a-\dfrac{1}{2}, a+\dfrac{1}{2}) \cap A$\\
	$\vdots$\\
	Побудували послідовність $\{a_n, n \geq 1\}$, таку, що $a_n \in (a-\dfrac{1}{n}, a+\dfrac{1}{n}) \cap A$.\\
	Тобто $a - \dfrac{1}{n} < a_n < a + \dfrac{1}{n}$.\\
	За теоремою про двох поліцаїв, якщо $n \to \infty$, то отримаємо, що $\exists \huge \lim_{n \to \infty} a_n = a$.
	\bigskip \\
	
	\leftproof Дано: $\exists \{a_n, n \geq 1\} \subset A: \forall n \geq 1: a_n \neq a: \huge \lim_{n \to \infty} a_n = a$\\
	$\implies \forall \varepsilon > 0: \exists N: \forall n \geq N: |a_n-a|<\varepsilon \implies a_n \in (a-\varepsilon,a+\varepsilon)$.\\
	А отже, $(a-\varepsilon,a+\varepsilon) \cap A$ - нескінченна множина, тож $a$ - гранична точка.
\end{proof}
	\subsection{Основні поняття про функції}
	\begin{definition}
	Задано дві множини $X,Y$.\\
	\textbf{Функцією} $f$ із множини $X$ в множину $Y$ називають правило, в якому кожному елементу з $X$ ставиться у відповідність елемент з $Y$.\\
	Позначення: $f: X \to Y$.
	\begin{figure}[H]
\centering {
\begin{tikzpicture}
\fill[red!40] (0,0) ellipse (1cm and 2cm);
\fill[blue!40] (4,0) ellipse (1cm and 2cm);
\node (A1) at (0.7,1) [circle,fill,inner sep=1.5pt]{};
\node (A2) at (-0.5,-0.7) [circle,fill,inner sep=1.5pt]{};
\node (A3) at (0.5,-0.1) [circle,fill,inner sep=1.5pt]{};

\node (B1) at (4+0,1) [circle,fill,inner sep=1.5pt]{};
\node (B2) at (4-0.8,0.2) [circle,fill,inner sep=1.5pt]{};
\node (B3) at (4+0.5,-0.9) [circle,fill,inner sep=1.5pt]{};
\node[anchor = south east] at (0,0) {$X$};
\node[anchor = north west] at (4,0) {$Y$};

\draw[thick, ->] (A1)--(B1);
\draw[thick, ->] (A2)--(B3); \draw[thick, ->] (A3)--(B1);
\end{tikzpicture}
}
\end{figure}
	\end{definition}
	
	\begin{example}
	Задано дві множини:\\
	$X = \{0; 1; 2; 3 \}$ \\ $Y = \{-1; \sqrt{2}, 17, \sqrt{101}, 124, 1111\}$\\ 
	Можна побудувати таку функцію $f: X \to Y$:\\
	$\begin{matrix}
	X & Y \\
	0 & \sqrt{2} \\
	1 & 1111 \\
	2 & \sqrt{2} \\
	3 & \sqrt{101} \\
	\end{matrix}$
	\end{example}
	
	\begin{example}
	Задано таку функцію: $f: [-4; 5] \to \mathbb{R}$ \hspace{1cm} $f(x) = x^3-x^2+4$.
	\end{example}
	
	\begin{definition}
	Задано дві функції: $f: X \to Y$, $g: Y \to Z$.\\
	\textbf{Композицією функцій} $f$ та $g$ називають таку функцію $h: X \to Z$, що:
	\begin{align*}
	h(x) = g(f(x)) \overset{\text{або}}{=} (g \circ f) (x)
	\end{align*}
	\begin{figure}[H]
\centering {
\begin{tikzpicture}
\fill[red!40] (0,0) ellipse (1cm and 2cm);
\fill[blue!40] (4,0) ellipse (1cm and 2cm);
\fill[green!40] (8,0) ellipse (1cm and 2cm);

\node[anchor = south east] at (0,1) {$X$};
\node[anchor = south east] at (4,1) {$Y$};
\node[anchor = south east] at (8,1) {$Z$};

\draw[thick, ->](0.5,0) .. controls (2,1) .. (4,0) node at (2,1.2) {$f$};
\draw[thick, ->](4.5,0) .. controls (6,1) .. (8,0) node at (6,1.2) {$g$};
\draw[thick, ->](0.5,-0.1) .. controls (4,-1) .. (8,-0.1) node at (6,-1) {$h = g \circ f$};;

\end{tikzpicture}
}
\end{figure}
	\end{definition}
	
	\begin{example}
	Маємо функцію $f: [0,+\infty) \to \mathbb{R} \hspace{0.2cm}$ $f(x) = x^2$ та $g: \mathbb{R} \to [-1,1] \hspace{0.2cm}$, $g(x) = \sin x$\\
	Визначимо композицію $h: [0,+\infty) \to [-1,1]$ так: $h(x) = g(f(x)) = \sin x^2$.
	\end{example}
	
	\begin{proposition}[Асоціативність композиції]
	Задано функції $f: W \to X$, $g: X \to Y$, $h: Y \to Z$,\\
	Тоді $h \circ (g \circ f) = (h \circ g) \circ f$.
	\end{proposition}
	
	\begin{proof}
	З одного боку, $(h \circ (g \circ f)) (x) = h((g \circ f) (x)) = h(g(f(x)))$.\\
	З іншого боку, $((h \circ g) \circ f) (x) = (h \circ g) (f(x)) = h(g(f(x)))$.
	Отже, $h \circ (g \circ f) = (h \circ g) \circ f$.
	\end{proof}
	\begin{figure}[H]
\centering {
\begin{tikzpicture}
\fill[yellow!40] (0,0) ellipse (1cm and 2cm);
\fill[red!40] (4,0) ellipse (1cm and 2cm);
\fill[blue!40] (8,0) ellipse (1cm and 2cm);
\fill[green!40] (12,0) ellipse (1cm and 2cm);

\node[anchor = south east] at (0,1) {$W$};
\node[anchor = south east] at (4,1) {$X$};
\node[anchor = south east] at (8,1) {$Y$};
\node[anchor = south east] at (12,1) {$Z$};

\draw[thick, ->](0.5,0) .. controls (2,1) .. (4,0) node at (2,1.2) {$f$};
\draw[thick, ->](4.5,0) .. controls (6,1) .. (8,0) node at (6,1.2) {$g$};
\draw[thick, ->](4+4.5,0) .. controls (4+6,1) .. (4+8,0) node at (4+6,1.2) {$h$};
\draw[thick, ->](0.5,-0.1) .. controls (4,-1) .. (8,-0.1) node at (6,0) {$g \circ f$};
\draw[thick, ->](4+0.5,-0.1-1) .. controls (4+4,-1-1) .. (4+8,-0.1-1) node at (4+6,-1-1) {$h \circ g$};
\draw[thick, ->] (0,1.5) .. controls (6,3) .. (12,1.5) node at(6,3.2) {$h \circ (g \circ f) = (h \circ g) \circ f$};
\end{tikzpicture}
}
\end{figure}

	\begin{definition}
	Функція $f: X \to Y$ називається: \\
	\begin{tabular}{ll}
	- \textbf{ін'єкцією}, якщо & $\forall x_1, x_2 \in X: x_1 \neq x_2 \implies f(x_1) \ne f(x_2)$.\\
	- \textbf{сюр'єкцією}, якщо & $\forall y \in Y: \exists x: f(x) = y$.\\
	- \textbf{бієкцією}, якщо & $\forall y \in Y: \exists! x: f(x) = y$.
	\end{tabular}
	\end{definition}
	
	\begin{remark}
	Означення ін'єкції можна переписати так: $\forall x_1,x_2 \in X: f(x_1)=f(x_2) \implies x_1 = x_2$
	\end{remark}
	
	\begin{example} Розглянемо такі приклади:\\
	1) $f: \mathbb{R} \to [-1,1]$: $f(x) = \cos x$ - сюр'єкція;\\
	2) $f: \mathbb{R} \to \mathbb{R}$: $f(x)=3^x$ - ін'єкція;\\
	3) $f: \mathbb{R} \to \mathbb{R}$: $f(x) = x^3$ - бієкція;\\
	4) $f: \mathbb{R} \to \mathbb{R}$: $f(x) = |x|$ - жодна з означень.
\begin{figure}[H]
\centering 
\resizebox{1\textwidth}{!} {
\begin{tikzpicture}
\fill[red!40] (0,0) ellipse (1cm and 2cm);
\fill[blue!40] (3,0) ellipse (1cm and 2cm);
\node (A1) at (0,0.8) [circle,fill,inner sep=1.5pt]{};
\node (A2) at (0,0) [circle,fill,inner sep=1.5pt]{};
%\node (A3) at (0,-0.8) [circle,fill,inner sep=1.5pt]{};

\node (B1) at (3+0,0.8) [circle,fill,inner sep=1.5pt]{};
\node (B2) at (3+0,0) [circle,fill,inner sep=1.5pt]{};
\node (B3) at (3+0,-0.8) [circle,fill,inner sep=1.5pt]{};
\node[anchor = south east] at (0,1) {$X$};
\node[anchor = south east] at (3,1) {$Y$};

\draw[thick, ->] (A1)--(B1);
\draw[thick, ->] (A2)--(B3);
\node at (1.5,-3) {$\textrm{ін'єкція}$};
\end{tikzpicture}
\qquad

\begin{tikzpicture}
\fill[red!40] (0,0) ellipse (1cm and 2cm);
\fill[blue!40] (3,0) ellipse (1cm and 2cm);
\node (A1) at (0,0.8) [circle,fill,inner sep=1.5pt]{};
\node (A2) at (0,0) [circle,fill,inner sep=1.5pt]{};
\node (A3) at (0,-0.8) [circle,fill,inner sep=1.5pt]{};

\node (B1) at (3+0,0.8) [circle,fill,inner sep=1.5pt]{};
%\node (B2) at (4+0,0) [circle,fill,inner sep=1.5pt]{};
\node (B3) at (3+0,-0.8) [circle,fill,inner sep=1.5pt]{};
\node[anchor = south east] at (0,1) {$X$};
\node[anchor = south east] at (3,1) {$Y$};

\draw[thick, ->] (A1)--(B1);
\draw[thick, ->] (A2)--(B3); \draw[thick, ->] (A3)--(B3);
\node at (1.5,-3) {$\textrm{сюр'єкція}$};
\end{tikzpicture}

\qquad
\begin{tikzpicture}
\fill[red!40] (0,0) ellipse (1cm and 2cm);
\fill[blue!40] (3,0) ellipse (1cm and 2cm);
\node (A1) at (0,0.8) [circle,fill,inner sep=1.5pt]{};
\node (A2) at (0,0) [circle,fill,inner sep=1.5pt]{};
\node (A3) at (0,-0.8) [circle,fill,inner sep=1.5pt]{};

\node (B1) at (3+0,0.8) [circle,fill,inner sep=1.5pt]{};
\node (B2) at (3+0,0) [circle,fill,inner sep=1.5pt]{};
\node (B3) at (3+0,-0.8) [circle,fill,inner sep=1.5pt]{};
\node[anchor = south east] at (0,1) {$X$};
\node[anchor = south east] at (3,1) {$Y$};

\draw[thick, ->] (A1)--(B1);
\draw[thick, ->] (A2)--(B2); \draw[thick, ->] (A3)--(B3);
\node at (1.5,-3) {$\textrm{бієкція}$};
\end{tikzpicture}
}
\end{figure}
\end{example}
	
	\begin{proposition}
	Задано функцію $f: X \to Y$. \\
	$f$ - бієкція $\iff$ $f$ - одначно сюр'єкція та ін'єкція
	\end{proposition}
	
	\begin{proof}
	\rightproof Дано: $f$ - бієкція, тобто $\forall y \in Y: \exists! x: f(x) = y \implies $\\
	1) $\forall y \in Y: \exists x: f(x) = y$ - сюр'єкція;\\
	2) якщо при $x_1 \neq x_2$ вважати, що $f(x_2) = f(x_1) = y$, то це суперечить умові бієкції. Тому $f(x_1) \neq f(x_2)$ - ін'єкція.\\
	Отже, $f$ - одночасно сюр'єкція та ін'єкція.
	\bigskip \\
	
	\leftproof Дано: $f$ - одночасно сюр'єкцієя та ін'єкція.\\
	Візьмемо довільне $y \in Y$. Тоді $\exists x_1, x_2: f(x_1) = f(x_2) = y$. Суперечить ін'єкції.\\ 
	Тоді $\exists! x: f(x) = y$ - бієкція.
	\end{proof}
	
	\begin{definition}
	Функції $f: X \to Y$, $g: Y \to X$ називаються \textbf{взаємно оберненими}, якщо
	\begin{align*}
	\forall x \in X: g(f(x)) = x \\
	\forall y \in Y: f(g(y)) = y
	\end{align*}
	Позначення: $g = f^{-1}$.
\begin{figure}[H]
\centering {
\begin{tikzpicture}
\fill[red!40] (0,0) ellipse (1cm and 2cm);
\fill[blue!40] (4,0) ellipse (1cm and 2cm);

\node[anchor = south east] at (0,1) {$X$};
\node[anchor = south east] at (4,1) {$Y$};

\draw[thick, ->](0.5,0) .. controls (2,1) .. (4,0) node at (2,1.2) {$f$};
\draw[thick, ->](3.5,0) .. controls (2,-1) .. (0,0) node at (2,-1.2) {$g = f^{-1}$};
\end{tikzpicture}
}
\end{figure}
	\end{definition}
	
	\begin{example} Розглянемо такі приклади:\\
	1) $f: \mathbb{R} \to \mathbb{R}$: \hspace{1cm} $f(x) = x^3, g(x) = \sqrt[3]{x}$. \\
	2) $f: \mathbb{R} \to \mathbb{R}$: \hspace{1cm}
	$f(x) = x^2$ - оберненої не має.
	\end{example}
	
	\begin{proposition}
	Функції $f,g$ - взаємно обернені $\iff$ вони є бієкціями.
	\end{proposition}
	
	\begin{proof}
	\rightproof Дано: $g = f^{-1}$, тоді $\forall y \in Y: f(g(y)) = y$\\
	Доведемо для функції $f$.\\
	Візьмемо $y \in Y$. Тоді $\exists x = g(y): f(x) = f(g(y)) = y$. Отже, $f$ - сюр'єкція.\\
	Дано $x_1 \neq x_2$ і припустимо $f(x_1) = f(x_2) = y_0$. Тоді $g(y_0) = g(f(x_1)) = x_1$ та $g(y_0) = g(f(x_2)) = x_2$ і вони рівні. Отже, суперечність. Тоді $x_1 \neq x_2 \Rightarrow f(x_1) \neq f(x_2)$. Отже, $f$ - ін'єкція.\\
	З $g$ все аналогічно.
	\bigskip \\
	$\boxed{\Leftarrow}$ Дано: $f,g$ - бієкції.\\
	Розглянемо функцію $f: X \to Y$, щоб $y = f(x)$. Визначимо функцію $g: Y \to X$ так, щоб $x = g(y)$.\\
	Нехай $y_0 \in Y$. Тоді $\exists! x \in X: y = f(x)$. А тепер візьмемо $x = g(y)$. Тоді для нього $\exists! y = y_0 \in Y: x = g(y) = g(f(x))$. І так для будь-якого $y_0$\\
	Аналогічно доводиться, що $y = f(g(y))$.\\
	Отже, $g = f^{-1}$.
	\end{proof}
	
	\begin{example} Знайдемо обернену функцію для $f(x) = \dfrac{x+4}{2x-5}$.\\
	Позначимо $f(x) = y$. А тепер букви $x,y$ змінимо місцями - отримаємо:\\
	$x = \dfrac{y+4}{2y-5} \implies (2y-5)x = y+4 \implies y(2x-1) = 4+5x \implies y = \dfrac{4+5x}{2x-1}$.\\
	Отримали обернену функцію $f^{-1}(x) = \dfrac{4+5x}{2x-1}$.\\
	Щоб пересвідчитись, можна порахувати $f(f^{-1}(x))$ та $f^{-1}(f(x))$.
	\bigskip \\
	Буде, скоріш, коректно, якщо задати функцію так: \\ $f: \mathbb{R} \setminus \{2.5\} \to \mathbb{R} \setminus \{0.5\}$, а також $f^{-1}: \mathbb{R} \setminus \{0.5\} \to \mathbb{R} \setminus \{2.5\}$.\\ 
	Бо якщо ми залишимо викинуті точки, то тоді ми не зможемо порахувати $f \circ f^{-1}$ та $f^{-1} \circ f$.
	\end{example}
	
	\begin{definition}
	Задано функцію $f: X \to Y$.\\
	\textbf{Образом} множини $X_0 \subset X$ називається така множина:
	\begin{align*}
	f(X_0) = \{f(x) \in Y: x \in X_0 \}
	\end{align*}
	\textbf{Повним прообразом} множини $Y_0 \subset Y$ називається така множина:
	\begin{align*}
	f^{-1}(Y_0) = \{x \in X:  f(x) \in Y_0 \}
	\end{align*}
	\end{definition}
	
	\begin{example} Задамо функцію $f: \mathbb{R} \to \mathbb{R}$: $f(x) = x^2$ та множину $A = [-5, 4)$.\\
	$f(A) = \{f(x) = x^2: x \in [-5, 4) \} = [0, 25]$\\
	$f^{-1}(A) = \{x: f(x) = x^2 \in [-5, 4) \} \overset{x^2 < 4}{=} (-2, 2)$
	\end{example}
	
	\begin{proposition}[Властивості повних прообразів]
	1) $f^{-1}(A \cup B) = f^{-1}(A) \cup f^{-1}(B)$\\
	2) $f^{-1}(A \cap B) = f^{-1}(A) \cap f^{-1}(B)$\\
	3) $f^{-1}(\overline{A}) = \overline{f^{-1}(A)}$\\
	\textit{Випливає з теорії множин.}
	\end{proposition}
	
	\begin{remark}
	Властивість образів не часто співпадають. Зокрема: $f(A \cap B) \neq f(A) \cap f(B)$.
	\end{remark}
	
	
	\subsection{Границі функції}
	\begin{definition}
	Задано функцію $f: A \to \mathbb{R}$ та $x_0 \in \mathbb{R}$ - гранична точка для $A$.\\
	Число $b$ називається \textbf{границею функції в т.} $x_0$, якщо
	\begin{align*}
	\forall \varepsilon > 0: \exists \delta(\varepsilon) > 0: \forall x \in A: x \neq x_0: |x-x_0|<\delta \Rightarrow |f(x)-b|<\varepsilon \textrm{ - def. Коші}
	\end{align*}
	\begin{align*}
	\forall \{x_n, n \geq 1\}\subset A: \forall n \geq 1: x_n \neq x_0: \lim_{n \to \infty} x_n = x_0 \Rightarrow \lim_{n \to \infty} f(x_n) = b \textrm{ - def. Гейне}
	\end{align*}
	Позначення: $\huge \lim_{x \to x_0} f(x) = b$.
	\end{definition}
	
	\begin{theorem}
	Означення Коші $\iff$ Означення Гейне.
	\end{theorem}
	
	\begin{proof}
	\rightproof Дано: означення Коші, тобто\\
	$\forall \varepsilon > 0: \exists \delta > 0: \forall x \in A: x \neq x_0: |x-x_0|<\delta \Rightarrow |f(x)-b|<\varepsilon$.\\
	Зафіксуємо послідовність $\{x_n, n \geq 1\} \subset A$ таку, що $\forall n \geq 1: x_n \neq x_0: \huge \lim_{n \to \infty} x_n = x_0$.\\
	На це ми мали права, оскільки $x_0$ - гранична точка $A$.\\
	Нехай $\varepsilon > 0$. Тоді для нашого заданого $\exists \delta$, а для нього $\exists N: \forall n \geq N: |x_n - x_0| < \delta \implies |f(x_n) - b| < \varepsilon$.\\
	Таким чином, $\huge \lim_{n \to \infty} f(x_n) = b$ - означення Гейне.
	\bigskip \\
	\leftproof Дано: означення Гейне, тобто\\
	$\huge \forall \{x_n, n \geq 1\}\subset A: x_n \neq x_0: \forall n \geq 1: \lim_{n \to \infty} x_n = x_0 \Rightarrow \lim_{n \to \infty} f(x_n) = b$.\\
	!Припустимо, що означення Коші не виконується, тобто\\
	$\exists \varepsilon^*>0: \forall \delta > 0: \exists x_{\delta} \in A: x_{\delta} \neq x_{0}: |x_{\delta} - x_0| < \delta \Rightarrow |f(x_{\delta}) - b| \geq \varepsilon^*$.\\
	Зафіксуємо $\delta = \huge \frac{1}{n}$. Тоді побудуємо послідовність $\{x_n, n \geq 1\}$ таким чином, що \\ $x_n \in A: |x_n-x_0| < \huge \frac{1}{n} \Rightarrow \exists \lim_{x \to \infty} x_n = x_0$ за теоремою про поліцаї, але водночас $|f(x_n) - b| \geq \varepsilon^*$.\\
	Отже, суперечність!
	\end{proof}
	
	\begin{remark}
	Границя функції має єдине значення.\\
	\textit{Випливає з означення Гейне, оскільки границя числової послідовності є єдиною.}
	\end{remark}
	
	\begin{example}
	Задано функцію $f: \mathbb{R} \setminus \{2\} \to \mathbb{R}$, $f(x) = \dfrac{x^3-2x^2}{x-2}$. Довести, що $\huge \lim_{x \to 2} \dfrac{x^3-2x^2}{x-2} = 4$.\\
	За означенням Коші, ми хочемо, щоб\\
	$\forall \varepsilon > 0: \exists \delta > 0: \forall x: x \neq 2: |x-2|<\delta \Rightarrow \abs{\dfrac{x^3-2x^2}{x-2} - 4} <\varepsilon$\\
	$\abs{\dfrac{x^3-2x^2}{x-2} - 4} = \abs{\dfrac{x^2(x-2)}{x-2} - 4} = |x^2-4| = |x-2||x+2| \text{ } \boxed{<}$\\
	Необхідно якось обмежити $|x+2|$, щоб було все чудово. Можемо попросити, щоб $|x-2| < \underset{=\delta^*}{1}$. Тоді $-1<x-2<1 \Rightarrow |x+2|<5$.\\
	$\boxed{<} \text{ } 5|x-2| \text{ } \boxed{\boxed{<}}$\\
	А щоб отримати бажану оцінку, ми додатково просимо, щоб $|x-2| < \underset{=\delta^{**}}{\dfrac{\varepsilon}{5}}$.\\
	$\boxed{\boxed{<}} \text{ } \varepsilon$\\
	Ми використали одночасно нерівності $|x-2| < 1$, а також $|x-2| < \dfrac{\varepsilon}{5}$. Тому щоб дістатись до оцінки $\abs{\dfrac{x^3-2x^2}{x-2}-4}<\varepsilon$, необхідно вказати $\delta = \huge \min \left\{1, \frac{\varepsilon}{5} \right\}$ - тоді наше означення Коші буде виконаним.\\
	Отже, $\huge\lim_{x \to 2} \dfrac{x^3-2x^2}{x-2} = 4$.
\begin{figure} [H]
\centering
\resizebox{0.4\textwidth}{!}
{
\begin{tikzpicture}[scale = 0.8]

\draw[thick, ->] (-1,0)--(4,0) node[anchor = north] {$x$};
\draw[thick, ->] (0,-1)--(0,7.1) node[anchor = east] {$y$};

\draw[thick, domain=-1:2.5, variable=\x] plot({\x}, {\x*\x}) node[anchor = west, scale = 0.7] {$f(x) = x^2$};
\draw (2 cm, 1pt) -- (2 cm, -1pt) node[anchor = north] {$2$};
\draw (1 pt, 4cm) -- (-1 pt, 4cm) node[anchor = east] {$4$};
\draw[thick, blue, dashed] (2.2,2.2*2.2)--(2.2,0) node[anchor = north west, scale=0.9] {$2+\delta$};
\draw[thick, blue, dashed] (1.8,1.8*1.8)--(1.8,0) node[anchor = north east, scale=0.9] {$2-\delta$};
\draw[thick, red, dashed] (1.8,1.8*1.8)--(0,1.8*1.8) node[anchor = east, scale=0.9] {$4-\varepsilon$};
\draw[thick, red, dashed] (2.2,2.2*2.2)--(0,2.2*2.2) node[anchor = east, scale=0.9] {$4+\varepsilon$};
\draw[thick, dashed] (2,0)--(2,4);
\draw[thick, dashed] (2,4)--(0,4);
\node[white] at (2,4) [circle,fill,inner sep=1.5pt, draw = black]{};

\end{tikzpicture}

}
\caption*{Наскільки ми б не відступали від $4$ по осі $OY$, ми завжди знайдемо окіл т. $2$ на осі $OX$.}
\end{figure}
\end{example}

	\begin{example}
	Задано функцію $f: \mathbb{R} \to \mathbb{R}$, $f(x) = \dfrac{|x|}{x}$. Довести, що не існує границі $\huge \lim_{x \to 0} \dfrac{|x|}{x}$.\\
	За означенням та запереченням Гейне, зафіксуємо наступну послідовність:\\
	$\huge \left\{x_n = \frac{(-1)^n}{2n}, n \geq 1\right\}$, де $\huge \lim_{n \to \infty} x_n = 0$.\\
	Але $\huge \lim_{n \to \infty} \dfrac{|x_n|}{x_n} = \left[ \begin{gathered} 1, n = 2k \\ -1, n = 2k-1 \end{gathered} \right.$ - не збіжна, бо має різні часткові границі.\\
	Таким чином, прийшли до висновку: границі не існує.
	\begin{figure} [H]
	\centering
	\begin{tikzpicture}[scale=1.2]
	\draw[thick, ->] (-2,0)--(2,0) node[anchor = north] {$x$};
	\draw[thick, ->] (0,-2)--(0,2.1) node[anchor = east] {$y$};
		\draw[thick, domain=0.01:2, variable=\x] plot({\x}, {1}) node[anchor = west] {$f(x) = \dfrac{|x|}{x}$};
		\draw[thick, domain=-2:-0.01, variable=\x] plot({\x}, {-1});
	
	\draw[fill = white](0,-1) circle(2pt);
	\draw[fill = white](0,1) circle(2pt);
	\foreach \x in {1,2,...,20}
		\fill[red] ({2*((-1)^(\x))/(\x)},{0}) circle (1pt);

	\end{tikzpicture}
	\caption*{Червоні точки наближаються до нуля, але функція в цих точках скаче.}
	\end{figure}
	\end{example}
	
		\begin{definition}
	Задано функцію $f: A \to \mathbb{R}$ та $x_0 \in \mathbb{R}$ - гранична точка для $A$.\\
	Функція \textbf{прямує до нескінченності в т.} $x_0$, якщо:
	\begin{align*}
	\forall E > 0: \exists \delta(E) > 0: \forall x \in A: x \neq x_0: |x-x_0|<\delta \Rightarrow |f(x)|>E \textrm{ - def. Коші}
	\end{align*}
	\begin{align*}
	\forall \{x_n, n \geq 1\}\subset A: \forall n \geq 1: x_n \neq x_0: \lim_{n \to \infty} x_n = x_0 \Rightarrow \lim_{n \to \infty} f(x_n) = \infty \textrm{ - def. Гейне}
	\end{align*}
	Позначення: $\huge \lim_{x \to x_0} f(x) = \infty$.
	\end{definition}
	
	\begin{example}
	Задано функцію $f: \mathbb{R} \to \mathbb{R}$, $f(x) = \dfrac{1}{x^2}$. Доведемо, що $\huge\lim_{x \to 0} \dfrac{1}{x^2} = +\infty$.\\
	За означенням Коші, що ми хочемо:\\
	$\forall E > 0: \exists \delta > 0: \forall x: x \neq 0: |x|<\delta \Rightarrow \dfrac{1}{x^2} > E$.\\
	Із останньої нерівності, $x^2 < \dfrac{1}{E}$, тому одразу встановимо $\delta = \dfrac{1}{\sqrt{E}}$.\\
	Отже, $\huge\lim_{x \to 0} \dfrac{1}{x^2} = +\infty$.
\begin{figure} [H]
\centering
\begin{tikzpicture}[scale=0.8]
\draw[thick, ->] (-2,0)--(2,0) node[anchor = north] {$x$};
\draw[thick, ->] (0,-0.25)--(0,6) node[anchor = east] {$y$};

\draw[thick, domain=-2:-0.4, variable=\x] plot({\x}, {1/(\x*\x)});
\draw[thick, domain=0.4:2, variable=\x] plot({\x}, {1/(\x*\x)}) node[anchor = west, scale = 0.7] {$f(x) = \dfrac{1}{x^2}$};

\node at (0.2,-0.2) {$0$};
\draw[thick, blue, dashed] (0.5,4)--(0.5,0) node[anchor = north west, scale=0.9] {$0+\delta$};
\draw[thick, blue, dashed] (-0.5,4)--(-0.5,0) node[anchor = north east, scale=0.9] {$0-\delta$};
\draw[thick, red, dashed] (-0.8,4)--(0.8,4) node at (0.1,4.2) {$E$};
\end{tikzpicture}
\caption*{Всі значення функції навколо околу т. $0$ знаходяться вище за червоної лінії.}
\end{figure}
	\end{example}
	
	\begin{example}
	Можна довести, що $\huge\lim_{x \to 0} \dfrac{1}{x} = \infty$. Однак не можна визначити, чи $+\infty$ або $-\infty$.
	\end{example}
	
	\begin{definition}
	Задано функцію $f: \mathbb{R} \to \mathbb{R}$. \\
	Число $b$ називається \textbf{границею функції} при $x \to \infty$, якщо
	\begin{align*}
	\forall \varepsilon > 0: \exists \Delta(\varepsilon) > 0: \forall x \in \mathbb{R}: |x|>\Delta \Rightarrow |f(x)-b|<\varepsilon \textrm{ - def. Коші}
	\end{align*}
	\begin{align*}
	\forall \{x_n, n \geq 1\} \subset \mathbb{R}: \forall n \geq 1: \lim_{n \to \infty} x_n = \infty \Rightarrow \lim_{n \to \infty} f(x_n) = b \textrm{ - def. Гейне}
	\end{align*}
	Позначення: $\huge \lim_{x \to \infty} f(x) = b$.
	\end{definition}
	
	\begin{example}
	Задано функцію $f: \mathbb{R} \to \mathbb{R}$, $f(x) = \dfrac{\sin x}{x}$. Доведемо, що $\huge\lim_{x \to \infty} \dfrac{\sin x}{x} = 0$.\\
	За означенням Коші, ми вимагаємо, щоб:\\
	$\forall \varepsilon > 0: \exists \Delta > 0: \forall x: x \neq 0: |x| > \Delta \Rightarrow \abs{\dfrac{\sin x}{x}} < \varepsilon$\\
	Маємо таку оцінку: $\abs{\dfrac{\sin x}{x}} = \dfrac{|\sin x|}{|x|} \leq \dfrac{1}{|x|} < \varepsilon$\\
	Із цієї оцінки ми можемо встановити $\Delta = \dfrac{1}{\varepsilon}$. А тому $\huge\lim_{x \to \infty} \dfrac{\sin x}{x} = 0$.
	\end{example}
	
	\begin{remark}
	Можна спробувати записати def. Коші та def. Гейне для випадку $\huge \lim_{x \to \infty} f(x) = \infty$.
	\end{remark}
	
	\begin{remark}
	Для інших варіації границь функції, еквівалентність двох означень, за Коші та за Гейне, залишається в силі.
	\end{remark}
	
	\begin{remark}
	Там надалі я буду розглядати $x \to x_0$ при $x_0$ - гранична точка. Для $x \to \infty$ все теж саме можна записати.
	\end{remark}
	
	\begin{definition}
	Задано функцію $f: A \to \mathbb{R}$ та $x_0 \in \mathbb{R}$ - гранична точка для $A$.\\
	Якщо $\huge \lim_{x \to x_0} f(x) = \infty$, то функцію $f(x)$ називають \textbf{нескінченно великою (н.в.) в т.} $x_0$.\\
	Якщо $\huge \lim_{x \to x_0} f(x) = 0$, то функцію $f(x)$ називають \textbf{нескінченно малою (н.м.) в т.} $x_0$.\\
	\end{definition}
	
	\subsection{Основні властивості}
	\begin{theorem}[Арифметичні властивості н.м. та н.в. великих функцій]
	Задано функції $f,g,h: A \to \mathbb{R}$ - відповідно н.м., н.в., обмежена в $x_0 \in \mathbb{R}$ - гранична точка для $A$. Тоді:\\
	1) $f(x) \cdot h(x)$ - н.м. в т. $x_0$;\\
	2) $\huge \frac{1}{f(x)}$ - н.в. в т. $x_0$;\\
	3) $\huge \frac{1}{g(x)}$ - н.м. в т. $x_0$.
	\end{theorem}
	
	\begin{proof}
	Зафіксуємо $\{x_n, n \geq 1\}$, таку, що $\huge \lim_{n \to \infty} x_n = x_0$. Тоді за Гейне, \\ 
	$\huge \lim_{n \to \infty} f(x_n) = 0$, $\huge \lim_{n \to \infty} g(x_n) = \infty$, отже:\\
	$\{f(x_n), n \geq 1\}$ - н.м.;\\
	$\{g(x_n), n \geq 1\}$ - н.в.;\\
	$\{h(x_n), n \geq 1\}$ - досі обмежена.\\
	За властивостями границь послідовності, $\left\{f(x_n) \cdot h(x_n) \right\}$ - н.м., $\left\{ \huge \frac{1}{f(x_n)} \right\}$ - н.в., $\left\{ \huge \frac{1}{g(x_n)} \right\}$ - н.м.\\
	Ну а тому існують відповідні границі: $\huge \lim_{n \to \infty} f(x_n) h(x_n) = 0$, $\huge \lim_{n \to \infty} \frac{1}{f(x_n)} = \infty$, $\huge \lim_{n \to \infty} \frac{1}{g(x_n)} = 0$.\\
	За Гейне, отримаємо бажане: $\huge\lim_{x \to x_0} f(x)h(x) = 0 \hspace{0.5cm} \huge\lim_{x \to x_0} \dfrac{1}{f(x)} = \infty \hspace{0.5cm} \huge\lim_{x \to x_0} \dfrac{1}{g(x)} = 0$.
	\end{proof}

	\begin{example}
	Знайти границю $\huge \lim_{x \to \infty} \frac{(x-1)(x-2)(x-3)}{(4x-5)^3}$.\\
	Завдяки щойно доведеної теореми, ми отримаємо наступне:\\
	$\huge \lim_{x \to \infty} \frac{(x-1)(x-2)(x-3)}{(4x-5)^3} = \lim_{x \to \infty} \frac{(1-\frac{1}{x})(1-\frac{2}{x})(1-\frac{3}{x})}{(4-\frac{5}{x})^3} = \frac{1 \cdot 1 \cdot 1}{4^3} = \dfrac{1}{64}$.
	\end{example}
	
	\begin{theorem} Задано функцію $f: A \to \mathbb{R}$, що містить границю в т. $x_0$. Тоді вона є обмеженою в околі т. $x_0$.
	\end{theorem}
	
	\begin{proof}
	$\exists \huge \lim_{x \to x_0} f(x) = b \implies \forall \varepsilon > 0: \exists \delta: \forall x \in A: x \neq x_0: |x-x_0|<\delta \Rightarrow |f(x)-b|<\varepsilon$\\
	Зафіксуємо $\varepsilon = 1$, тоді $|f(x) -b| < 1$.\\
	$|f(x)| = |f(x) - b + b| \leq |f(x) - b| + |b| < 1 + |b|$.\\
	Покладемо $c = \max\{1 + |b|, f(x_0) \}$. А тому отримаємо:\\
	$\forall x \in A: |x-x_0| < \delta \Rightarrow |f(x)| < c$. Отже, обмежена.
	\end{proof}
	
	\begin{theorem}[Арифметика границь]
	Задано функції $f,g: A \to \mathbb{R}$, такі, що $\exists \huge \lim_{x \to x_0} f(x) = b_1$, $\exists \huge \lim_{x \to x_0} g(x) = b_2$. Тоді:\\
	1) $\forall c \in \mathbb{R}: \exists \huge \lim_{x \to x_0} cf(x) = c b_1$;\\
	2) $\exists \huge \lim_{x \to x_0} (f(x)+g(x)) = b_1 + b_2$;\\
	3) $\exists \huge \lim_{x \to x_0} f(x)g(x) = b_1 b_2$;\\
	4) $\exists \huge \lim_{x \to x_0} \frac{f(x)}{g(x)} = \frac{b_1}{b_2}$ при $b_2, g(x) \neq 0$.\\
	\textit{Випливають з властивостей границь числової послідовності, якщо доводити за Гейне. Доведу лише перший підпункт для прикладу.}
	\end{theorem}
	
	\begin{proof}
	$\forall \{x_n, n \geq 1\} \subset A: \forall n \geq 1: x_n \neq x_0: \huge \lim_{n \to \infty} x_n = x_0 \Rightarrow \lim_{n \to \infty} f(x_n) = b$\\
	Тоді $\huge \forall c \in \mathbb{R}: \lim_{n \to \infty} c f(x_n) = c \lim_{n \to \infty} f(x_n) = c b_1$.\\
	Таким чином, $\exists \huge \lim_{x \to x_0} cf(x) = c b_1$.
	\end{proof}
	
	\begin{example}
	Обчислити границю: $\huge \lim_{x \to 0} \frac{x^2-1}{2x^2-2x-1}$.\\
	$\huge \lim_{x \to 0} \frac{x^2-1}{2x^2-x-1} = \frac{\huge \lim_{x \to 0} (x^2-1)}{\huge \lim_{x \to 0}(2x^2-x-1)} = \frac{\huge \lim_{x \to 0}x^2 - \lim_{x \to 0}1}{\huge 2\lim_{x \to 0}x^2 - \lim_{x \to 0}x - \lim_{x \to 0}1} = \frac{0-1}{0-0-1} = 1$\\
	Ми пояснюємо ці рівності рівність справа наліво, як було з послідовностями.
	\end{example}
	
	\begin{theorem}
	Задано функцію $f: A\to \mathbb{R}$ та $x_0 \in \mathbb{R}$ - гранична точка для $A$.\\
Відомо, що в околі т. $x_0$ функція $f(x) < c$ та $\exists \huge \lim_{x \to x_0} f(x) = b$. Тоді $b \leq c$.
	\end{theorem}

	\begin{proof}
За Гейне, $\huge \forall \{x_n, n \geq 1\} \subset A:  \lim_{n \to \infty} x_n = x_0 \Rightarrow \lim_{n \to \infty} f(x_n) = b$. За властивостями границь числової послідовності, $b \leq c$.
	\end{proof}
	
	\begin{corollary}
	Задано функції $f,g: A \to \mathbb{R}$ такі, що в околі т. $x_0 \in \mathbb{R}$ - гранична точка для $A$ - справедлива $f(x) \leq g(x)$. Також $\exists \huge \lim_{x \to x_0} f(x) = b_1$, $\exists \huge \lim_{x \to x_0} g(x) = b_2$. Тоді $b_1 \leq b_2$.\\
\textit{Вказівка: розглянути функцію} $h(x) = f(x) - g(x)$.
	\end{corollary}
	
	\begin{theorem}[Теорема про 3 функції]
Задано функції $f,g,h: A \to \mathbb{R}$ та $x_0 \in \mathbb{R}$ - гранична точка для $A$.\\
Відомо, що в околі т. $x_0$ виконується: $f(x) \leq g(x) \leq h(x)$ та $\exists \huge \lim_{x \to x_0} f(x) = \lim_{x \to x_0} h(x) = a$.\\
Тоді $\exists \huge \lim_{x \to x_0} g(x) = a$.\\
\textit{Випливає з теореми про поліцаїв в числової послідовності.}
	\end{theorem}
	
	\begin{theorem}[Критерій Коші]
Задано функцію $f: A \to \mathbb{R}$ та $x_0 \in \mathbb{R}$ - гранична точка для $A$.\\
$\exists \huge \lim_{x \to x_0} f(x) \iff \forall \varepsilon > 0: \exists \delta(\varepsilon): \forall x_1,x_2 \in A: x_1,x_2 \neq x_0:$ 
$\begin{cases} |x_1-x_0|<\delta \\ |x_2-x_0|<\delta \end{cases} \Rightarrow |f(x_1)-f(x_2)|<\varepsilon
$
	\end{theorem}
	
	\begin{proof}
\rightproof Дано: $\exists \huge \lim_{x \to x_0} f(x) = b$, тобто за def. Коші,\\
$\forall \varepsilon > 0: \exists \delta: \forall x \in A: x \neq x_0: |x-x_0|<\delta \Rightarrow |f(x)-b|< \huge \frac{\varepsilon}{2}$.\\
Тоді $\forall x_1, x_2 \in A: |x_1 - x_0| < \delta$ і одночачно $|x_2 - x_0| < \delta \Rightarrow$\\
$|f(x_1)-f(x_2)| = |f(x_1)-b + b - f(x_2)| \leq |f(x_1) - b| + |f(x_2)-b| < \varepsilon$.\\
Отримали праву частину критерія.
\bigskip \\
\leftproof Дано: $\forall \varepsilon > 0: \exists \delta(\varepsilon): \forall x_1,x_2 \in A:, x_1,x_2 \neq x_0:$
$\begin{cases} |x_1-x_0|<\delta \\ |x_2-x_0|<\delta \end{cases} \Rightarrow |f(x_1)-f(x_2)|<\varepsilon$.\\
Розглянемо послідовність $\{t_n, n \geq 1\}$, таку, що $\huge \lim_{n \to \infty} t_n = x_0$.\\
Тоді за означенням, $\exists N: \forall n,m \geq N: \begin{cases} |t_n-x_0|<\delta \\ |t_m-x_0|<\delta \end{cases} \Rightarrow |f(t_n)-f(t_m)|<\varepsilon$.\\
Отримаємо, що $\{f(t_n),n \geq 1\}$ - фундаментальна послідовність, а тому є збіжною, тобто\\
$\exists \huge \lim_{n \to \infty} f(t_n) = b$.\\
А тепер час відповісти на питання, чи буде границя функції залежати від вибіру послідовності. Бо критерій Коші дає відповідь на збіжність, але не знає куди.\\
!Припустимо, що є послідовність $\{s_n, n \geq 1\}$, таку, що $\huge \lim_{n \to \infty} s_n = x_0$.\\
Тоді за аналогічними міркуваннями, $\exists \huge \lim_{n \to \infty} f(s_n) = a$, уже інша границя.
\bigskip \\
І нарешті, побудуємо послідовність $\{p_n, n \geq 1\}$ таким чином, що $p_{2k} = t_k$, $p_{2k-1} = s_k$. Тобто $\{s_1, t_1, s_2, t_2, \dots \}$.\\
Тут $\exists \huge \lim_{n \to \infty} p_n = x_0$. Тоді знову за аналогічними міркуваннями, $\exists \huge \lim_{n \to \infty} f(p_n)$, але чому буде дорівнювати, зараз побачимо.\\
Оскільки $\exists \huge \lim_{n \to \infty} f(p_n)$, то одночасно $\exists \huge \lim_{k \to \infty} f(p_{2k}) = b$, $\exists \huge \lim_{k \to \infty} f(p_{2k-1}) = a$.\\
У збіжної послідовності є лише одна часткова послідовність, тому $a = b$. Суперечність!\\
Це означає, що результат не залежить від вибору послідовності.\\
Тому за Гейне, отримаємо, що $\exists \huge \lim_{x \to x_0} f(x) = b$.
\end{proof}

\begin{theorem}[Границя від композиції функції]
Задано функції $f: A \to B$, $g: B \to \mathbb{R}$ та композиція $h = g(f(x))$. Більш того, $x_0, y_0 \in \mathbb{R}$ - граничні точки відповідно для $A,B$ та $\exists \huge \lim_{x \to x_0} f(x) = y_0$ та $\exists \huge \lim_{y \to y_0} g(y) = b$.\\
Тоді $\exists \huge \lim_{x \to x_0} h(x) = b$.\\
\textit{Це ще називають \lq\lq заміною в границях\rq\rq}
\end{theorem}

\begin{proof}
$\exists \huge \lim_{y \to y_0} g(y) = b \overset{\textrm{def.}}{\Rightarrow} \forall \varepsilon > 0: \exists \delta: \forall y \in B: y \neq y_0: |y-y_0|<\delta \Rightarrow |g(y)-b|<\varepsilon$\\
$\exists \huge \lim_{x \to x_0} f(x) = y_0 \overset{\textrm{def.}}{\Rightarrow} \forall \delta > 0: \exists \tilde{\delta}: \forall x \in A: x \neq x_0: |x-x_0|<\tilde{\delta} \Rightarrow |f(x)-y_0|<\delta$\\
Таким чином, можемо отримати: $\forall \varepsilon > 0: \exists \delta > 0 \Rightarrow \exists \tilde{\delta}: \forall x \in A: x \neq x_0: |x-x_0| < \tilde{\delta} \Rightarrow \\ |f(x)-y_0| = |y-y_0|<\delta \Rightarrow |g(y)-b|=|g(f(x))-b| = |h(x)-b|<\varepsilon$\\
Отже, $\exists \huge \lim_{x \to x_0} h(x) = b$.
\end{proof}

\begin{example}
Обчислити границю: $\huge\lim_{x \to 1} \dfrac{x^3-2x^8+1}{x^{40}-3x^{10}+2}$.\\
Ми не розпишемо це арифметичними властивостями, тому що (поки що за означенням Коші) ліміт чисельника - нуль, ліміт знаменика - нуль. І це - невизначеність.\\
Проведемо заміну: $x = t+1$. Оскільки $x \to 1$, то тоді $t \to 0$. А далі порахуємо таку границю:\\
$\huge\lim_{t \to 0} \dfrac{(t+1)^3-2(t+1)^8+1}{(t+1)^{40}-3(t+1)^{10}+2} \overset{\text{ф-ла бінома}}{=} \lim_{t \to 0} \dfrac{(t^3+3t^2+3t+1) - 2(t^8+8t^7+\dots+8t+1)+1}{(t^{40}+40t^{39}+\dots+40t+1)-3(t^{10}+10t^9+\dots+10t+1)+2} = \\ = \lim_{t \to 0} \dfrac{(t^3+3t^2+3t)-2(t^8+8t^7+\dots+8t)}{(t^{40}+40t^{39}+\dots+40t)-3(t^{10}+10t^9+\dots+10t)} = \lim_{t \to 0} \dfrac{(t^2+3t+3)-2(t^7+8t^6+\dots+8)}{(t^{39}+40t^{38}+\dots+40)-3(t^{9}+10t^8+\dots+10)} = \dfrac{3-2 \cdot 8}{40-3 \cdot 10} = -\dfrac{13}{10}$\\
Отже, $\huge\lim_{x \to 1} \dfrac{x^3-2x^8+1}{x^{40}-3x^{10}+2} = -\dfrac{13}{10}$.
\bigskip \\
Як тут використалась теорема про композицію. \\ У нас $h(x) = g(f(x)) = \dfrac{x^3-2x^8+1}{x^{40}-3x^{10}+2}$, від якої ми шукаємо ліміт. \\
Далі, $f(x) = x-1$, \hspace{0.5cm} $g(y) = \dfrac{(y+1)^3+2(y+1)^8 + 1}{(y+1)^{40}-3(y+1)^{10}+2}$.\\
Знаємо, що $\huge\lim_{x \to 1} f(x) = 0$, тобто $y \to 0$.\\
Знаємо, що $\huge\lim_{y \to 0} g(y) = -\dfrac{13}{10}$ - цей ліміт ми вже рахували через арифметичні властивості.\\
А тому початковий ліміт, тобто $\huge\lim_{x \to 0} h(x) = -\dfrac{13}{10}$.
\end{example}

\subsection{Односторонні границі та границі монотонних функцій}
\begin{definition}
Задано функцію $f: A \to \mathbb{R}$, та $x_0 \in \mathbb{R}$ - гранична точка для $A$.\\
Числом $b$ називають \textbf{границею справа}, якщо
\begin{align*}
\forall \varepsilon > 0: \exists \delta(\varepsilon)>0: \forall x \in A: x_0<x<x_0+\delta \Rightarrow |f(x)-b|<\varepsilon \textrm{ - def. Коші}\\
\forall \{x_n,n\geq 1\} \subset A: \forall n \geq 1: x_n > x_0: \lim_{n \to \infty}x_n = x_0 \Rightarrow \lim_{n \to \infty}f(x_n) = b \textrm{ - def. Гейне}
\end{align*}
Позначення: $\huge f(x_0+0) = \lim_{x \to x_0^+} f(x) \overset{\text{або}}{=} \lim_{x \to x_0+0} f(x) = b$.\\
Числом $\tilde{b}$ називають \textbf{границею зліва}, якщо
\begin{align*}
\forall \varepsilon > 0: \exists \delta(\varepsilon)>0: \forall x \in A:  x_0-\delta<x<x_0 \Rightarrow |f(x)-\tilde{b}|<\varepsilon \textrm{ - def. Коші}\\
\forall \{x_n,n\geq 1\} \subset A: \forall n \geq 1: x_n < x_0: \lim_{n \to \infty}x_n = x_0 \Rightarrow \lim_{n \to \infty}f(x_n) = \tilde{b} \textrm{ - def. Гейне}
\end{align*}
Позначення: $\huge f(x_0-0) = \lim_{x \to x_0^-} f(x) \overset{\text{або}}{=} \lim_{x \to x_0-0} f(x) = \tilde{b}$.
\end{definition}

\begin{theorem}
Задано функцію $f: A \to \mathbb{R}$, та $x_0 \in \mathbb{R}$ - гранична точка для $A$.\\
$\exists \huge \lim_{x \to x_0} f(x) = b \iff \exists \begin{cases} \huge \lim_{x \to x_0^+} f(x) = b \\ \huge \lim_{x \to x_0^-} f(x) = b \end{cases}$
\end{theorem}

\begin{proof}
$\exists \huge \lim_{x \to x_0} f(x) = b \iff$
$\forall \varepsilon > 0: \exists \delta: \forall x \in A: |x-x_0|<\delta \Rightarrow |f(x)-b|<\varepsilon$\\
$\iff \forall \varepsilon > 0: \exists \delta: \forall x \in A: |x-x_0|<\delta \Rightarrow \begin{cases} x-x_0<\delta \\ x_0-x<\delta \end{cases} \Rightarrow |f(x)-b|<\varepsilon$ \\ $\iff \exists \begin{cases} \huge \lim_{x \to x_0^+} f(x) = b \\ \huge \lim_{x \to x_0^-} f(x) = b \end{cases}$
\end{proof}

\begin{remark}
Для функції $f(x) = \sqrt{x}$ є границя $\huge\lim_{x \to 0+0} \sqrt{x} = 0$, але не існує $\huge\lim_{x \to 0-0} \sqrt{x}$. Тобто не існує $\huge\lim_{x \to 0} \sqrt{x}$ - звісно, ні.\\
Справа в тому, що попередню теорему можна застосовувати, коли т. $x_0 = 0$ була б визначена одночасно десь лівіше й правіше. А область визначення $A = [0,+\infty)$, тобто ми вже не можемо розглядати границі зліва.\\
Коротше, все гуд тут, $\huge\lim_{x \to 0} \sqrt{x} = 0$.
\end{remark}

\begin{example}
Повернімось до функції $f(x) = \dfrac{|x|}{x}$. Ми довели, що границя в т. $x=0$ не існує, проте\\
$\huge\lim_{x \to 0+0} \dfrac{|x|}{x} = 1 \hspace{1cm} \lim_{x \to 0-0} \dfrac{|x|}{x} = -1$.
\end{example}

\begin{definition}
Задано функцію $f: (a,b) \to \mathbb{R}$.\\
Функцію називають \textbf{монотонно}:\\
\begin{tabular}{ll}
\textbf{- строго зростаючою}, якщо & $\forall x_1,x_2 \in (a,b): x_1 > x_2 \Rightarrow f(x_1)>f(x_2)$\\
\textbf{- не спадною}, якщо & $\forall x_1,x_2 \in (a,b): x_1 > x_2 \Rightarrow f(x_1) \geq f(x_2)$\\
\textbf{- строго спадною}, якщо & $\forall x_1,x_2 \in (a,b): x_1 > x_2 \Rightarrow f(x_1) < f(x_2)$\\
\textbf{- не зростаючою}, якщо & $\forall x_1,x_2 \in (a,b): x_1 > x_2 \Rightarrow f(x_1) \leq f(x_2)$
\end{tabular}\\
Функцію називають \textbf{обмеженою}, якщо $\exists M>0: \forall x \in (a,b): |f(x)| \leq M$
\end{definition}

\begin{theorem}
Задано функцію $f: (a,b) \to \mathbb{R}$ - монотонна та обмежена.\\
Тоді $\huge \exists \lim_{x \to b^-} f(x) = d$ та $\huge \exists \lim_{x \to a^+} f(x) = c$.
\end{theorem}

\begin{proof}
Доведу лише першу границю і буду вважати, що функція строго спадна. Для решти аналогічно.
\bigskip \\
Отже, $f$ - строго спадає, тобто $\forall x_1,x_2 \in (a,b): x_1>x_2 \Rightarrow f(x_1)<f(x_2)$.\\
Більш того, $f$ - обмежена, тому $\exists \huge \inf_{x \in (a,b)} f(x) = d$\\
Доведемо, що вона є границею зліва. За критерієм $\inf$:\\
1) $\forall x \in (a,b): f(x) \geq d$\\
2) $\forall \varepsilon > 0: \exists x_{\varepsilon} \in (a,b): f(x_{\varepsilon})< d + \varepsilon$.\\
Оберемо $\delta = b - x_{\varepsilon} > 0$. Тоді $\forall x \in (a,b): b-x<\delta \Rightarrow x > b - (b-x_{\varepsilon}) = x_{\varepsilon} \Rightarrow f(x) < f(x_\varepsilon)$.\\
Звідси справедлива наступна нерівність:\\
$d - \varepsilon < d \leq f(x) < f(x_\varepsilon) < d + \varepsilon \Rightarrow |f(x)-d| < \varepsilon$.\\
Остаточно, за def. Коші, $\exists \huge \lim_{x \to b^-} f(x) =  d$.
\end{proof}

\subsection{Перша чудова границя}
Розглянемо такий геометричний малюнок:\\
\begin{figure}[H]
\centering
%\resizebox{0.5\textwidth}{!} 
{
\begin{tikzpicture}[scale = 0.5]
\fill[thick, fill = blue!40] (1,0) arc (0:45:1cm) -- (0,0) -- (1,0) -- cycle node[anchor = south west] {$x$};
\draw[thick, ->] (-1.5*3cm,0)--(1.5*3cm,0);
\draw[thick, ->] (0,-1.5*3cm)--(0,1.5*3cm);
\draw[thick] (0,0) circle (1*3cm);
\draw[thick] (0,0)--(1*3,1*3) node[anchor = south west] {$K$};
\draw[thick] (1*3,1*3)--(1*3,0) node[anchor = south west] {$C$};
\draw[thick] ({cos(45)*3},{sin(45)*3})--({cos(45)*3},0) node[anchor = north] {$B$};
\node[anchor = north west] at (0,0) {$0$};
\node[anchor = south] at ({cos(45)*3},{sin(45)*3}) {$A$};
\draw (1pt, 1*3 cm) -- (-1pt, 1*3 cm) node[anchor = south east] {$1$};
\draw (1*3 cm, 1pt) -- (1*3 cm, -1pt) node[anchor = north west] {$1$};
\end{tikzpicture}
}
\caption*{Коло радіусом $1$. Вважаємо поки $x \in \left(0, \dfrac{\pi}{2} \right)$.}
\end{figure}
Виділимо з малюнку наступні дані:
	$|AB|=\sin x$;\\
	$|AC|=x$;\\
	$|KC|=\tg x$.\\
	Зрозуміло, що $|AB|<|AC|<|KC| \implies \sin x < x < \tg x$.\\
	Розглянемо обидва сторони нерівності:\\
	$\displaystyle \sin x < x \implies \frac{\sin x}{x} < 1$.\\
	$\displaystyle x < \tg x = \frac{\sin x}{\cos x} \Rightarrow \frac{\sin x}{x} > \cos x = 1-2 \sin^2 \frac{x}{2} > 1 - 2 \frac{x^2}{4} = 1 - \frac{x^2}{2}$.\\
	$\displaystyle 1- \frac{x^2}{2}<\frac{\sin x}{x} < 1$.\\
Можна розширити інтервал до $\huge \left(-\frac{\pi}{2},\frac{\pi}{2} \right)$, оскільки нерівність не змінюється. Тому за теоремою про 3 функції, маємо наступне:
\begin{theorem}[I чудова границя]
$\huge \lim_{x \to 0} \frac{\sin x}{x} = 1$
\end{theorem}

\begin{corollary}[Наслідки І чудової границі]
0) $\huge\lim_{x \to 0} \cos x = 1$\\
1) $\huge \lim_{x \to 0} \frac{\tg x}{x} = 1$\\
2) $\huge \lim_{x \to 0} \frac{\arcsin x}{x} = 1$\\
3) $\huge \lim_{x \to 0} \frac{\arctg x}{x} = 1$
\end{corollary}

\begin{proof}
0) $\huge\lim_{x \to 0} \cos x = \lim_{x \to 0} \left( 1-\sin^2 \dfrac{x}{2} \right) = 1 - \lim_{x \to 0} \dfrac{\sin \dfrac{x}{2} \sin \dfrac{x}{2} \cdot \dfrac{x}{2} \cdot \dfrac{x}{2}}{\dfrac{x}{2} \cdot \dfrac{x}{2}} = 1 - \lim_{x \to 0} \dfrac{\sin \dfrac{x}{2}}{\dfrac{x}{2}} \lim_{x \to 0} \dfrac{\sin \dfrac{x}{2}}{\dfrac{x}{2}} \lim_{x \to 0} \dfrac{x^2}{4} \boxed{=}$\\
В перших двох лімітів заміна: $\dfrac{x}{2} = t$. Оскільки $x \to 0$, то $t \to 0$, тоді $\huge\lim_{x \to 0} \dfrac{\sin \dfrac{x}{2}}{\dfrac{x}{2}} = \lim_{t \to 0} \dfrac{\sin t}{t} =1$\\
$\boxed{=} 1 - 1 \cdot 1 \cdot 0 = 1$
\end{proof}

1) \textit{Вказівка: $\tg x = \dfrac{\sin x}{\cos x}$.} \\
2) \textit{Вказівка: $\arcsin x = t$.} \\
3) \textit{Вказівка: $\arctg x = t$.}

\subsection{Друга чудова границя}
Відомо, що $\forall x \in \mathbb{R}$ справедлива нерівність: $[x] \leq x < [x]+1$.\\
Тоді можна дійти до цієї нерівності:\\
$\huge \left(1 + \frac{1}{[x]+1} \right)^{[x]} < \left(1 + \frac{1}{x} \right)^x < \left(1 + \frac{1}{[x]} \right)^{[x]+1}$.\\
Вважаємо, що $x \to +\infty$, тоді відповідно $[x] \to + \infty$ та $[x]+1 \to + \infty$.\\
Також $[x] \in \mathbb{N}$, тому за визначенням числа Ейлера,
$\huge \lim_{[x] \to +\infty} \left(1 + \frac{1}{[x]} \right)^{[x]} = e$.\\
Скористаємось цим фактом в нашої нерівності:\\
$\huge \left(1 + \frac{1}{[x]+1} \right)^{[x]} = \frac{\huge \left(1 + \frac{1}{[x]+1} \right)^{[x]+1}}{\huge 1 + \frac{1}{[x]+1}} \to \frac{e}{1} = e$\\
$\huge \left(1 + \frac{1}{[x]} \right)^{[x]+1} = \left(1 + \frac{1}{[x]} \right)^{[x]} \left(1 + \frac{1}{[x]} \right) \to e \cdot 1 = e$.\\
І це все при $x \to +\infty$. Тоді за теоремою про поліцаїв, отримаємо ще одну чудову границю:
\begin{theorem}[II чудова границя]
$\huge \lim_{x \to +\infty} \left(1 +\frac{1}{x} \right)^x = e$
\end{theorem}

\begin{corollary}[Наслідки ІІ чудової границі]
1) $\huge \lim_{x \to -\infty} \left(1 +\frac{1}{x} \right)^x = e$\\
2) $\huge \lim_{x \to 0} \left(1 +x \right)^{\textstyle \frac{1}{x}} = e$\\
3) $\huge \lim_{x \to 0} \frac{\ln(1+x)}{x} = 1$\\
4) $\huge \lim_{x \to 0} \frac{e^x - 1}{x} = 1$\\
5) $\huge \lim_{x \to 0} \frac{(1+x)^\alpha - 1}{x} = \alpha$

\begin{proof}
1) $\huge\lim_{x \to -\infty} \left(1 + \dfrac{1}{x} \right)^x \overset{x =-t}{=} \lim_{t \to +\infty} \left(1 - \dfrac{1}{t} \right)^{-t} = \lim_{t \to \infty} \left(1 + \dfrac{1}{t-1} \right)^t \overset{t-1 = y}{=} \lim_{y \to +\infty} \left(1 + \dfrac{1}{y} \right)^{y+1} = \\ = \lim_{y \to +\infty} \left( 1 + \dfrac{1}{y} \right) \lim_{y \to +\infty} \left( 1 + \dfrac{1}{y} \right)^y = 1 \cdot e = e$
\end{proof}

2) \textit{Вказівка: $\dfrac{1}{x} = t$.}\\
3) \textit{Вказівка: використати властивість логарифма. Тут я використав все ж таки факт про неперервність функції $\ln x$.}\\
4) \textit{Вказівка: $x = \ln(1+t)$.}\\
5) \textit{Вказівка: $1 + x = e^t$.}
\end{corollary}

\begin{example}
Обчислити границю $\huge\lim_{x \to 0} \dfrac{\ln (\cos 2x)}{\ln (\cos 3x)}$ - універсальний приклад.\\
$\huge\lim_{x \to 0} \dfrac{\ln (\cos 2x)}{\ln (\cos 3x)} = \lim_{x \to 0} \dfrac{\ln (1 + (\cos 2x - 1))}{\ln (1 + (\cos 3x - 1))} = \lim_{x \to 0} \dfrac{\dfrac{\ln (1 + (\cos 2x - 1))}{\cos 2x - 1}}{\dfrac{\ln (1 + (\cos 3x - 1))}{\cos 3x - 1}} \cdot \dfrac{\cos 2x -1}{\cos 3x - 1} = \\ = \dfrac{\huge \lim_{x \to 0} \dfrac{\ln (1 + (\cos 2x - 1))}{\cos 2x - 1}}{\huge \lim_{x \to 0} \dfrac{\ln (1 + (\cos 3x - 1))}{\cos 3x - 1}} \lim_{x \to 0} \dfrac{\cos 2x -1}{\cos 3x - 1} \boxed{=}$\\
Заміна для першої границі: $\cos 2x - 1 = t$. Оскільки $x \to 0$, то звідси $t \to 0$.\\
Заміна для другої границі: $\cos 3x - 1 = t$. Оскільки $x \to 0$, то звідси $t \to 0$.\\
Звели ці ліміти до ІІ чудових границь.\\
$\boxed{=} \huge \dfrac{\huge\lim_{t \to 0} \dfrac{\ln (1+t)}{t}}{\huge \lim_{t \to 0} \dfrac{\ln(1+t)}{t}} \lim_{x \to 0} \dfrac{\cos 2x -1}{\cos 3x - 1} = \dfrac{1}{1} \lim_{x \to 0} \dfrac{\cos 2x -1}{\cos 3x - 1} = \lim_{x \to 0} \dfrac{1 - \cos 2x}{1 - \cos 3x} = \\ = \lim_{x \to 0} \dfrac{2 \sin^2 x}{2 \sin^2 \dfrac{3x}{2}} = \lim_{x \to 0} \dfrac{\dfrac{\sin^2 x}{x^2}}{\dfrac{\sin^2 \dfrac{3x}{2}}{\dfrac{9x^2}{4}}} \cdot \dfrac{x^2}{\dfrac{9x^2}{4}} = \dfrac{\huge \lim_{x \to 0} \dfrac{\sin^2 x}{x^2}}{\huge \lim_{x \to 0} \dfrac{\sin^2 \dfrac{3x}{2}}{\dfrac{9x^2}{4}}} \huge\lim_{x \to 0} \dfrac{x^2}{\dfrac{9x^2}{4}} = \dfrac{\huge \lim_{x \to 0} \dfrac{\sin x}{x} \lim_{x \to 0} \dfrac{\sin x}{x}}{\huge \lim_{x \to 0} \dfrac{\sin \dfrac{3x}{2}}{\dfrac{3x}{2}} \lim_{x \to 0} \dfrac{\sin \dfrac{3x}{2}}{\dfrac{3x}{2}}} \cdot \dfrac{4}{9} \boxed{\boxed{=}}$\\
Заміна для границь в знаменнику: $\dfrac{3x}{2} = t$. Оскільки $x \to 0$, то звідси $t \to 0$.\\
Звели ці ліміти до І чудових границь.\\
$\boxed{\boxed{=}} \dfrac{\huge \lim_{x \to 0} \dfrac{\sin x}{x} \lim_{x \to 0} \dfrac{\sin x}{x}}{\huge \lim_{t \to 0} \dfrac{\sin t}{t} \lim_{t \to 0} \dfrac{\sin t}{t}} \cdot \dfrac{4}{9} = \dfrac{1 \cdot 1}{1 \cdot 1} \cdot \dfrac{4}{9} = \dfrac{4}{9}$
\end{example}

\subsection{Порівняння функцій, відношення О-велике, о-маленьке та еквівалентності}
\begin{definition}
Задано функції $f,g: A \to \mathbb{R}$ та $x_0 \in \mathbb{R}$ - гранична точка для $A$.\\
Функція $f$ називається \textbf{порівнянною} з функцією $g$, якщо
\begin{align*}
\exists L>0: \exists \delta > 0: \forall x \in A: x \neq x_0: |x-x_0| < \delta \Rightarrow |f(x)| \leq L |g(x)|
\end{align*}
Позначення: $f(x) = O(g(x)), x \to x_0$\\
Інакше називають, що $f$ - \textbf{обмежена відносно} $g$ при $x \to x_0$.
\end{definition}

\begin{theorem}[Властивості]
1) $f(x) = O(g(x)), x \to x_0 \iff \huge \frac{f(x)}{g(x)}$ - обмежена в околі т. $x_0$.
\bigskip \\
2) Якщо $\exists \huge \lim_{x \to x_0} \frac{f(x)}{g(x)} = c$, то $f(x) = O(g(x)), x \to x_0$.
\bigskip \\
3) Нехай $f_1(x) = O(g(x)), f_2(x) = O(g(x))$. Тоді:\\
a) $f_1(x) + f_2(x) = O(g(x))$;\\
b) $\forall \alpha \in \mathbb{R}: \alpha f_1(x) = O(g(x))$;\\
c) $\forall \alpha \neq 0: f_1(x) = O(\alpha g(x))$;\\
Всюди $x \to x_0$.
\bigskip \\
4) Нехай $f(x) = O(g(x))$, $g(x) = O(h(x))$. Тоді $f(x) = O(h(x)), x \to x_0$.
\end{theorem}

\begin{proof}
Доведу лише 3 а). Інші зрозуміло.\\
$f_1(x) = O(g(x)) \Rightarrow \exists L_1: \exists \delta_1: \forall x : |x-x_0| < \delta_1 \Rightarrow |f_1(x)| \leq L_1 |g(x)| \\
f_2(x) = O(g(x)) \Rightarrow \exists L_2: \exists \delta_2: \forall x : |x-x_0| < \delta_2 \Rightarrow |f_2(x)| \leq L_2 |g(x)|$\\
Тоді $\exists \delta = \min\{\delta_1, \delta_2 \}: \forall x: |x-x_0|<\delta \Rightarrow$\\
$|f(x_1)+f(x_2)| \leq |f(x_1)|+|f(x_2)| \leq (L_1+L_2)|g(x)|$.\\
А тому $f_1(x) + f_2(x) = O(g(x))$.
\end{proof}

\begin{example}
Довести, що $x+x^2 = O(x), x \to 0$.\\
Знайдемо наступну границю:\\
$\huge \lim_{x \to 0} \frac{x+x^2}{x} = \lim_{x \to 0}(1+x) = 1$\\
Отже, $x+x^2 = O(x), x \to 0$.
\end{example}

\begin{remark}
В математичному аналізі О-велике не використовується часто, це більше вже для дослідження алгоритмів в комп'ютерних науках\\
Зокрема існує такий алгоритм Binary Search для пошуку елемента в відсортованому масиві. Складність алгоритму оцінюється в $O(\log_2 n)$, де $n$ - кількість елементів.
\end{remark}

\begin{definition}
Задано функції $f,g: A \to \mathbb{R}$ та $x_0 \in \mathbb{R}$ - гранична точка для $A$.\\
Функція $f$ називається \textbf{знехтувально малою} відносно $g$, якщо
\begin{align*}
\forall \varepsilon>0: \exists \delta > 0: \forall x \in A: x \neq x_0: |x-x_0| < \delta \Rightarrow |f(x)| < \varepsilon |g(x)|
\end{align*}
Позначення: $f(x) = o(g(x)), x \to x_0$.\\
Інакше кажуть, що $f$ - \textbf{нескінченно мала/великою більш високого порядку, ніж} $g$ при $x \to x_0/\infty$.
\end{definition}

\begin{theorem}[Властивості]
1) $f(x) = o(g(x)), x \to x_0 \iff \huge \exists \lim_{x \to x_0} \frac{f(x)}{g(x)} = 0$.
\bigskip \\
2) Нехай $f_1(x) = o(g(x)), f_2(x) = o(g(x))$. Тоді:\\
a) $f_1(x) + f_2(x) = o(g(x))$;\\
b) $\forall \alpha \in \mathbb{R}: \alpha f_1(x) = o(g(x))$;\\
c) $\forall \alpha \neq 0: f_1(x) = o(\alpha g(x))$;\\
Всюди $x \to x_0$.
\bigskip \\
3) Нехай $f(x) = o(g(x))$, $g(x) = o(h(x))$. Тоді $f(x) = o(h(x)), x \to x_0$.
\end{theorem}

\begin{proof}
Доведу лише 1), Інші Зрозуміло.\\
$f(x) = o(g(x)),x \to x_0 \iff \forall \varepsilon>0: \exists \delta: \forall x \in A: |x-x_0| < \delta \Rightarrow |f(x)| < \varepsilon |g(x)| \iff \\ \huge \abs{\frac{f(x)}{g(x)} - 0} < \varepsilon \iff \exists \lim_{x \to x_0} \frac{f(x)}{g(x)} = 0$.
\end{proof}

\begin{example}
Довести, що $x^3 - x^2 - x + 1 = o(x-1), x \to 1$.\\
Знайдемо наступну границю:\\
$\huge \lim_{x \to 1} \dfrac{x^3-x^2-x+1}{x-1} = \lim_{x \to 1} \dfrac{x^2(x-1)-(x-1)}{x-1} = \lim_{x \to 1} (x^2-1) = 0$.\\
Отже, $x^3 - x^2 - x + 1 = o(x-1), x \to 1$.
\begin{figure} [H]
\centering
{
\begin{tikzpicture}
\draw[thick, ->] (-1.2,0)--(4,0) node[anchor = north] {$x$};
\draw[thick, ->] (0,-2)--(0,5) node[anchor = east] {$y$};


\draw[thick, domain=-1.2:2.2, variable=\x, samples = 1000] plot({\x}, {(\x)^3 - (\x)^2 - \x + 1}) node[anchor = south west, scale = 0.7] {$f(x) = x^3 - x^2 - x + 1$};
\draw[thick, domain=-1:2.2, variable=\x, samples = 1000] plot({\x}, {\x - 1}) node[anchor = south west, scale = 0.7] {$g(x) = x - 1$};
\end{tikzpicture}
\caption*{Тут $x-1$ миттєво стала нулем і миттєво пішла далі. А $x^3-x^2-x+1$ набагато довше була близька в нулі.}
}
\end{figure}
\end{example}

\begin{theorem}[Інші властивості]
1.1) Нехай $f(x) = o(g(x))$ та $g(x) = O(h(x))$. Тоді $f(x) = o(h(x)), x \to x_0$.\\
1.2) Нехай $f(x) = O(g(x))$ та $g(x) = o(h(x))$. Тоді $f(x) = o(h(x)), x \to x_0$.
\bigskip \\
2) Нехай $f(x) = o(g(x))$. Тоді $f(x) = O(g(x)), x \to x_0$.
\end{theorem}

\begin{proof}
1) для обох випадків\\
$\huge \lim_{x \to x_0} \frac{f(x)}{h(x)} = \lim_{x \to x_0} \frac{f(x)}{g(x)} \frac{g(x)}{h(x)} =$ (обм *н.м.) $= 0 \Rightarrow f(x) = o(h(x)), x \to x_0$.
\bigskip \\
2) \textit{Випливає з властивості 2 О-великого.}
\end{proof}

\begin{definition}
Задано функції $f,g: A \to \mathbb{R}$ та $x_0 \in \mathbb{R}$ - гранична точка для $A$.\\
Функція $f$ називається \textbf{еквівалентною} $g$, якщо
\begin{align*}
\exists \huge\lim_{x \to x_0} \dfrac{f(x)}{g(x)} = 1
\end{align*}
Позначення: $f(x) \sim g(x), x \to x_0$.\\
Тобто функції $f(x)$ та $g(x)$ в околі т. $x_0$ мають однакову поведінку.
\end{definition}

\begin{theorem}
$f(x) \sim g(x), x \to x_0 \iff f(x)-g(x) = o(g(x))$
\end{theorem}

\begin{proof}
$f(x) \sim g(x), x \to x_0 \iff \huge \lim_{x \to x_0} \frac{f(x)}{g(x)} = 1 \iff \lim_{x \to x_0} \frac{f(x)-g(x)}{g(x)} = 0 \iff f(x)-g(x) = o(g(x))$
\end{proof}

\begin{theorem}[Граничний перехід]
Задано $f_1(x) \sim g_1(x)$ та $f_2(x) \sim g_2(x)$, $x \to x_0$. Тоді:\\
1) $\huge \lim_{x \to x_0} f_1(x) f_2(x) = \lim_{x \to x_0} g_1(x) g_2(x)$;\\
2) $\huge \lim_{x \to x_0} \frac{f_1(x)}{f_2(x)} = \lim_{x \to x_0} \frac{g_1(x)}{g_2(x)}$.\\
За умовою, що принаймні один з чотирьох лімітів існує, не обов'язково скінченний
\end{theorem}

\begin{proof}
Із початкових умов отримаємо, що:\\
$\huge \exists \lim_{x \to x_0} \frac{f_1(x)}{g_1(x)} = 1$, $\huge \exists \lim_{x \to x_0} \frac{f_2(x)}{g_2(x)} = 1$. Тоді маємо:\\
1) $\huge \lim_{x \to x_0} f_1(x) f_2(x) = \lim_{x \to x_0} \frac{f_1(x) f_2(x) g_1(x) g_2(x)}{g_1(x) g_2(x)} = \lim_{x \to x_0} \frac{f_1(x) f_2(x)}{g_1(x) g_2(x)} \lim_{x \to x_0} g_1(x) g_2(x) = \lim_{x \to x_0} g_1(x) g_2(x)$.\\
2) $\huge \lim_{x \to x_0} \frac{f_1(x)}{f_2(x)} = \lim_{x \to x_0} \frac{f_1(x)g_1(x)g_2(x)}{f_2(x)g_1(x)g_2(x)} = \lim_{x \to x_0} \frac{f_1(x)g_2(x)}{f_2(x)g_1(x)} \lim_{x \to x_0} \frac{g_1(x)}{g_2(x)} = \lim_{x \to x_0} \frac{g_1(x)}{g_2(x)}$.
\end{proof}

\begin{remark}
Еквівалентні функції задають відношення еквівалентності: \\
рефлексивність, симетричність, транзитивність.
\end{remark}

Використовуючи всі наслідки від чудових границь, ми можемо отримати наступні еквівалентні функції, коли $x \to 0$
\begin{center}
\begin{tabular}{ c c }
 $\sin x \sim x$ & $\ln(1+x) \sim x$ \\ 
 $\tg x \sim x$ & $e^x - 1 \sim x$ \\
 $\arcsin x \sim x$ & $(1+x)^\alpha - 1 \sim \alpha x$ \\
 $\arctg x \sim x$ & $a^x - 1 \sim x \ln a$ \\ 
\end{tabular}
\end{center}

\begin{example}
Обчислити границю $\huge \lim_{x \to 0} \frac{\arcsin x \cdot (e^x - 1)}{1 - \cos x}$.\\
Маємо, з таблиці еквівалентності:\\
$\huge \lim_{x \to 0} \frac{\arcsin x \cdot (e^x - 1)}{1 - \cos x} = \lim_{x \to 0} \frac{x \cdot x}{2 \sin^2 \frac{x}{2}} = \lim_{x \to 0} \frac{x \cdot x}{2 \frac{x^2}{4}} = 2$.
\end{example}

\begin{remark}
Узагальнене зауваження:\\
$f(x) = O(1), x \to x_0 \iff f(x)$ - обмежена в околі т. $x_0$.\\
$f(x) = o(1), x \to x_0 \iff f(x)$ - н.м. функція.
\end{remark}
\begin{figure} [H]
\centering
\begin{tikzpicture}
\draw[thick, ->] (-4,0)--(4.2,0) node[anchor = north] {$x$};
\draw[thick, ->] (0,-4)--(0,4.2) node[anchor = east] {$y$};


\draw[thick, domain=-4:4, variable=\x, samples = 1000] plot({\x}, {sin(deg(\x))}) node[anchor = north west, scale = 0.7] {$f(x) = \sin x$};
\draw[thick, domain=-4:4, variable=\x, samples = 1000] plot({\x}, {\x}) node[anchor = south west, scale = 0.7] {$g(x) = x$};
\end{tikzpicture}
\caption*{В околі т. $x_0 = 0$ функція $\sin x$ дуже схожа на $x$, тобто однакова поведінка}
\end{figure}
\newpage
%\fi %IMPORTANT COMMENT

%\iffalse %IMPORTANT COMMENT
\section{Неперервність функції}
Щоб було повноцінне визначення неперервності, зробимо відступ.
\begin{definition}
Задано множину $A \subset \mathbb{R}$ та т. $x \in A$.\\
Точка $x$ називається \textbf{ізольованою}, якщо
\begin{align*}
\exists \varepsilon > 0: U_{\varepsilon}(x) \cap A = \{x\}
\end{align*}
\end{definition}

\begin{example}
Маємо множину $A = [0,2] \cup \{4\}$. Тут т. $x = 4 \in A$ - ізольована.
\end{example}

Якщо придивитись уважно на означення, то тут записано заперечення того, що $x$ - гранична точка. Отже:
\begin{corollary}
Точка $x \in A$ - ізольована $\iff x$ - не гранична для $A$.
\end{corollary}

\subsection{Неперервність в точці}
\begin{definition}
Задано функцію $f: A \to \mathbb{R}$ та т. $x_0 \in A$.\\
Функція $f(x)$ називається \textbf{неперервною в т.} $x_0$, якщо
\begin{align*}
\forall \varepsilon > 0: \exists \delta(\varepsilon) > 0: \forall x \in A: |x-x_0| < \delta \Rightarrow |f(x)-f(x_0)| < \varepsilon \text{ def. Коші} \\
\forall \{x_n, n \geq 1\} \subset A: \huge\lim_{n \to \infty} x_n = x_0 \Rightarrow \huge\lim_{n \to \infty} f(x_n) = f(x_0) \text{ def. Гейне}
\end{align*}
\end{definition}

\begin{theorem}
Означення Коші $\iff$ Означення Гейне.\\
\textit{Доведення є аналогічним з означеннями Коші, Гейне в границях.}
\end{theorem}

\begin{proposition}
Задано функцію $f: A \to \mathbb{R}$ та т. $x_0 \in A$ - ізольована. Тоді $f$ - неперервна в т. $x_0$.
\end{proposition}

\begin{proof}
Якщо $x_0$ - ізольована, то $\exists \delta^* > 0: U_{\delta^*} \cap A = \{x_0\}$.\\
Нехай $\varepsilon > 0$. Тоді $\exists \delta = \delta^* > 0: \forall x \in A: |x-x_0|< \delta \Rightarrow |f(x)-f(x_0)|<\varepsilon$.\\
Якщо $x \in A$ та $|x-x_0|< \delta$, то звідси $x=x_0$. А для нього $|f(x)-f(x_0)| = 0 < \varepsilon$.
\end{proof}

\begin{proposition}[Стандартне означення неперервності функції в точці]
Задано функцію $f: A \to \mathbb{R}$ та т. $x_0 \in A$ - гранична точка.\\
$f$ - неперервна в т. $x_0 \iff$ $\huge\lim_{x \to x_0} f(x) = f(x_0)$.
\end{proposition}

\begin{proof}
\leftproof Дано: $f$ - неперервна в т. $x_0$. Оскільки $x_0 \in A$ - гранична, то $\exists \{x_n, n \geq 1\} \subset A: x_n \neq x_0: x_n \to x_0$.\\
Оскільки $f$ - неперервна, то $f(x_n) \to f(x_0)$. Отже, за Гейне, $\exists \huge\lim_{x \to x_0} f(x) = f(x_0)$.
\bigskip \\
\rightproof Дано: $\exists \huge\lim_{x \to x_0} f(x) = f(x_0)$. Тоді за Коші,
$\forall \varepsilon > 0: \exists \delta > 0: \forall x \in A: 0 < |x-x_0| < \delta \Rightarrow |f(x)-f(x_0)| < \varepsilon$.
За Коші означення неперервності, отримали, що $f$ - неперервна в т. $x_0$.
\end{proof}

\begin{example}
Пояснювальний приклад, навіщо ми створили нестандартне означення.\\
Маємо функцію $f(x) = \sqrt{x^2(x^2-1)}$, яка визначена на $(-\infty,-1] \cup [1,+\infty)$, а також в т. $x = 0$.\\
І ось точка $x_0 = 0$ - ізольована точка. Отже, можна вважати, що $f$ - неперервна в т. $x_0$.
\begin{figure}[H]
\centering
\begin{tikzpicture}
\draw[thick,->] (-2.1,0)--(2.1,0) node[anchor = north] {$x$};
\draw[thick,->] (0,-0.5)--(0,4) node[anchor = east] {$y$};
\draw[thick, domain=-2:-1, variable=\x, samples = 500] plot({\x}, {sqrt(\x*\x*(\x*\x-1))});
\fill(0,0) circle (2pt);
\draw[thick, domain=1:2, variable=\x, samples = 500] plot({\x}, {sqrt(\x*\x*(\x*\x-1))});
\end{tikzpicture}
\end{figure}
\end{example}

\begin{definition}
Задано функція $f: A \to \mathbb{R}$ та т. $x_0 \in A$.\\
Функція $f$ називається \textbf{розривною в т.} $x_0$, якщо в цій точці функція не є неперервною. А сама т. $x_0$ називається \textbf{точкою розриву}.
\end{definition}

\begin{remark}
Лише граничні точки можуть бути точками розриву, а в ізольованій завжди функція неперервна.
\end{remark}

\subsubsection*{Класифікації точок розриву}
\subsubsection*{І роду}
- \textbf{усувна}, якщо $\exists \huge \lim_{x \to x_0} f(x) \neq f(x_0)$\\
- \textbf{стрибок}, якщо $\exists \huge \lim_{x \to x_0^+} f(x)$, $\exists \huge \lim_{x \to x_0^-} f(x)$, але при цьому $\huge \lim_{x \to x_0^+} f(x) \neq \lim_{x \to x_0^-} f(x)$

\subsubsection*{ІI роду}
якщо виконується один з 4 випадків:\\
1) $\huge \lim_{x \to x_0^-} f(x) = \infty$ \hspace{3cm} 3) $\huge \not\exists \lim_{x \to x_0^-} f(x)$\\
2) $\huge \lim_{x \to x_0^+} f(x) = \infty$ \hspace{3cm} 4) $\huge \not\exists \lim_{x \to x_0^+} f(x)$

\begin{example}
Розглянемо функцію $f(x) = \begin{cases} \huge \frac{\sin x}{x}, x \neq 0 \\ 1, x = 0 \end{cases}$.\\
В т. $x_0$ функція $f(x)$ є неперервною, оскільки
$\huge \lim_{x \to 0} \frac{\sin x}{x} \overset{\textrm{I чудова границя}}{=} 1 = f(0)$.
\end{example}

\begin{example}
Розглянемо тепер функцію $f(x) = \begin{cases} \huge \frac{\sin x}{x}, x \neq 0 \\ 0, x = 0 \end{cases}$.\\
У цьому випадку в т. $x_0$ буде розривом I роду, усувною, оскільки\\
$\huge \lim_{x \to 0} \frac{\sin x}{x} \overset{\textrm{I чудова границя}}{=} 1 \neq f(0) = 0$.\\
\begin{figure} [H]
\centering
\resizebox{1\textwidth}{!}
{
\begin{tikzpicture}

\draw[thick, ->] (-7,0)--(7.5,0) node[anchor = north] {$x$};
\draw[thick, ->] (0,-1)--(0,2) node[anchor = east] {$y$};

\draw[thick, domain=-7:7, variable=\x, samples = 1000] plot({\x}, {sin(deg(\x))/\x}) node[anchor = south east, scale = 0.8] {$f(x) = \dfrac{\sin x}{x}$};
\node[white] at (0,1) [circle,fill,inner sep=1.5pt, draw = black]{};
\node[black] at (0,0) [circle,fill,inner sep=1.5pt, draw = black]{};
\end{tikzpicture}
}
\end{figure}
\end{example}

\begin{example} Розглянемо функцію
$f(x) = \begin{cases} 
2x - \dfrac{x-2}{|x-2|}, x \neq 2 \\
2, x = 2
\end{cases}$.\\
Тут проблема виникає в т. $x_0 = 2$. Розглянемо границі в різні сторони:\\
$\huge \lim_{x \to 2^-} \left(2x - \dfrac{x-2}{2-x}\right) = \lim_{x \to 2^-} (2x-1) = 3$\\
$\huge \lim_{x \to 2^+} \left(2x - \dfrac{x-2}{x-2}\right) = \lim_{x \to 2^+} (2x+1) = 5$\\
Обидва ліміти не рівні, а отже, $x_0 = 2$ - розрив I роду, стрибок.
\begin{figure}[H]
\centering
{
\begin{tikzpicture}

\draw[thick, ->] (-2,0)--(4,0) node[anchor = north] {$x$};
\draw[thick, ->] (0,-3)--(0,6) node[anchor = east] {$y$};

\draw[thick, domain=-2:1.99, variable=\x, samples = 1000] plot({\x}, {2*\x - (\x-2)/(abs(\x-2))});
\draw[thick, domain= (2.01:3.5, variable=\x, samples = 1000] plot({\x}, {2*\x - (\x-2)/(abs(\x-2))}) node[anchor = south, scale = 0.8] {$f(x) = 2x - \dfrac{x-2}{|x-2|}$};
\node[white] at (2,5) [circle,fill,inner sep=1.5pt, draw = black]{};
\node[white] at (2,3) [circle,fill,inner sep=1.5pt, draw = black]{};
\draw[thick, dashed] (2,5)--(2,0) node[anchor = north] {$2$};
\draw (1 pt, 5cm) -- (-1 pt, 5cm) node[anchor = east] {$5$};
\draw (1 pt, 3cm) -- (-1 pt, 3cm) node[anchor = east] {$3$};
\draw (1 pt, 2cm) -- (-1 pt, 2cm) node[anchor = east] {$2$};
\fill(2,2) circle (2pt);
\end{tikzpicture}
}
\end{figure}
\end{example}

\begin{example} Маємо функцію $f(x) = \dfrac{1}{x+1}$.
Проблема в т. $x_0 = -1$. Але принаймні по одну сторону, наприклад $\huge \lim_{x \to -1^+0} =\dfrac{1}{x+1} = +\infty$, матимемо нескінченність.\\
Тому одразу т. $x_0 = -1$ - розрив II роду.
\begin{figure}[H]
\centering
{
\begin{tikzpicture}

\draw[thick, ->] (-3,0)--(1,0) node[anchor = north] {$x$};
\draw[thick, ->] (0,-3)--(0,3) node[anchor = east] {$y$};

\draw[thick, domain=-3:-1.35, variable=\x, samples = 1000] plot({\x}, {1/(\x+1)});
\draw[thick, domain=-0.65:1, variable=\x, samples = 1000] plot({\x}, {1/(\x+1)});
\draw[dashed] (-1,-3)--(-1,3);
\draw (-1,1pt)--(-1,-1pt) node[anchor = north] {$-1$};
\end{tikzpicture}
}
\end{figure}
\end{example}

\begin{theorem}[Арифметичні властивості неперервних функцій]
Задано функції $f,g: A \to \mathbb{R}$ та $x_0 \in A$. Відомо, що $f,g$ - неперервні в т. $x_0$. Тоді:\\
1) $\forall c \in \mathbb{R}: (cf)(x)$ - неперервна в т. $x_0$;\\
2) $(f+g)(x)$ - неперервна в т. $x_0$;\\
3) $(fg)(x)$ - неперервна в т. $x_0$;\\
4) $\dfrac{f}{g}(x)$ - неперервна в т. $x_0$ при $g(x_0) \neq 0$.\\
\textit{1),2),3),4) - всі вони випливають із означення. Але в 4) більш детально розпишу одну штуку.}
\end{theorem}
Переконаємось, що все буде коректно визначено в 4)\\
$g$ - неперервна в $x_0$, тобто $\forall \varepsilon > 0: \exists \delta: \forall x \in A: |x-x_0|<\delta \Rightarrow |g(x)-g(x_0)|<\varepsilon$.\\
Оберемо $\varepsilon = \dfrac{|g(x_0)|}{2}$.\\
Тоді $g(x_0)-\varepsilon <g(x) <g(x_0)+\varepsilon$.\\
Якщо $g(x_0) > 0$, то $\varepsilon = \dfrac{g(x_0)}{2} \Rightarrow 0 < g(x) < \dfrac{3}{2}g(x_0)$.\\
Якщо $g(x_0) < 0$, то $\varepsilon = -\dfrac{g(x_0)}{2} \Rightarrow \dfrac{3}{2}g(x_0) < g(x) < \dfrac{1}{2}g(x_0) < 0$.\\
Тобто $\exists \delta: \forall x \in A: |x-x_0|<\delta \Rightarrow g(x) \neq 0$.\\
Отже, наше означення є коректним.

\begin{theorem}[Неперервність композиції]
Задано функції $f: A \to B, g: B \to \mathbb{R}$ та $h = g \circ f$. Відомо, що $f$ неперервна в т. $x_0 \in A$; та $g$ - неперервна в т. $f(x_0) = y_0 \in B$.\\
Тоді $h$ - неперервна в т. $x_0$.\\
\textit{Випливає з означення та властивості композиції.}
\end{theorem}

\begin{definition}
Функція $f: A \to \mathbb{R}$ називається \textbf{неперервною на множині} $A$, якщо вона є неперервною $\forall x \in A$.\\
Позначення: $C(A)$ - множина неперервних функцій в $A$.\\
\end{definition}

\subsection{Неперервність функції на відрізку}
Надалі ми розглядаємо лише функції $f \in C([a,b])$, тобто неперервні функції на відрізку. Саме для них будуть працювати такі теореми:

\begin{theorem}[Теорема Вейєрштрасса 1]
Задано функцію $f \in C([a,b])$. Тоді вона є обмеженою на $[a,b]$.
\end{theorem}

\begin{proof}
!Припустимо, що $f$ не є обмежено, тобто\\
$\forall n \geq 1: \exists x_n \in [a,b]: |f(x_n)| > n$.\\
Отримаємо послідовність $\{x_n,n \geq 1\}$. Є два випадки, тому виділимо 2 підпослідовності:\\
1) $\{x_{n_k}, k \geq 1\}: f(x_{n_k})>n_k$;\\
2) $\{x_{n_m}, m \geq 1\}: f(x_{n_m})<-n_m$.\\
Розглянемо другу. Вона є обмеженою, оскільки $\{x_{n_m}, m \geq 1\} \subset [a,b]$.\\
Тоді за Вейєрштраса, для підпослідовності $\{x_{n_{m_p}}, p \geq 1\}: \exists \huge \lim_{n \to \infty} x_{n_{m_p}} =x_*$.\\
Тому за означенням Гейне і за неперервністю, $\exists \huge \lim_{p \to \infty} f(x_{n_{m_p}}) = f(x_*)$.\\
Але водночас ми маємо, що функція не є обмеженою знизу, тобто $\exists \huge \lim_{p \to \infty} f(x_{n_{m_p}}) = -\infty$. Суперечність!\\
Для першого пункту все аналогічно і теж є суперечність.\\
Отже, $f$ - все ж таки обмежена на $[a,b]$.
\end{proof}

\begin{theorem}[Теорема Вейєрштрасса 2]
Задано функцію $f \in C([a,b])$. Тоді:\\
- $\huge \exists x_* \in [a,b]: f(x_*) = \inf_{x \in [a,b]} f(x)$\\
- $\huge \exists x^* \in [a,b]: f(x^*) = \sup_{x \in [a,b]} f(x)$
\end{theorem}

\begin{proof}
Доведемо перший випадок, другий є аналогічним.\\
Нехай $\huge \inf_{x \in [a,b]} f(x) = c$. За означенням:\\
1) $\forall x \in [a,b]: f(x) \geq c$;\\
2) $\forall \varepsilon > 0: \exists x_{\varepsilon} \in [a,b]: f(x_{\varepsilon}) < c + \varepsilon$.\\
Зафіксуємо $\varepsilon = \dfrac{1}{n}$. Тоді $\exists x_n \in [a,b]: c \leq f(x_n) < c + \dfrac{1}{n}$.\\
Ми також маємо обмежену послідовність $\{x_n, n \geq 1\} \subset [a,b]$.\\
Тому за Вейєрштрасом, для $\{x_{n_k},k \geq 1\}: \exists \huge \lim_{n \to \infty} x_{n_k} = x_*$.\\
Отже, за Гейне і за неперервністю, $\huge \exists \lim_{k \to \infty} f(x_{n_k}) = f(x_*)$.\\
Але в той самий час $\exists x_{n_k} \in [a,b]: c \leq f(x_{n_k}) < c + \dfrac{1}{n_k}$.\\
Коли $k \to \infty$, то за теоремою про поліцаїв, $\exists \huge \lim_{k \to \infty} f(x_{n_k}) = c$.\\
Таким чином, отримали, що $c = f(x_*) = \huge \inf_{x \in [a,b]} f(x) = \min_{x \in [a,b]} f(x)$.
\end{proof}

\begin{theorem}[Теорема Коші про нульове значення]
Задано функцію $f \in C([a,b])$, причому $f(a) \cdot f(b) < 0$. Тоді $\exists x_0 \in (a,b): f(x_0) = 0$.
\end{theorem}

\begin{proof}
Будемо доводити випадок, коли $f(a) < 0$, $f(b) > 0$.\\
Розглянемо множину $M= \{x \in [a,b], f(x) < 0\}$\\
Оскільки $f$ - неперервна, то $\huge \exists \lim_{x \to a} f(x) = f(a)$\\
$\Rightarrow$ для $\varepsilon = -\dfrac{f(a)}{2}: \exists \delta: \forall x: |x-a|<\delta \Rightarrow |f(x)-f(a)|< -\dfrac{f(a)}{2}$\\
$\Rightarrow \dfrac{3f(a)}{2} < f(x) < \dfrac{f(a)}{2} \Rightarrow \forall x: |x-a|<\delta \Rightarrow f(x) < 0$.\\
Отже, $M \neq \emptyset$.\\
З іншого боку, ми маємо $\huge \lim_{x \to b} f(x) = f(b)$\\
$\Rightarrow$ для $\tilde{\varepsilon} = \dfrac{f(b)}{2}: \exists \tilde{\delta}: \forall x: |x-b| < \tilde{\delta} \Rightarrow |f(x)-f(b)|<\dfrac{f(b)}{2}$\\
$\Rightarrow \dfrac{f(b)}{2} < f(x) < \dfrac{3f(b)}{2} \Rightarrow \forall x: |x-b| < \tilde{\delta} \Rightarrow f(x) > 0$.\\
Жодна з цих значень аргументів не потрапляє в нашу множину $M$.\\
А оскільки $M \subset [a,b]$, то вона є обмеженою.
\bigskip \\
Із двох міркувань випливає, що $\exists \sup M \overset{\textrm{позн.}}{=} x_0$. А тепер перевіримо, що дійсно $f(x_0) = 0$.\\
За критерієм $\sup$:\\
$\forall x \in M: x \leq x_0$\\
Для $\varepsilon = \dfrac{1}{n}: \exists x_n \in M: x_n > x_0 - \dfrac{1}{n}$.\\
Тобто $\forall n \geq 1: x_0 - \dfrac{1}{n} < x_n \leq x_0$.\\
Розглянемо послідовність $\{x_n, n \geq 1\} \subset M: \exists \huge \lim_{n \to \infty} x_n = x_0$.\\
Отже, за Гейне та неперервністю, $\huge \exists \lim_{n \to \infty} f(x_n) = f(x_0) \leq 0$.
\\
Оскільки ми маємо $\sup M = x_0$, то тоді $\forall n \geq 1: x_0 + \dfrac{1}{n} \not\in M$.\\
Тому розглянемо послідовність $\{\tilde{x_n} = x_0 + \dfrac{1}{n}, n \geq 1\}$.\\
Тут $\huge \lim_{n \to \infty} \tilde{x_n} = x_0 \Rightarrow \exists \lim_{n \to \infty} f(\tilde{x_n}) = f(x_0) > 0$.\\
Остаточно, $f(x_0) = 0$.
\end{proof}

\begin{corollary}[Теорема Коші про проміжкове значення]
Задано функцію $f \in C([a,b])$.
Тоді $\forall L \in \left[ \begin{gathered} (f(a),f(b)) \\ (f(b),f(a)) \end{gathered} \right.: \exists x_L \in (a,b): f(x_L) = L$.\\
\textit{Вказівка: розглянути функцію $g(x) = f(x) - L$.}
\end{corollary}

\subsection{Неперервність функції на інтервалі}
\begin{theorem}[Про існування оберненої функції]
Заданао функцію $f: (a,b) \to (c,d)$ - строго монотонна і неперервна.\\
Відомо, що $\huge \lim_{x \to a^+} f(x) = c$, $\huge \lim_{x \to b^-} f(x) = d$.\\
Тоді існує функція $g: (c,d) \to (a,b)$ - строго монотонна (як і $f$) і неперервна, яка є оберненою до $f$.
\end{theorem}

\begin{proof}
Розглянемо випадок монотонно зростаючої функції $f$. Тоді $c < d$. Для спадної аналогічно.\\
За теоремою про проміжкове значення, $\forall y \in (c,d): \exists x \in (a,b): y = f(x)$.\\
Покажемо, що $\forall y \in (c,d): \exists ! x \in (a,b): y = f(x)$.\\
!Припустимо, не єдиний $x$ існує, тобто $\exists x_1, x_2: f(x_1) = y, f(x_2) = y$, але при цьому $x_1 \neq x_2$.\\
Тоді якщо $x_1 < x_2$, то через монотонно зростаючу функцію $f(x_1) < f(x_2)$.\\
Тоді якщо $x_1 > x_2$, то через монотонно зростаючу функцію $f(x_1) > f(x_2)$.\\
Суперечність!\\
Таким чином, $\exists ! x \in (a,b): y = f(x)$ - бієкція.\\
Ба більше, $\forall x \in (a,b): f(x) \in (c,d)$.\\
Тоді створімо функцію $g: (c,d) \to (a,b)$, що є оберненою до $f$.\\
1. Покажемо, що $g(x)$ - монотонно зростає.\\
$\forall y_1,y_2: y_1 > y_2$\\
$x_1 = g(y_1), x_2 = g(y_2)$\\
$y_1 \neq y_2 \iff x_1 \neq x_2$\\
Якщо $x_1 < x_2$, то тоді $y_1 = f(x_1) < f(x_2) = y_2$, що не є можливим.\\
Отже, $x_1 > x_2 \implies g(y_1) > g(y_2)$.\\
Це й є ознака строгого зростання.
\bigskip \\
2. Покажемо, що $g \in C((c,d))$.\\
!Припустимо, що це не так, тобто $\exists y_0: g(y)$ - не є неперервною в т. $y_0$.\\
Зафіксуємо дві послідовності, що збігаються до т. $y_0$.\\
$\exists \{y_n^1, n \geq 1\}, \{y_n^2, n \geq 1\}: \huge \lim_{n \to \infty} y_n^1 = y_0, \huge \lim_{n \to \infty} y_n^2 = y_0$\\
Але водночас $\huge \lim_{n \to \infty} g(y_n^1) \neq g(y_0), \huge \lim_{n \to \infty} g(y_n^1) \neq g(y_0)$.\\
А це означає, що $\huge \lim_{n \to \infty} g(y_n^1) \neq \lim_{n \to \infty} g(y_n^2)$.\\
Позначимо $\{x_n^1 = g(y_n^1), n \geq 1 \}$, $\{x_n^2 = g(y_n^2), n \geq 1 \}$.\\
Тоді $\huge \lim_{n \to \infty} x_n^1 \neq \lim_{n \to \infty} x_n^2$.\\
Позначимо $\huge \lim_{n \to \infty} x_n^1 = u_1$, $\huge \lim_{n \to \infty} x_n^2 = u_2$.\\
Тоді з неперервності $f(x)$ отримаємо, що:\\
$f(u_1) = \huge \lim_{n \to \infty} f(x_n^1) = \lim_{n \to \infty} f(g(y_n^1)) = \lim_{n \to \infty} y_n^1 = y_0 = \lim_{n \to \infty} y_n^2 = \lim_{n \to \infty} f(g(y_n^2)) = \lim_{n \to \infty} f(x_n^2) = f(u_2)$.\\
Тобто $f(u_1) = f(u_2)$. Суперечність! Оскільки $f$ - строго монотонно зростаюча функція.\\
Отже, наше припущення - невірне. Тоді $g \in C((c,d))$.
\bigskip \\
Фінальний висновок: $g \in C((c,d))$ та строго монотонно зростаюча на $(c,d)$. \end{proof}

\subsection{Неперервність елементарних функцій}
0) Задано функцію $f(x) = x$.  $f \in C(\mathbb{R})$.
\begin{proof}
$\forall \varepsilon > 0: \exists \delta = \varepsilon: \forall x: |x-x_0|<\delta \Rightarrow |f(x)-f(x_0)| = |x-x_0|<\delta = \varepsilon$
\end{proof}

1) Задано функцію $f(x) = a_0 + a_1 x + \dots + x_n x^n$.  $f \in C(\mathbb{R})$.
\begin{proof}
Оскільки $g(x) = x \in C(\mathbb{R})$, то \\ $h(x)=x^n = x \cdot \dots \cdot x \in C(\mathbb{R})$ як добуток функцій $\forall n \geq 1$.\\
Отже, $f(x) = a_0 + a_1 x + \dots + x_n x^n \in C(\mathbb{R})$ як сума неперервних функцій, множених на константу.
\end{proof}

2) Задано функцію $f(x) = \sin x$. $f \in C(\mathbb{R})$.
\begin{proof}
Вже відомо давно нерівність:\\
$1 - \dfrac{x^2}{2} < \dfrac{\sin x}{x} < 1 \Rightarrow x - \dfrac{x^3}{2} < \sin x < x$.\\
Якщо $x \to 0$, то за теоремою про 2 поліцая, $\huge \lim_{x \to 0} \sin x = 0 = \sin 0$.\\
Отже, $\sin x$ - неперервна лише в т. $0$.\\
Перевіримо неперервність в т. $a \in \mathbb{R}$:\\
$\huge \lim_{x \to a} (\sin x - \sin a) = \lim_{x \to a} 2 \sin \dfrac{x-a}{2} \cos \dfrac{x+a}{2} \boxed{=} $\\
Проведемо заміну: $\dfrac{x-a}{2} = t$. Тоді $t \to 0$\\
$\boxed{=} \huge \lim_{t \to 0} 2 \sin t \cos (t+a) \overset{\text{н.м. * обм.}}{=} 0$
$\Rightarrow \huge \lim_{x \to a} \sin x = \sin a$.\\
Остаточно, $f(x) = \sin x \in C(\mathbb{R})$.
\end{proof}

3) Задано функцію $f(x) = \cos x$. $f \in C(\mathbb{R})$.
\begin{proof}
$f \in C(\mathbb{R})$ як композиція, бо $\cos x = \sin\left(\dfrac{\pi}{2} -x \right)$.
\end{proof}

4.1) Задано функцію $f(x) = \tg x$. $f \in C\left(\mathbb{R} \setminus \left\{\dfrac{\pi}{2} + \pi k, k \in \mathbb{Z}\right\} \right)$.\\
4.2) Задано функцію $f(x) = \ctg x$. $f \in C\left(\mathbb{R} \setminus \left\{\pi k, k \in \mathbb{Z}\right\} \right)$.
\begin{proof}
1.$f \in C$ як частка, бо $\tg x = \dfrac{\sin x}{\cos x}$.\\
2.$f \in C$ за аналогічними міркуваннями.
\end{proof}
Час перервати на ішний підрозділ...

\subsection{Зведення в дійсну степінь}
Починалось зі зведення числа в натуральну степінь
$$x^n = x \cdot x \cdot \dots \cdot x, \hspace{0.5cm} n \in \mathbb{N}$$
Потім виникла ціла степінь
$$x^{-n} = \dfrac{1}{x^n}, \hspace{0.5cm} -n \in \mathbb{Z}$$
А далі в школі мали розглядати раціональну степінь
$$x^{\frac{m}{n}} = \sqrt[n]{x^m}, \hspace{0.5cm} q = \dfrac{m}{n} \in \mathbb{Q}$$
У всіх зберігається один клас властивостей:\\
1) $q_1 > q_2 \implies \left[ \begin{gathered} x^{q_1} > x^{q_2}, x>1 \\ x^{q_1} < x^{q_2}, 0<x<1 \end{gathered} \right.$;\\
2) $x^{q_1} x^{q_2} = x^{q_1+q_2}$;\\
3) $(x^{q_1})^{q_2} = x^{q_1q_2}$;\\
4) $(xy)^{q} = x^q y^q$.\\
Зазначимо, що саме в цьому моменті ми вимагаємо, щоб основа $x > 0$, оскільки виникає суперечність з раціональними степенями. Наприклад:\\
$\sqrt[3]{-2} = (-2)^{\frac{1}{3}} = (-2)^{\frac{2}{6}} = \sqrt[6]{(-2)^2} = \sqrt[6]{4} = \sqrt[3]{2}$.\\
Тепер ми хочемо навчитися зводити в дійсну степінь певне число, але наведу спочатку корисні твердження.
\begin{lemma}
$|a^q-1| \leq 2|q|(a-1)$, якщо $a>1$ та $q \in \mathbb{Q}: |q| \leq 1$.
\end{lemma}

\begin{proof}
1) Розглянемо дріб $q = \dfrac{1}{n}, n \in \mathbb{N}$. Позначимо $\alpha = a^{\frac{1}{n}}-1 > 0$. Тоді за нерівністю Бернуллі,\\
$(1+\alpha)^n \geq 1 + \alpha n \implies a \geq 1 + \alpha n$.\\
$\alpha = a^{\frac{1}{n}} - 1 \leq \dfrac{1}{n}(a-1) < 2 \dfrac{1}{n} (a-1)$ - та сама нерівність, що в лемі.
\bigskip \\
2) Нехай тепер маємо будь-яке раціональне число $0 < r < 1$. Тоді знайдеться таке $n \in \mathbb{N}$, що $\dfrac{1}{n+1} < r < \dfrac{1}{n}$. А тому $|a^r - 1| < |a^{\frac{1}{n}} - 1|$ - вже було.
\bigskip \\
3) Нарешті, $-1 < q < 0$, але ми робимо заміну $r = -q$ - приходимо до 2)
\end{proof}

\begin{theorem}
	Для будь-якого дійсного числа знайдеться збіжна до неї послідовність раціональних чисел.
\end{theorem}
	
	\begin{proof}
	Спочатку випадок, коли $q \in \mathbb{Q}$. Тоді будуємо стаціонарну послідовність $\{q_n = q, n \geq 1\}$ - готово.
	\bigskip \\
	Тепер розглянемо $x \in \mathbb{R} \setminus \mathbb{Q}$. А тепер розглянемо числа $x - \dfrac{1}{n}, x + \dfrac{1}{n} \in \mathbb{R}, \forall n \geq 1$. Тоді за щільністю раціональних чисел,\\
	$\exists q_n \in \mathbb{Q}: x - \dfrac{1}{n} < q_n < x + \dfrac{1}{n}$. Тепер спрямуємось $n \to \infty$. \\
	Тоді за теоремою про двох поліцаїв, для послідовності $\{q_n, n \geq 1\} \subset \mathbb{Q}: \hspace{0.2cm} \exists \huge\lim_{n \to \infty} q_n = x$.
	\end{proof}
	
	\begin{theorem}
	Для будь-якого раціонального числа знайдеться збіжна до неї послідовність ірраціональних чисел.\\
	\textit{Доводиться аналогічно, із використанням щільності ірраціональних чисел.}
	\end{theorem}

\begin{definition}
Задано $x \in \mathbb{R}, a > 0$ та послідовність $\{q_n, n \geq 1\} \subset \mathbb{Q}: q_n \overset{n \to \infty}{\longrightarrow} x$.\\
\textbf{Зведення числа до дійсної степіні} визначається ось так:
\begin{align*}
a^x = \huge\lim_{n \to \infty} a^{q_n}
\end{align*}
\end{definition}
Виникає віднині дуже багато питань, які треба розв'язати:
\bigskip \\
1. Існування границі\\
Розглянемо $a > 1$ та послідовність раціональних чисел $q_n \to x$. Тоді за Коші,\\
$\forall \varepsilon > 0: \exists N_1: \forall m,n \geq N: |q_m-q_n| <\varepsilon$.\\
Зокрема якщо $\varepsilon = 1$, то тоді $\exists N_2: \forall m,n \geq N_2: |q_m-q_n| < 1$.\\
Тоді, зафіксувавши $N = \max\{N_1,N_2\}$, отримаємо ось таку оцінку $\forall n \geq N:$\\
$|a^{q_m}-a^{q_n}| = a^{q_m}|a^{q_m-q_n}-1| \leq a^{q_m} 2 |q_m-q_n|(a-1) \boxed{<} 2 a^C \varepsilon (a-1)$.\\
Додаткове пояснення: $q_n$ - збіжна послідовність, а тому - обмежена, тобто $|q_n| < C$. Беремо лише $C \in \mathbb{Q}$. А оскільки $a > 1$, то, $a^{q_n} < a^C$\\
Лишилось $0 < a < 1$. Але якщо розглянути $b = \dfrac{1}{a}$, то $b>1$ - вже доведено. А випадок $a = 1$ зрозумілий.\\
Отже, дійсно, за критерієм Коші, $\exists \huge\lim_{n \to \infty} a^{q_n}$.
\bigskip \\
2. Незалежніть від послідовності раціональних чисел\\
Задано $q_n \to x$ та $q_n' \to x$. Доведемо, що $\huge\lim_{n \to \infty} a^{q_n} = \huge\lim_{n \to \infty} a^{q_n'}$.\\
Зауважимо, що $q_n - q_n' \to 0$, тоді для $\varepsilon = 1: \exists N: \forall n \geq N: |q_n-q_n'| < 1$.\\
Розглянемо $a > 1$. Тоді $|a^{q_n} - a^{q_n'}| = a^{q_n'}|a^{q_n-q_n'}-1| \leq a^{q_n'} 2|q_n-q_n'|(a-1) \overset{n \to \infty}{\longrightarrow} 0$.\\
Розглянемо $0 < a < 1$. Тоді $b = \dfrac{1}{a}, b > 1$ - доведено. А для $a = 1$ - зрозуміло.\\
Отже, дійсно,  $\huge\lim_{n \to \infty} a^{q_n} = \huge\lim_{n \to \infty} a^{q_n'}$, що підтверджує незалежність.
\bigskip \\
3. Що буде, якщо степінь - раціональна\\
Ми тоді беремо стаціонарну послідовність $\{q_n = q, q \geq 1\}$ - все.
\bigskip \\
Час тепер показати, що властивості степеней зберігаються. Я буду розглядати лише випадки $a > 1$. Якщо $0 < a < 1$, то тоді робимо заміну $b = \dfrac{1}{a}$, що зводить до випадку $b > 1$.\\
1) $x > y \implies \left[ \begin{gathered} a^{x} > a^{y}, a>1 \\ a^{x} < a^{y}, 0<a<1 \end{gathered} \right.$
\begin{proof}
Зафіксуємо дві раціональні числи $q_1,q_2 \in \mathbb{Q}$, щоб була ситуація $y < q_1 < q_2 < x$.\\
Розглянемо послідовності $\{x_n\} \subset \mathbb{Q}$, щоб всі члени були правіші за $q_2$, та $\{y_n\} \subset \mathbb{Q}$, щоб всі члени були лівіші за $q_1$, таким чином, щоб $x_n \to x, y_n \to y$. Тоді\\
$a^{y_n} < a^{q_1} < a^{q_2} < a^{x_n}$. Тоді за граничним переходом, $a^y \leq a^{q_1} < a^{q_2} \leq a^x \implies a^x > a^y$.
\end{proof}

2) $a^x a^y = a^{x+y}$
\begin{proof}
Розглянемо $\{x_n\} \subset \mathbb{Q}, \{y_n\} \subset \mathbb{Q}$ так, щоб $x_n \to x, y_n \to y$. Причому зауважу, що $x_n + y_n \to x +y$. Тоді $a^x a^y = \huge\lim_{n \to \infty} a^{x_n} \huge\lim_{n \to \infty} a^{y_n} = \huge\lim_{n \to \infty} a^{x_n} a^{y_n} = \huge\lim_{n \to \infty} a^{x_n+y_n} = a^{x+y}$.
\end{proof}

3) $(a^x)^y = a^{xy}$
\begin{proof}
Корисно знати: $a_1 \leq a_2 \implies \left[ \begin{gathered} a_1^x \leq a_2^x, x \geq 0 \\ a_1^x > a_2^x, x < 0 \end{gathered} \right.$. Це можна довести через раціональні степені\\
Розглянемо $\{x_n\} \subset \mathbb{Q}, \{y_n\} \subset \mathbb{Q}$ так, щоб $x_n \to x, y_n \to y$. Зафіксуємо чотири послідовності раціональних чисел $\{q_{1n}\},\{q_{2n}\},\{q_{3n}\},\{q_{4n}\}$, щоб $q_{1n},q_{2n} \to x$, $q_{3n},q_{3n} \to y$, а також \\
$q_{1n} < x < q_{2n}$, $q_{3n} < y < q_{4n}$. Тоді\\
$a^{q_{1n}} < a^x < a^{q_{2n}}$.\\
А використовуючи початкові знання доведення, маємо, що при $y > 0$ (для інших аналогічно) маємо\\
$a^{q_{1n} q_{3n}} = (a^{q_{1n}})^{q_{3n}} < (a^{q_{1n}})^y \leq (a^x)^y \leq (a^{q_{2n}})^y < (a^{q_{2n}})^{q_{4n}} = a^{q_{2n} q_{4n}}$.\\
А тепер за теоремою про двох поліцаїв, маємо, що $(a^x)^y = a^{xy}$.
\end{proof}
До речі, далі вже визначають \textbf{логарифм} $\huge\log_a b$ як таке число $x$, щоб $a^x = b$
\vspace{1cm} \\
\textbf{Повернімось до п. 1.4.} \\
5) Задано функцію $f(x) = e^x$. $f \in C(\mathbb{R})$.
\begin{proof}
$\huge \lim_{x \to 0} (e^x - 1) = \huge \lim_{x \to 0} \frac{e^x-1}{x}\cdot x = 0 \implies \huge \lim_{x \to 0} e^x = 1 = e^0$, тобто неперервна в т. $0$.\\
Тоді
$\huge \lim_{x \to a} (e^x - e^a) = \lim_{x \to a} e^a(e^{x-a}-1) \overset{x-a=t}{=} \lim_{t \to 0} e^a(e^t-1) = 0$.\\
$\Rightarrow \huge \lim_{x \to a} e^x = e^a$\\
Отже, $f(x) = e^x \in C(\mathbb{R})$.
\end{proof}

6) Задано функцію $f(x) = a^x$. $f \in C(\mathbb{R})$.
\begin{proof}
$f(x) = a^x = e^{\ln a^x} = e^{x \ln a} \in C(\mathbb{R})$ як композиція.
\end{proof}

7) Задано функцію $f(x) = \arcsin x$. $f \in C([-1,1])$.
\begin{proof}
Маємо функцію $g(x) = \sin x$, що визначена на $\left[ -\dfrac{\pi}{2}, \dfrac{\pi}{2} \right]$. На цьому проміжку вона монотонно строго зростає, неперервна.\\
Отже, за \textbf{Th. 1.3.1.}, $g^{-1}(x) = f(x) = \arcsin x \in C([-1,1])$. Теж, до речі, зростає.
\end{proof}
8) Задано функцію $f(x) = \arccos x$. $f \in C([-1,1])$.\\
9) Задано функцію $f(x) = \arctg x$. $f \in C(\mathbb{R})$.\\
10) Задано функцію $f(x) = \arcctg x$. $f \in C(\mathbb{R})$.\\
11) Задано функцію $f(x) = \log_a x$. $f \in C((0,+\infty))$.\\
\textit{Всі вони довдяться аналогічно як 7)}
\bigskip \\
12) Задано функцію $f(x) = \sh x$. $f \in C(\mathbb{R})$.\\
13) Задано функцію $f(x) = \ch x$. $f \in C(\mathbb{R})$.\\
14) Задано функцію $f(x) = \th x$. $f \in C(\mathbb{R})$.\\
15) Задано функцію $f(x) = \cth x$. $f \in C(\mathbb{R} \setminus \{0\})$.\\
\textit{Перші дві функції розписуються через експоненту. Останні два через тотожності попередніх функцій.}

\subsection{Рівномірна неперервність}
\begin{definition}
Функція $f$ називається \textbf{рівномірно неперервною на множині} $A$, якщо
\begin{align*}
\forall \varepsilon > 0: \exists \delta(\varepsilon) > 0: \forall x_1,x_2 \in A: |x_1-x_2|<\delta \Rightarrow |f(x_1) - f(x_2)| < \varepsilon
\end{align*}
Позначення: $C_{unif}(A)$ - множина рівномірно неперервних функцій на $A$.
\end{definition}

\begin{proposition}
Задано функцію $f \in C_{unif}(A)$. Тоді $f \in C(A)$.\\
\textit{Випливає з означення рівномірної неперервності.}
\end{proposition}

\begin{example}
Доведемо, що функція $f(x) = \sqrt{x} \in C_{unif}([0,+\infty))$.\\
Розглянемо нерівність для т. $x_1,x_2 \in [0,+\infty)$ так, щоб $|x_1-x_2|<\delta$.\\
$|f(x_1) - f(x_2)| = |\sqrt{x_1} - \sqrt{x_2}| = \sqrt{|\sqrt{x_1} - \sqrt{x_2}|^2} \leq \sqrt{|\sqrt{x_1}-\sqrt{x_2}| |\sqrt{x_1}+\sqrt{x_2}|} = \sqrt{|x_1-x_2|} < \sqrt{\delta} = \varepsilon$.\\
Якщо зафіксуємо $\delta = \varepsilon^2$, то отримаємо, що $f \in C_{unif}$.
\begin{figure}[H]
\centering
\begin{tikzpicture}
\pgfmathsetmacro{\eps}{0.4};
\pgfmathsetmacro{\delt}{(\eps)^2};
\draw[thick, ->] (-0.5,0)--(4.5,0) node[anchor = north] {$x$};
\draw[thick, ->] (0,-0.5)--(0,2.5) node[anchor = east] {$y$};

%RECTANGLE
\fill[blue!30] (1-\delt,{1-\eps})--(1-\delt,{1+\eps})--(1+\delt,{1+\eps})--(1+\delt,{1-\eps}) -- cycle;

%RECTANGLE
\fill[blue!30] (3-\delt,{sqrt(3)-\eps})--(3-\delt,{sqrt(3)+\eps})--(3+\delt,{sqrt(3)+\eps})--(3+\delt,{sqrt(3)-\eps}) -- cycle;

\draw[thick, domain=0:4, variable=\x, samples = 500] plot({\x}, {sqrt(\x)}) node[anchor = south] {$f(x) = \sqrt{x}$};
\draw[thick, red, domain=1-\delt:1+\delt, variable=\x, samples = 40] plot({\x}, {sqrt(\x)});
\draw[thick, red, domain=3-\delt:3+\delt, variable=\x, samples = 40] plot({\x}, {sqrt(\x)});

\draw[dashed] (1-\delt,{sqrt(1)+\eps})--(1-\delt,0);
\draw[dashed] (1+\delt,{sqrt(1)+\eps})--(1+\delt,0);
\draw[dashed] (1+\delt,{sqrt(1)-\eps})--(0,{sqrt(1)-\eps});
\draw[dashed] (1+\delt,{sqrt(1)+\eps})--(0,{sqrt(1)+\eps});

\draw[dashed] (3-\delt,{sqrt(3)+\eps})--(3-\delt,0);
\draw[dashed] (3+\delt,{sqrt(3)+\eps})--(3+\delt,0);
\draw[dashed] (3+\delt,{sqrt(3)-\eps})--(0,{sqrt(3)-\eps});
\draw[dashed] (3+\delt,{sqrt(3)+\eps})--(0,{sqrt(3)+\eps});
\end{tikzpicture}
\caption*{Рівномірна неперервність означає таке: якщо візьмемо $\varepsilon>0$, ми знайдемо $\delta = \varepsilon^2$ в нашому випадку. Утворимо блакитний прямокутик - цей прямокутник не змінить довжини, тому що всюди червона функція потрапляє в цей прямокутник.}
\end{figure}
\end{example}

\begin{example}
Розглянемо функцію $f(x) = \ln x$, де $x \in (0,1)$. Вона є неперервною. Доведемо проте, що не рівномірно неперервна.\\
Заперечення рівномірної неперервності має такий вигляд:\\
$\exists \varepsilon^* > 0: \forall \delta > 0: \exists x_{1 \delta},x_{2 \delta} \in A: |x_{1 \delta}-x_{2 \delta}|<\delta$, але $|f(x_{1 \delta}) - f(x_{2 \delta})| \geq \varepsilon^*$.\\
Маємо ось що:\\
$|\ln x_{1 \delta} - \ln x_{2 \delta}| = \abs{\ln \dfrac{x_{1 \delta}}{x_{2 \delta}}} \geq 1 = \varepsilon^*$, якщо $\dfrac{x_{1 \delta}}{x_{2 \delta}} \geq e$.\\
Ми вже зафіксували $\varepsilon^* = 1$, а тепер лишилось надати $x_{1 \delta}, x_{2 \delta}$.\\
Маємо $x_{1 \delta} \geq e x_{2 \delta}$, а також $|x_{1 \delta} - x_{2 \delta}| < \delta$.\\
Оскільки $\delta$ в нас задовільне, то $\exists n: \dfrac{1}{n} < \delta$. Тоді надамо $x_{1 \delta} = \dfrac{e}{3n}, x_{2 \delta} = \dfrac{1}{3n}$.
$x_{1 \delta} \geq e x_{2 \delta}$ буде виконана.\\
$|x_{1 \delta} - x_{2 \delta}| = \dfrac{e}{3n} - \dfrac{1}{3n} = \dfrac{e-1}{3n} < \dfrac{1}{n} < \delta$.\\
Що ми отримали:\\
$\exists \varepsilon^* = 1: \forall \delta: \exists n: \exists x_{1 \delta} = \dfrac{e}{3n}, x_{2 \delta} = \dfrac{1}{3n}: |x_{1 \delta} - x_{2 \delta}| < \dfrac{1}{n} < \delta$, але $|f(x_{1\delta}) - f(x_{2 \delta})| \geq 1$.\\
Що й доводить те, що функція НЕ є рівномірно неперервною.
\begin{figure}[H]
\centering
\begin{tikzpicture}
\fill[blue!30] ({1/6 - 1/2},{ln(1/6)-1})--({1/6 + 1/2},{ln(1/6)-1})--({1/6 + 1/2},{ln(1/6)+1})--({1/6 - 1/2},{ln(1/6)+1})--cycle;

\draw[thick,->] (-0.5,0)--(2.5,0) node[anchor = north] {$x$};
\draw[thick, ->] (0,-3)--(0,1) node[anchor = east] {$y$};

\draw[thick, domain=0.05:2, variable=\x, samples = 500] plot({\x}, {ln(\x)}) node[anchor = south] {$f(x) = \ln x$};
\draw[dashed] ({1/6 - 1/2},{ln(1/6)-1})--({1/6+0.5+0.2},{ln(1/6)-1});
\draw[dashed] ({1/6 - 1/2},{ln(1/6)+1})--({1/6+0.5+0.2},{ln(1/6)+1});
\draw[dashed] ({1/6 - 1/2}, {ln(1/6)-1})--({1/6-1/2},0);
\draw[dashed] ({1/6 + 1/2}, {ln(1/6)-1})--({1/6+1/2},0);

\draw[thick, red, domain=0.05:{1/6+1/2}, variable=\x, samples = 40] plot({\x}, {ln(\x)});
\end{tikzpicture}
\caption*{В заданий прямокутник не потрапляють всі значення функції в точці з околу $x$. Тобто нам необхідно її розмір зменшати для неперервності. Тобто зміна розміру в залежності від точки $x$ - вже не буде рівномірно неперервною.}
\end{figure}

Проте в звортньому напрямку твердження буде працювати, якщо зробити додаткове обмеження. Це буде записано в наступній теоремі:
\end{example}

\begin{theorem}[Теорема Кантора]
Задано функцію $f \in C([a,b])$. Тоді $f \in C_{unif}([a,b])$.
\end{theorem}

\begin{proof}
!Припустимо, що вона не є рівномірно неперервною, тобто\\
$\exists \varepsilon^* > 0: \forall \delta: \exists x_{1 \delta}, x_{2 \delta} \in [a,b]: |x_{1 \delta} - x_{2 \delta}| < \delta \Rightarrow |f(x_{1 \delta}) - f(x_{2 \delta})| \geq \varepsilon^*$.\\
Розглянемо $\delta = \dfrac{1}{n}$. Тоді $x_{1 \delta}, x_{2 \delta} = x_{1n}, x_{2n}$.\\
Створимо послідовність $\{x_{1n}, n \geq 1\}$ - обмежена, бо всі в відрізку $[a,b]$, тому \\ 
для $\{x_{{1n}_k}, k \geq 1\}: \exists \huge \lim_{k \to \infty} x_{{1n}_k} = x_0$.\\
Оскільки $|x_{1n} - x_{2n}| < \dfrac{1}{n}$, то маємо, що $|x_{1n_k} - x_{2n_k}| < \dfrac{1}{n_k}$.\\
Тоді $x_{1n_k} - \dfrac{1}{n_k} < x_{2n_k} < x_{1n_k} + \dfrac{1}{n_k}$\\
Якщо $k \to \infty$, то за теоремою про поліцаї, $\exists \huge \lim_{k \to \infty} x_{2n_k} = x_0$.\\
За умовою неперервності, отримаємо, що $\huge\lim_{k \to \infty} f(x_{1n_k}) = \lim_{k \to \infty} f(x_{2n_k}) = f(x_0)$.\\
Але $\varepsilon \leq |f(x_{1n_k}) - f(x_{2n_k})| \to 0$, коли $k \to \infty$. Суперечність!
\end{proof}
\newpage
%\fi %IMPORTANT COMMENT


%\iffalse %IMPORTANT COMMENT
\section{Диференціювання}
\subsection{Основні означення}
\begin{definition}
Задано функцію $f: A \to \mathbb{R}$ та $x_0 \in A$ - гранична точка для $A$.\\
Функцію $f$ називають \textbf{диференційованою} в т. $x_0$, якщо
\begin{align*}
\exists L \in \mathbb{R}: f(x) - f(x_0) = L(x-x_0)+o(x-x_0),x \to x_0
\end{align*}
\end{definition}

\begin{proposition}
Задано функцію $f$ - диференційована в т. $x_0$. Тоді вона в т. $x_0$ неперервна.
\end{proposition}

\begin{proof}
$\huge \lim_{x \to x_0} (f(x)-f(x_0)) = \lim_{x \to x_0}(L(x-x_0)+o(x-x_0)) = 0 \implies \lim_{x \to x_0} f(x) = f(x_0)$
\end{proof}

\begin{proposition}
Функція $f$ - диференційована в т. $x_0 \iff \exists \huge \lim_{x \to x_0} \frac{f(x)-f(x_0}{x-x_0} = L = f'(x_0)$.
\end{proposition}

\begin{definition}
Тут число $f'(x_0)$ називають \textbf{похідною} функції в т. $x_0$, якщо ліміт існує.
\end{definition}

\begin{proof}
$f$ - диференційована в т. $x_0 \overset{\textrm{def.}}{\iff} \exists L: f(x)-f(x_0)=L(x-x_0)+o(x-x_0), x \to x_0 \iff \\ \iff \exists L: o(x-x_0) = f(x)-f(x_0)-L(x-x_0), x \to x_0 \iff \\ \iff \huge \lim_{x \to x_0} \frac{f(x)-f(x_0)-L(x-x_0)}{x-x_0} = 0 \iff \lim_{x \to x_0} \frac{f(x)-f(x_0)}{x-x_0} = L = f'(x_0)$.
\end{proof}

\begin{remark}
Задамо $\Delta x = x - x_0$, яку називають \textbf{прирістом аргумента}. Тоді похідну функції в т. $x_0$ можна записати іншою формулою: $f'(x_0) = \huge\lim_{\Delta x \to 0} \dfrac{f(x_0+\Delta x) - f(x_0)}{\Delta x}$. \\
А диференційованість ось так: $f(x_0+\Delta x) - f(x_0) = L \Delta x + o(\Delta x), \Delta x \to 0$.
\end{remark}

\begin{proposition}[Арифметичні властивості]
Задано функції $f,g$ - диференційовані в т. $x_0$, \hspace{0.3cm} $f'(x_0),g'(x_0)$ - їхні похідні. Тоді:\\
1) $\forall c \in \mathbb{R}: cf$ - диференційована в т. $x_0$, а її похідна
$(cf)'(x_0) = cf'(x_0)$;\\
2) $f \pm g$ - диференційована в т. $x_0$, а її похідна
$(f+g)'(x_0)=f'(x_0)+g'(x_0);$\\
3) $f \cdot g$ - диференційована в т. $x_0$, а її похідна
$(f \cdot g)(x_0) = f'(x_0)g(x_0)+f(x_0)g'(x_0)$;\\
4) $\huge \frac{f}{g}$ - диференційована в т. $x_0$ при $g(x_0) \neq 0$, а її похідна
$\huge \left(\frac{f}{g}\right)'(x_0) = \frac{f'(x_0)g(x_0)-f(x_0)g'(x_0)}{(g(x_0))^2}$.
\end{proposition}

\begin{proof}
Доведення буде проводитись за допомогою минуло доведеного твердження:\\
1) $(cf)'(x_0) = \huge \lim_{x \to x_0} \frac{cf(x)-cf(x_0)}{x-x_0} = c f'(x_0) \\ \Rightarrow cf$ - диференційована в т. $x_0$.
\bigskip \\
2) $(f+g)'(x_0) = \huge \lim_{x \to x_0} \frac{f(x)+g(x) - f(x_0)-g(x_0)}{x-x_0} = \lim_{x \to x_0} \frac{f(x) - f(x_0)}{x-x_0} + \lim_{x \to x_0} \frac{g(x) - g(x_0)}{x-x_0} = \\ = f'(x_0) + g'(x_0) \\ \Rightarrow f+g$ - диференційована в т. $x_0$.
\bigskip \\
3) $(f \cdot g)'(x_0) = \huge \lim_{x \to x_0} \frac{f(x)g(x) - f(x_0)g(x_0)}{x-x_0} = \lim_{x \to x_0} \frac{f(x)g(x) - f(x_0)g(x) + f(x_0)g(x) - f(x_0)g(x_0)}{x-x_0} = \\ = \lim_{x \to x_0} g(x) \frac{f(x)-f(x_0)}{x-x_0} + f(x_0) \lim_{x \to x_0} \frac{g(x)-g(x_0)}{x-x_0} = f'(x_0)g(x_0) + f(x_0)g'(x_0) \\ \Rightarrow fg$ - диференційована в т. $x_0$.
\bigskip \\
4) $\huge \left(\frac{f}{g}\right)'(x_0) = \lim_{x \to x_0} \frac{\textstyle \frac{f(x)}{g(x)} - \frac{f(x_0)}{g(x_0)}}{x-x_0} = \lim_{x \to x_0} \frac{f(x)g(x_0)-f(x_0)g(x)}{g(x)g(x_0)(x-x_0)} \overset{\textrm{як в 3)}}{=} \frac{1}{(g(x_0))^2} (f'(x_0)g(x_0) - f(x_0)g'(x_0)) \\ \Rightarrow \frac{f}{g}$ - диференційована в т. $x_0$.
\end{proof}

\begin{proposition}[Похідна від композиції функцій]
Задано функції $f,g$ та $h=g \circ f$. Відомо, що $f$ - диференційована в т. $x_0$, а $g$ - диференційована в т. $y_0 = f(x_0)$.\\
Тоді функція $h$ - диференційована в т. $x_0$, а її похідна $h'(x) = g'(f(x_0)) \cdot f'(x_0)$.
\end{proposition}

\begin{proof}
$h'(x) = \huge \lim_{x \to x_0} \frac{h(x)-h(x_0)}{x-x_0} = \lim_{x \to x_0} \frac{g(f(x))-g(f(x_0))}{x-x_0} = \lim_{x \to x_0} \frac{g(f(x))-g(f(x_0))}{f(x)-f(x_0)} \frac{f(x)-f(x_0)}{x-x_0} \boxed{=}$\\
Розіб'ємо дві дроби на окремі границі. В першому дробі заміна: $y=f(x)$\\
Якщо $x \to x_0$, то в силу диференційованості, а внаслідок - неперервності, $f(x) \to f(x_0)$ або $y \to y_0$.\\
$\boxed{=} \huge \lim_{y \to y_0} \frac{g(y)-g(y_0)}{y-y_0} \lim_{x \to x_0} \frac{f(x)-f(x_0)}{x-x_0}= g'(y_0) f'(x_0)=g'(f(x_0))f'(x_0) \\ \Rightarrow h$ - диференційована в т. $x_0$.
\end{proof}

\begin{proposition}[Похідна від оберненої функції]
Задано функції $f,g$ - взаємно обернені. Відомо, що $f$ - диференційована в т. $x_0$. \\ Тоді $g$ - диференційована в т. $y_0 = f(x_0)$, а її похідна 
$g'(y_0) = \huge \frac{1}{f'(x_0)}$.
\end{proposition}

\begin{proof}
$g'(y_0) = \huge \lim_{y \to y_0} \frac{g(y)-g(y_0)}{y-y_0} \boxed{=}$\\
Заміна: $y = f(x)$. Через взаємну оберненість $g(y)=g(f(x))=x$. Якщо $y \to y_0$, то $g(y) \to g(y_0) \Rightarrow x \to x_0$.\\
$\boxed{=} \huge \lim_{x \to x_0}\frac{x-x_0}{f(x)-f(x_0)} = \frac{1}{f'(x_0)} \\ \Rightarrow g$ - диференційована в т. $y_0$.
\end{proof}

\begin{definition}
Функція $f$ є \textbf{диференційованою на множині} $A$, якщо
\begin{align*}
\forall x_0 \in A: f \text{ - диференційована в т. } x_0
\end{align*}
\end{definition}

\subsubsection*{Таблиця похідних елементарних функцій}
\begin{center}
\begin{tabular}{ c|c } 
 $f(x)$ & $f'(x)$ \\
 \hline 
 $const$ & $0$ \\ [2ex]
 \hline 
 $x^\alpha, \alpha \neq 0$ & $\alpha \cdot x^{\alpha-1}$ \\ [2ex]
 \hline 
 $e^x$ & $e^x$ \\ [2ex]
 \hline 
 $a^x$ & $a^x \cdot \ln a$ \\ [2ex]
 \hline 
 $\sin x$ & $\cos x$\\ [2ex]
 \hline 
 $\cos x$ & $-\sin x$\\ [2ex]
 \hline 
 $\tg x$ & $\huge \frac{1}{\cos^2 x}$\\ [2ex]
 \hline 
 $\ctg x$ & $-\huge \frac{1}{\sin^2 x}$\\ [2ex]
 \hline 
 $\ln x$ & $\huge \frac{1}{x}$\\ [2ex]
 \hline 
 $\log_a x$ & $\huge \frac{1}{x \cdot \ln a}$\\ [2ex]
 \hline 
 $\arcsin x$ & $\huge \frac{1}{\sqrt{1-x^2}}$\\ [2ex]
 \hline 
 $\arccos x$ & $\huge -\frac{1}{\sqrt{1-x^2}}$\\ [2ex]
 \hline 
 $\arctg x$ & $\huge \frac{1}{1+x^2}$\\ [2ex]
 \hline 
 $\arcctg x$ & $\huge -\frac{1}{1+x^2}$\\ [2ex]
 \hline 
 $\ln(x+\sqrt{1+x^2})$ & $\huge \frac{1}{\sqrt{1+x^2}}$\\ [2ex]
 \hline 
\end{tabular}
\end{center}

Почергово доведемо кожну похідну:\\
1. $f(x) = const$\\
$f'(x_0) = \huge \lim_{x \to x_0} \frac{c-c}{x-x_0} = \lim_{x \to x_0} 0 = 0$
\bigskip \\

2. $f(x) = x^{\alpha}$\\
$f'(x_0) = \huge \lim_{x \to x_0} \frac{x^{\alpha} - x_0^{\alpha}}{x-x_0} \overset{x-x_0 = t \to 0}{=} \lim_{t \to 0} \frac{(t+x_0)^{\alpha} - x_0^{\alpha}}{t} = x_0^{\alpha-1} \lim_{t \to 0} \frac{\left(1 + \frac{t}{x_0}\right)^{\alpha} - 1}{\frac{t}{x_0}} = \alpha x_0^{\alpha-1}$
\bigskip \\

3. $f(x) = e^x$\\
$f'(x_0) = \huge \lim_{x \to x_0} \frac{e^x-e^{x_0}}{x-x_0} = \lim_{x \to x_0} \frac{e^{x_0}(e^{x-x_0}-1)}{x-x_0} = e^{x_0}$
\bigskip \\

4. $h(x) = a^x$\\
Перепишемо інакше: $h(x) = e^{x \cdot \ln a}$\\
Побачимо, що $y = f(x) = x \cdot \ln a$, а в той час $g(y) = e^y \Rightarrow h(x)=g(f(x))$\\
Тоді за композицією, $h'(x_0) = g'(y_0) f'(x_0) = e^{y_0} \ln a = e^{x_0 \ln a} \ln a = a^{x_0} \ln a$
\bigskip \\

5. $f(x) = \sin x$\\
$f'(x_0) = \huge \lim_{x \to x_0} \frac{\sin x - \sin x_0}{x-x_0} = \lim_{x \to x_0} \frac{2 \sin \frac{x-x_0}{2} \cos \frac{x-x_0}{2}}{x-x_0} = \lim_{x \to x_0} \frac{\sin \frac{x-x_0}{2}}{\frac{x-x_0}{2}} \cos \frac{x-x_0}{2} = \cos x_0$
\bigskip \\

6. $h(x) = \huge \cos x = \sin \left(\frac{\pi}{2} - x \right)$\\
$f(x) = \huge \frac{\pi}{2} - x$, $g(y) = \sin y \Rightarrow h(x) = g(f(x))$\\
Отже, $h'(x_0) = g'(y_0)f'(x_0) = \cos y_0 (-1) = \huge -\cos \left(\frac{\pi}{2} - x \right) = -\sin x$
\bigskip \\

7. $f(x) = \tg x$\\
Або $f(x) = \huge \frac{\sin x}{\cos x}$\\
Тоді $f'(x) = \huge \frac{(\sin x)' \cos x - \sin x (\cos x)'}{\cos^2 x} = \frac{\cos^2 x + \sin^2x}{\cos^2 x} = \frac{1}{\cos^2 x}$
\bigskip \\

8. $f(x) = \ctg x$\\
\textit{За аналогічними міркуваннями до 7.}
\bigskip \\

9. $g(y) = \ln y$\\
Маємо функцію $f(x) = e^x$, тоді $f,g$ - взаємно обернені\\
Тоді оскільки $f'(x_0) = e^{x_0}$, то $g'(y_0) = \huge \frac{1}{f'(x_0)} = \frac{1}{e^{x_0}} = \frac{1}{e^{\ln y_0}} = \frac{1}{y_0}$
\bigskip \\

10. $f(x) = \log_a x$\\
Або $f(x) = \huge \frac{\ln x}{\ln a} \Rightarrow f'(x_0) = \frac{1}{\ln a} \frac{1}{x_0}$
\bigskip \\

11. $g(y) = \arcsin y$\\
Маємо функцію $f(x) = \sin x$, тоді $f,g$ - взаємно обернені\\
Тоді оскільки $f'(x_0) = \cos x_0$, то $g'(y_0) = \huge \frac{1}{f'(x_0)} = \frac{1}{\cos x_0}= \frac{1}{\cos (\arcsin y_0)} = \\ = \frac{1}{\sqrt{1- \sin^2(\arcsin y_0)}} = \frac{1}{\sqrt{1-y_0^2}}$\\
Важливо, що тут функція $f: \huge \left[-\frac{\pi}{2},\frac{\pi}{2}\right] \to [-1,1]$
\bigskip \\

12. $f(x) = \arccos x$\\
Або $f(x) = \huge \frac{\pi}{2} - \arcsin x \Rightarrow f'(x_0) = - \frac{1}{\sqrt{1-x_0^2}}$
\bigskip \\

13. $g(y) = \arctg y$\\
\textit{За аналогічними міркуваннями до 11.}, але тут вже $f: \huge \left(-\frac{\pi}{2},\frac{\pi}{2}\right) \to \mathbb{R}$, $f(x) = \tg x$
\bigskip \\

14. $f(x) = \arcctg x$\\
\textit{За аналогічними міркуваннями до 12.}, але $\arcctg x = \huge \frac{\pi}{2} - \arctg x$
\bigskip \\

15. $f(x) = \ln(x + \sqrt{1+x^2})$\\
$f'(x_0) = \huge \frac{1}{x_0+ \sqrt{1+x_0^2}} \cdot (x+ \sqrt{1+x^2})'_{x = x_0} = \frac{1+ \frac{1}{2\sqrt{1+x_0^2}} \cdot (1+x^2)'_{x=x_0}}{x_0+ \sqrt{1+x_0^2}} = \frac{1+ \frac{x_0}{\sqrt{1+x_0^2}}}{x_0 + \sqrt{1+x_0^2}} = \\ = \frac{\sqrt{1+x_0^2}+x_0}{x_0 + \sqrt{1 + x_0^2}} \frac{1}{\sqrt{1+x_0^2}} = \frac{1}{\sqrt{1+x_0^2}}$
\bigskip \\

Тут треба більш детально про $f(x) = \ln(x+\sqrt{1+x^2})$ сказати.\\
Розглянемо рівняння $\sh x = y$.\\
Розв'яжемо її відносно $x$.\\
$\huge \frac{e^x-e^{-x}}{2} = y \Rightarrow e^x-e^{-x}=2y \Rightarrow e^{2x}- 2y e^x - 1 = 0$\\
$\Rightarrow e^x = y \pm \sqrt{1+y^2} \Rightarrow e^x = y + \sqrt{1+y^2}$\\
$\Rightarrow x = \ln(y + \sqrt{1+y^2})$\\
Таким чином, можна стверджувати, що $\ln(y+\sqrt{1+y^2}) = \textrm{arcsh } y$.\\
Але найбільше застосування все ж таки виявляється згодом (коли підуть інтеграли).

\begin{example} Обчислити похідну функції $f(x) = \sqrt[3]{\dfrac{1+x^2}{1-x^2}} + 2022$.\\
$f'(x) = \left( \sqrt[3]{\dfrac{1+x^2}{1-x^2}} + 2022 \right)' + \left(\left( \dfrac{1+x^2}{1-x^2} \right)^{\frac{1}{3}} \right)' + (2022)' = \dfrac{1}{3} \left( \dfrac{1+x^2}{1-x^2} \right)^{-\frac{2}{3}} \cdot \left( \dfrac{1+x^2}{1-x^2} \right)' + 0 = \\ = \dfrac{1}{3} \left( \dfrac{1+x^2}{1-x^2} \right)^{-\frac{2}{3}} \dfrac{2x(1-x^2)+2x(1+x^2)}{(1-x^2)^2} = \dfrac{1}{3} \left( \dfrac{1+x^2}{1-x^2} \right)^{-\frac{2}{3}} \dfrac{4x}{(1-x^2)^2}$
\end{example}

\subsection{Похідні по один бік}
\begin{definition}
\textbf{Односторонню похідну} функції $f(x)$ в т. $x_0$ називають:\\
-якщо \textbf{справа}: $\huge f'(x_0^+) \overset{\text{або}}{=} f'(x_0+0) = \lim_{x \to x_0^+} \frac{f(x)-f(x_0)}{x-x_0}$.\\
- якщо \textbf{зліва}: $\huge f'(x_0^-) \overset{\text{або}}{=} f'(x_0-0) = \lim_{x \to x_0^-} \frac{f(x)-f(x_0)}{x-x_0}$.
\end{definition}

\begin{theorem}
Функція $f$ - диференційована в т. $x_0$ $\iff$ вона містить похідну зліва та справа, а також $f'(x_0^+) = f'(x_0^-)$.
\end{theorem}

\begin{proof}
$f$ - диференційована в т. $x_0$ $\iff$ $\exists f'(x_0)$, тобто $\exists$ границя $\iff$ $\exists$ та сама границя зліва та справа, які рівні $\iff$ вона містить похідну зліва та справа та $f'(x_0^+) = f'(x_0^-)$.
\end{proof}

\begin{example}
Знайти похідну функції $f(x) = |x|$.\\
Якщо $x>0$, то $f(x) = x \Rightarrow f'(x) = 1$.\\
Якщо $x<0$, то $f(x) = -x \Rightarrow f'(x) = -1$.\\
Перевіримо існування похідної в т. $x_0 = 0$.\\
$f'(0+0) = \huge \lim_{x \to 0^+} \frac{|x|-|0|}{x-0} = 1 \hspace{1cm}$
$f'(0-0) = \huge \lim_{x \to 0^-} \frac{|x|-|0|}{x-0} = -1$\\
$\Rightarrow f'(0^+) \neq f'(0^-)$, отже $\not \exists f'(0)$. До речі кажучи, похідну функції можна переписати інакше:\\
$f'(x) = \huge \frac{|x|}{x}$.\\
Також приклад того, що $f$ в т. $0$ неперервна, але не диференційована - контрприклад для \textbf{Prp 5.1.2}.
\end{example}

\begin{remark}
У першому означенні розділу взагалі треба вимагати т. $x_0 \in A$ бути внутрішньою. Утім в рамках аналізу $\mathbb{R}$ гранична точка теж припустима, оскільки ми маємо таке поняття як похідна справа та зліва, $f'(x_0+0),f'(x_0-0)$. Чого не можна сказати в аналізі $\mathbb{R}^n$.\\
Якщо мені дадуть функцію $f: [0,1] \to \mathbb{R}$, де $f(x) = e^x$, то з похідними в внутрішніх точок все зрозуміло. А ось на кінцях, що не є вже внутрішніми, але граничними, $\exists f'(0) = f'(0+0)$, а також $\exists f'(1) = f'(1-0)$.
\end{remark}

\subsection{Дотична та нормаль до графіку функції}
\begin{definition}
Пряма $y = k (x-x_0) + f(x_0)$ називається \textbf{дотичною до графіку функції} $f(x)$ \textbf{в т.} $x_0$, якщо
\begin{align*}
f(x) - [k(x-x_0)+f(x_0)] = o(x-x_0), x\to x_0
\end{align*}
\end{definition}

\begin{proposition}
Функція $f$ має дотичну в т. $x_0 \iff f$ - диференційована в т. $x_0$. \\ При цьому $k = f'(x_0)$.
\end{proposition}

\begin{proof}
$f$ має дотичну в $x_0 \iff f(x) - [k(x-x_0)+f(x_0)] = o(x-x_0), x \to x_0 \iff \\ \iff f(x)-f(x_0) = k(x-x_0) + o(x-x_0), x \to x_0 \overset{\textrm{def.}}{\iff}$ $f$ - диференційована в т. $x_0$, $k=f'(x_0)$.
\end{proof}
Таким чином, рівняння дотичної задається формулою $$y - f(x_0) = f'(x_0)(x-x_0)$$
\bigskip \\
Є ще інше пояснення дотичної:\\
Нехай є фіксована точка $(x_0,f(x_0))$ та точка $(x^*,f(x^*))$. Через ці дві точки проведемо пряму - її ще називають \textbf{січною}. Маємо таке рівняння:\\
$\dfrac{x-x_0}{x^*-x_0} = \dfrac{y-f(x_0)}{f(x^*) - f(x_0)} \Rightarrow \dfrac{f(x^*)-f(x_0)}{x^*-x_0}(x-x_0) = y - f(x_0)$.\\
Ну а далі спрямуємо $x^* \to x_0$. І якщо функція $f$ - диференційована в т. $x_0$, то одразу маємо\\
$f - f(x_0) = f'(x_0)(x-x_0)$.\\
Що й хотіли.

\begin{definition}
Пряма, яка проходить через т. дотику $(x_0, f(x_0))$ та перпендикулярна до дотичної, називається \textbf{нормаллю до графіку функції} $f(x)$ \textbf{в т.} $x_0$.
\end{definition}

Знайдемо безпосередньо рівняння нормалі. Маємо рівняння дотичної:
$f'(x_0)(x-x_0) - (y-f(x_0)) = 0$.\\
Нормальний вектор дотичної задається координатами $\vec{n} = (f'(x_0); -1)$.\\
Тоді для рівняння нормалі даний вектор буде напрямленим. Нам також відомо, що нормаль проходить через т. $(x_0,f(x_0))$, а отже,\\
$\huge \frac{x-x_0}{f'(x_0)} = \frac{y-f(x_0)}{-1} \Rightarrow f'(x_0)(y-f(x_0)) = -(x-x_0)$.\\
Таким чином, рівняння нормалі задається формулою $$y-f(x_0) = \huge -\frac{1}{f'(x_0)}(x-x_0)$$
\begin{figure}[H]
\centering
\begin{tikzpicture}
\draw[thick, ->] (-3,0)--(4,0) node[below = 2pt] {$x$};
\draw[thick, ->] (0,-2)--(0,4) node[left = 2pt] {$y$};
\draw[thick] plot [smooth, tension=1] coordinates { (-2,1) (-1,2) (1,1) (3,3.5) } node[align=center, below, left =4pt] {$f(x)$};
\draw[thick, red] (-1,1/3)--(4,2);
\draw[thick, blue] (0.5,2.5)--(2,-2);
\filldraw (1,1) circle (2pt) node[align=center, below = 4pt, right = 2pt] {$(x_0, f(x_0))$};
\end{tikzpicture}
	\caption*{Графік функції, до якої проведена дотична (червоний) та нормаль (синій).}
\end{figure}

\begin{example}
Знайти дотичну до графіку функції $f(x) = 2 \cos x + 5$ в т. $x_0 = \huge \frac{\pi}{2}$.\\
$y = f'(x_0)(x-x_0)+f(x_0)$\\
$f(x_0) = \huge f(\frac{\pi}{2}) = 5$\\
$f'(x_0) = \huge f'(\frac{\pi}{2}) = -2 \sin x |_{x = \frac{\pi}{2}} = -2$\\
Отже, маємо:\\
$y = \huge -2(x-\frac{\pi}{2}) + 5 = -2x + (5 - \pi)$.
\end{example}

\subsection*{Ліричний відступ}
Тут вже виникає необхідність поговорити про похідну функції, якщо вона раптом стане рівною нескінченність. І дійсно, ми можемо допускати такий випадок.\\
$f'(x_0) = \huge \lim_{x \to x_0} \dfrac{f(x)-f(x_0)}{x-x_0} = \pm \infty$\\
Одразу зауважу, що просто $\infty$ границі бути не може.

\begin{example}
Нехай є функція $f(x) = \sqrt[3]{x^2}$. Знайдемо похідну цієї штуки в т. $x_0 = 0$ за означенням.\\
$f'(0) = \huge \lim_{x \to 0} \dfrac{\sqrt[3]{x^2} - 0}{x} = \lim_{x \to 0} \dfrac{1}{\sqrt[3]{x}} = \infty$\\
Проте для існування похідної необхідно і достатньо існування похідних з різних боків, а тут\\
$f'(0^-) = \huge \lim_{x \to 0^-} \dfrac{\sqrt[3]{x^2} - 0}{x} = -\infty$\\
$f'(0^+) = \huge \lim_{x \to 0^+} \dfrac{\sqrt[3]{x^2} - 0}{x} = +\infty$.\\
Зрозуміло, що жодним чином $f'(0-0) \neq f'(0+0)$, тож похідна в $\infty$ існувати точно не може.
\begin{figure}[H]
\centering
{
\begin{tikzpicture}

\draw[thick, ->] (-3,0)--(3,0) node[anchor = north] {$x$};
\draw[thick, ->] (0,-0.5)--(0,2) node[anchor = east] {$y$};

\draw[thick, domain=0.001:3, variable=\x, samples = 1000] plot({\x}, {((\x)^(2/3)}) node[anchor = north west] {$f(x) = \sqrt[3]{x^2}$};
\draw[thick, domain=-3:-0.001, variable=\x, samples = 1000] plot({\x}, {((-\x)^(2/3)});
\end{tikzpicture}
}
\end{figure}
\end{example}

А тепер повернімось до геометричних застосувань. Вже відомо, що $f'(x_0) = \tg \alpha$ для дотичних.\\
Якщо $f'(x_0) \to \pm \infty$, тобто $\tg \alpha \to \pm \infty$, то тоді кут $\alpha \to \pm \dfrac{\pi}{2}$. Тобто це означає, що ми матимемо справу з дотичною, яка є вертикальною прямою в т. $x_0$, тобто\\
$x=x_0$.

\begin{example}
Нехай є функція $f(x) = \sqrt[3]{x-1}$. Знайдемо похідну цієї штуки в т. $x_0 = 1$ за означенням.\\
$f'(1) = \huge \lim_{x \to 1} \dfrac{\sqrt[3]{x-1}-0}{x-1} = \lim_{x \to 1} \dfrac{1}{\sqrt[3]{(x-1)^2}} = + \infty$\\
Похідна існує. Це можна навіть перевірити, пошукавши похідну зліва та справа\\
Тоді дотичною графіка функції $f$ в т. $x_0 = 0$ буде вертикальна пряма\\
$x_0 = 1$.\\
\begin{figure}[H]
\centering
{
\begin{tikzpicture}

\draw[thick, ->] (-3,0)--(3,0) node[anchor = north] {$x$};
\draw[thick, ->] (-1,-2)--(-1,2) node[anchor = east] {$y$};

\draw[thick, domain=0.001:3, variable=\x, samples = 1000] plot({\x}, {((\x)^(1/3)}) node[anchor = north] {$f(x) = \sqrt[3]{x-1}$};
\draw[thick, domain=-3:-0.001, variable=\x, samples = 1000] plot({\x}, {(-(-\x)^(1/3)});

\draw[thick, red] (0,-1)--(0,1);
\end{tikzpicture}
}
\end{figure}
\end{example}

\subsection{Диференціал функції}
\begin{definition}
Задано функцію $f$ - диференційована.\\
\textbf{Диференціалом} функції $f$ називають
\begin{align*}
df(x,\Delta x) = f'(x) \Delta x
\end{align*}
\end{definition}

\begin{example}
Розглянемо функцію $f(x) = x$. Вона має похідну $f'(x) = 1$, тому диференційована. Тоді диференціал $df(x,\Delta x)$ запишеться так:\\
$d(x,\Delta x) = \Delta x$
\end{example}
Зазвичай надалі опускають другий аргумент диференціалу та пишут уже так: $dx = \Delta x$. А тому диференціал функції $f$ можна записати іншим чином:
$$ df = f'(x)\,dx $$

\begin{remark}
Геометричний зміст диференціала функції $f(x)$ в т. $x_0$ - це приріст дотичної.
\end{remark}
\begin{figure}[H]
\centering
\begin{tikzpicture}[spy using outlines={circle,yellow,magnification=5,size=6cm, connect spies}]
\draw[thick, ->] (-3,0)--(2.5,0) node[below = 2pt] {$x$};
\draw[thick, ->] (0,-0.5)--(0,4) node[left = 2pt] {$y$};
\draw[thick, domain=-2.5:2, variable=\x] plot({\x}, {exp(\x*ln(2))}) node[anchor = west, scale = 0.7] {$f(x)$};
\draw[thick, domain=-0.5:1.5, red, variable=\x] plot({\x}, {sqrt(2)*ln(2)*(\x-0.5)+sqrt(2)});
\draw[dashed] (0.5, {sqrt(2)})--(0.5,0) node [anchor = north, scale = 0.7] {$x_0$};
\draw[dashed] (1, {exp(1*ln(2))})--(1,0) node [anchor = north, scale = 0.7] {$x$};
\draw[blue] (1, {2*ln(2)*(-0.1)+2})--(1, {sqrt(2)});
\draw[green] (1, {exp(1*ln(2))})--(1, {2*ln(2)*(-0.1)+2});
\draw[dashed] (1, {sqrt(2)})--(0.5, {sqrt(2)});
\draw (0.5+0.2, {sqrt(2)}) arc (0:{atan(2*ln(2))}:0.2) node [anchor = west,scale = 0.3] {$\alpha$};
\node at (0.8, {sqrt(2)}) [scale = 0.3,anchor = north] {$dx$};
\node at (1.15, {2*ln(2)*(-0.1)+2}) [scale = 0.2, anchor = north, blue] {$df(x_0)$};
\spy on (0.7, {exp(0.7*ln(2))}) in node[left] at (12,2);
\end{tikzpicture}
\caption*{Синій - це $df(x_0)$: приблизна різниця між функціями в двох точках. А синій + зелений - це $\Delta f(x_0)$: точна різниця між функціями в двох точках.}
\end{figure}

\subsection{Інваріантність форми першого диференціалу}
Задано функцію $f(x)$ - диференційована. Тоді диференціал $df(x) = f'(x)\,dx$\\
Нехай задано функцію $x = x(t)$ - теж диференційована. Отримаємо складену функцію $f(x(t))$, від якої знайдемо диференціал.\\
$df(x(t)) = (f(x(t)))'\,dt = f'(x(t))x'(t)\,dt = f'(x(t))\,dx(t)$\\
Отримали, що $df(x(t)) = f'(x(t))\,dx(t)$.\\
Коли $x$ - залежна змінна, то формула диференціалу все рівно залишається такою самою. Це й є \textbf{інваріантність форми першого диференціалу}

\subsection{Приблизне обчислення значень для диференційованих функцій}
Задано функцію $f$ - диференційована в т. $x_0$. \\ Тоді за твердженням, функція має дотичну $y = f'(x_0)(x-x_0)+f(x_0)$, для якого:\\
$f(x)-y = o(x-x_0), x \to x_0$.\\
Права частина - якесь нескінченно мале число, яким можна знехтувати.
Тому коли $x$ 'близьке' до $x_0$, тобто $|x-x_0| <<1$, то маємо: $f(x) -y \approx 0 \Rightarrow$ маємо таку формулу: 
$$f(x) \approx f'(x_0)(x-x_0)+f(x_0)$$

\begin{example}
Знайти приблизно значення $\sqrt{65}$.\\
Перетворимо значення іншим чином:\\
$\sqrt{65} \huge = \sqrt{64 \cdot \frac{65}{64}} = 8 \sqrt{\frac{65}{64}} = 8 \sqrt{1 + \frac{1}{64}}$.\\
А тепер розглянемо функцію $f(x) = 8\sqrt{x}$. Тут $x = \huge \frac{65}{64}$, в той час $x_0 = 1$.\\
$|x-x_0| = \huge \abs{\frac{65}{64} - 1} = \frac{1}{64} <<1$\\
Знайдемо значення функції та похідну в т. $x_0$:\\
$f(x_0) = f(1) = 8$\\
$f'(x_0) = f'(1) = \huge 8\frac{1}{2 \sqrt{x}} |_{x = 1} = 4$\\
Таким чином, отримаємо:\\
$\sqrt{65} \approx \huge 4\left(\frac{65}{64}-1\right)+8 = \frac{1}{16} + 8 = 8.0625$.
\end{example}

\subsection{Похідна та диференціал вищих порядків}
\begin{definition}
Задано функцію $f$, для якої $\exists f'(x)$.\\
\textbf{Похідною $2$-го порядку від} $f(x)$ називають $f''(x) = (f'(x))'$, якщо вона існує.
\end{definition}

\begin{definition}
Задано функцію $f$, для якої $\exists f^{(n)}(x)$.\\
\textbf{Похідною $(n+1)$-го порядку від} $f(x)$ називають $f^{(n+1)}(x) = (f^{(n)}(x))'$, якщо вона існує.
\end{definition}

\begin{example}
Знайдемо похідну $n$-го порядку функції $f(x) = \cos x$.\\
$g(x) = \cos x \Rightarrow g'(x) = -\sin x \Rightarrow g''(x) = -\cos x \Rightarrow g'''(x) = \sin x \Rightarrow g^{(4)}(x) = \cos x \Rightarrow \dots$\\
Продовжувати можна довго, але можемо помітити, що:\\
$\cos x = \cos x$\\
$- \sin x = \huge \cos \left(x + \frac{1\pi}{2} \right) = (\cos x)'$\\
$- \cos x = \huge \cos \left(x + \pi \right) = \cos \left(x + \frac{2 \pi}{2} \right) = (\cos x)''$\\
$ \sin x = \huge \cos \left(x + \frac{3\pi}{2} \right) = (\cos x)'''$\\
$\vdots$\\
Спробуємо ствердити, що працює формула: $(\cos x)^{(n)} = \huge \cos \left(x + \frac{n\pi}{2} \right)$. Покажемо, що для $(n+1)$-го члену це теж виконується.\\
$(\cos x)^{(n+1)} = \left((\cos x)^{(n)}\right)' = \huge \left( \cos \left(x + \frac{n\pi}{2} \right) \right)' = -\sin \left(x + \frac{n\pi}{2} \right) = \cos \left(x + \frac{n\pi}{2} + \frac{\pi}{2} \right) = \cos \left(x + \frac{(n+1)\pi}{2} \right)$\\
Остаточно отримаємо, що для функції $f(x) = \cos x$ існують похідні\\
$\forall n \geq 1: f^{(n)}(x) = \huge \cos \left(x + \frac{n\pi}{2} \right)$.
\end{example}

А тепер уявімо собі іншу проблему: задано функції $f,g$, для яких існують $n$ похідних. Спробуємо знайти $(fg)^{(n)}$\\
Будемо робити по черзі:\\
$(fg)' = f'g+fg'$\\
$(fg)''=((fg)')'=(f'g+fg')'=(f'g)'+(fg')'=(f''g+f'g')+(f'g'+fg'')= f''g + 2f'g' + fg''$\\
$(fg)''' = ((fg)'')' = (f''g + 2f'g' + fg'')' = f'''g+f''g+2f''g'+2f'g''+f'g''+fg''' = \\ = f'''g + 3f''g' + 3f'g'' + fg'''$\\
Це можна продовжувати до нескінченності, але можна зробити деякі зауваження, що форма виразу схожа дуже на формулу Бінома-Ньютона, якщо порядок похідної замінити уявно на степінь.\\
Тоді якщо посилатись на МІ, то доведемо таку формулу:

\begin{theorem}[Формула Лейбніца]
$(f(x)g(x))^{(n)} = \huge \sum_{k=0}^n C_n^k f^{(k)}(x) g^{(n-k)}(x)$
\end{theorem}

\begin{example}
Знайти похідну $n$-го порядку функції $y = x^2 \cos x$.\\
$f(x) = x^2 \Rightarrow f'(x) = 2x \Rightarrow f''(x) = 2 \Rightarrow f'''(x) = 0 \Rightarrow \dots$\\
Коротше, $\forall n \geq 3: f^{(n)}(x) = 0$.\\
$g(x) = \cos x \overset{\textrm{попередній приклад}}{\Rightarrow} \forall n \geq 1: g^{(n)}(x) = \huge \cos \left(x + \frac{n\pi}{2} \right)$\\
Скористаємось ф-лою Лейбніца:\\
$y^{(n)} = (f(x) g(x))^{(n)} = \huge \sum_{k=0}^n C_n^k f^{(k)}(x)g^{(n-k)}(x) = 
\\ = C_n^0 f(x)g^{(n)}(x) + C_n^1 f'(x)g^{(n-1)}(x) + C_n^2 f''(x)g^{(n-2)}(x) + C_n^3 f'''(x)g^{(n-3)}(x) + \dots + C_n^n f^{(n)}(x)g(x) = \\
= f(x)g^{(n)}(x) + n f'(x)g^{(n-1)}(x) + \frac{n(n-1)}{2} f''(x)g^{(n-2)}(x) + 0 = \\
= x^2 \cos \left(x + \frac{n\pi}{2} \right) + 2nx \cos \left(x + \frac{(n-1)\pi}{2} \right) + n(n-1)\cos \left(x + \frac{(n-2)\pi}{2} \right)\\
$.
Тут зауважу, що \\ $\huge \cos \left(x + \frac{(n-1)\pi}{2} \right) = \cos \left(x + \frac{n\pi}{2} - \frac{\pi}{2} \right) = \sin \left(x + \frac{n\pi}{2} \right)$\\
$\huge \cos \left(x + \frac{(n-2)\pi}{2} \right) = \cos \left(x + \frac{n\pi}{2} - \pi \right) = - \cos \left(x + \frac{n\pi}{2} \right)$\\
$= \huge [x^2 - n(n-1)]\cos \left(x + \frac{n\pi}{2} \right) + 2nx \sin \left(x + \frac{n\pi}{2} \right)$.\\
Остаточно,\\
$y^{(n)} = \huge [x^2 - n(n-1)]\cos \left(x + \frac{n\pi}{2} \right) + 2nx \sin \left(x + \frac{n\pi}{2} \right)$.
\end{example}

\begin{definition}
\textbf{Диференціалом $n$-го порядку} функції $f(x)$ називають такий диференціал:
\begin{align*}
d^n f = d(d^{n-1} f)
\end{align*}
\end{definition}
Це можна переписати трошки інакше:\\
$df = f'\,dx$\\
$d^2 f = d(df) = d(f'\,dx) = dx \, d(f') = dx \, f'' \, dx = f'' \, (dx)^2$\\
Частіше позначають $(dx)^2 = dx^2$ таким чином. Тоді\\
$d^2 f = f'' \,dx^2$\\
\vdots
Продовжуючи за МІ, отримаємо:
$$ d^n f = f^{(n)} \,dx^n $$

\begin{example}
Маємо функцію $f(x) = \cos x$, знайдемо диференціал $n$-го порядку.\\
Знаємо похідну $f^{(n)}(x) = \cos \left(x + \dfrac{n \pi}{2} \right)$, тому диференціал\\
$d^n \cos x = \cos \left(x + \dfrac{n \pi}{2} \right) \,dx^n$.
\end{example}

\subsection{Неінваріантність форми другого диференціалу}
Задано функцію $f(x)$ - диференційована. Тоді другий диференціал $d^2f(x) = f''(x)\,dx^2$.\\
Нехай задано функцію $x = x(t)$ - теж диференційована. Отримаємо складену функцію $f(x(t))$, від якої знайдемо другий диференціал.\\
$d^2 f(x(t)) = (f(x(t)))'' \,dt^2 = [f'(x(t))x'(t)]'\,dt^2 = [f''(x(t))(x'(t))^2 + f'(x(t))x''(t)]\,dt^2 = \\ 
= f''(x(t))(x'(t))^2\,dt^2 + f'(x(t))x''(t)\,dt^2 = f''(x(t)) dx(t)^2 + f'(x(t))\,d^2 x(t)$\\
Отримали, що $d^2f(x(t)) \neq f''(x(t))\,dx(t)^2$.\\
Маємо уже випадок \textbf{неінваріантності} \\ Єдине, що якщо $x$ - якась лінійна функція, то тоді інваріантність залишається.

\subsection{Похідна від параметрично заданої функції}
Задано параметричну функцію $y: \begin{cases} y = y(t) \\ x = x(t) \end{cases}$.\\
Мета: знайти $y'_x$ - похідну функції за $x$.\\
Ми знаємо, що $dy = y'_x \,dx \Rightarrow y'_x = \dfrac{dy}{dx}$. Знайдемо ці диференціали:\\
$\begin{cases} dx = x'_t\,dt \\ dy = y'_t\,dt \end{cases} \Rightarrow \huge y'_x = \frac{dy}{dx} = \frac{y'_t}{x'_t}$. Таким чином:
\begin{align*}
y'_x: \begin{cases} y'_x = \dfrac{y'_t(t)}{x'_t(t)} \\ x = x(t) \end{cases}
\end{align*}

\begin{example}
Знайти похідну від функції: $y: \begin{cases} x = \ln t \\ y = t^3 \end{cases}$\\
$x'_t = \huge \frac{1}{t}$,    $y'_t = 3t^2$\\
$\Rightarrow \huge  y'_x: \begin{cases} x = \ln t \\ y'_x = \huge \frac{3t^2}{\frac{1}{t}} = 3t^3 \end{cases}$\\
Сюди ми ще повернемось
\end{example}

Знайдемо другу похідну:\\
$\huge y''_{x^2}(t) = (y'_x(t))'_x = \frac{(y'_x(t))'_t}{x'_t(t)} = \frac{y''_{t^2}(t)x'_t(t)-x''_{t^2}(t)y'_t(t)}{(x'_t(t))^3}$.
\bigskip \\
Складно виглядає, тому краще повернемось до прикладу.\\
Маємо $y: \begin{cases} x = \ln t \\ y = t^3 \end{cases}$\\
$x'_t = \huge \frac{1}{t}$,    $y'_t = 3t^2 \Rightarrow y'_x = 3t^3$\\
Тоді отримаємо, що $y''_{x^2} = \huge \frac{(y'_x)'_t}{x'_t} = \frac{9t^2}{t^3} = \frac{9}{t}$.\\
$y''_{x^2}: \begin{cases} x = \ln t \\ y''_{x^2} = \dfrac{9}{t} \end{cases}$

\subsection{Основні теореми}
\begin{theorem}[Лема Ферма]
Задано функцію $f: (a,b) \to \mathbb{R}$ - диференційована в т. $x_0 \in (a,b)$. Більш того, в т. $x_0$ функція $f$ приймає найбільше (або найменше) значення. \\ Тоді $f'(x_0)=0$.
\end{theorem}

\begin{proof}
Розглянемо випадок $\max$. Для $\min$ аналогічно.\\
В т. $x_0$ функція $f$ приймає найбільше значення, тобто $\forall x \in (a,b): f(x_0) \geq f(x)$.\\
Оскільки $\exists f'(x_0)$, то тоді $\exists f'(x_0^+), \exists f'(x_0^-)$\\
$f'(x_0^+) \huge \overset{\textrm{def.}}{=} \lim_{x \to x_0^+} \frac{f(x)-f(x_0)}{x-x_0}$  $\left( \frac{\leq 0}{> 0} \right)$ $\leq 0$.\\
$f'(x_0^-) \huge \overset{\textrm{def.}}{=} \lim_{x \to x_0^-} \frac{f(x)-f(x_0)}{x-x_0}$ $\left( \frac{\leq 0}{< 0} \right)$ $\geq 0$.\\
Таким чином, $0 \leq f'(x_0^-) = f'(x_0^+) \leq 0 \Rightarrow f'(x_0^-) = f'(x_0^+) = 0 \implies f'(x_0) = 0$.
\end{proof}

\begin{figure}[H]
\centering
\begin{tikzpicture}
\draw[thick,->] (2,0)--(5,0) node[anchor = north west] {$x$};
\draw[thick, domain=3:4.7, variable=\x, samples = 100] plot({\x}, {-(4-\x)^2 + 2});
\draw (3,-1pt)--(3,1pt) node [anchor = north] {$a$};
\draw (4.7,-1pt)--(4.7,1pt) node [anchor = north] {$b$};
\draw [dashed] (4,2)--(4,0) node [anchor = north] {$x_0$};
\draw [thick, red] (3.5,2)--(4.5,2) node [anchor = south, black] {$f'(x_0) = 0$};
\end{tikzpicture}
\end{figure}

\begin{theorem}[Теорема Ролля]
Задано функцію $f: [a,b] \to \mathbb{R}$, $f \in C([a,b])$ та диференційована на $(a,b)$. Більш того, $f(a) = f(b)$. Тоді $\exists \xi \in (a,b): f'(\xi) = 0$.
\end{theorem}

\begin{proof}
Оскільки $f \in C([a,b])$, то за Th. Вейерштраса,\\
$\exists x_1 \in [a,b]: f(x_1) = \huge \min_{x \in [a,b]} f(x)$.\\
$\exists x_2 \in [a,b]: f(x_2) = \huge \max_{x \in [a,b]} f(x)$.\\
Розглянемо два випадки:\\
I. $f(x) = const \Rightarrow f'(x) = 0, \forall x \in (a,b)$, $\xi = x$.\\
II. $f(x) \neq const \Rightarrow$ або є $x_1$, або є $x_2$, або навіть обидва.\\
Якщо беремо $x_2$, то функція $f$ приймає найбільше значення, тому за лемою Ролля, $f'(x_2) = 0 \Rightarrow \xi = x_2$.\\
Для $x_1$ - аналогічно.
\end{proof}

\begin{remark}
Диференційованість в т. $x_0=a,x_0=b$ не обов'язкова.\\
Маємо функцію $f(x) = \sqrt{1-x^2}$. Що ми маємо: $f(-1) = f(1)$, $f \in C([-1,1])$, диференційована всюди, але не в т. $x_0 = \pm 1$. При цьому $\exists \xi = 0: f'(\xi) = 0$.
\end{remark}

\begin{theorem}[Теорема Лагранжа]
Задано функцію $f: [a,b] \to \mathbb{R}$, $f \in C([a,b])$ та диференційована на $(a,b)$. \\ Тоді $\exists c \in (a,b): f'(c) = \huge \frac{f(b)-f(a)}{b-a}$.
\end{theorem}

\begin{proof}
Розглянемо функцію $h(x) = \huge (f(x)-f(a))- \frac{f(b)-f(a)}{b-a}(x-a)$.\\
За сумою та добутками, маємо, що $h \in C([a,b])$ і теж диференційована на $(a,b)$.\\
$h'(x) = \huge f'(x) - \frac{f(b)-f(a)}{b-a}$.\\
Зауважимо, що $h(a) = 0$ та $h(b) = 0 \Rightarrow h(a) = h(b)$. Тому за теоремою Ролля, \\
$\exists \xi = c \in (a,b): f'(c) = 0 \implies f'(c) = \huge \frac{f(b)-f(a)}{b-a}$.
\end{proof}

\begin{figure}[H]
\centering
\begin{tikzpicture}
\pgfmathsetmacro{\const}{(3*ln(6)-3*ln(2.3))/(6-2.3)};
\draw[thick,->] (2,0)--(6.5,0) node[anchor = north west] {$x$};
\draw[thick, domain=2.3:6, variable=\x, samples = 100] plot({\x}, {3*ln(\x)});
\draw (2.3,-1pt)--(2.3,1pt) node [anchor = north] {$a$};
\draw (6,-1pt)--(6,1pt) node [anchor = north] {$b$};
\draw (2.3, {3*ln(2.3)})--(6,{3*ln(6)});
\draw [dashed] ({(3)/(\const)},{3*ln((3)/(\const))})--({(3)/(\const)},0) node [anchor = north] {$c$};
\draw[thick, domain=3:5, variable=\x, samples = 100, red] plot({\x}, {\const*(\x-(3)/(\const)) + 3*ln((3)/(\const))});
\end{tikzpicture}
\caption*{Для $f$ в т. $c$ проведемо дотичну. І в цій точці відрізок, що сполучає початкову та кінцеву точку, буде паралельна дотичній.}
\end{figure}

\begin{corollary}[Наслідки з теореми Лагранжа]
1. Якщо $\forall x \in (a,b): f'(x) = 0$, то $f(x) = const$.\\
2. Якщо $\forall x \in (a,b): f'(x) = k$, то $f(x) = kx + q$.\\
3. Нехай $g$ - така ж за властивостями як і $f$. Якщо $\forall x \in (a,b): f'(x) = g'(x)$, то $f(x) = g(x) + C$.\\
4. Якщо додатково $f'$ - обмежена, то $f$ задовільняє умові Ліпшиця.
\end{corollary}

\begin{definition}
Функція $f$ \textbf{задовільняє умовою Ліпшиця}, якщо
\begin{align*}
\exists L \in \mathbb{R}: \forall x_1,x_2 \in [a,b]: |f(x_1)-f(x_2)| \leq L|x_1-x_2|
\end{align*}
\end{definition}

\begin{proof}
1. $\exists c: f(b)-f(a) = f'(c)(b-a) \Rightarrow f(b) = f(a)$.\\
Але взагалі-то кажучи $\exists c \in (x_1,x_2) \subset (a,b): f(x_1) = f(x_2)$.\\
Коротше, $f(x) = const$.
\bigskip \\

2. Розглянемо функцію $g(x) = f(x) - kx$, теж неперервна і диференційована на $(a,b)$.\\
Тоді $g'(x) = f'(x) - k \Rightarrow g'(x) = 0 \overset{\textrm{насл 1.}}{\Rightarrow} g(x) = q$.\\
Отже, $g(x) = kx + q$.
\bigskip \\

3. Розглянемо функцію $h(x) = f(x) - g(x)$, теж неперервна і диференційована на $(a,b)$.\\
Тоді $h'(x) = f'(x) - g'(x) = 0 \overset{\textrm{насл 1.}}{\Rightarrow} h(x) = C \Rightarrow f(x) = g(x) + C$.
\bigskip \\

4. $\exists c \in (x_1,x_2) \subset (a,b): f(x_2)-f(x_1)=f'(c)(x_2-x_1)$\\
$\Rightarrow |f(x_2)-f(x_1)|=|f'(c)||x_2-x_1| \leq M|x_2-x_1|$. Тоді встановлюючи $L=M$, маємо умову Ліпшиця.
\end{proof}

\begin{theorem}[Теорема Коші]
Задано функції $f,g: [a,b] \to \mathbb{R}$, $f \in C([a,b])$ та диференційовані на $(a,b)$. При цьому $g(x) \not\equiv 0$.\\
Тоді $\exists c \in (a,b): \huge \frac{f'(c)}{g'(c)}  =  \frac{f(b)-f(a)}{g(b)-g(a)}$.\\
\textit{Доводиться аналогічно як теорема Лагранжа.}\\
\textit{Вказівка: розглянути функцію $h(x) = (f(x) - f(a)) - \dfrac{f(b)-f(a)}{g(b)-g(a)}(g(x)-g(a))$.}
\end{theorem}

\begin{example}
Довести нерівність: $|\arctg a - \arctg b| \leq |a-b|$.\\
Оскільки $\arctg x$ - неперервна та диференційована на $(a,b)$, то за теоремою Лагранжа,\\
$\exists c \in (a,b): (\arctg x)'_{x = c} = \dfrac{\arctg b - \arctg a}{b-a}$.\\
Тобто $\dfrac{1}{1+c^2} =\dfrac{\arctg b - \arctg a}{b-a}$.\\
Тоді $|\arctg a - \arctg b| = \abs{\dfrac{1}{1+c^2}} |a-b| \leq |a-b|$.
\end{example}

\subsection{Дослідження функції на монотонність}
Означення монотонної функції можна побачити в розділу про границі функції. Тому приступимо безпосередньо до теорем.
\begin{theorem}
Задано функцію $f: (a,b) \to \mathbb{R}$ та $f$ диференційована на $(a,b)$.\\
Функція $f$ нестрого монотонно $\left[ \begin{gathered} \textrm{зростає} \\ \textrm{спадає} \end{gathered} \right. \iff \forall x \in (a,b): \left[ \begin{gathered} f'(x) \geq 0 \\ f'(x) \leq 0 \end{gathered} \right.$.
\end{theorem}

\begin{proof}
Розглянемо випадок зростаючої функції. Для спадної аналогічно.\\
\rightproof Дано: $f$ - нестрого зростає, тобто $x_2 > x_1 \implies f(x_2) \geq f(x_1)$\\
Оскільки диференційована $\forall x_0 \in (a,b)$, то $\exists f'(x_0)$, а тому \\ 
$\exists f'(x_0^+) = \huge \lim_{x \to x_0^+} \frac{f(x)-f(x_0)}{x-x_0} \left( \frac{\geq 0}{\geq 0} \right) \geq 0$.\\
$\exists f'(x_0^-) = \huge \lim_{x \to x_0^-} \frac{f(x)-f(x_0)}{x-x_0} \left( \frac{\leq 0}{\leq 0} \right) \geq 0$.\\
Також $f'(x_0^+) = f'(x_0^-)$, а отже, $\forall x_0 \in (a,b): f'(x_0) \geq 0$.
\bigskip \\
\leftproof Дано: $\forall x \in (a,b): f'(x) \geq 0$.\\
Зафіксуємо такі $x_1,x_2$, щоб $x_2 > x_1$. Розглянемо функцію тепер на відрізку $[x_1,x_2] \subset (a,b)$.\\
В кожній точці цього відрізку є похідна, тож $f \in C([x_1,x_2])$. Також можна розглядати диференційованість на $(x_1,x_2)$. Тоді за Лагранжом,\\
$\exists c \in (x_1,x_2): f(x_2)-f(x_1) = f'(c)(x_2-x_1) \geq 0$.\\
Остаточно, $f(x_2) \geq f(x_1)$, тобто монотонно нестрого зростає.
\end{proof}

\begin{theorem}
Задано функцію $f: (a,b) \to \mathbb{R}$ та $f$ диференційована на $(a,b)$\\
Функція $f$ строго монотонно $\left[ \begin{gathered} \textrm{зростає} \\ \textrm{спадає} \end{gathered} \right. \impliedby \forall x \in (a,b): \left[ \begin{gathered} f'(x) > 0 \\ f'(x) < 0 \end{gathered} \right.$.\\
\textit{Доведення є аналогічним.}
\end{theorem}

\begin{remark}
А тепер питання, куди зник знак $\implies$ в цій теоремі.\\
Нехай задано функцію $f: (-1,1) \to \mathbb{R}$, де $f(x) = x^3$. Вона строго монотонно зростає.\\
А тепер знайдемо похідну: $f'(x) = 3x^2$. Вона не для всіх точках строго додатня, для $x = 0$ маємо, що $f'(x) = 0$.
\end{remark}

\subsection{Екстремуми функції}
%\subsubsection{Локальні}
\begin{definition}
Задано функцію $f: A \to \mathbb{R}$ та $x_0 \in A$.\\
Точку $x_0$ називають точкою \textbf{локального}\\
\begin{tabular}{ll}
- \textbf{максимуму}, якщо & $\exists \delta > 0: \forall x \in (x_0-\delta, x_0 +\delta) \cap A: f(x_0) \geq f(x)$\\
- \textbf{мінімуму}, якщо & $\exists \delta > 0: \forall x \in (x_0-\delta, x_0 +\delta) \cap A: f(x_0) \leq f(x)$
\end{tabular}\\
Ці дві точки називають ще \textbf{точками локального екстремуму}.\\
Якщо нерівність строга, то екстремуми називають \textbf{строгими} та не розглядаємо в околі т. $x_0$.
\end{definition}

\begin{definition}
Якщо в т. $x_0$ маємо $f'(x_0) = 0$ або $\not\exists f'(x_0)$, то таку точку називають \textbf{критичною}.
\end{definition}

\begin{theorem}[Необхідна умова для екстремума]
Задано функцію $f: (a,b) \to \mathbb{R}$ та т. $x_0 \in (a,b)$ - локальний екстремум. Тоді ця точка є критичною.\\
\end{theorem}

\begin{proof}
Розглянемо випадок точки максимуму. Для мінімума аналогічно.\\
$x_0$ - локальна точка максимуму - тобто, приймає в околі т. $x_0$ функція $f$ приймає найбільшого значення. Тоді за лемою Ферма, при існування похідної в т. $x_0$, $f'(x_0) = 0$.\\
Або $\not\exists f'(x_0)$.
\end{proof}

\begin{remark}
Пояснювальний приклад, чому нас точки з неіснуючою похідною цікавить.\\
$f(x) = |x|$ - в т. $x_0 = 0$ похідної нема, проте вона є точкою локального мінімуму.
\end{remark}

\begin{remark}
Інший приклад, чому ця умова не є достатньою.\\
$f(x) = x^3$, маємо похідну $f'(x) = 3x^2 \overset{f'(x) = 0}{\Rightarrow} x_0 = 0$, але вона не є естремумом, оскільки минулого разу дізнались, що така функція зростає всюди.
\end{remark}

\begin{theorem}[Достатня умова для екстремума]
Задано функцію $f: (a,b) \to \mathbb{R}$ та т. $x_0 \in (a,b)$ - критична точка. Відомо, що \\ $\exists \delta > 0: \forall x \in \begin{cases} (x_0-\delta,x_0): f'(x) \leq 0 \\ (x_0,x_0+\delta): f'(x) \geq 0 \end{cases}$ (або нерівності навпаки).\\
Тоді $x_0$ - точка локального мінімуму (максимуму).\\
При строгої нерівності екстрему буде строгим.
\end{theorem}

\begin{proof}
Розглянемо випадок, коли $\forall x \in \begin{cases} (x_0-\delta,x_0): f'(x) \leq 0 \\ (x_0,x_0+\delta): f'(x) \geq 0 \end{cases}$. Для нерівностей навпаки все аналогічно.\\
Тоді звідси $f$ - спадає на $(x_0 - \delta, x_0)$ і зростає на $(x_0, x_0 + \delta)$.\\
Або математично, $\forall x \in (x_0-\delta, x_0): f(x_0) \leq f(x)$ та $\forall x \in (x_0, x_0+\delta): f(x_0) \leq f(x)$.\\
За означенням, це й є точка локального мінімуму.
\end{proof}

\begin{remark}
Робимо такий висновок: щоб знайти локальний екстремум, треба спочатку знайти всі критичні точки, а потім дослідити, які значення похідним вона приймає навколо.
\end{remark}

\begin{example}
Задано функцію $f(x) = x^3-3x^2-9x+2$. Знайдемо всі локальні екстремуми.\\
Спочатку шукаємо критичні точки:\\
$f'(x) = 3x^2-6x-9$\\
$f'(x) = 0 \Rightarrow x^2-2x-3 = 0 \Rightarrow x = -1,x = 2$\\
Перевіримо екстремуми на інтервалі.
\bigskip \\
\begin{tikzpicture}
\draw[thick, ->] (-3,0)--(4,0) node[below = 2pt] {$x$};
\draw[thick] (-1,-0.25)--(-1,0.25) node[below = 6pt] {$-1$};
\draw[thick] (2,-0.25)--(2,0.25) node[below = 6pt] {$2$};
\draw[thick, ->] (-2,0.5)--(-1.5,1);
\draw[thick, ->] (0,1)--(0.5,0.5);
\draw[thick, ->] (2.5,0.5)--(3,1);
\end{tikzpicture}
\\
Стрілки вказують на зростання або на спадання функції на даному інтервалі. Тоді можемо зробити висновок, що $x=-1$ - локальний максимум, а $x=2$ - локальний мінімум.
\end{example}

\subsection{Правила Лопіталя}
\begin{theorem}[I правило Лопіталя]
Задані функції $f,g$ - диференційовані на $(a,b)$ та $\forall x \in (a,b): g'(x) \neq 0$. Також відомо, що:\\
1. $\exists \huge \lim_{x \to b^-} f(x) = 0$, $\exists \huge \lim_{x \to b^-} g(x) = 0$\\
2. $\exists \huge \lim_{x \to b^-} \frac{f'(x)}{g'(x)} = L$\\
Тоді $\exists \huge \lim_{x \to b^-} \frac{f(x)}{g(x)} = L$.\\
\textit{Тут можна замість $x \to b^-$ записати $x \to a^+$, доведення аналогічне.}
\end{theorem}

\begin{proof}
$\huge \lim_{x \to b^-} \frac{f(x)}{g(x)} = \lim_{x \to b^-} \frac{f(x)-f(b)}{g(x)-g(b)} \boxed{=}$\\
Функцію $f$ довизначимо, щоб $f \in C([x,b])$, бо існує ліміт. Тоді за теоремою Коші, $\exists c \in (x, b): \huge \frac{f(x)-f(b)}{g(x)-g(b)} = \frac{f'(c)}{g'(c)}$. Тут $x < c < b$. Коли $x \to b^-$, $b \to b^-$. Отже, $c \to b^-$.\\
$\boxed{=} \huge \lim_{c \to b^-} \frac{f'(c)}{g'(c)} = L$.
\bigskip \\
Випадок, коли $L = \infty$, маємо, що $\dfrac{g'(x)}{f'(x)} \to 0$, а тому $\dfrac{g(x)}{f(x)} \to 0 \implies \dfrac{f(x)}{g(x)} \to \infty$.
\end{proof}

\begin{example}
Обчислити границю $\huge\lim_{x \to 0} \dfrac{x-\sin x}{x^3}$.\\
Маємо $f(x) = x - \sin x$, $g(x) = x^3$ - обидва неперервні та диференційовані. Також $f(x) \to 0, g(x) \to 0$, якщо $x \to 0$\\
Тепер з'ясуємо, куди прямує $\dfrac{f'(x)}{g'(x)}$.\\
$\huge\lim_{x \to 0} \dfrac{1-\cos x}{3x^2} = \lim_{x \to 0} \dfrac{2 \sin^2 \dfrac{x}{2}}{3x^2} = \huge\lim_{x \to 0} \dfrac{2x^2}{3x^2 \cdot 4} = \dfrac{1}{6}$.\\
Всі пункти І правила Лопіталя виконуються. Отже, $\huge\lim_{x \to 0} \dfrac{x-\sin x}{x^3} = \dfrac{1}{6}$.
\end{example}

\begin{theorem}[II правило Лопіталя]
Задані функції $f,g$ - диференційовані на $(a,b)$ та $\forall x \in (a,b): g'(x) \neq 0$. Також відомо, що:\\
1. $\exists \huge \lim_{x \to b^-} g(x) = \infty$\\
2. $\exists \huge \lim_{x \to b^-} \frac{f'(x)}{g'(x)} = L$\\
Тоді $\exists \huge \lim_{x \to b^-} \frac{f(x)}{g(x)} = L$.\\
\textit{Тут можна замість $x \to b^-$ записати $x \to a^+$, доведення аналогічне.}
\end{theorem}

\begin{proof}
Одразу нехай $\varepsilon > 0$, далі знадобиться\\
Маємо $\huge\lim_{x \to b^-} \dfrac{f'(x)}{g'(x)} = L \implies \exists \delta: \forall x: b-\delta < x < b \Rightarrow \abs{\dfrac{f'(x)}{g'(x)}-L}<\varepsilon$.\\
Оберемо фіксоване $x_0 \in (b-\delta,b)$ та будь-яке $x \in (x_0,b)$. Розглядається функція $f,g \in C([x_0,x])$ та диференційовані на $(x_0,x)$. Отже, за теоремою Коші, $\exists c \in (x_0,x): \dfrac{f'(c)}{g'(c)} = \dfrac{f(x)-f(x_0)}{g(x)-g(x_0)}$.\\
Оскільки $x_0 < c < x < b$, для будь-якого $x \in (b-\delta,b)$, то тоді $\forall x \in (b-\delta,b) \Rightarrow \abs{\dfrac{f(x)-f(x_0)}{g(x)-g(x_0)} - L} < \varepsilon$.\\
Дріб поділимо на $g(x)$. Ми це можемо, оскільки $g$ - нескінченно велика для $x \in (b-\delta,b)$\\
$\dfrac{f(x)-f(x_0)}{g(x)-g(x_0)} = \dfrac{\dfrac{f(x)}{g(x)} - \dfrac{f(x_0)}{g(x)}}{1 - \dfrac{g(x_0)}{g(x)}} \implies \dfrac{f(x)}{g(x)} - \dfrac{f(x)-f(x_0)}{g(x)-g(x_0)} = \dfrac{f(x_0)}{g(x)} - \dfrac{f(x)-f(x_0)}{g(x)-g(x_0)} \dfrac{g(x_0)}{g(x)}$.\\
З'ясуємо, куди прямує права частина, якщо $x \to b^-$. Ми знаємо, що $g(x) \to \infty$, ну а $f(x_0),g(x_0)$ - визначені, тоді $\dfrac{f(x_0)}{g(x)} \to 0, \dfrac{g(x_0)}{g(x)} \to 0$. Дріб $\dfrac{f(x)-f(x_0)}{g(x)-g(x_0)}$ обмежена за використанною теоремою Коші, тому все чудово.\\
Отже, $\dfrac{f(x)}{g(x)} - \dfrac{f(x)-f(x_0)}{g(x)-g(x_0)} \to 0$, тож $\abs{\dfrac{f(x)}{g(x)} - \dfrac{f(x)-f(x_0)}{g(x)-g(x_0)}} < \varepsilon$.\\
За нерівністю трикутника, маємо, що $\abs{\dfrac{f(x)}{g(x)} - L} < 2\varepsilon$.\\
Остаточно, $\exists \huge\lim_{x \to b^-} \dfrac{f(x)}{g(x)} = L$.
\bigskip \\
Випадок, коли $L = +\infty$ (для $-\infty$ аналогічно). Ми задаємо $E > 0$, тоді $\exists \delta: \forall x \in (b-\delta,b) \Rightarrow \dfrac{f'(x)}{g'(x)} > E$.\\
За аналогічними міркуваннями, $\forall x \in (b-\delta,b) \Rightarrow \dfrac{f(x)-f(x_0)}{g(x)-g(x_0)} > E$.\\
Оскільки $g(x) \to \infty$, то $\dfrac{1}{g(x)} \to 0$, тобто $-1 < \dfrac{1}{g(x)} < 1$ для деякого $\delta'$.\\
$\implies \dfrac{f(x)}{g(x)} = \dfrac{f(x)-f(x_0)}{g(x)-g(x_0)} + \dfrac{f(x_0)}{g(x)} - \dfrac{f(x)-f(x_0)}{g(x)-g(x_0)} \dfrac{g(x_0)}{g(x)} > E - f(x_0) + Eg(x_0) \implies \dfrac{f(x)}{g(x)} \to +\infty$.
\end{proof}

\begin{example}
Обчислити границю $\huge \lim_{x \to 0^+} x^x$.\\
$\huge \lim_{x \to 0^+} x^x = \lim_{x \to 0^+} e^{x \ln x} = e^{\displaystyle \lim_{x \to 0^+} \frac{\ln x}{\frac{1}{x}}} \boxed{=}$\\
Перевіримо цю границю за Лопіталем:\\
$\huge \lim_{x \to 0^+} \frac{(\ln x)'}{\left(\frac{1}{x}\right)'} = \lim_{x \to 0^+} \frac{\frac{1}{x}}{-\frac{1}{x^2}} = 0$.\\
Отже, можемо продовжувати наш ланцюг обчислення:\\
$\boxed{=} e^0 = 1$
\end{example}

\begin{remark}
Якщо виникає $x \to \pm \infty$, то можна застосувати правило Лопіталя, використавши заміну $t = \dfrac{1}{x}$, де $t \to 0^{\pm}$.
\end{remark}

\begin{remark}
Границю $\huge \lim_{x \to 0} \frac{\sin x}{x}$ в жодному (!) випадку не можна рахувати за Лопіталем, хоча й результат буде таким самим. Все це тому, що $(\sin x)'$ ми отримали завдяки цієї границі, ми посилаємось на те, що ми знаємо цю границю уже (!). Коротше, замнений круг відносно логічної послідовності виклада.
\end{remark}

\subsection{Формула Тейлора}
Задача цього підрозділу полягає в тому, що ми хочемо навчитись апроксимувати функцію в вигляді многочлена навколо певній точці.\\
Маємо функцію $f(x)$ та т. $x_0$.\\
Перше наближення до многочлену - це буде $y = f(x_0)$. Досить грубе наближення.\\
Друге наближення до многочлену - це буде $y = f(x_0) + f'(x_0)(x-x_0)$, якщо функція диференційована. А це вже - дотична, яка дає вже нормальне наближення.\\
Третє наближення до многочлену - це буде $y = f(x_0) + f'(x_0)(x-x_0) + \dfrac{f''(x_0)}{2}(x-x_0)^2$, якщо функція двічі диференційована. Ділю я навпіл, тому що я вимагаю, щоб $y''=f''(x_0)$. Це вже краще наближення, використовуючи знання випуклості функції.\\
\begin{figure}[H]
\begin{tikzpicture}
\draw[thick, domain=-2:1, variable=\x, samples = 100] plot({\x}, {exp(\x)}) node[anchor = north east] {$f(x)$};
\draw[thick,->] (-2,0)--(1.5,0) node [anchor = north west] {$x$};
\draw[thick,->] (-1,-0.2)--(-1,3) node[anchor = east] {$y$};
\node at (0,-0.2) {$x_0$};
\draw[red] (-2,1)--(1.5,1) node at (0,-2) {$y = f(x_0) \phantom{+ \dfrac{1}{1}}$};
\draw[dashed] (0,0)--(0,1);
\end{tikzpicture}
\qquad
\begin{tikzpicture}
\draw[thick, domain=-2:1, variable=\x, samples = 100] plot({\x}, {exp(\x)}) node[anchor = north east] {$f(x)$};
\draw[thick,->] (-2,0)--(1.5,0) node [anchor = north west] {$x$};
\draw[thick,->] (-1,-0.2)--(-1,3) node[anchor = east] {$y$};
\node at (0,-0.2) {$x_0$};
\draw[thick, red, domain=-2:1, variable=\x, samples = 100] plot({\x}, {1+\x}) node at (0,-2) {$y=f(x_0) + f'(x_0)(x-x_0) \phantom{+ \dfrac{1}{1}}$};
\draw[dashed] (0,0)--(0,1);
\end{tikzpicture}
\qquad
\begin{tikzpicture}
\draw[thick, domain=-2:1, variable=\x, samples = 100] plot({\x}, {exp(\x)}) node[anchor = north east] {$f(x)$};
\draw[thick,->] (-2,0)--(1.5,0) node [anchor = north west] {$x$};
\draw[thick,->] (-1,-0.2)--(-1,3) node[anchor = east] {$y$};
\node at (0,-0.2) {$x_0$};
\draw[thick, red, domain=-2:1, variable=\x, samples = 100] plot({\x}, {1+\x + (\x)^2/2}) node at (0,-2) {$y=f(x_0) + f'(x_0)(x-x_0) + \dfrac{f''(x_0)}{2}(x-x_0)^2$};
\draw[dashed] (0,0)--(0,1);
\end{tikzpicture}
\end{figure}
Тощо, тощо, тощо
\begin{definition}
Задано функцію $f$ - диференційована $n$ разів в т. $x_0$.\\
\textbf{Многочленом Тейлора} функції $f$ в т. $x_0$ називається такий многочлен порядка $n$:
\begin{align*}
P_n(x,x_0) = f(x_0) + \dfrac{f'(x_0)}{1!}(x-x_0) + \dfrac{f''(x_0)}{2!}(x-x_0)^2 + \dots + \dfrac{f^{(n)}(x_0)}{n!}(x-x_0)^n
\end{align*}
\end{definition}
Оскільки ми на кожному наближенні вимагали рівність похідних в т. $x_0$, то для многочлена Тейлора має бути теж саме.
\begin{lemma}
$f^{(k)}(x_0) = (P_n(x,x_0))^{(k)}(x_0)$\\
\textit{Зрозуміло.}
\end{lemma}

Я буду собі наближувати щоразу - і тоді в мене виникне певна похибка. Для цієї похибки є теорема, яку наведу після розмови, бо сприйняти буде важко.
\bigskip \\
Розглянемо функцію $f$ - $n$ разів диференційована в т. $x_0$ та многочлен Тейлора $P_n(x,x_0)$.\\
Розглянемо функцію $g(t) = f(x) - P_n(x,t)$, або більш розгорнуто \\ 
$g(t) = f(x) - \left( f(t) + \dfrac{f'(t)}{1!}(x-t) + \dfrac{f''(t)}{2!}(x-t)^2 + \dots + \dfrac{f^{(n)}(t)}{n!}(x-t)^n \right)$.\\
$g(x) = 0$\\
$g(x_0) \overset{\text{позн.}}{=} r_n(x,x_0) = f(x) - P(x,x_0)$, позначимо це як залишковий член - та сама похибка.\\
Тут вимагаємо, щоб функція $f$ була $n$ разів диференційована на відрізку $[x_0,x]$, коли в нас $x_0 < x$. (*) \\
Також вимагатимемо, щоб функція $f$ мала похідну $n+1$ порядку на інтервалі $(x_0,x)$. (**)\\
Маючи (*),(**), ми можемо знайти похідну функції $g$, тоді:\\
$g'(t) = - \left(f'(t) - \dfrac{f'(t)}{1!} + \dfrac{f''(t)}{1!}(x-t) - \dfrac{2f''(t)}{2!}(x-t) + \dfrac{f'''(t)}{2!}(x-t)^2 + \dots - \dfrac{n f^{(n)}(t)}{n!}(x-t)^{n-1} + \dfrac{f^{(n+1)}(t)}{n!}(x-t)^n \right)$\\
$g'(t) = -\dfrac{f^{(n+1)}(t)}{n!}(x-t)^n$.\\
Згідно з (*),(**) ми можемо сказати, що $g \in C([x_0,x])$ та диференційована в $(x_0,x)$. Додамо ще функцію $\varphi \not\equiv 0$ з такими самими умовами. Тоді за теоремою Коші,\\
$\exists c \in (x_0,x): \dfrac{g(x)-g(x_0)}{\varphi(x)-\varphi(x_0)} = \dfrac{g'(c)}{\varphi'(c)} \implies r(x,x_0) = \dfrac{\varphi(x)-\varphi(x_0)}{\varphi'(c)} \dfrac{f^{(n+1)}(c)}{n!}(x-c)^n$.\\
Отримали загальну формулу залишкового члена, але мене буде цікавити інший формат.\\
Тому нехай задано функцію $\varphi(t) = (x-t)^{n+1}$, яка потрапляє під всіма умовами.\\
Тоді маємо, що \\ $r_n(x,x_0) = \dfrac{\varphi(x) - \varphi(x_0)}{\varphi'(c)} \dfrac{f^{(n+1)}(c)}{n!}(x-c)^n = \dfrac{-(x-x_0)^{n+1}}{-(n+1)(x-c)^{n}} \dfrac{f^{(n+1)}(c)}{n!}(x-c)^n = \dfrac{f^{(n+1)}(c)}{(n+1)!}(x-x_0)^{n+1}$.\\
Таким чином, ми можемо сформулювати теорему:
\begin{theorem}[Теорема Тейлора (у формі Лагранжа)]
Задано функцію $f$ - диференційована $n$ разів на $[x_0,x]$ при $x_0 < x$ та має похідну $n+1$ порядка на $(x_0,x)$.\\
Тоді $\exists c \in (x_0,x)$, така, що функція $f$ представляється у вигляді\\
$f(x) = f(x_0) + \dfrac{f'(x_0)}{1!}(x-x_0) + \dots + \dfrac{f^{(n)}(x_0)}{n!}(x-x_0)^n + \dfrac{f^{(n+1)}(c)}{(n+1)!}(x-x_0)^{n+1}$.
\end{theorem}
Інше представлення формули Тейлора буде таким:\\
Ми знову розглянемо функцію $g(x) = f(x) - P_n(x,x_0)$, але цього разу ми спробуємо довести, що $f(x) - P_n(x,x_0) = o((x-x_0)^n), x \to x_0$.\\
Зрозуміло, що $g(x_0) = g'(x_0) = \dots = g^{(n)}(x_0) = 0$. Тепер обчислимо таку границю:\\
$\huge\lim_{x \to x_0} \dfrac{g(x)}{(x-x_0)^n} = \lim_{x \to x_0} \dfrac{g'(x)}{n(x-x_0)^{n-1}} = \lim_{x \to x_0} \dfrac{g''(x)}{n(n-1)(x-x_0)^{n-2}} = \dots = \lim_{x \to x_0} \dfrac{g^{(n)}(x)}{n!} = 0$.\\
Тут ми використовували $n$ разів I правило Лопіталя. Таким чином, ми сформували теорему:
\begin{theorem}[Теорема Тейлора (у формі Пеано)]
Задано функцію $f$ - диференційована $n$ разів в т. $x_0$. Тоді \\
$f(x) = f(x_0) + \dfrac{f'(x_0)}{1!}(x-x_0) + \dots + \dfrac{f^{(n)}(x_0)}{n!}(x-x_0)^n + o((x-x_0)^n), x \to x_0$.
\end{theorem}

\begin{remark}
Існують такі функції, де в певній точці апроксимація не спрацьовує. Такі функції називають \textbf{неаналітичними}.\\
Зокрема $f(x) = \begin{cases} e^{\textstyle -\frac{1}{x^2}}, x \neq 0 \\ 0, x = 0 \end{cases}$\\
В т. $x_0 = 0$ вийде многочлен Тейлора $P_n(x,0) \equiv 0$.\\
\begin{figure}[H]
\centering
\begin{tikzpicture}
\draw[thick,->] (-2,0)--(2,0) node [anchor = north west] {$x$};
\draw[thick,->] (0,-0.2)--(0,1.5) node[anchor = east] {$y$};
\draw[thick, domain=-2.1:-0.01, variable=\x, samples = 100] plot({\x}, {exp(-1/(\x)^2)});
\draw[thick, domain=0.01:2.1, variable=\x, samples = 100] plot({\x}, {exp(-1/(\x)^2)});
\end{tikzpicture}
\end{figure}
\end{remark}

\subsubsection*{Основні розклади в Тейлора}
Всі вони розглядатимуться в т. $x_0 = 0$, всюди $x \to x_0$.\\
I. $e^x \huge = 1 + \frac{x}{1!} + \frac{x^2}{2!} + \dots + \frac{x^n}{n!} + o(x^n)$\\
II. $\sin x \huge = \frac{x}{1!} - \frac{x^3}{3!} + \dots + \frac{(-1)^n}{(2n+1)!}x^{2n+1} + o(x^{2n+2})$\\
III. $\cos x \huge = 1 - \frac{x^2}{2!} + \frac{x^4}{4!} + \dots + \frac{(-1)^n}{(2n)!}x^{2n} + o(x^{2n+1})$\\
IV. $(1+x)^{\alpha} \huge = \frac{\alpha x}{1!} + \frac{\alpha (\alpha-1) x^2}{2!} + \dots + \frac{\alpha (\alpha-1)\dots(\alpha-(n-1)) x^n}{n!} + o(x^n)$\\
V. $\ln(1+x) = x - \dfrac{x^2}{2} + \dots + (-1)^{n-1} \dfrac{x^n}{n} + o(x^n)$
\bigskip \\
А тепер полягає питання, який розклад використовувати: за Лагранжем чи Пеано. Відповідь на ці питання дадуть приклади нижче.
\begin{example}
Обчислити границю функції $\huge \lim_{x \to 0} \frac{\huge e^x-1-\sin x - \frac{x^2}{2}}{x(1-\cos x)}$\\
Маємо, що:\\
$\huge \lim_{x \to 0} \frac{\huge e^x-1-\sin x - \frac{x^2}{2}}{x(1-\cos x)} = \lim_{x \to 0} \frac{\huge e^x - 1 - \sin x - \frac{x^2}{2}}{2x \sin^2 \frac{x}{2}} = 2\lim_{x \to 0} \frac{\huge e^x - 1 - \sin x - \frac{x^2}{2}}{x^3} \boxed{=} $\\
Розкладемо $e^x$ та $\sin x$ до степеня знаменника:\\
$\boxed{=} \huge 2\lim_{x \to 0} \frac{\displaystyle 1 + x + \frac{x^2}{2} + \frac{x^3}{6} + o(x^3) - 1 - x + \frac{x^3}{6} + o(x^4) - \frac{x^2}{2}}{x^3} = 2\lim_{x \to 0} \frac{\displaystyle \frac{x^3}{6} + o(x^3) + \frac{x^3}{6} + o(x^4)}{x^3} = \\ = 2 \lim_{x \to 0} \left(\frac{1}{3} + \frac{o(x^3)}{x^3} + \frac{o(x^4)}{x^3} \right) = 2 \lim_{x \to 0} \left(\frac{1}{3} + \frac{x^4}{x^3} + \frac{x^5}{x^3} \right) = \frac{2}{3}$.\\
\textit{Тобто коли обчислюються ліміти, то тоді краще через Пеано розписувати.}
\end{example}

\begin{example}
Обчислити $\sin 1^\circ$ із точністю до $10^{-6}$.\\
Для дурних як я: із точністю до $10^{-6}$' означає, що реальна відповідь відрізняється від приблизної відповіді не більше ніж на $10^{-6}$.\\
Маємо $f(x) = \sin x$. Розклад цієї формули має такий вигляд:\\
$\sin x = \dfrac{x}{1!} - \dfrac{x^3}{3!} + \dots + \dfrac{(-1)^n}{(2n-1)!}x^{2n-1} + \dfrac{f^{(2n+1)}(c)}{(2n+1)!} x^{2n+1}$.\\
У нашому випадку $1^{\circ} = \dfrac{\pi}{180}$, тоді $c \in \left( 0, \dfrac{\pi}{180} \right)$.
Щоб порахувати з точністю до $10^{-6}$, треба, щоб залишковий член був менше за цю похибку, тобто\\
$\abs{\dfrac{f^{(2n+1)}(c)}{(2n+1)!} x^{2n+1}} < 10^{-6}$\\
$\abs{\dfrac{\cos c}{(2n+1)!} x^{2n+1}} = \dfrac{|\cos c| |x^{2n+1}|}{(2n+1)!} \leq \dfrac{|x^{2n+1}|}{(2n+1)!} \leq \dfrac{\pi^{2n+1}}{(2n+1)! 180^{2n+1}} < \dfrac{1}{(2n+1)! 45^{2n+1}} < \dfrac{1}{1000000}$.\\
Методом перебора можна отримати, що $n = 2$. Тоді\\
$\sin \dfrac{\pi}{180} \approx \dfrac{\pi}{180} - \dfrac{\pi^3}{180^3 3!} = a$.\\
Дійсно, якщо порахувати $|\sin 1^{\circ} - a|$, то різниця не є більше $10^{-6}$.\\
\textit{Тобто коли треба приблизне обчислення, то тоді краще через Лагранжа розписувати.}
\bigskip \\
Перше зауваження: насправді, для $n=1$ різниця вже не перебільшує нашу похибку. Проте це дуже складно перевірити в нерівностях.\\
Друге зауваження: якщо оцінювати нерівності дуже грубо, то тоді $n$ було б великим числом, що не є гарно. Нас не цікавить дуже точне значення.
\end{example}

\subsection{Опуклі функції та точки перегину}
Розглянемо графік функції $f(x)$ на множині $A$. Із множини $A$ розглядаються дві точки $x_1,x_2$, так, що $x_1 > x_2$.
\begin{figure}[H]
\centering
\begin{tikzpicture}
\draw[thick, ->] (-2,0)--(4,0) node[anchor = north] {$x$};
\draw[thick, ->] (0,-0.5)--(0,5) node[anchor = east] {$y$};

\draw[thick, domain=-1.5:4, variable=\x, samples = 1000] plot({\x}, {0.5*(\x-1)^2}) node at (2,1.2) {$f(x)$};
\node[black] at (-1,2) [circle,fill,inner sep=1pt, draw = black]{};
\node[black] at (3.5,3.125) [circle,fill,inner sep=1pt, draw = black]{};
\draw (-1,2)--(3.5,3.125) node[anchor = south east] {$l(x)$};
\draw[dashed] (-1,2)--(-1,0) node [anchor = north] {$x_1$};
\draw[dashed] (3.5,3.125)--(3.5,0) node [anchor = north] {$x_2$};

\end{tikzpicture}
\end{figure}
Це приклад так називаємої \textbf{опуклої функції донизу} (або просто опуклої), коли на множині $A$ справедлива нерівність:
\begin{align*}
\forall x \in A: f(x) \leq l(x)
\end{align*}
Прийнято трошки інше означення, а це просто пояснення, звідки все це береться.\\
Знайдемо рівняння прямої, що проходить через т. $(x_1,f(x_1)), (x_2,f(x_2))$\\
$\dfrac{x-x_1}{x_2-x_1} = \dfrac{l(x)-f(x_1)}{f(x_2)-f(x_1)} \Rightarrow l(x) = f(x_1) + \dfrac{f(x_2)-f(x_1)}{x_2-x_1}(x-x_1)$.\\
Її підставити можна в нерівність, проте таке означення все рівно не є зручним.\\
Зафіксуємо $\lambda \in [0,1]$ та розглянемо точку $x = \lambda x_1 + (1-\lambda) x_2$.\\
Для довільних $\lambda$ точка $x \in [x_1,x_2]$. \\
А якщо це рівняння розв'язти відносно $\lambda$, ми отримаємо, що:\\
$\lambda = \dfrac{x_2-x}{x_2-x_1} \hspace{1cm} 1-\lambda = \dfrac{x-x_1}{x_2-x_1}$.\\
Отримане $\lambda \in (0,1)$. Тоді\\
$x = \dfrac{x_2-x}{x_2-x_1} x_1 + \dfrac{x-x_1}{x_2-x_1} x_2$, це все $\forall x_1 < x < x_2$.\\
Але поки що обмежимось першим виглядом.\\
Підставимо цю точку в рівняння прямої.\\
$l(x) = l(\lambda x_1 + (1-\lambda) x_2) = f(x_1) + \dfrac{f(x_2)-f(x_1)}{x_2-x_1} (\lambda x_1 + (1-\lambda)x_2 - x_1) = \\
= f(x_1) + (f(x_2)-f(x_1))(1-\lambda) = \lambda f(x_1) + (1-\lambda)f(x_2)$.\\
Таким чином, якщо повернутись до нерівності, то отримаємо наступне:
\begin{align*}
\forall \lambda \in [0,1]: f(\lambda x_1 + (1-\lambda)x_2)) \leq \lambda f(x_1) + (1-\lambda) f(x_2)
\end{align*}
А ось таке означення можна використовувати подалі для інших досліджень.\\
Аналогічні міркування будуть для \textbf{опуклої функції догори} (або просто угнутої), але тут нерівність навпаки.

\begin{definition}
Задано функцію $f: A \to \mathbb{R}$.\\
Цю функцію називають \textbf{опуклою $\underset{\textrm{догори}}{\textrm{донизу}}$}, якщо
\begin{align*}
\forall x_1,x_2 \in A: \forall \lambda \in [0,1]: f(\lambda x_1 + (1-\lambda)x_2) \underset{\geq}{\leq} \lambda f(x_1) + (1-\lambda)f(x_2)
\end{align*}
\end{definition}

\begin{remark}
Якщо $\lambda \in (0,1)$ нерівність строга.
\end{remark}

\begin{lemma}[Лема про 3 хорди]
Функція $f: A \to \mathbb{R}$ опукла донизу $\iff$ справедлива нерівність\\
$\dfrac{f(x)-f(x_1)}{x-x_1} \leq \dfrac{f(x_2)-f(x_1)}{x_2-x_1} \leq \dfrac{f(x_2)-f(x)}{x_2-x}$\\
де $x \in (x_1,x_2) \subset A$.
\end{lemma}
\begin{figure}[H]
\centering
\begin{tikzpicture}
\draw[thick, ->] (-2,0)--(4,0) node[anchor = north] {$x$};
\draw[thick, ->] (0,-0.5)--(0,5) node[anchor = east] {$y$};

\draw[thick, domain=-1.5:4, variable=\x, samples = 1000] plot({\x}, {0.5*(\x-1)^2}) node at (3,4) {$f(x)$};
\node[black] at (-1,2) [circle,fill,inner sep=1pt, draw = black]{};
\node[black] at (3.5,3.125) [circle,fill,inner sep=1pt, draw = black]{};
\node[black] at (2,0.5) [circle,fill,inner sep=1pt, draw = black]{};

\node at (-1,2+0.5) {$P_1$};
\node at (3.5,3.125+0.5) {$P_2$};
\node at (2-0.1,0.5+0.4) {$P$};

\draw (-1,2)--(3.5,3.125);
\draw (-1,2)--(2,0.5);
\draw (2,0.5)--(3.5,3.125);
\draw[dashed] (-1,2)--(-1,0) node [anchor = north] {$x_1$};
\draw[dashed] (3.5,3.125)--(3.5,0) node [anchor = north] {$x_2$};
\draw[dashed] (2,0.5)--(2,0) node [anchor = north] {$x$};

\end{tikzpicture}
\end{figure}
Нерівність означає наступне: кутовий коефіцієнт $PP_1 \leq$ кутовий коефіцієнт $P_2P_1 \leq$ кутовий коефіцієнт $P_2P$.

\begin{remark}
Для опуклої догори нерівність навпаки. Для строгої опуклості нерінвість буде строгою.
\end{remark}

\begin{proof}
Зафіксуємо точки $x_1,x_2 \in A$ та точку $x \in (x_1,x_2)$.\\
$f$ - опукла донизу $\iff f(x) \leq \dfrac{x_2-x}{x_2-x_1}f(x_1) + \dfrac{x-x_1}{x_2-x_1}f(x_2)$\\
$\iff (x_2-x_1) f(x) \leq (x_2-x)f(x_1) + (x-x_1)f(x_1)$\\
$\iff (f(x)-f(x_1))(x_2-x_1) \leq (f(x_2)-f(x))(x-x_1)$\\
$\iff \dfrac{f(x)-f(x_1)}{x-x_1} \leq \dfrac{f(x_2)-f(x)}{x_2-x}$ \\
Середня нерівність мене поки що не цікавить, це я так, щоб було.
\end{proof}

\begin{lemma}
Задано функцію $f: A \to \mathbb{R}$ - диференційована на $A$\\
$f$ - опукла $\underset{\textrm{догори}}{\textrm{донизу}}$ $\iff$ $f'$ $\underset{\textrm{не зростає}}{\textrm{не спадає}}$ на $A$.
\end{lemma}

\begin{proof}
\rightproof Дано: $f$ - опукла донизу.\\
Розглянемо т. $x_1, x_2 \in A$, тоді\\
$\dfrac{f(x)-f(x_1)}{x-x_1} \leq \dfrac{f(x_2)-f(x)}{x_2-x}$.\\
Ба більше, оскільки $f$ - диференційована, то $\exists f'(x_1), \exists f'(x_2)$.\\
Тоді отримаємо ось що, використовуючи границі в нерівностях:\\
$f'(x_1) = \huge \lim_{x \to x_1^+} \dfrac{f(x)-f(x_1)}{x-x_1} \leq \dfrac{f(x_2)-f(x_1)}{x_2-x_1}$\\
$\dfrac{f(x_2)-f(x_1)}{x_2-x_1} \leq \huge \lim_{x \to x_2^-} \dfrac{f(x_2)-f(x)}{x_2-x} = f'(x_2)$.\\
Отже, $\forall x_1,x_2 \in A: x_2 > x_1 \Rightarrow f'(x_2) \geq f'(x_1)$.
\bigskip \\
\leftproof Дано: $f'$ - неспадна на $A$, тобто\\
$\forall x_1,x_2 \in A: x_1 < x_2 \Rightarrow f'(x_1) \leq f'(x_2)$.\\
Оскільки $f$ - диференційована на $A$, то за теоремою Лагранжа,\\
$f'(x_1) = \dfrac{f(c) - f(c_1)}{c- c_1}$\\
$f'(x_2) = \dfrac{f(c_2) - f(c)}{c_2 - c}$\\
$\Rightarrow \dfrac{f(c) - f(c_1)}{c- c_1} \leq \dfrac{f(c_2) - f(c)}{c_2 - c}$.\\
Тоді маємо, що $f$ - випукла донизу.
\end{proof}

\begin{remark}
Майже аналогічно для строгої опуклості.\\
Єдине, що в першій частині доведення треба застосувати теореми Лагранжа для точок $z_1 \in (x_1,x)$ та $z_2 \in (x,x_2)$.
\end{remark}

\begin{theorem}
Задано функцію $f: (a,b) \to \mathbb{R}$ - $f \in C([a,b])$ та двічі диференційована на $(a,b)$.\\
$f$ - опукла $\left[ \begin{gathered} \textrm{догори} \\ \textrm{донизу} \end{gathered} \right.$ $\iff$\\
1. $\forall x \in (a,b): \left[ \begin{gathered} f''(x) \leq 0 \\ f''(x) \geq 0 \end{gathered} \right.$;\\
2. $\not\exists (\alpha, \beta) \subset (a,b): f''(x) = 0$.
\end{theorem}

\begin{proof}
$f$ - опукла догори $\iff$ $f'$ - спадає $\iff$ \\
1. $\forall x \in (a,b): \left[ \begin{gathered} f''(x) \leq 0 \\ f''(x) \geq 0 \end{gathered} \right.$;\\
2. $\not\exists (\alpha, \beta) \subset (a,b): f''(x) = 0$.
\end{proof}

\begin{example}
Функція $f(x) = x^2$ буде опуклою донизу, оскільки $f''(x) = 2 > 0$.
\end{example}

\begin{theorem}
Задано функцію $f: (a,b) \to \mathbb{R}$, $f \in C([a,b])$ та диференційована на $(a,b)$.\\
$f$ - опукла донизу на $(a,b) \iff \forall x_0 \in (a,b):$ дотична в т. $x_0$ лежить ничже графіка функції $f$.
\end{theorem}

\begin{proof}
\rightproof Дано: $f$ - опукла донизу на $(a,b)$\\
Зафіксуємо т. $x_0 \in (a,b)$, тоді маємо рівняння дотичної.\\
$\tau(x) = f(x_0) + f'(x_0)(x-x_0) \Rightarrow f(x) - \tau(x) = f(x) - f(x_0) - f'(x_0)(x-x_0)$.\\
За теоремою Лагранжа, отримаємо: $\exists z \in (x,x_0):$ \\
$f(x) - f(x_0) = f'(z)(x-x_0) \Rightarrow f(x) - \tau(x) = (f'(z) - f'(x_0))(x-x_0)$.\\
За дано, маємо: $x < x_0 \Rightarrow z \leq x_0 \Rightarrow f'(z) \leq f'(x_0)$.\\
Тоді маємо, що $f(x) - \tau(x) \geq 0 \Rightarrow \tau(x) \leq f(x)$.\\
Тобто дійсно, дотична знаходиться нижче за графіка функції.\\
Для $(x_0,x)$ ситуація є аналогічною.
\bigskip \\
\leftproof Дано: $\forall x_0 \in (a,b)$, дотична в т. $x_0$ нижче $f$, тобто\\
$\forall x \in [a,b]: \tau(x) = f(x_0) + f'(x_0)(x-x_0) \leq f(x)$\\
$\Rightarrow f(x) - \tau(x) = f(x) - f(x_0) - f'(x_0)(x-x_0) \geq 0$.\\
Тоді отримаємо:\\
$\begin{cases}
\dfrac{f(x)-f(x_0)}{x-x_0} \geq f'(x_0), x > x_0 \\
\dfrac{f(x)-f(x_0)}{x-x_0} \leq f'(x_0), x < x_0 \\
\end{cases}
$.\\
Візьмемо точки $x_1 < x < x_2$, тоді матимемо таку нерівність\\
$\dfrac{f(x_1)-f(x)}{x_1-x} \leq \dfrac{f(x_2)-f(x)}{x_2-x}$.\\
Що й свідчить про випуклість функції $f$ донизу.
\end{proof}

\begin{definition}
Задано функцію $f: A \to \mathbb{R}$ - диференційована в т. $x_0 \in A$.\\
Точку $x_0$ називають \textbf{точкою перегину}, якщо в лівому та правому околі т. $x_0$ вони мають протилежні напрямки опуклості.\\
Варто уточнити, що може існувати $f'(x_0) = \pm \infty$.
\end{definition}

\begin{example}
Маємо $f(x) = \huge \frac{(x-1)^3}{4} + 2$.\\
$f''(x) = \huge \frac{3}{2}(x-1) = 0$\\
Тут буде т. $x_0 = 1$ - точка перегину.\\
Якщо $x > 1$, то $f''(x) > 0$. А якщо $x < 1$, то $f''(x) < 0$.\\
Отже, на $(-\infty,1)$ - випукла догори, а на $(1,+\infty)$ - випукла донизу.
\end{example}

\begin{example}
Маємо $f(x) = \sqrt[3]{x}$.\\
$f''(x) = \dfrac{1}{3} \left( -\dfrac{2}{3} \right) x^{-\frac{5}{3}}$.\\
Тут буде т. $x_0 = 0$ - точка перегину.\\
Водночас $\exists y'(0^+) = \huge \lim_{x \to 0^+} \dfrac{\sqrt[3]{x} - 0}{x - 0} = +\infty \hspace{0.3cm} \exists y'(0^-) = \huge \lim_{x \to 0^-} \dfrac{\sqrt[3]{x} - 0}{x - 0} = +\infty$.
\begin{figure}[H]
\centering
\begin{tikzpicture}
\draw[thick, ->] (-2,0)--(2,0) node[anchor = north] {$x$};
\draw[thick, ->] (0,-2)--(0,2) node[anchor = east] {$y$};

\draw[thick, domain=0.001:1.8, variable=\x, samples = 1000] plot({\x}, {(\x)^(1/3)}) node at (2,1.5) {$f(x)$};
\draw[thick, domain=-1.8:-0.001, variable=\x, samples = 1000] plot({\x}, {-(-\x)^(1/3)});
\end{tikzpicture}
\end{figure}
\end{example}

\begin{example}
Маємо $f(x) = \sqrt{|x|}$.\\
Тут т. $x_0 = 0$ не може бути точкою перегину, оскільки $\not \exists f'(0)$.
\end{example}

\begin{theorem}[Необхідна умова для перегину]
Задано функцію $f: A \to \mathbb{R}$ та т. $x_0 \in A$ - точка перегину.\\
Тоді $f''(x_0) = 0$.\\
\textit{Тут все зрозуміло.}
\end{theorem}

\begin{theorem}[Достатня умова для перегину]
Задано функцію $f: A \to \mathbb{R}$, $f \in C(A)$ та диференційована в околі т. $x_0$ та має другу похідну. Якщо по обидва боки від точки $x_0$ маємо протилежні знаки, то тоді $x_0$ - точка перегину.\\
\textit{Тут теж все зрозуміло.}
\end{theorem}

\begin{theorem}[Нерівність Єнсена]
Задано функцію $f:(a,b) \to \mathbb{R}$ - опукла $\underset{\textrm{догори}}{\textrm{донизу}}$. Тоді\\
$\forall \alpha_1, \dots, \alpha_n \in (0,1): \huge \alpha_1 + \dots + \alpha_n = 1:$\\
$\huge f(\alpha_1 x_1 + \dots + \alpha_n x_n) \underset{>}{<} \alpha_1 f(x_1) + \dots + \alpha_n f(x_n)$.
\end{theorem}

\begin{pfMI}
$n = 2$. Тоді $\forall \alpha_1, \alpha_2: \alpha_1 + \alpha_2 = 1 \Rightarrow \alpha_2 = 1- \alpha_1:$\\
$f(\alpha_1 x_2 + \alpha_2 x_2) = f(\alpha_1 x_2 + (1-\alpha_1)x_2) < \alpha_1 f(x_1) + (1-\alpha_1)f(x_2)$, оскільки наша функція опукла донизу.\\
Припустимо, що для $n-1$ нерівність виконана. Доведемо для $n$:\\
$\forall \alpha_1,\dots,\alpha_n \in (0,1): \forall x \in (a,b):$\\
$f(\alpha_1 x_1 + \dots + \alpha_n x_n) = \huge f\left(\alpha_n x_n + (1-\alpha_n)\left(\frac{\alpha_1}{1-\alpha_n}x_1 + \dots + \frac{\alpha_{n-1}}{1-\alpha_{n-1}}x_{n-1} \right)\right) <$\\
Зауважу, що $\huge \frac{\alpha_1}{1-\alpha_n} + \dots + \frac{\alpha_{n-1}}{1-\alpha_{n-1}} = 1$ та всі доданки $>0$.\\
$< \huge \alpha_n f(x_n) + (1-\alpha_n)\left(\frac{\alpha_1}{1-\alpha_n}x_1 + \dots + \frac{\alpha_{n-1}}{1-\alpha_{n-1}}x_{n-1} \right) = \\ = \alpha_1 f(x_1) + \dots + \alpha_n f(x_n)$\\
МІ доведено
\end{pfMI}

\begin{example}
Розглянемо функцію $f(x) = \ln x$.\\
Вона є опуклою догори, тому що $f''(x) = -\dfrac{1}{x^2} < 0$.\\
Тоді за нерівністю Єнсена, отримаємо:\\
$\ln(\alpha_1 x_1 + \dots + \alpha_n x_n) > \alpha_1 \ln x_1 + \dots + \alpha_n \ln x_n$\\
де $\alpha_1 + \dots + \alpha_n = 1$.\\
Можемо встановити $\alpha_1 = \dots = \alpha_n = \dfrac{1}{n}$, сума буде також рівна одинички. Прийдемо до такої нерівності:\\
$\ln \dfrac{x_1+\dots+x_n}{n} \geq \dfrac{1}{n} \left( \ln x_1 + \dots + \ln x_n \right)$.
\end{example}
\newpage
%\fi %IMPORTANT COMMENT

\section*{Додаткові матеріали на згодом}
\begin{enumerate}
\item Теореми Штольца
\item Ірраціональність числа $e$
\item Функція Діріхле, Рімана та їхня поведінка на неперервність
\item Формула Фаа-ді-Бруно
\item Другу достатню умову випуклості функції
\end{enumerate}

\section*{Література}
\begin{enumerate}
\item КН ММСА: Подколзін Г.Б., Богданський Ю.В.
\item \href{https://www.math.brown.edu/reschwar/INF/handout3.pdf}{Про дійсні числа та дедекіндовий переріз}
\item Б.В. Трушин
\item Лекторий ФПМИ: Лукашов А.Л.
\end{enumerate}
\end{document}