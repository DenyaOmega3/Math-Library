\documentclass[a4paper, 14pt]{extarticle}
\usepackage[margin=1in]{geometry}
\usepackage{amsfonts, amsmath, amssymb}
\usepackage[none]{hyphenat}
\usepackage{fancyhdr} %create a custom header and footer
\usepackage[utf8]{inputenc}
\usepackage[english, main=ukrainian]{babel}
\usepackage{pgfplots}
\usepgfplotslibrary{fillbetween}
\usepackage{tikz}
\usepackage{graphicx}
\usepackage{caption}
\usepackage{float}
\usepackage{physics}
\usepackage[unicode]{hyperref}
\usetikzlibrary{spy}

\fancyhead{}
\fancyfoot{}
\parindent 0ex
\DeclareMathOperator*\uplim{\overline{lim}}
\DeclareMathOperator*\downlim{\underline{lim}}
\def\stackbelow#1#2{\underset{\displaystyle\overset{\displaystyle\shortparallel}{#2}}{#1}}
\def\huge{\displaystyle}
\def\bigline{\vspace{5mm}\\}
\def\defin#1{\textbf{Definition {#1}}}
\def\ex#1{\textbf{Example {#1}}}
\def\rm#1{\textbf{Remark {#1}}}
\def\prp#1{\textbf{Proposition {#1}}}
\def\lm#1{\textbf{Lemma {#1}}}
\def\th#1{\textbf{Theorem {#1}}}
\def\crl#1{\textbf{Corollary {#1}}}
\def\proof{\textbf{Proof.}\\}
\def\proofMI{\textbf{Proof MI.}\\}
\def\contra{\textbf{!Proof.}\\}
\def\bigline{\vspace{5mm}\\}
\def\qed{$\blacksquare$}
\def\dim#1{\textrm{dim} {#1}}
\def\sequence#1{$\displaystyle \left\{ {#1}, n\geq1 \right\}$}
\def\subsequence#1{$\displaystyle \left\{ {#1}, k\geq1 \right\}$}
\def\limitdef#1#2#3#4#5{$\displaystyle \forall #1 > 0: \exists #2(#1): \forall #3 \geq #2: \left|#4 - #5\right| < #1$}
\def\bounded#1#2#3{$\exists #1>0: \forall #2\geq1: \abs{#3} < #1$}

\def\rightproof{$\boxed{\Rightarrow}$ }
\def\leftproof{$\boxed{\Leftarrow}$ }

\begin{document}
	\begin{titlepage}
		\begin{center}
		\hfill
		\vfill
		\line(1,0){400}\\
		\large{\textbf{Математичний аналіз}}\\[1mm]
		{\textbf{1 семестр}}\\[1mm]
		\line(1,0){400}\\
		\vfill
        	\end{center}
    	\end{titlepage}
\tableofcontents
\newpage

\section*{Момент Богданського (або с вікі)}
\subsection*{Зведення числа в дійсну степінь}
Нехай $a \in (0,+\infty), a \neq 1$. Надалі для визначеності вважатимемо $a > 1$\\
Починали з натуральних степененй\\
$a^n = a \cdot a \cdot \dots \cdot a$ - множення $n$ разів\\
$a^1 = a$\\
Потім цілі степені\\
$a^{-n} = \dfrac{1}{a^n}$\\
$a^0 = 1$\\
Далі розглядали властивості, які вже зі школи ми знаємо\\
Тепер розглянемо раціональну степінь $q \in \mathbb{Q}$
\bigline
\lm{} Для $a > 0$ існує єдине число $b > 0$, що $b^n = a$\\
\proof
Розглянемо множину $X = \{x \geq 0 | x^n < a\}$ та $Y = \{x \geq 0 | x^n > a\}$\\
Оскільки $0 \in X$, то $X$ - непорожня. Оскільки при $a>1$ $a \in Y$, а при $a < 1$ $1 \in Y$, $Y$ - непорожня\\
Тоді для $x \in X, y \in Y \Rightarrow x^n < a < y^n \Rightarrow x < y$\\
Остання імплікація від супротивного доводиться\\
Тоді за принципом Дедекінда, $\exists ! b \in \mathbb{R}: x \leq b \leq y$\\
Лишилось показати, що $b^n = a$\\
Ми припустимо, що $b^n > a$, але тоді $\huge \lim_{m \to \infty} \left(b - \dfrac{1}{m} \right)^n = b^n > a$\\
Отже, $\exists m \in \mathbb{N}: \left(b - \dfrac{1}{m} \right)^n > a$, тому елемент $b - \dfrac{1}{m} \in Y$, проте $b \leq b - \dfrac{1}{m}$. Суперечність!\\
Випадок $b^n < a$ аналогічно
\bigline
Останній крок: перевірити єдиність\\
Припускаємо протилежне, але тоді $b_1 < b_2 \Rightarrow b_1^n < b_2^n \Rightarrow a < a$ \qed
\bigline
Тепер відповідаємо, що таке раціональна степінь. Якщо $q = \dfrac{m}{n}$, то тоді\\
$a^q = \left(\sqrt[n]{a} \right)^m = \sqrt[n]{a^m}$\\
Властивості досі зберігаються
\bigline
Нарешті, дістались до дійсної степені. Візьмемо $x_0 \in \mathbb{R}$\\
Нехай також $X = \mathbb{Q} \cap (-\infty, x_0), a > 1$ - тобто множина всіх раціональних чисел, лівіша за дійсне число $x_0$\\
На множині $X$ функція $a^x$ - зростаюча\\
$x_0 \not\in X$ - гранична точка. Тоді існує границя\\
$\huge \lim_{Q \ni x \to x_0^-} a^x = \lim_{X \ni x \to x_0} a^x = \sup_{X} a^x$\\
Аналогічно визначається множина $Y = \mathbb{Q} \cap (x_0,+\infty)$ та $\exists \huge\lim_{Y \ni x \to x_0} a^x = \inf_{Y} a^x$\\
Доведемо, що $\sup_X a^x = \inf_Y a^x$\\
Зрозуміло, що для $r_1 \in X, r_2 \in Y$ справедлива нерівність\\
$r_1 < x_0 < r_2$\\
Але ми візьмемо такі раціональні числа, що $r_2 - r_1 < \dfrac{1}{n}$. При цьому\\
$a^{r_1} < \sup_X a^x \leq \inf_Y a^x < a^{r_2}$\\
Тоді\\
$1 \leq \dfrac{\huge \inf_Y a^x}{\huge \sup_X a^x} \leq a^{r_2-r_2} < a^{\frac{1}{n}}$\\
Оскільки $\huge \lim_{n \to \infty} a^{\frac{1}{n}} = 1$, тоді маємо, що $\huge \inf_Y a^x = \sup_X a^x$\\
Достатньо лише показати, що $f(x) = a^x$ буде неперевною на $\mathbb{Q}$\\
Достатьно показати тоді, що для послідовності $x_n \in \mathbb{Q}$, для якої $x_n \to x_0^+$ має збіжність $a^{x_n} \to a^{x_0}$\\
$a^{x_n-x_)} = a^{x_0}(a^{x_n-x_0}-1)$\\
Останню дужку можна довести через епсілон означення. Остаточно отримаємо\\
$\huge\lim_{n \to \infty} a^{x_n} = a^{x_0}$
\bigline
Отже, $\huge \lim_{x \to x_0} a^x = a^{x_0}$
\newpage
    	
	\section{Перші дрібниці. Дійсні числа}
	Вже з такими числами було більш-менш ознайомлено в школі. Починалось все з натуральних чисел \\
	$\mathbb{N} = \{1,2,3,\dots\}$ \\
	Далі пішли цілі  числа \\
	$\mathbb{Z} = \{\dots,-3,-2,-1,0,1,2,3,\dots\}$\\
	Саме в цілих числах ми змогли визначити вже операцію $+$, але цього недостатньо\\
	Потім раціональні числа \\
	$\mathbb{Q} = \left\{ \dfrac{m}{n} | m \in \mathbb{Z}, n \in \mathbb{N} \right\}$ \\
	А тут вже ми змогли визначити операцію $\cdot$, і цього теж мало\\
	Настав саме час дослідити поле дійсних чисел - $\mathbb{R}$\\
	Одна з головних мотивацій зробити - це прямокутний трикутник зі сторонами $1$
	\begin{figure}[H]
	\centering
	\begin{tikzpicture}
	\draw[thick] (0,0)--(2,0) node at (-0.3,1) {$1$};
	\draw[thick] (0,0)--(0,2) node at (1,-0.3) {$1$};
	\draw[thick] (2,0)--(0,2) node at (1.2,1.2) {$x$};
	\end{tikzpicture}
	\end{figure}
	За теоремою Піфагора, ми вже знаємо, що\\
	$x^2 = 1^2 + 1^2 \implies x^2 = 2$\\
	І от тут виникли проблеми\\
	\prp{1.} Не існує числа $x \in \mathbb{Q}$, щоб $x^2 = 2$\\
	\proof
	!А давайте припустимо, що все ж таки існує $x \in \mathbb{Q}$, тобто $x= \dfrac{m}{n}, m \in \mathbb{Z}, n \in \mathbb{N}$, для якого\\
	$x^2 = 2 \implies \dfrac{m^2}{n^2} = 2 \implies m^2 = 2n^2$\\
	Оскільки $2n^2$ - це парне число, то $m^2$ - також парне, а відтоді $m$ - парне, тоді таке число представмо у вигляді $m = 2k, k \in \mathbb{Z}$\\
	$4k^2 = 2n^2 \implies 2k^2 = n^2$\\
	Оскільки $2k^2$ - це парне число, то $n^2$ - також парне, а відтоді $n$ - парне, тоді таке число представимо у вигляді $n = 2l, l \in \mathbb{Z}$\\
	Проте $m,n$ одночасно не можуть бути парними, оскільки ми отримаємо скоротиму дріб, а за умовою ми не брали таких. Суперечність!\\
	Таким чином, наше припущення було невірним \qed
	\bigline
	Є два варіанти, як розвиватись далі. Проте я піду за принципом Дедекінда: він дуже просто визначив $\mathbb{R}$
	
	\subsection{Аксіоматика множини дійсних чисел, принцип Дедекінда}
	\textbf{Axiom 1.1.1.} Візьмемо якісь числа $a,b,c \in\mathbb{R}$. Тоді наступні аксіоми справедливі:\\
	Відносно операції $+$:\\
	$a+b=b+a$ - комутативність\\
	$(a+b)+c=a+(b+c)$ - асоціативність\\
	$\exists 0 \in\mathbb{R}: a+0=a$ - існування нейтрального елементу\\
	$\exists (-a) \in\mathbb{R}: a+(-a)=0$ - існування оберненого елементу\\
	\\
	Відносно операції $\cdot$:\\
	$a \cdot b=b \cdot a$ - комутативність\\
	$(a \cdot b) \cdot c=a \cdot (b \cdot c)$ - асоціативність\\
	$\exists 1 \in\mathbb{R}: a \cdot 1=a$ - існування нейтрального елементу\\
	$\huge \exists \left(\frac{1}{a}\right) \in\mathbb{R}: a \cdot \frac{1}{a}=1$ - існування оберненого елементу\\
	\\
	$(a+b) \cdot c = a \cdot c + b \cdot c$ - дистрибутивність\\
	\\
	Відношення порядка:\\
	Якщо $a>b$, то $a+c>b+c$\\
	Якщо $a>b$, $c>0$, то $a+c>b$\\
	Якщо $a>b$, $c>0$, то $a \cdot c>b \cdot c$\\
	\\
	\textbf{Axiom 1.1.2. Принцип Дедекінда}\\
	Нехай є дві множини $A,B \subset \mathbb{R}$. Відомо, що $\forall a \in A$, $\forall b \in B: a \leq b$. Тоді $\exists c \in \mathbb{R}: a \leq c \leq b$
	
\subsection{Принцип мат. індукції}
	\defin{1.2.1.} Числова множина $E$ називається \textbf{індуктивною}, якщо $\forall x \in E: x+1 \in E$
	\bigline
	\th{1.2.2.} Множина натуральних чисел $\mathbb{N}$ - мінімальна індуктивна множина, що містить $1$\\
	Інакше кажучи про другу частину, $\forall E$ - індуктивна: $1 \in E \Rightarrow \mathbb{N} \subset E$\\
	\proof
	1) Те, що $\mathbb{N}$ індуктивна, зрозуміло, тому що $\forall k \in \mathbb{N}: k+1 \in \mathbb{N}$\\
	2) Оскільки $1 \in E$ і більш того, вона є індуктивною, то $2 \in E, 3 \in E, \dots, k \in E$\\
	$\forall k \in \mathbb{N} \Rightarrow k \in E$\\
	Таким чином, $\mathbb{N} \subset E$ \qed
	\bigline
	\crl{1.2.3. Принцип мат. індукції}\\
	Розглянемо числову множину $E = \{n \in \mathbb{N}: P(n)\}$\\
	Тут $P(n)$ - це деяка умова\\
	Тоді якщо $1 \in E$ та індуктивна, то $E = \mathbb{N}$\\
	\proof
	За умовою наслідка, маємо, що $E \subset \mathbb{N}$\\
	Оскільки $1 \in E$ та індуктивна, то $\mathbb{N} \subset E$\\
	Отже, $E = \mathbb{N}$ \qed
	\bigline
	Інакшою мовою: хочемо стверджитись, що $P(n)$ виконується при будь-яких $n \in \mathbb{N}$\\
	1. База індукції\\
	$P(1)$ виконується
	\\
	2. Крок індукції\\
	Вважаємо, що $P(n)$ - виконано. Показуємо, що $P(n+1)$ виконується\\
	Тоді наша множина $E$ - індуктивна, що містить одиницю. Отже, МІ перевірено
	\bigline
	\ex{1.2.4} Довести, що\\
	$1 + 2 + \dots + n = \dfrac{n(n+1)}{2}$\\
	Тут множина $E = \left\{n \in \mathbb{N}: 1 + 2 + \dots + n = \dfrac{n(n+1)}{2} \right\}$\\
	1. База індукції\\
	$1 \in E \Rightarrow 1 = \dfrac{1(1+1)}{2} = 1$
	\bigline
	2. Крок індукції\\
	Нехай $k \in E$, тобто\\
	$1 + 2 + \dots + k = \dfrac{k(k+1)}{2}$\\
	Доведемо, що $k+1 \in E$\\
	$1+ 2 + \dots + k + (k+1) = \dfrac{k(k+1)}{2} + k = \dfrac{k(k+1)+2k}{2} = \dfrac{(k+1)(k+2)}{2}$\\
	Отже, $k+1 \in E$\\
	А значить, $E = \mathbb{N}$, тобто наше твердження виконується $\forall n \in \mathbb{N}$. МІ доведено\\
	
\subsection{Основні нерівності}
	\textbf{Theorem 1.3.1. Нерівність Бернуллі}\\ Для всіх $x>-1$ виконується нерівність:
	\begin{align*}
	(1+x)^n \geq 1+nx
	\end{align*}
	\textbf{Proof MI.}\\
	1. База індукції: при $n=1$: $(1+x)^1 \geq 1+1\cdot x$. Нерівність виконується\\
	2. Крок індукції: нехай для фіксованого $n$ дана нерівність виконується. Доведемо для значення $n+1$\\
	$(1+x)^{n+1}=(1+x)(1+x)^n \geq (1+x)(1+nx)=1+(n+1)x+nx^2 \geq \\ \geq 1+(n+1)x$\\
	Отже, така нерівність справедлива $\forall n \geq 1$. МІ доведено \qed
	\bigline
	\textbf{Theorem 1.3.2. Нерівність Коші}\\
	Для всіх $a_1, \cdots, a_n \geq 0$ виконується нерівність:
	\begin{align*}
	\frac{a_1+\cdots+a_n}{n} \geq \sqrt[n]{a_1 \cdots a_n}
	\end{align*}
	\textbf{Proof.}\\
	Тимчасове перепозначення: $\huge A_n = \frac{a_1+\cdots+a_n}{n}$, $\huge G_n = \sqrt[n]{a_1 \cdots a_n}$\\
	Зрозуміло, що $\huge \frac{A_n}{A_{n-1}} > 0 \Rightarrow \frac{A_n}{A_{n-1}}-1>-1$. Тоді за нерівністю Бернуллі\\
	$\huge \left(1+ \left(\frac{A_n}{A_{n-1}} -1 \right) \right)^n \geq 1 + n \cdot \left(\frac{A_n}{A_{n-1}} -1 \right)$\\
	$\Rightarrow \huge \frac{(A_n)^n}{(A_{n-1})^n} \geq \frac{a_n}{A_{n-1}}$\\
	$\Rightarrow \huge (A_n)^n \geq a_n (A_{n-1})^{n-1}$, $\forall n \geq 1$. Тоді\\
	$(A_n)^n \geq a_n (A_{n-1})^{n-1} \geq \cdots \geq a_n a_{n-1} \cdots a_1$. \\ Отже,
	$A_n \geq G_n$, що й хотіли довести \qed
	\bigline
	\textbf{Theorem 1.3.3. Нерівність трикутника}\\
	Для довільних $x,y \in \mathbb{R}$ справедлива нерівність:
	\begin{align*}
	|x+y|\leq|x|+|y|
	\end{align*}
	\textbf{Proof.}\\
	$(|x+y|)^2=(x+y)^2=x^2+2xy+y^2 = |x|^2 + 2xy + |y|^2 \leq |x|^2 + 2|x||y|+|y|^2 = \\ = (|x|+|y|)^2$\\
	$\Rightarrow |x+y| \leq |x|+|y|$ \qed
	\\
	
	\subsection{Відкриті, замкнені множини}
	\defin{1.4.1.} $U(x)$ - \textbf{окіл} т. $x$, тобто довільний інтервал, що містить т. $x$
\bigline
\defin{1.4.2.} $U_{\varepsilon}(x)$ - $\varepsilon$-\textbf{окіл} т. $x \in \mathbb{R}$ називають інтервал:\\
$(x-\varepsilon,x+\varepsilon), \forall \varepsilon > 0$
\bigline
\prp{1.4.3.} Задано $U(x)$ - окіл т. $x$. Тоді $\exists \varepsilon > 0: U_{\varepsilon}(x) \subset U(x)$\\
\proof
Задамо будь-який окіл $U = (a,b)$\\
\begin{tikzpicture}
\draw[thick,->] (0,0)--(5,0);
\fill[black] (2,0) circle (1pt) node [anchor = north] {$x$};
\node[black] at (1,0) {$($};
\node[black] at (4,0) {$)$};
\node at (1,0) [anchor = north] {$a$};
\node at (4,0) [anchor = north] {$b$};
\end{tikzpicture}
\\
Тоді існує $\varepsilon = \min \{|x-a|,|x-b|\}$. І такий окіл $(x-\varepsilon,x+\varepsilon) \subset U(x)$ \qed
\bigline
\defin{1.4.4} Задана множина $A \subset \mathbb{R}$ та елемент $a \in A$\\
$a$ - \textbf{внутрішня точка} множини $A$, якщо $\exists U(a) \subset A$
\bigline
\prp{1.4.5.} $a$ - внутрішня т. $A \iff \exists \varepsilon > 0: (a-\varepsilon,a+\varepsilon) \subset A$\\
\proof
$\boxed{\Rightarrow}$ Дано: $a$ - внутрішня т. $A$\\
Тоді $\exists U$ - окіл т. $a$, а отже, $\exists \varepsilon > 0: (a-\varepsilon,a+\varepsilon) \subset U \subset A$
\bigline
$\boxed{\Leftarrow}$ Дано: $\exists \varepsilon > 0: (a-\varepsilon,a+\varepsilon) \subset A$\\
Автоматично означає, що $A$ - окіл т. $a$, а тому $a$ - внутрішня точка \qed
\bigline
\defin{1.4.6.} Множина $A \subset \mathbb{R}$ називається \textbf{відкритою}, якщо \\ $\forall a \in A$ - внутрішня
\bigline
\ex{1.4.7.} Розглянемо множини: $(a,b), [a,b], (a,+\infty), [a,+\infty), \emptyset, \mathbb{R}$\\
$(a,b)$ - відкрита, оскільки $\forall x \in (a,b): \exists \varepsilon = \min\{|x-a|,|x-b|\}: \\ U_{\varepsilon}(x) \subset (a,b) \Rightarrow \forall x \in (a,b): x$ - внутрішня точка
\bigline
$[a,b]$ - НЕ відкрита. Якщо припустити, що $a$ - внутрішня точка, то $\exists \varepsilon > 0: (a-\varepsilon, a+\varepsilon) \subset [a,b]$, проте $a-\dfrac{\varepsilon}{2} \in (a - \varepsilon, a + \varepsilon), \not \in [a,b]$, тому т. $a$ не може бути внутрішньою. Аналогічно для $b$. Решта - внутрішні, задавши той самий $\varepsilon$, як попередього разу
\bigline
$(a,+\infty)$ - відкрита, тому що $\forall x: \exists \varepsilon = |x-a|$
\bigline
$[a,+\infty)$ - НЕ відкрита через т. $a$: не є внутрішньою. Решта - внутрішні
\bigline
$\emptyset$ - відкрита. Оскільки порожня множина не містить точок, ми не зможемо знайти точку в порожній множині, яка НЕ Є внутрішньою, щоб зруйнувати означення
\bigline
$\mathbb{R}$ - відкрита
\bigline
\prp{1.4.8.} Якщо $\{A_{\lambda}\}$ - сім'я відкритих підмножин, то $\huge \bigcup_{\lambda} A_{\lambda}$ - відкрита\\
\proof
Візьмемо довільну т. $a \in \huge \bigcup_{\lambda} A_{\lambda} \Rightarrow$ принаймні одному з сімей множин $a \in A_{\lambda}$\\
Така множина є відкритою, а тому $a$ - внутрішня точка\\
Із нашого ланцюга отримаємо: $\forall a \in \huge \bigcup_{\lambda} A_{\lambda} \Rightarrow a - $ внутрішня. Тобто $\huge \bigcup_{\lambda} A_{\lambda}$ - відкрита \qed
\bigline
\ex{1.4.9.} Маємо $A = (1,2) \cup (4,16) \cup (32, 64)$. Попередньо ми знаємо, що будь-який інтервал є відкритою множиною. Тому їхнє об'єднання буде відкритою множиною, тобто $A$
\bigline

\defin{1.4.10.} Задана множина $A \subset \mathbb{R}$, $a \in \mathbb{R}$\\
Число $a$ - \textbf{гранична точка} множини $A$, якщо:\\
$\forall U(a): \exists x \in A: x \neq a$
\bigline
\prp{1.4.11.} Задана множина $A \subset \mathbb{R}$, $a \in \mathbb{R}$\\
$a$ - гранична точка $A \iff \forall \varepsilon > 0: \exists x \in A: |x-a|<\varepsilon$\\
\proof
$\boxed{\Rightarrow}$ Дано: $a$ - гранична точка $A$\\
Тобто $\forall U(a): \exists x \in A: x \neq a$\\
Тоді зокрема $\forall U_{\varepsilon}(a): \exists x = a + \dfrac{\varepsilon}{2} \in A: x \neq a$\\
$\Rightarrow \forall \varepsilon > 0: \exists x \in A: |x-a| = \dfrac{\varepsilon}{2} < \varepsilon$
\bigline
$\boxed{\Leftarrow}$ Дано: $\forall \varepsilon > 0: \exists x \in A: |x-a| < \varepsilon$\\
Тоді $\forall U(a): \exists \varepsilon > 0: U_{\varepsilon}(a) \subset U(a) \Rightarrow \exists x \in A: x = a+ \dfrac{\varepsilon}{2} \neq a$ \qed
\bigline
\defin{1.4.12.} \textbf{Проколений окіл} т. $a$: $\overset{\circ}{U}(a) = U(a) \setminus \{a\}$
\bigline
\prp{1.4.13.} Наступні твердження є еквівалетними між собою:\\
1) $a$ - гранична т. $A$\\
2) $\forall \overset{\circ}{U}(a): A \cap \overset{\circ}{U}(a) \neq \emptyset$\\
3) $\forall \overset{\circ}{U}_{\varepsilon}(a): A \cap \overset{\circ}{U}_{\varepsilon}(a) \neq \emptyset$\\
4) $\forall \varepsilon > 0: A \cap (a-\varepsilon, a+\varepsilon)$ - нескінченна множина\\
\proof
$\boxed{1) \Rightarrow 2)}$ Дано: $a$ - гранична т. $A$\\
Тоді $\forall U(a): \exists x \in A: x \neq a$\\
Зафіксуймо $\overset{\circ}{U}{a}$. Для неї існує т. $x \in A$, отже $A \cap \overset{\circ}{U}{a} \neq \emptyset$
\bigline
$\boxed{2) \Rightarrow 3)}$ Дано: $\forall \overset{\circ}{U}(a): A \cap \overset{\circ}{U}(a) \neq \emptyset$\\
Оскільки для окілу $\overset{\circ}{U}(a)$ існує $\varepsilon > 0: \overset{\circ}{U_\varepsilon}(a) \subset \overset{\circ}{U}(a)$, тоді $A \cap \overset{\circ}{U_\varepsilon}(a) \neq \emptyset$. Тоді й для довільного спрацьовує
\bigline
$\boxed{3) \Rightarrow 4)}$ Дано: $\forall \overset{\circ}{U_\varepsilon}(a): A \cap \overset{\circ}{U_\varepsilon}(a) \neq \emptyset$\\
!Припустімо, що не виконується, тобто $\exists \varepsilon^* > 0: A \cap (a-\varepsilon^*,a+\varepsilon^*) = \{x_1,\dots,x_n\}$ - скінченна множина\\
Тоді справедливі нерівності\\
$\begin{cases}
|x_1 - a| < \varepsilon^* \\
\vdots \\
|x_n - a| < \varepsilon^*
\end{cases}$\\
Якщо обрати $\varepsilon = \min \{|x_1-a|, \dots, |x_n-a|\}$, то в цьому проколеному околі нема жодних точок, а отже, $A \cap \overset{\circ}{U_\varepsilon}(a) = \emptyset$, суперечність!
\bigline
$\boxed{4) \Rightarrow 1)}$ Дано: $\forall \varepsilon > 0: A \cap (a-\varepsilon,a+\varepsilon)$ - нескінченна множина\\
Зафіксуймо т. $x = a - \dfrac{\varepsilon}{2} \in A \cap (a-\varepsilon,a+\varepsilon)$, тоді $a$ - гранична т. $A$ \qed
\bigline

\defin{1.4.14.} Множниа $A \subset \mathbb{R}$ називається \textbf{замкненою}, якщо вона містить всі свої граничні точки
\bigline
\ex{1.4.15.} Розглянемо множини: $(a,b), [a,b], (a,+\infty), [a,+\infty), \emptyset, \mathbb{R}$\\
$(a,b)$ - НЕ замкнена, оскільки кожний окіл т. $a$, $b$ - граничні т. для $(a,b)$, які не належать цієї множини
\bigline
$[a,b]$ - замкнена, тому що $\forall x \in [a,b]: \forall \varepsilon > 0: [a,b] \cap (x-\varepsilon,x+\varepsilon) = \left[
\begin{gathered}
\left[a,x+\varepsilon \right) \\
\left(x-\varepsilon, x + \varepsilon \right) \\
\left(x - \varepsilon, b\right] \\
\left[a,b\right]
\end{gathered}
 \right.$ - всі вони нескінченні множини
\bigline
$(a,+\infty)$ - НЕ замкнена, тому що точка $a$ - гранична для $(a,+\infty)$, але не належить
\bigline
$[a,+\infty)$ - замкнена (аналогічно)
\bigline
$\emptyset$ - замкнена: вона містить всі свої граничні точки, яких просто нема
\bigline
$\mathbb{R}$ - замкнена
\bigline
\rm{1.4.15.} Єдині множини, які є одночасно відкритими та замкненими - це $\emptyset, \mathbb{R}$
\bigline
\prp{1.4.16.} $A$ - відкрита множина $\iff \overline{A}$ - замкнена\\
\proof
$\boxed{\Rightarrow}$ Дано: $A$ - відкрита множина\\
!Припустімо, що $\overline{A}$ - НЕ замкнена множина, тобто вона містить НЕ всі свої граничні точки, тобто $\exists a' \in A$, яка буде граничною для $\overline{A}$\\
Оскільки $a' \in A$, то вона є внутрішньою, тобто $\exists \varepsilon > 0: (a'-\varepsilon,a'+\varepsilon) \subset A \Rightarrow (a'-\varepsilon,a'+\varepsilon) \cap \overline{A} = \emptyset$. Суперечність! Бо тут, навпаки, не має виконуватись рівність
\bigline
$\boxed{\Leftarrow}$ Дано: $\overline{A}$ - замкнена множина\\
!Припустімо, що $A$ - НЕ відкрита множина, тобто $\exists a \in A$, яка НЕ є внутрішньою, тобто\\
$\forall \varepsilon > 0: U_{\varepsilon}(a) \not\subset A \Rightarrow U_{\varepsilon}(a) \cap \overline{A} \neq \emptyset$, тобто $a$ - гранична точка $\overline{A}$\\
Оскільки $\overline{A}$ - замкнена, то вона містить всі свої граничні точки, проте $a \not\in \overline{A}$. Суперечність!
\qed
\bigline
\prp{1.4.17.} Якщо $\{A_{\lambda}\}$ - сім'я замкнених підмножин, то $\huge \bigcap_{\lambda} A_{\lambda}$ - замкнена\\
\textit{Випливає із} \textbf{Prp. 1.4.8.}, \textbf{Prp. 1.4.16.} \textit{та правила де Моргана}
\bigline
\subsection{Точкові межі}
	\textbf{Definition 1.5.1.(1)} Задана множина $A \subset \mathbb{R}$\\
	Множина $A$ називається \textbf{обмеженою зверху}, якщо
	\begin{align*}
	\exists c \in \mathbb{R}: \forall a \in A: a \leq c
	\end{align*}
		\textbf{Definition 1.5.1.(2)} Задана множина $B \subset \mathbb{R}$\\
	Множина $B$ називається \textbf{обмеженою знизу}, якщо
	\begin{align*}
	\exists d \in \mathbb{R}: \forall b \in B: b \geq d
	\end{align*}
	Множину всіх чисел, що обмежують множину зверху, позначу за $UpA$, тобто
	\begin{align*}
	UpA = \{c \in \mathbb{R}: \forall a \in A: a \leq c \}
	\end{align*}
	Множину всіх чисел, що обмежують множину зверху, позначу за $DownB$, тобто
	\begin{align*}
	DownB = \{d \in \mathbb{R}: \forall b \in B: b \geq d \}
	\end{align*}
	\ex{1.5.2.} Задана множина $A = \{1-2^{-n} | n \in \mathbb{N}\} = \left\{\dfrac{1}{2}, \dfrac{3}{4}, \dfrac{7}{8}, \dots \right\}$\\
	Є обмеженою зверху, наприклад, числом $2 \in \mathbb{R}$, тобто $\forall a \in A: a < 2$\\
	Є обмеженою знизу, наприклад, числом $0 \in \mathbb{R}$, тобто $\forall a \in A: a > 0$
	\bigline
	\textbf{Proposition 1.5.3.(1)} Якщо $c \in UpA$ та $c_1 > c$, то $c_1 \in UpA$\\
	\textbf{Proposition 1.5.3.(2)} Якщо $d \in DownB$ та $d_1 < d$, то $d_1 \in DownB$\\
	\textit{Обидва твердження випливають з визначення множин}
	\bigline
	\textbf{Proposition 1.5.4.} Множина $UpA$ обмежена знизу, а множина $DownB$ обмежена зверху\\
	\textit{Випливає з означень обмеженості}
	\bigline
	\textbf{Proposition 1.5.5.} Для множини $UpA$ існує мінімальний елемент, а для множини $DownB$ існує максимальний елемент (причому вони єдині)\\
	\textbf{Proof.}\\
	$UpA = \{c \in \mathbb{R}: \forall a \in A: a \leq c \}$\\ За аксіомою відокремленості, $\exists c' \in \mathbb{R}: a \leq c' \leq c \Rightarrow c' \in UpA$\\
	$\forall c \in UpA: c' \leq c \Rightarrow c' = \min UpA$\\
	Доведемо єдиність:\\
	!Припустимо, що $\exists c'' = \min UpA$\\
	Але це автоматично не є можливо, оскільки якщо $c'' > c'$, то $c''$ не є більше мінімальним елементом, а якщо $c'' < c'$, то вже $c'$ не є мінімальним елементом. Суперечність!\\
	Для $DownB$ доведення аналогічне \qed
	\bigline
	
	\textbf{Definition 1.5.6.(1)} \textbf{Точковою верхньою межею} називають наступне число:
	\begin{align*}
	\sup A = \min UpA
	\end{align*}
	\textbf{Definition 1.5.6.(2)} \textbf{Точковою нижньою межею} називають наступне число:
	\begin{align*}
	\inf B = \max DownB
	\end{align*}
	
	\textbf{Theorem 1.5.7.(1) Критерій супремуму}\\
	$c' = \sup A \iff \begin{cases} 
	 \forall a \in A: a \leq c' \\
	 \forall \varepsilon > 0: \exists a_{\varepsilon} \in A: a_{\varepsilon} > c' - \varepsilon
	\end{cases}$\\
	\textbf{Proof.}\\
	\boxed{\Rightarrow} Дано: $c' = \sup A$\\
	Тоді автоматично $c' \in UpA$, тобто $\forall a \in A: a \leq c'$\\
	Оскільки це мінімальне значення, то\\ $\forall \varepsilon > 0: c' - \varepsilon \notin UpA \Rightarrow \exists a_{\varepsilon} \in A: a_{\varepsilon} > c' - \varepsilon$\\
	(остання умова - це заперечення означення обмеженості зверху)
	\\
	\\
	\boxed{\Leftarrow} Дано: система з двох умов\\
	З другої умови випливає, що не лише $c' \in UpA$, а ще й\\$c' = \min UpA = \sup A$ \qed
	\bigline
	\textbf{Theorem 1.5.7.(2) Критерій інфімуму}\\
	$d' = \inf B \iff \begin{cases} 
	 \forall b \in B: b \geq d'\\
	 \forall \varepsilon > 0: \exists b_{\varepsilon} \in B: b_{\varepsilon} < d' + \varepsilon
	\end{cases}$\\
	\textit{Доведення є аналогічним до критерію супремуму}
	\bigline
	\ex{1.5.8.} Повернемось до множини $A = \{1-2^{-n} | n \in \mathbb{N}\} = \left\{\dfrac{1}{2}, \dfrac{3}{4}, \dfrac{7}{8}, \dots \right\}$\\
	Доведемо, що $\sup A = 1$\\
	Дійсно, $\forall a \in A: a = 1 - \dfrac{1}{2^n} < 1$\\
	Залишилось довести, що $\forall \varepsilon > 0: \exists a_{\varepsilon}: a_{\varepsilon} > 1 - \varepsilon$\\
	Або $\exists n: 1 - 2^{-n} > 1 -\varepsilon$\\
	Або $1 - \dfrac{1}{2^n} > 1 - \dfrac{1}{n} > 1 - \varepsilon \Rightarrow \dfrac{1}{n} < \varepsilon \Rightarrow n > \dfrac{1}{\varepsilon}$\\
	Можна обрати $n = \left[ \dfrac{1}{\varepsilon} \right] + 1$, і тоді елемент з цим номером задовільнятиме умові
	\bigline
	\textbf{Definition 1.5.9.} Множина $F \subset \mathbb{R}$ називається \textbf{обмеженою}, якщо
	\begin{align*}
	\exists p>0: \forall f \in F: |f| \leq p
	\end{align*}
	Якщо $A$ не є обмеженою зверху, то вважаємо $\sup A = +\infty$\\
	Якщо $B$ не є обмеженою знизу, то вважаємо $\inf B = -\infty$\\
	
	
	
	\subsection{Принцип Архімеда та основні твердження мат. аналізу}
	\th{1.6.1. Принцип Архімеда} \\ 
	$\forall x \in \mathbb{R}: \forall h > 0: \exists! n \in \mathbb{Z}: nh \leq x < (n+1)h$\\
\proof
Задамо множину $M = \left\{k \in \mathbb{Z}: k \leq \dfrac{x}{h} \right\}$\\
Така множина є обмеженою, тоді $\exists \sup M = \max M = n$\\
$\Rightarrow n+1 \not \in M$\\
$\Rightarrow n \leq \dfrac{x}{h} < n+1$ \qed
\bigline
\crl{1.6.1.(1).} $\forall \varepsilon > 0: \exists n \in \mathbb{N}: \dfrac{1}{n} < \varepsilon$\\
\proof
Встановимо за принципом Архімеда $x = 1$, $h = \varepsilon$\\
Тоді $\exists! m = n-1 \in \mathbb{Z}: (n-1)\varepsilon \leq 1 < n \varepsilon$\\
$\Rightarrow \dfrac{1}{n} < \varepsilon$ \qed
\bigline
\crl{1.6.1.(2).} $\forall x \in \mathbb{R}: x \geq 0$. $\forall \varepsilon > 0: x < \varepsilon \Rightarrow x = 0$\\
\contra
Припустимо, що $x > 0$. Тоді за попереднім наслідком, $\exists n: \dfrac{1}{n} < x$\\
Розглянемо $\varepsilon = \dfrac{1}{n} > 0 \Rightarrow \varepsilon < x$\\
Суперечність! \qed
\bigline
\crl{1.6.1.(3).} $\forall a,b \in \mathbb{R}: a < b: \exists q \in \mathbb{Q}: a < q \leq b$\\
\proof
$a < b$, тоді розглянемо число $b-a>0 \Rightarrow \exists n: \dfrac{1}{n} < b-a$\\
В принципі Архімеда встановимо $x = a, h = \dfrac{1}{n}$\\
Тоді $\exists! m: \dfrac{m}{n} \leq a < \dfrac{m+1}{n}$\\
А тепер покажемо, що $q = \dfrac{m+1}{n} \leq b$\\
!Від супротивного, припускаємо, що $\dfrac{m+1}{n} > b$\\
Тоді $b-a \leq \dfrac{m+1}{n} - a \leq \dfrac{m+1}{n} - \dfrac{m}{n} = \dfrac{1}{n}$\\
Але ми мали, що $b-a > \dfrac{1}{n}$, суперечність!\\
Тож, $q \leq b$\\
Остаточно $a < q \leq b$ \qed
\bigline
	\defin{1.6.2.} \textbf{Цілою частиною числа} $x \in \mathbb{R}$ називають найближче менше ціле число $[x]$\\
	Для $h = 1: \exists! n \in \mathbb{Z}: n \leq x < n+1$\\
	$[x] = n$
	\bigline
	\\
	\th{1.6.3. Лема Кантора про вкладені відрізки}\\
	Задані відрізки наступним чином: $\forall n \geq 1: [a_n, b_n] \supset [a_{n+1}, b_{n+1}]$. Тоді\\
	1) $\exists c \in \mathbb{R}: \forall n \geq 1: c \in [a_n,b_n]$\\
	2) Якщо $\forall \varepsilon > 0: \exists n \in \mathbb{N}: b_n - a_n < \varepsilon$, то $\exists! c \in \mathbb{R}: \forall n \geq 1: c \in [a_n,b_n]$
	\begin{figure}[H]
	\centering
	\begin{tikzpicture}
	\draw[thick, ->] (0,0)--(6,0);
	\draw node at (0.5,0) {$[$}; \draw node at (5.5,0) {$]$};
	\draw node at (1.5,0) {$[$}; \draw node at (4.5,0) {$]$};
	\draw node at (2,0.5) {$\dots$}; \draw node at (4,0.5) {$\dots$};
	\draw node at (2.5,0) {$[$}; \draw node at (3.5,0) {$]$};
	
	\draw node at (0.5,-0.5) {$a_1$}; \draw node at (5.5,-0.5) {$b_1$};
	\draw node at (1.5,-0.5) {$a_2$}; \draw node at (4.5,-0.5) {$b_2$};
	\draw node at (2.5,-0.5) {$a_n$}; \draw node at (3.5,-0.5) {$b_n$};
	\end{tikzpicture}
	\end{figure}
	\proof
	1) Із умови випливає, що $\forall n,m \in \mathbb{N}:$\\
	$a_1 \leq a_2 \leq \dots \leq a_n \leq \dots < \dots \leq b_n \leq \dots \leq b_2 \leq b_1$\\
	Отже, $\forall n,m \in \mathbb{N}: a_n \leq b_m$\\
	Тому що:\\
	- $n < m: a_n \leq \dots \leq a_m < b_m \leq \dots \leq b_n$\\
	- $n > m: a_n < b_n \leq \dots \leq b_m$\\
	Розглянемо множини $A = \{a_1,\dots,a_n\}, B = \{b_1, \dots, b_m\}$\\
	Тоді за принципом Дедекінда, $\exists c \in \mathbb{R}: \forall n,m \in \mathbb{N}: a_n \leq c \leq b_m$\\
	Таким чином, $\forall n \geq 1: c \in [a_n,b_n]$
	\bigline
	2) Розглянемо окремо, коли $\forall \varepsilon > 0: \exists n: b_n - a_n < \varepsilon$\\
	!Припустимо, що $\exists c' \in \mathbb{R}: \forall n \geq 1: c' \in [a_n,b_n]$, але $c \neq c'$\\
	Задамо $\varepsilon = |c' - c| > 0$\\
	Тоді $\exists n: b_n - a_n < \varepsilon$, але $c,c' \in [a_n,b_n]$ для заданого $n$\\
	Тому $\varepsilon = |c'-c| < a_n-b_n < \varepsilon$ - суперечність!\\
	Отже, така точка є єдиною, причому\\
	$[a_1, b_1] \cap [a_2, b_2] \cap \dots = \{c\}$ \qed
	\bigline
	\th{1.6.4. Теорема Больцано-Вейєрштраса}\\
	Задана множина $A$ - обмежена множина з нескінченною кількістю елементів. Тоді вона містить принаймні одну граничну точку\\
	\proof
	Оскільки $A$ - обмежена, то:\\
	$\exists a \in \mathbb{R}: \forall x \in A: x \geq a$\\
	$\exists b \in \mathbb{R}: \forall x \in A: x \leq b$\\
	Тобто маємо множину $A \subset [a,b]$\\
	Розіб'ємо множину $[a,b]$ навпіл: $\left[a, \dfrac{a+b}{2}\right]$ та $\left[\dfrac{a+b}{2},b \right]$\\
	Оскільки $A$ має нескінченну кількість чисел, то принаймні одна з множин $\left[a, \dfrac{a+b}{2}\right] \cap A$ або $\left[\dfrac{a+b}{2}, b\right] \cap A$ - нескінченна множина. Ту половину позначимо за множину $[a_1,b_1]$ (якщо обидва нескінченні, то вибір довільний). Тоді $A \cap [a_1,b_1]$ - нескінченна множина\\
	Розіб'ємо множину $[a_1,b_1]$ навпіл: $\left[a_1, \dfrac{a_1+b_1}{2}\right]$ та $\left[\dfrac{a_1+b_1}{2},b_1 \right]$\\
	І за аналогічними міркуваннями одна з множин нескінченна, позначу за $[a_2,b_2]$. Тоді $A \cap [a_2,b_2]$ - нескінченна множина\\
	Розіб'ємо множину $[a_2,b_2]$ навпіл: $\left[a_2, \dfrac{a_2+b_2}{2}\right]$ та $\left[\dfrac{a_2+b_2}{2},b_2 \right]$\\
	$\dots$\\
	В результаті матимемо вкладені відрізки: $[a,b] \supset [a_1,b_1] \supset [a_2,b_2] \supset \dots$\\
	Причому $\forall n: b_n - a_n = \dfrac{b-a}{2^n}$\\
	Зафіксуємо $\varepsilon > 0$ та перевіримо, чи існує $n$, що $b_n - a_n < \varepsilon$\\
	Маємо: $\dfrac{b-a}{2^n} < \dfrac{b-a}{n} < \varepsilon \Rightarrow n > \dfrac{b-a}{\varepsilon}$\\
	Тоді за лемою Кантора, $\exists! c \in \mathbb{R}: \forall n \geq 1: c \in [a_n,b_n]$\\
	А далі покажемо, що $c$ - дійсно гранична точка множини $A$\\
	Зафіксуємо $\varepsilon > 0$. Знайдемо, чи існує $n$: $b_n - a_n = \dfrac{b-a}{2^n} < \dfrac{\varepsilon}{2} \Rightarrow \dots \Rightarrow n > \dfrac{2(b-a)}{\varepsilon}$\\
	Тоді $[a_n,b_n] \subset (c-\varepsilon, c+\varepsilon)$, оскільки $c-a_n \leq \dfrac{\varepsilon}{2}$ та $b_n -c \leq \dfrac{\varepsilon}{2}$\\
	І це все виконується $\forall \varepsilon > 0$\\
	Таким чином, $A \cap (c-\varepsilon, c+\varepsilon) \supset A \cap [a_n,b_n]$ - нескінченна множина, а отже, $c$ - гранична точка $A$ \qed
	\newpage
	
	
	\section{Границі числової послідовності}
	\subsection{Основні означення}
	\textbf{Definition 2.1.0. Числовою послідовністю} називають якийсь набір чисел \sequence{a_n}\\
	Тобто кожному номеру $n$ буде зіставлено якесь число $a_n$\\
	Можна її задати або довільним чином, або формулою, або рекурсивно за початковими умовами
	\bigline
	\textbf{Definition 2.1.1.} Число $a$ називається \textbf{границею числової послідовності} \sequence{a_n}, якщо справедливе таке твердження:
	\begin{align*}
	\forall \varepsilon > 0: \exists N(\varepsilon) \in \mathbb{N}: \forall n \geq N: |a_n - a| < \varepsilon
	\end{align*}
	Позначення: $\displaystyle \lim_{n \to \infty} a_n = a$ або $a_n \overset{n \to \infty}{\longrightarrow} a$\\
	Якщо в деякої послідовності існує чисельна границя, то така послідовність називається \textbf{збіжною}. В інакшому випадку - \textbf{розбіжна}
	\bigline
	\textbf{Theorem 2.1.2.} Для збіжної границі існує єдина границя\\
	\textbf{!Proof.}\\
	Нехай задана збіжна числова послідовність \sequence{a_n}, для якої існують дві границі:\\
	$\displaystyle \lim_{n \to \infty} a_n = a, \lim_{n \to \infty} a_n = b$\\
	Врахуємо, що $a<b$ (для $a>b$ міркування є аналогічними)\\
	Оскільки границі існують, ми можемо задати $\displaystyle \varepsilon = \frac{b-a}{3}$. Тоді\\
	$\displaystyle \exists N_1: \forall n \geq N_1: |a_n-a|< \frac{b-a}{3} \Rightarrow a_n < a + \frac{b-a}{3}$\\
	$\displaystyle \exists N_2: \forall n \geq N_2: |a_n-b|< \frac{b-a}{3} \Rightarrow a_n > b - \frac{b-a}{3}$\\
	Аби обидві нерівності працювали одночасно, ми зафіксуємо новий \\ $N= \max\{N_1,N_2\}$. Тоді:\\
	$\displaystyle \forall n \geq N: a_n < \frac{a+(a+b)}{3} < \frac{b+(a+b)}{3}<a_n$\\
	Отримали суперечність! Отже, обидва ліміти не існують одночасно \qed
	\bigline
	\textbf{Example 2.1.3!} Доведемо за означенням, що $\displaystyle\lim_{n \to \infty} \frac{1}{n} = 0$\\
	Задано довільне $\varepsilon > 0$. Необхідно знайти $\displaystyle N: \forall n \geq N: \left|\frac{1}{n}-0 \right|<\varepsilon$\\
	$\huge \abs{\frac{1}{n} - 0} < \varepsilon \iff \displaystyle \frac{1}{n}<\varepsilon \iff n > \frac{1}{\varepsilon}$\\
	Зафіксуймо $\displaystyle N = \left[\frac{1}{\varepsilon} \right] + 1$. Тоді маємо:\\
	$\forall \varepsilon > 0: \exists N = \huge \left[\frac{1}{\varepsilon} \right] + 1: \forall n \geq N: n > \frac{1}{\varepsilon} \Rightarrow \abs{\frac{1}{n} - 0} < \varepsilon$\\
	Отже, означення виконується, тому $\displaystyle\lim_{n \to \infty} \frac{1}{n} = 0$\\
\begin{figure}[H]
\centering
\resizebox{0.8\textwidth}{!} {
\begin{tikzpicture}
\draw[thick, ->] (-2,0)--(17,0) node[anchor = north west] {$a_n = \dfrac{1}{n}$};
\foreach \i [evaluate=\i as \x using int(16/ \i)] in {16,8,4,2,1}
	\filldraw (\i,0) circle (1pt) node[anchor = north] {$\dfrac{1}{\x}$};
\draw (0,-1pt)--(0,1pt) node[anchor = north] {$0$};
\node[red] at (1.6,0) {$)$};
\node[red] at (-1.6,0) {$($};
\node[anchor = south, red] at (1.6,0) {$0+\varepsilon$};
\node[anchor = south, red] at (-1.6,0) {$0-\varepsilon$};
\end{tikzpicture}
}
\caption*{Тут на малюнку я обрав $\varepsilon = 0.1$. Тоді починаючи з $n=11$ (або з $12$, $13$,...), всі решта члени не покидатимуть червоні лінії. Якщо члени не будуть покидати ці лінії для будь-якого заданого $\varepsilon$, то тоді границя існує \\}
\end{figure}
	\hypertarget{ex2.1.4.}{\textbf{Example 2.1.4!}} Доведемо за означенням, що $\displaystyle\lim_{n \to \infty} \sqrt[n]{n}=1$\\
	Знову задамо довільне $\varepsilon > 0$. Знову необхідно знайти $\displaystyle N: \forall n \geq N: \left|\sqrt[n]{n}-1  \right|<\varepsilon \iff \sqrt[n]{n}<1+\varepsilon$\\
	Використовуючи нерівність Коші, ми отримаємо таку оцінку:\\
	$\displaystyle \sqrt[n]{n}= \sqrt[n]{\sqrt{n}\cdot\sqrt{n}\cdot 1 \cdots 1} \leq \frac{\sqrt{n}+\sqrt{n}+1+\cdots+1}{n} = \frac{2\sqrt{n}+n-2}{n} = \\ = \frac{2}{\sqrt{n}}+1-\frac{2}{n}<\frac{2}{\sqrt{n}}+1$. Тоді:\\
	$\displaystyle \sqrt[n]{n} < \frac{2}{\sqrt{n}} + 1 < 1 + \varepsilon \iff \frac{2}{\sqrt{n}} < \varepsilon \iff n > \frac{4}{\varepsilon^2}$\\
	Тепер зафіксуємо $\displaystyle N = \left[\frac{4}{\varepsilon^2} \right] + 2021$. Ну тоді $\forall n \geq N$ всі нерівності виконуються, зокрема $\left|\sqrt[n]{n}-1  \right|<\varepsilon$\\
	Остаточно, $\displaystyle\lim_{n \to \infty} \sqrt[n]{n}=1$
	\bigline
	\textbf{Example 2.1.5!} Доведемо за означенням, що $\displaystyle\lim_{n \to \infty} \frac{n^k}{b^n} = 0$, $b>1$\\
	Вже було доведено в \hyperlink{ex2.1.4.}{\textbf{Ex. 2.1.4.}}, що $\displaystyle\lim_{n \to \infty} \sqrt[n]{n}=1$, а тому означення працює:
	\limitdef{\varepsilon'}{N_0}{n}{\sqrt[n]{n}}{1} $\iff \sqrt[n]{n}<1+\varepsilon'$\\
	Оскільки границя існує, ми оберемо $\displaystyle \varepsilon' = \sqrt[2k]{b}-1$. Тоді:\\
	$\displaystyle \sqrt[n]{n} < 1 + \sqrt[2k]{b}-1 \iff n^k < b^{\frac{n}{2}}$\\
	Отже, ми отримали, що $\forall n \geq N_0: n^k < b^{\frac{n}{2}}$. Дану оцінку використаємо для доведення бажаного ліміту\\
	\\
	Зафіксуємо інше $\varepsilon>0$. Хочемо знайти $N_1: \forall n \geq N_1: \displaystyle \abs{\frac{1}{b^{\frac{n}{2}}}} < \varepsilon$\\
	$\displaystyle \iff \cdots \iff n>2 \log_{b} \frac{1}{\varepsilon}$. Тоді $\displaystyle N_1 = \left[\log_{b} \frac{1}{\varepsilon} \right] + 2^2$\\
	Нарешті, якщо зафіксувати $N=\max\{N_0, N_1\}$, то $\forall n \geq N$ справедлива оцінка:\\
	$\displaystyle \abs{\frac{n^k}{b^n}} < \abs{\frac{1}{b^{\frac{n}{2}}}} < \varepsilon'$\\
	\\
	Остаточно, \limitdef{\varepsilon}{N}{n}{\frac{n^k}{b^n}}{0} $\overset{\textrm{def.}}{\iff} \displaystyle\lim_{n \to \infty} \frac{n^k}{b^n} = 0$, $b>1$
	\bigline
	\hypertarget{ex2.1.6.}{\textbf{Example 2.1.6.}} Доведемо, що не існує $\displaystyle \lim_{n \to \infty} (-1)^n$\\
	Припускаємо, що даний ліміт збіжний, тобто $\displaystyle \lim_{n \to \infty} (-1)^n = a$, тобто\\
	\limitdef{\varepsilon}{N}{n}{(-1)^n}{a}. Тоді\\
	$\displaystyle 2=\abs{(-1)^n-(-1)^{n+1}}=\abs{(-1)^n-a+a-(-1)^{n+1}} \leq \abs{(-1)^n-a}+\abs{a-(-1^{n+1})} < 2\varepsilon \Rightarrow \varepsilon > 1$\\
	Прийшли до суперечності. Тому даний ліміт існувати не може\\
\begin{figure}[H]
\centering
\resizebox{0.5\textwidth}{!} {
\begin{tikzpicture}

\draw[thick, ->] (-2,0)--(2,0) node[anchor = north west] {$a_n = (-1)^n$};
\filldraw (1,0) circle (1pt) node[anchor = north] {$1$};
\filldraw (-1,0) circle (1pt) node[anchor = north] {$-1$};
\node[anchor = south, red] at (1+0.8,0) {$1+\varepsilon$};
\node[anchor = south, red] at (1-0.8,0) {$1-\varepsilon$};
\node[red] at (1+0.5,0) {$)$};
\node[red] at (1-0.5,0) {$($};
\end{tikzpicture}
}
\caption*{Тут на малюнку я встановил границю $a=1$. Лише для деяких $\varepsilon$ всі члени потраплятимуть всередину. Однак, скажімо, не для $\varepsilon = 0.5$ як на малюнку, ось чому ліміт не може бути рівним $1$. І так для кожного $a$}
\end{figure}

	\textbf{Definition 2.1.7.} Послідовність \sequence{a_n} називається \textbf{обмеженою}, якщо \begin{align*}
	\exists C>0: \forall n \geq 1: |a_n|\leq C
	\end{align*}
	\textbf{Theorem 2.1.8.} Будь-яка збіжна послідовність є обмеженою\\
	\textbf{Proof.}\\
	Нехай задана збіжна послідовність \sequence{a_n}, тобто для неї\\ $\displaystyle \exists \lim_{n \to \infty} a_n = a \overset{\textrm{def.}}{\iff}$ \limitdef{\varepsilon}{N}{n}{a_n}{a}\\
	Оскільки ліміт існує, то задамо $\varepsilon = 1$. Тоді: $\forall n \geq N: \abs{a_n-a}<1$
	Спробуємо оцінити вираз $|a_n|$ для нашого бажаного:\\
	$|a_n| = |a_n - a + a| \leq |a_n-a|+|a| < 1 + |a|$. Це виконується $\forall n \geq N$. Інакше кажучи, всі числа, починаючи з $N$, є обмеженими.\\
	Покладемо $C=\max\{|a_1|,|a_2|,\cdots, |a_{N-1}|, 1+|a|\}$. Тоді отримаємо, що\\
	$\forall n\geq1: |a_n|\leq C$, що й позначає обмеженість \qed
	\\ \\
	\textbf{Remark 2.1.8.} Обернене твердеження не є вірним\\
	В \hyperlink{ex2.1.4.}{\textbf{Ex. 2.1.4.}} послідовність \sequence{(-1)^n} є обмеженою, але не збіжна
	\bigline
	\textbf{Definition 2.1.9.} Посідовність \sequence{a_n} \textbf{має границю} $\infty$, якщо виконується твердження: \begin{align*}
	\forall E>0: \exists N(E) \in \mathbb{N}: \forall n \geq N: |a_n|>E
	\end{align*}
	Якщо $+\infty$, то $a_n > E$ \hspace{0.5cm} Якщо $-\infty$, то $-a_n > E$
	\bigline
	\textbf{Example 2.1.10.} Доведемо за означенням, що $\displaystyle\lim_{n \to \infty} 2^n = +\infty$\\
	Задано довільне $E>0$. Необхідно знайти $N: \forall n \geq N: 2^n>E$\\
	Думаю, зрозуміло, що $2^n > n$. Вимагатимемо тепер, щоб $n > E$\\
	Фіксуймо $N=\left[ E \right] + 2$. Тоді $\forall n \geq N: n > E$, а тим паче\\
	$2^n > n > E$\\
	Тому $\displaystyle\lim_{n \to \infty} 2^n = +\infty$\\
	\begin{figure}[H]
\centering
\resizebox{0.9\textwidth}{!} {
\begin{tikzpicture}
\draw[thick, ->] (-2,0)--(17,0) node[anchor = north west] {$a_n = 2^n$};
\foreach \i in {16,8,4,2}
	\filldraw (\i,0) circle (1pt) node[anchor = north] {$\i$};
\node[red] at (6,0) {$|$};
\node[anchor = south, red] at (6,0) {$E$};
\end{tikzpicture}
}
\caption{Тут на малюнку $E = 6$. Тоді починаючи з $n=3$ (або з $4$, $5$,...), всі решта члени будуть правіше за червону лінію}
\end{figure}
\ex{2.1.11.} Доведемо, що $\huge \lim_{n \to \infty} (-1)^n n = \infty$\\
Задано довільне $E > 0$. Необхідно знайти $N: \forall n \geq N: \\ |(-1)^n n| = n > E$.
Але це ми вже доводили зверху
\bigline
\rm{2.1.11.} Тут не можна визначитись, чи $+\infty$, чи $-\infty$ через знакочередованість\\
Дійсно, $\huge \lim_{n \to \infty} (-1)^n n \neq +\infty$, оскільки\\
$\exists E = 10: \forall N: \exists n = 2N-1 \geq N: (-1)^n n = (-1)^{2N-1} (2N-1) = 1-2N \leq E$\\
Аналогічно, $\huge \lim_{n \to \infty} (-1)^n n \neq -\infty$
	
	
	\subsection{Нескінченно малі/великі послідовності}
	\textbf{Definition 2.2.1.(1).} Якщо послідовність \sequence{a_n} містить границю $\displaystyle\lim_{n \to \infty} a_n = 0$, то така послідовність називається \textbf{нескінченно малою (н.м.)}\\
	\textbf{Definition 2.2.1.(2).} Якщо послідовність \sequence{a_n} містить границю $\displaystyle\lim_{n \to \infty} a_n = \infty$, то така послідовність називається \textbf{нескінченно великою (н.в.)}
	\bigline
	\textbf{Example 2.2.2.} Зокрема \sequence{\frac{1}{n}} є нескінченно малою, а \sequence{2^n} є нескінченно великою, виходячи з минулих прикладів
	\bigline
	\textbf{Theorem 2.2.3. Арифметика н.м. та н.в.}\\
	Задані такі послідовності:\\
	1. \sequence{a_n} - н.м.;\\
	2. \sequence{b_n} - н.м.;\\
	3. \sequence{c_n} - обмежена;\\
	4. \sequence{d_n} - н.в.;\\
	5. \sequence{p_n} - послідовність, що віддалена від 0 \\ ($\exists \delta>0: \forall n\geq 1: |p_n|\geq \delta$)\\
	Тоді наступні послідовності:\\
	1) \sequence{a_n+b_n} - н.м.\\
	2) \sequence{C \cdot a_n} - н.м.\\
	3) \sequence{c_n \cdot a_n} - н.м.\\
	4) \sequence{\frac{1}{a_n}} - н.в.\\
	5) \sequence{\frac{1}{d_n}} - н.м.\\
	6) \sequence{p_n \cdot d_n} - н.в.\\
	\textbf{Proof.}\\
	1) $\displaystyle \lim_{n \to \infty} a_n = 0, \lim_{n \to \infty} b_n = 0 \overset{\textrm{def.}}{\iff}$\\
	\limitdef{\varepsilon}{N_1}{n}{a_n}{0} $\displaystyle \Rightarrow |a_n|<\frac{\varepsilon}{2}$\\
	\limitdef{\varepsilon}{N_2}{n}{b_n}{0} $\displaystyle \Rightarrow |b_n|<\frac{\varepsilon}{2}$\\
	Нехай існує $N=\max\{N_1,N_2\}$. Тоді $\forall n \geq N:$\\
	$|a_n+b_n| \leq |a_n|+|b_n| < \varepsilon$\\
	Отже, \sequence{a_n+b_n} - н.м.
	\bigline
	3) $\displaystyle \lim_{n \to \infty} a_n = 0 \overset{\textrm{def.}}{\iff}$ \limitdef{\varepsilon}{N}{n}{a_n}{0} $\Rightarrow \displaystyle |a_n| < \frac{\varepsilon}{M}$\\ 
	Також \bounded{M}{n}{c_n}\\
	Тоді $\forall n \geq N: |a_n \cdot c_n| = |a_n| \cdot |c_n| < \varepsilon$\\
	Отже, \sequence{a_n \cdot c_n} - н.м.
	\bigline
	4) $\displaystyle \lim_{n \to \infty} a_n = 0 \overset{\textrm{def.}}{\iff}$ \limitdef{\varepsilon}{N}{n}{a_n}{0} $\Rightarrow \displaystyle |a_n| < \varepsilon$\\
	Зафіксуємо $\displaystyle \varepsilon = \frac{1}{E}$. Тоді $\displaystyle \forall n \geq N: |a_n|<\frac{1}{E} \iff \abs{\frac{1}{a_n}} > E$\\
	Отже, \sequence{\frac{1}{a_n}} - н.в.\\
	2), 6) доводиться як 3). 5) доводиться аналогічно як 4) \qed
	\bigline
	\textbf{Theorem 2.2.4. Про характеризацію збіжної послідовності}\\
	Послідовність \sequence{a_n} є збіжною $\iff a_n=a+\alpha_n$, \\ де \sequence{\alpha_n} - н.м.\\
	\textbf{Proof.}\\
	\boxed{\Rightarrow} Дано: \sequence{a_n} - збіжна, тобто\\
	\limitdef{\varepsilon}{N}{n}{a_n}{a}\\
	Позначимо $a_n-a=\alpha_n$. Тоді $a_n=a+\alpha_n$ та послідовність \sequence{\alpha_n} - н.м., оскільки $|\alpha_n - 0| = |\alpha_n| = |a_n - a| < \varepsilon$
	\bigline
	\boxed{\Leftarrow} Дано: \sequence{\alpha_n} - н.м., де $a_n = a + \alpha_n$. Тоді\\
	\limitdef{\varepsilon}{N}{n}{\alpha_n}{0} $\Rightarrow |a_n-a|<\varepsilon$\\
	Отже, \sequence{a_n} - збіжна \qed
	\bigline
	\textbf{Theorem 2.2.5. Арифметика границь}\\
	Задані \sequence{a_n}, \sequence{b_n} та $\exists \huge \lim_{n \to \infty} a_n = a, \exists \huge \lim_{n \to \infty} b_n = b$ Тоді:\\
	1) $\displaystyle \exists \lim_{n \to \infty} (a_n+b_n) = \lim_{n \to \infty} a_n+\lim_{n \to \infty} b_n$\\
	2) $\displaystyle \forall C \in \mathbb{R}: \exists \lim_{n \to \infty} C \cdot a_n = C \lim_{n \to \infty} a_n$\\
	3) $\displaystyle \exists \lim_{n \to \infty} (a_n \cdot b_n) = \lim_{n \to \infty} a_n \cdot \lim_{n \to \infty} b_n$\\
	4) $\displaystyle \exists \lim_{n \to \infty} \frac{a_n}{b_n} = \frac{\displaystyle \lim_{n \to \infty} a_n}{\displaystyle \lim_{n \to \infty} b_n}_{\neq 0}$\\
	\textbf{Proof.}\\
	Обидва послідовності збіжні $\iff$ $a_n = a + \alpha_n$, \sequence{\alpha_n} - н.м., а також $b_n=b+\beta_n$, \sequence{\beta_n} - н.м.\\
	Тоді:\\
	1) $a_n+b_n=a+\alpha_n+b+\beta_n=(a+b)+(\alpha_n+\beta_n)$, причому \sequence{\alpha_n + \beta_n} - н.м. $\iff$ послідовність \sequence{a_n+b_n} має границю:\\ $\displaystyle \lim_{n \to \infty} (a_n+b_n) = a+b = \lim_{n \to \infty} a_n+\lim_{n \to \infty} b_n$\\ \\
	2) довести самостійно\\ \\
	3) $a_n b_n - ab = (a+\alpha_n)(b+\beta_n) - ab = \alpha_n b + \alpha_n \beta_n + a \beta_n = \gamma_n$, причому послідовність \sequence{\gamma_n} - н.м. $\iff$ послідовність \sequence{a_n b_n} має границю:\\
	$\displaystyle \lim_{n \to \infty} (a_n \cdot b_n) = ab = \lim_{n \to \infty} a_n \cdot \lim_{n \to \infty} b_n$\\ \\
	4) В принципі, це є наслідком 3), якщо представити послідовність \\ $\dfrac{a_n}{b_n} = a_n \cdot \dfrac{1}{b_n}$\\
	Треба лишень довести, що $\dfrac{1}{b_n} \to \dfrac{1}{b}, n \to \infty$\\
	Відомо, що $b_n \to b \iff \forall \varepsilon > 0: \exists N': \forall n \geq N: |b_n-b| < \varepsilon$\\
	Зафіксую $\varepsilon = \dfrac{|b|}{2}$, тоді $\exists N: \forall b \geq N'': \forall n \geq N'': \\ |b| = |b - b_n + b_n| \leq |b - b_n| + |b_n| < \dfrac{|b|}{2} + |b_n| \Rightarrow |b_n| > \dfrac{|b|}{2}$\\
	Я хочу одночасно $|b_n| > \dfrac{|b|}{2}$ та $|b_n - b| < \varepsilon$, тож нехай $N = \max \{N', N'' \}$, тоді\\
	$\forall n \geq N: \abs{\dfrac{1}{b_n} - \dfrac{1}{b}} = \dfrac{|b_n -b|}{|b_n| |b|} < \dfrac{\varepsilon}{ \dfrac{|b|}{2} |b|} = \dfrac{2}{|b|^2} \varepsilon$\\
	Таким чином, можна твердити, що $\dfrac{1}{b_n} \to \dfrac{1}{b}, n \to \infty \Rightarrow \dfrac{a_n}{b_n} \to \dfrac{a}{b}$ \qed
	\bigline
	\textbf{Example 2.2.6.} Знайти границю $\displaystyle\lim_{n \to \infty} \frac{2n^2-3n+5}{1-n-3n^2}$\\
	$\displaystyle\lim_{n \to \infty} \frac{2n^2-3n+5}{1-n-3n^2} =\lim_{n \to \infty} \frac{\displaystyle 2-\frac{3}{n}+\frac{5}{n^2}}{\displaystyle \frac{1}{n^2}-\frac{1}{n}-3} = \frac{\displaystyle \lim_{n \to \infty} 2-\frac{3}{n}+\frac{5}{n^2}}{\displaystyle \lim_{n \to \infty} \frac{1}{n^2}-\frac{1}{n}-3} =\\ = \frac{\displaystyle \lim_{n \to \infty} 2- \lim_{n \to \infty} \frac{3}{n}+ \lim_{n \to \infty} \frac{5}{n^2}}{\displaystyle \lim_{n \to \infty} \frac{1}{n^2}- \lim_{n \to \infty}\frac{1}{n}- \lim_{n \to \infty} 3} = \frac{2-0+5}{0-0-3} = -\frac{2}{3}$\bigline
	\rm{2.2.6.} Більш детально, чому рівності спрацьовують:\\
	Оскільки існують ліміти в четвертому дробі, то існують ліміти в третьому дробі (як сума), то тоді існує ліміт в другому дробі (як частка)\\
	
	
	\subsection{Нерівності в границях}	
	\textbf{Theorem 2.3.1.} Задані дві збіжні числові послідовності \sequence{a_n}, \sequence{b_n} таким чином, що $\exists N': \forall n \geq N': a_n \leq b_n$. Тоді\\ $\displaystyle \lim_{n \to \infty} a_n \leq \lim_{n \to \infty} b_n$\\
	\textbf{!Proof.}\\
	Задані дві збіжні послідовності, для яких $\displaystyle \lim_{n \to \infty} a_n =a, \lim_{n \to \infty} b_n = b$\\
	Припустимо, що $a>b$ та розглянемо $\displaystyle \varepsilon = \frac{a-b}{2}$. Тоді за означенням границі,\\
	$\exists N_1: \forall n \geq N_1:|a_n-a|<\varepsilon \Rightarrow a_n>a-\varepsilon$\\
	$\exists N_2: \forall n \geq N_2:|b_n-b|<\varepsilon \Rightarrow b_n<b+\varepsilon$\\
	Задамо $N=\max \{N_1, N_2\}$. Тоді\\ $\displaystyle b_n < b+\varepsilon = b + \frac{a-b}{2}=\frac{a+b}{2}=a-\frac{a-b}{2} = a-\varepsilon<a_n \Rightarrow b_n<a_n$. Суперечність! \qed
	\bigline
	\textbf{Corollary 2.3.1.} Задана збіжна числова послідовность \sequence{b_n} таким чином, що $\exists N': \forall n \geq N': a \leq b_n$. Тоді $\displaystyle a \leq \lim_{n \to \infty} b_n$\\
	\textit{Вказівка: розглянути послідовність \sequence{a_n=a} - так звана стаціонарна послідовність}\\
	\bigline
	\textbf{Remark 2.3.1.} Для інших нерівності $\geq$ аналогічно все. А також ця теорема спрацьовує для $<$ або $>$, проте нерівність з границями залишається нестрогою
	\bigline
	\textbf{Theorem 2.3.2. Теорема про 3 послідовності}\\
	Задані три послідовності: \sequence{a_n},\sequence{b_n},\sequence{c_n} таким чином, що $\displaystyle \lim_{n \to \infty} a_n = \displaystyle \lim_{n \to \infty} b_n = a$. Більш того, $\exists N': \forall n \geq N': a_n \leq c_n \leq b_n$.\\
	Тоді $\exists \displaystyle \lim_{n \to \infty} c_n = a$\\
	\textbf{Proof.}\\
	$\displaystyle \lim_{n \to \infty} a_n = a \overset{\textrm{def.}}{\iff}$ \limitdef{\varepsilon}{N_1}{n}{a_n}{a} $\Rightarrow a_n > a - \varepsilon$\\
	$\displaystyle \lim_{n \to \infty} b_n = a \overset{\textrm{def.}}{\iff}$ \limitdef{\varepsilon}{N_2}{n}{b_n}{a} $\Rightarrow b_n<a+\varepsilon$\\
	Зафіксуємо $N=\max\{N_1, N_2, N'\}$. Тоді $\forall n \geq N:$\\
	$a-\varepsilon< a_n \leq c_n \leq b_n < a+\varepsilon \Rightarrow |c_n - a|<\varepsilon$\\
	Отже, $\displaystyle \lim_{n \to \infty} c_n = a$ \qed
	\bigline
	\textbf{Example 2.3.3.} Знайти границю $\displaystyle \lim_{n \to \infty} \sqrt[n]{2^n+7^n}$\\
	Можна отримати наступну оцінку:\\
	$\sqrt[n]{7^n} \leq \sqrt[n]{2^n+7^n} \leq \sqrt[n]{n\cdot 7^n}$\\
	Ця нерівність виконується завжди, починаючи з якогось номера $n$. Рахуємо ліміти з обох сторін\\
	$\displaystyle \lim_{n \to \infty} \sqrt[n]{7^n} = 7$\\
	$\displaystyle \lim_{n \to \infty} \sqrt[n]{n \cdot 7^n} = 7 \lim_{n \to \infty} \sqrt[n]{n} = 7$\\
	Тому з цього випливає, що $\displaystyle \lim_{n \to \infty} \sqrt[n]{2^n+7^n} = 7$\\
	
	\subsection{Монотонні послідовності. Число $e$}
	\textbf{Definition 2.4.1.} Послідовність \sequence{a_n} називається:\\
	- \textbf{строго монотонно зростаючою}, якщо $\forall n \geq 1: a_{n+1} > a_n$\\
	- \textbf{монотонно не спадною}, якщо $\forall n \geq 1: a_{n+1} \geq a_n$\\
	- \textbf{строго монотонно спадною}, якщо $\forall n \geq 1: a_{n+1} < a_n$\\
	- \textbf{монотонно не зростаючою}, якщо $\forall n \geq 1: a_{n+1} \leq a_n$\\
	\\
	\textbf{Example 2.4.2.} Дослідимо послідовність \sequence{a_n = \sqrt{n}} на монотонність\\
	$\displaystyle a_{n+1} - a_n = \sqrt{n+1} - \sqrt{n} = \frac{n+1-n}{\sqrt{n+1} + \sqrt{n}} = \frac{1}{\sqrt{n+1} + \sqrt{n}} > 0$\\
	$\Rightarrow a_{n+1}>a_n$, тобто дана послідовність зростає\\
	\bigline
	\textbf{Theorem 2.4.3. Теорема Вейєрштрасса}\\
	Будь-яка обмежена та монотонна (принаймні починаючи з якогось номера) послідовність є збіжною\\
	\textbf{Proof.}\\
	Нехай задана послідовність \sequence{a_n}, що задовільняє умові теореми. Нехай вона монотонно не спадає\\
	Оскільки вона монотонна, а ще - обмежена, то $\exists \sup\{a_n\} = a < +\infty$.\\
	За критерієм sup: \\
	$\forall n \geq 1: a_n \leq a$\\
	$\forall \varepsilon > 0: \exists N(\varepsilon): a_{N} > a - \varepsilon$\\
	Отримаємо наступний ланцюг нерівностей: $\forall n \geq N:$\\
	$a-\varepsilon < a_N \leq a_n \leq a < a + \varepsilon \Rightarrow |a_n-a|<\varepsilon$\\
	Отже, $\displaystyle \exists \lim_{n \to \infty} a_n = \sup\{a_n\}$\\
	Для інших випадків монотонності все аналогічно \qed \\
	\bigline
	\textbf{Example 2.4.4.} Довести, що для послідовності \sequence{a_n = \frac{2000^n}{n!}} існує границя\\
	Перевіримо на монотонність:\\
	$\displaystyle \frac{a_{n+1}}{a_n} = \frac{2000^{n+1} n!}{(n+1)! 2000^n} = \frac{2000}{n+1}$\\
	Отримаємо, що $a_{n+1} < a_n$ принаймні $\forall n \geq 2000$\\
	Послідовність обмежена принаймні знизу, тобто $a_n > 0$. Тоді для цієї послідовності існує ліміт:\\
	$\displaystyle a = \lim_{n \to \infty} a_n$. Тоді також $\displaystyle a = \lim_{n \to \infty} a_{n+1}$\\
	$\displaystyle a = \lim_{n \to \infty} a_{n+1} = \lim_{n \to \infty} \frac{2000}{n+1} a_n = 0$. Отже, $\displaystyle \lim_{n \to \infty} a_n = 0$ \bigline
	\textbf{Remark 2.4.4.} Такими самими міркуваннями можна довести, що \\ $\displaystyle \frac{n^k}{b^n}, \frac{b^n}{n!}, \frac{n!}{n^n} \to 0$, якщо $n \to \infty$\\
	\bigline
	\ex{2.4.5.} Дізнатись, який вираз більший при надто великих $n$\\
	$2^n$ або $n^{1000}$\\
	Відомо, що $\huge \lim_{n \to \infty} \dfrac{n^{1000}}{2^n} = 0$\\
	Якщо зафіксую $\varepsilon = 1$, то $\exists N: \forall n \geq N: \dfrac{n^{1000}}{2^n} < 1$\\
	Значить, $2^n > n^{1000}$
	\bigline
	Розглянемо послідовність \sequence{a_n = \left(1+\frac{1}{n} \right)^n}. Спробуємо для неї знайти границю\\
	1. Покажемо, що вона є монотонно зростаючою\\
	$\displaystyle \frac{a_{n+1}}{a_n} = \frac{\displaystyle \left(1+\frac{1}{n+1} \right)^{n+1}}{\displaystyle \left(1+\frac{1}{n} \right)^n} = \left(1 + \frac{1}{n+1} \right) \left( \frac{\displaystyle 1+\frac{1}{n+1}}{\displaystyle 1 + \frac{1}{n}} \right)^n = \frac{n+2}{n+1} \cdot \left( \frac{n(n+2)}{(n+1)^2} \right)^n = \\ = \frac{n+2}{n+1} \cdot \left(1 - \frac{1}{(n+1)^2} \right)^n = \frac{\displaystyle \frac{n+2}{n+1}}{\displaystyle 1-\frac{1}{(n+1)^2}} \cdot \left( 1 - \frac{1}{(n+1)^2} \right)^{n+1} = \\ = \frac{n+2}{n+1} \cdot \frac{(n+1)^2}{n^2+2n} \cdot \left( 1 - \frac{1}{(n+1)^2} \right)^{n+1} \boxed{\geq}$\\
	Тут ми маємо права на третю дужку використати нерівність Бернуллі, оскільки $\displaystyle - \frac{1}{(n+1)^2} > -1$\\
	$\displaystyle \boxed{\geq} \frac{n+2}{n+1} \cdot \frac{(n+1)^2}{n^2+2n} \cdot \left(1 - \frac{n+1}{(n+1)^2} \right) = \frac{n+1}{n} \left(1-\frac{1}{n+1} \right) = 1$\\
	Коротше, $\displaystyle \frac{a_{n+1}}{a_n} \geq 1 \Rightarrow a_{n+1} \geq a_n$. Тобто наша послідовність монотонно зростає\\
	\\
	2. Доведемо, що вона є обмеженою\\
	Для цього требя розглянути \sequence{b_n = \left(1+\frac{1}{n} \right)^{n+1}} і довести, що:\\
	а) $a_n < b_n \forall n \geq 1$\\
	b) вона є монотонно спадною\\
	\\
	a) Перший пункт очевидний, оскільки $\displaystyle \left(1+\frac{1}{n} \right)^n < \left(1+\frac{1}{n} \right)^{n+1}$ через однакову основу степені, що є більше одинички\\
	b) А це розпишу:\\
	$\displaystyle \frac{b_{n-1}}{b_n} = \frac{\displaystyle \left(1+\frac{1}{n-1} \right)^{n}}{\displaystyle \left(1+\frac{1}{n} \right)^{n+1}} = \frac{1}{\displaystyle \left(1+\frac{1}{n}\right)} \cdot \left(\frac{n^2}{n^2-1}\right)^n = \frac{n}{n+1} \cdot \left(1+\frac{1}{n^2-1} \right)^n \boxed{\geq}$\\
	За аналогічними причинами я можу користатися нерівностю Бернуллі для другої дужки\\
	$\displaystyle \boxed{\geq} \frac{n}{n+1} \left(1+\frac{n}{n^2-1}\right) = \frac{n}{n+1} + \frac{n^2}{(n+1)(n^2-1)} = \frac{n^3+n^2-n}{n^3+n^2-n-1} > 1$\\
	Коротше, $\displaystyle \frac{b_{n-1}}{b_n} > 1 \Rightarrow b_n < b_{n-1}$. Тобто ця послідовність монотонно спадає\\
	В результаті всього можемо отримати наступну обмеженність:\\
	$2=a_1 \leq a_2 \leq \cdots \leq a_n < b_n \leq \cdots \leq b_2 \leq b_1 = 4$. Обмежена\\
	А це означає, що для послідовності \sequence{a_n = \left(1+\frac{1}{n}\right)^n} існує границя:\\
	\textbf{Theorem. 2.4.6.}
	$\huge \lim_{n \to \infty}\left(1+\frac{1}{n} \right)^n = e \approx 2.71...$
	\bigline
	До речі, оскільки $\{a_n \}$ зростає, а $\{b_n \}$ спадає та обидва обмежені, то:\\
	$\forall n \geq 1: a_n<e<b_n$\\
	$\displaystyle \left(1+\frac{1}{n} \right)^n < e < \left(1+\frac{1}{n} \right)^{n+1}$\\
	Зробимо нове позначення: $\log_{e} a =\ln a$. Тоді:\\
	$\displaystyle n \ln \left(1+\frac{1}{n} \right) < 1 < (n+1) \ln \left(1+\frac{1}{n} \right)$\\
	В результаті ми можемо отримати одну оцінку:
	\begin{align*}
	 \frac{1}{1+n} < \ln (1+\frac{1}{n}) < \frac{1}{n}
	\end{align*}\\
	
	\subsection{Підпослідовності}
	\textbf{Definition 2.5.1.} Задана послідовність \sequence{a_n}\\
	Послідовність \subsequence{a_{n_k}} називається \textbf{підпослідовністю}
	\bigline
	\textbf{Definition 2.5.2. Послідовністю натуральних чисел} називають строго зростаючу послідовність \subsequence{n_k} $\subset \mathbb{N}$
	\bigline
	\ex{2.5.3.} $\{n_k = 2k, k \geq 1 \} = \{2,4,6,8,\dots \} \subset \mathbb{N}$
	\bigline
	\textbf{Proposition 2.5.3.} Якщо для послідовності \sequence{a_n} $\displaystyle \exists \lim_{n \to \infty} a_n = a, \textrm{то} \\ \exists \lim_{k \to \infty} a_{n_k} = a$\\
	\textbf{Proof.}\\
	$\displaystyle \exists \lim_{n \to \infty} a_n = a \iff$ \limitdef{\varepsilon}{N}{n}{a_n}{a}\\
	Візьмемо підпослідовність $\{a_{n_k}, k \geq 1\}$. Оскільки послідовність \\ $\{n_k, k \geq 1\}$ - строга зростаюча послідовність натуральних чисел, то $\exists \huge \lim_{k \to \infty} n_k = +\infty$\\
	Тоді для $E = N(\varepsilon): \exists K(\varepsilon): \forall k \geq K: n_k > N$\\
	Зокрема оскільки $n_k > N$, то одразу $|a_{n_k} - a| < \varepsilon$\\
	$\Rightarrow \displaystyle \lim_{k \to \infty} a_{n_k} = a$ \qed
	\bigline
	\textbf{Theorem 2.5.4. Теорема Больцано-Вейєрштрасса}\\
	Для будь-якої обмеженої послідовності існує збіжна підпослідовність\\
	\textbf{Proof.}\\
	Розглянемо послідовність \sequence{a_n}. Існують 2 випадки:\\
	1. Послідовність - скінченна (наприклад, як в \hyperlink{ex2.1.6.}{\textbf{Ex. 2.1.6.}}) Тоді одне із значень послідовності буде прийматись нескінченну кількість разів. Отримаємо стаціонарну підпослідовність, яка є збіжною\\
	\\
	2. Послідовність - нескінченна (наприклад, як \hyperlink{ex2.6.6.}{\textbf{Ex. 2.5.6.}})\\
	Оскільки вона є обмеженою, то за іншою теоремою Больцано-Вейерштрасса, в неї існує гранична точка $b_* \iff \forall \varepsilon > 0:\{a_n\}\cap (b_*-\varepsilon, b_*+\varepsilon)$ - нескінченна множина\\
	Розглянемо $\varepsilon = \displaystyle \frac{1}{k}$\\
	$\displaystyle k = 1: \{a_n\}\cap(b_*-1, b_*+1) \rotatebox[origin=c]{180}{$\in$}a_{n_1}$\\
	$\displaystyle k = 2: \{a_n\}\cap(b_*-\frac{1}{2}, b_*+\frac{1}{2}) \rotatebox[origin=c]{180}{$\in$}a_{n_2}, n_2>n_1$\\
	$\cdots$\\
	Побудовали підпослідовність \subsequence{a_{n_k}} таким чином, що\\ $\displaystyle b_*-\frac{1}{k} < a_{n_k} < b_*+\frac{1}{k}$\\
	А далі спрямуємо $k$ до нескінченності. В результаті чого отримаємо:\\
	$\displaystyle \underset{\rotatebox[origin=c]{-45}{$\rightarrow$}}{b_*-\frac{1}{k}} < \underset{\underset{\displaystyle b_*}{\rotatebox[origin=c]{90}{$\leftarrow$}}}{a_{n_k}} < \underset{\rotatebox[origin=c]{45}{$\leftarrow$}}{b_*+\frac{1}{k}}, k \to \infty$\\
	Тоді за теоремою про 2 поліцая, $\displaystyle \exists \lim_{k \to \infty} a_{n_k} = b_*$ \qed
	\bigline
	\textbf{Corollary 2.5.4.} Множина всіх часткових границь не є порожньою\\
	Таку множину позначу за $A$
	\bigline
	\textbf{Definition 2.5.5.(1) Верхньою границею} називають число:
	\begin{align*}
	\displaystyle \uplim_{n \to \infty} a_n \overset{\textrm{або}}{=} \limsup_{n \to \infty} a_n = \sup A
	\end{align*}
	\textbf{Definition 2.5.5.(2) Нижньою границею} називають число:
	\begin{align*}
	\displaystyle \downlim_{n \to \infty} a_n \overset{\textrm{або}}{=} \liminf_{n \to \infty} a_n = \inf A
	\end{align*}
	\bigline
	
	\hypertarget{ex2.6.6.}{\textbf{Example 2.5.6.}} Знайдемо часткові границі для послідовнонсті\\ $\{a_n, n \geq 1\}$, де $a_n = (-1)^{n-1} \left(2 + \dfrac{3}{n} \right)$\\
	Якщо $n = 2k-1$, то маємо послідовність $\left\{a_{n_k} = 2 + \dfrac{3}{2k-1} \right\}$\\
	$\huge\lim_{k \to \infty} \left( 2 + \dfrac{3}{2k-1} \right) = 2$\\
	Якщо $n = 2k$, то маємо послідовність $\left\{a_{n_k} = -2 - \dfrac{3}{2k}\right\}$\\
	$\huge\lim_{k \to \infty} \left(-2 - \dfrac{3}{2k} \right) = -2$\\
	Множина часткових границь: $A = \{-2, 2\}$ - не порожня\\
	Тоді за означенням верхньої та нижньої границі, \\ $\huge \uplim_{n \to \infty} a_n = 2, \huge \downlim_{n \to \infty} a_n = -2$\\
	Зауважимо одразу, що $\huge \sup_{n \geq 1} \{a_n\} = 5$ та $\huge \inf_{n \geq 1} \{a_n\} = -3.5$
	\bigline
	\ex{2.5.6.(2).} Є ще така послідовність $\left\{ 0, 1, \dfrac{1}{2}, \dfrac{1}{4}, \dfrac{3}{4}, \dfrac{1}{8}, \dfrac{3}{8}, \dfrac{5}{8}, \dfrac{7}{8}, \dots \right\}$\\
	У неї множина часткових границь задається так: $A = [0,1]$
	\bigline
	\textbf{Remark 2.5.5.} Якщо послідовність \sequence{a_n} не є обменежою:\\
	- зверху, то $\displaystyle \uplim_{n \to \infty} = +\infty$\\
	- знизу, то $\displaystyle \downlim_{n \to \infty} = -\infty$\\
	\bigline
	\textbf{Theorem 2.5.7.} Будь-яка обмежена послідовність має верхню/нижню границю\\
	\rm{2.5.7.} Зазвичай, коли ми говоримо про верхні, нижні грані деякої множини $X$, то не обов'язково, щоб елемент $\sup X, \inf X$ лежали безпосередньо в множині $X$. Проте ця теорема каже, що верхня та нижня грані завжди будуть лежать в цій множині\\
	\proof
	Наша мета: показати, що існує така $\{a_{n_k}, k \geq 1\}$, що \\ $\huge \lim_{k \to \infty} a_{n_k} = \downlim_{n \to \infty} a_n = x_* = \inf X$\\
	Оскільки $X$ - множина часткових границь, то\\
	$\forall \varepsilon > 0: \exists x_{\varepsilon} \in X: x_* \leq x_{\varepsilon} < x_* + \dfrac{\varepsilon}{2}$\\
	Оскільки $x_\varepsilon \in X$, то тоді це - часткова границя для послідовності \\ $\{a_n, n \geq 1\}$. Тому за Больцано-Вейерштрасса, $\exists \{a_{n_m}^{(\varepsilon)}, m \geq 1\}: \huge \\ \lim_{m \to \infty} a_{n_m}^{(\varepsilon)} = x_{\varepsilon}$\\
	$\Rightarrow \exists M(\varepsilon): \forall m \geq M: |a_{n_m}^{(\varepsilon)} - x_{\varepsilon}| < \varepsilon$\\
	$\Rightarrow |a_{n_m}^{(\varepsilon)} - x_*| = |a_{n_m}^{(\varepsilon)} - x_{\varepsilon} + x_{\varepsilon} - x_*| \leq |a_{n_m}^{(\varepsilon)} - x_{\varepsilon}| + |x_{\varepsilon} - x_*| < \dfrac{\varepsilon}{2} + \dfrac{\varepsilon}{2} = \varepsilon$\\
	При $\varepsilon = 1$ маємо: $|a_{n_{M(1)}}^{(1)} - x_*| < 1$\\
	При $\varepsilon = \dfrac{1}{2}$ маємо: $|a_{n_{M(\frac{1}{2})}}^{(\frac{1}{2})} - x_*| < \dfrac{1}{2}$\\
	$\dots$\\
	А тепер розглянемо підпослідовність $\{a_{n_k}, k \geq 1\}$, таку, що $a_{n_k} = a_{n_{M(\frac{1}{k})}}^{(\frac{1}{k})}$\\
	За побудовою, $|a_{n_k} - x_*| < \dfrac{1}{k} \Rightarrow$ $\displaystyle \underset{\rotatebox[origin=c]{-45}{$\rightarrow$}}{x_*-\frac{1}{k}} < \underset{\underset{\displaystyle x_*}{\rotatebox[origin=c]{90}{$\leftarrow$}}}{a_{n_k}} < \underset{\rotatebox[origin=c]{45}{$\leftarrow$}}{x_*+\frac{1}{k}}, k \to \infty$\\
	Таким чином, для $\{a_{n_k}, k \geq 1\}$ існує $\huge \lim_{k \to \infty} a_{n_k} = x_* = \downlim_{n \to \infty} a_n$ \qed
	\bigline
	\th{2.5.8.} Задана $\{a_n, n \geq 1\}$ - обмежена та $L^* \in \mathbb{R}$. Наступні твердження еквівалентні:\\
	I. $L^* = \huge \uplim_{n \to \infty} a_n$\\
	II. $\forall \varepsilon > 0:$ проміжок $(L^*+\varepsilon, + \infty)$ містить скінченну кількість елементів та проміжок $(L^*-\varepsilon, + \infty)$ містить нескінченну кількість елементів\\
	III. Нехай задана послідовність $\{b_m, m \geq 1\}$, де $b_m = \huge \sup_{n \geq m} \{a_n\}$. \\ Тоді $\exists \huge \lim_{m \to \infty} b_m = L^*$\\
	\proof
	$\boxed{\textrm{I} \Rightarrow \textrm{II}}$ Дано: $L^* =  \huge \uplim_{n \to \infty} a_n$\\
	Тоді $L^* = \huge \sup A$. За попередньою теоремою, $L^* \in A$, тож існує \\ $\{a_{n_k}, k \geq 1\}$, для якої $\huge \lim_{k \to \infty} a_{n_k} = L^* \\ \Rightarrow \forall \varepsilon > 0: \exists K: \forall n_k \geq K: L^*-\varepsilon < a_{n_k} < L^*+\varepsilon$\\
	Звідси ми вже маємо, що на проміжку $(L^*-\varepsilon, +\infty)$ маємо нескінченну кількість елементів\\
	!А далі припустимо, що $\exists \varepsilon^* > 0:$ проміжок $(L^*+\varepsilon, + \infty)$ має НЕскінченну кількість елементів\\
	Оскільки $\{a_n, n \geq 1\}$ - обмежена, то за Больцано-Вейєрштрасса, маємо підпослідовність $\{a_{n_m}, m \geq 1\}$ таку, що $a_{n_m} > L^* + \varepsilon$\\
	Тоді звідси $\huge \lim_{m \to \infty} a_{n_m} = L^{**} \geq L^* + \varepsilon$\\
	Тобто $L^{**} > L^*$, але $L^*$ - верхня границя. Суперечність!\\
	Висновок: $\forall \varepsilon > 0:$ проміжок $(L^*+\varepsilon, + \infty)$ має скінченну кількість елементів
	\bigline
	$\boxed{\textrm{II} \Rightarrow \textrm{III}}$ Дано: $\forall \varepsilon > 0:$ проміжок $(L^*+\varepsilon, + \infty)$ містить скінченну кількість елементів та проміжок $(L^*-\varepsilon, + \infty)$ містить нескінченну кількість елементів\\
	Для початку розглянемо $\{b_m, m \geq 1\}$ та покажемо, що в неї дійсно є границя\\
	$b_{m+1} \leq b_m$, тобто $\huge \sup_{n \geq m+1} \{a_n\} \leq \sup_{n \geq m} \{a_n\}$. Думаю, зрозуміло\\
	Також оскільки $\{a_n, n \geq 1\}$ - обмежена, то $b_m$ будуть теж обмеженими\\
	Тоді за Вейєрштрассом, $\exists \huge \lim_{m \to \infty} b_m = \inf_{m \geq 1} \{b_m\}$\\
	Оскільки $(L^*+\varepsilon, + \infty)$ має скінченну кількість елементів, то $\exists M: \forall m \geq M: x_m \leq L^* + \dfrac{\varepsilon}{2}$\\
	Тоді $\forall m \geq M: b_m < L^* + \varepsilon$\\
	Також оскільки $(L^*-\varepsilon, + \infty)$ має нескінченну кількість елементів, то\\
	$b_m > L^* - \varepsilon$, $\forall m \geq 1$\\
	Остаточно, $\forall \varepsilon > 0: \exists M: \forall m \geq M: |b_m - L^*| < \varepsilon$\\
	Отже, $\huge \lim_{m \to \infty} b_m = L^*$
	\bigline
	$\boxed{\textrm{III} \Rightarrow \textrm{I}}$ Дано: $\huge \lim_{m \to \infty} b_m = L^*$\\
	Візьмемо деяку підпослідовність $\{a_{n_k}, k \geq 1\}$. Маємо нерівність: \\ $a_{n_k} \leq b_{n_k} \leq b_k \Rightarrow \huge \lim_{k \to \infty} a_{n_k} \leq L^*$\\
	$\huge \lim_{m \to \infty} b_m = L^* \Rightarrow \forall \varepsilon > 0: \exists M: \forall m \geq M: L^*-\varepsilon < b_m < L^*+\varepsilon$\\
	Для номера $M$ виконано $b_M > L^*-\varepsilon$. Але не всі $a_n$, де $n \geq M$, можуть виконувати нерівність\\
	Отже, виділимо підпослідовність $\{a_{n_k}^{\varepsilon}, k \geq 1\}$, для яких\\
	$a_{n_k}^{\varepsilon} > L^* - \varepsilon$\\
	А тоді $a_{\varepsilon} > L^* - \varepsilon$\\
	Таким чином, ми отримали:\\
	1)$\forall a \in A: a \leq L^*$\\
	2)$\forall \varepsilon > 0: \exists a_{\varepsilon}: a_{\varepsilon} > L^* - \varepsilon$\\
	Тобто $L^* = \huge \sup A = \uplim_{n \to \infty} a_n$ \qed
	\bigline
	
	\subsection{Фундаментальна послідовність}
	\textbf{Definition 2.6.1.} Послідовність \sequence{a_n} називається \textbf{фундаментальною}, якщо
	\begin{align*}
	\forall \varepsilon > 0: \exists N \in \mathbb{N}: \forall n,m \geq N: |a_n - a_m| < \varepsilon
	\end{align*}
	\textbf{Theorem 2.6.2. Критерій Коші}\\
	Послідонвість \sequence{a_n} є збіжною $\iff$ вона є фундаментальною\\
	\textbf{Proof.}\\
	$\boxed{\Rightarrow}$ Дано: \sequence{a_n} - збіжна, тобто: $\forall \varepsilon >0: \exists N: $\\
	$\huge \forall n \geq N: |a_n - a| < \frac{\varepsilon}{2}$\\
	$\huge \forall m \geq N: |a_m - a| < \frac{\varepsilon}{2}$\\
	А тоді отримаємо,\\
	$|a_n - a_m| = |a_n - a + a - a_m| \leq |a_n - a| + |a_m - a| < \varepsilon$\\
	Отже, послідовність є фундаментальною\\
	\\
	$\boxed{\Leftarrow}$ Дано: \sequence{a_n} - фундаментальна, тобто\\
	$\forall \varepsilon > 0: \exists N \in \mathbb{N}: \forall n,m \geq N: |a_n - a_m| < \varepsilon$\\
	І. Доведемо, що вона є обмеженою\\
	Для $\varepsilon = 1: \exists N: \forall n \geq N, m = N: |a_n - a_N| < 1$\\
	$\Rightarrow |a_n| = |a_n - a_N + a_N| \leq |a_n - a_N| + |a_N| < 1 + |a_N|$\\
	Задамо $C = \max\{|a_1|, \dots, |a_{N-1}|, |1|+|a_N|\}$\\
	Тоді $\forall n \geq 1: |a_n| \leq C$, тобто обмежена\\
	II. Доведемо її збіжність\\
	Оскільки наша послідовність обмежена, виділимо збіжну підпослідовність \subsequence{a_{n_k}}, $\huge \lim_{n \to \infty} a_{n_k} = a \Rightarrow$\\
	$\huge \forall \varepsilon > 0: \exists K: \forall n_k \geq K: |a_{n_k} - a| < \frac{\varepsilon}{2}$\\
	Покладемо $m = n_k$. Тоді:\\
	$|a_n - a| = |a_n - a_{n_k} + a_{n_k} - a| \leq |a_n - a_{n_k}| + |a_{n_k} - a| < \varepsilon$\\
	Тобто $\huge \exists \lim_{n \to \infty} a_n = a$ \qed
	\bigline
	\rm{2.6.1.} Означення фундаментальної послідовності можна записати й таким чином
	\begin{align*}
	\forall \varepsilon > 0: \exists N \in \mathbb{N}: \forall n \geq N: \forall p \geq 1: |a_{n+p} - a_n| < \varepsilon
	\end{align*}
	Дійсно, якщо покласти $m = n + p$, де $p \in \mathbb{N}$, то отримаємо бажане
	\bigline
	\ex{2.6.3.} Розглянемо послідовність $\{a_n, n \geq 1\}:$\\
	$a_n = \dfrac{\sin 1}{1^2} + \dfrac{\sin 2}{2^2} + \dots + \dfrac{\sin n}{n^2}$\\
	Доведемо її фундаментальність за означенням\\
	$|x_{n+p} - x_n| \leq \dfrac{1}{(n+1)^2} + \dots + \dfrac{1}{(n+p)^2} \leq \dfrac{1}{n(n+1)} + \dots + \dfrac{1}{(n+p-1)(n+p)} = \\ = \dfrac{1}{n} - \dfrac{1}{n-1} + \dots + \dfrac{1}{n+p-1} - \dfrac{1}{n+p} = \dfrac{1}{n} - \dfrac{1}{n+p} \leq \dfrac{1}{n} < \varepsilon$\\
	$\Rightarrow n > \dfrac{1}{\varepsilon}$\\
	Встановимо $N = \left[ \dfrac{1}{\varepsilon} \right] + 1$. Тоді $\forall n \geq N: \forall p \geq 1: |x_{n+p} - x_n| < \varepsilon$\\
	Отже, наша послідовність - фундаментальна
	\newpage
	\subsection{*Деякі теоретичні факти}
	\prp{2.7.1.} Задано $\{a_n, n \geq 1 \}$ - збіжний. Тоді $\{|a_n|,n \geq 1\}$ - збіжний\\
	\proof
	$\exists \huge \lim_{n \to \infty} a_n = a \iff \forall \varepsilon > 0: \exists N: \forall n \geq N: |a_n-a|<\varepsilon$\\
	Тоді $||a_n| - |a|| \leq |a_n - a| < \varepsilon$\\
	А це означає, що $\exists \huge \lim_{n \to \infty} |a_n| = |a|$ \qed
	\bigline
	\rm{2.7.1.} В зворотньому порядку не працює, принаймні для \\ $\{a_n = (-1)^n, n \geq 1\}$
	\bigline
	\th{2.7.2.} Задано множина $A \subset \mathbb{R}$\\
	$a$ - гранична точка $A$ $\iff \exists$ $\{a_n, n \geq 1\}$ $\subset A: \huge \lim_{n \to \infty} a_n = a$, причому $\forall n \geq 1: a_n \neq a$
	\\
	\proof
	\rightproof Дано: $a$ - гранична т. $A$, тоді $\forall \varepsilon > 0: (a-\varepsilon, a + \varepsilon) \cap A$ - нескінченна множина\\
	$\varepsilon = 1: \exists a_1 \in (a-1,a+1) \cap A$\\
	$\varepsilon = \dfrac{1}{2}: \exists a_2 \in (a-\dfrac{1}{2}, a+\dfrac{1}{2}) \cap A$\\
	$\dots$\\
	Побудували послідовність \sequence{a_n}, таку, що $a_n \in (a-\dfrac{1}{n}, a+\dfrac{1}{n}) \cap A$\\
	Тобто $a - \dfrac{1}{n} < a_n < a + \dfrac{1}{n}$\\
	За теоремою про двох поліцаїв, якщо $n \to \infty$, то отримаємо, що \\ $\exists \huge \lim_{n \to \infty} a_n = a$
	\bigline
	
	\leftproof Дано: $\exists$ \sequence{a_n} $\subset A: \forall n \geq 1: a_n \neq a: \huge \lim_{n \to \infty} a_n = a$\\
	$\implies \forall \varepsilon > 0: \exists N: \forall n \geq N: |a_n-a|<\varepsilon \implies a_n \in (a-\varepsilon,a+\varepsilon)$\\
	А отже, $(a-\varepsilon,a+\varepsilon) \cap A$ - нескінченна множина, тож $a$ - гранична точка \qed
	
	\subsection{*Константа Ейлера-Маскероні}
	Розглянемо дві послідовності $\{a_n, n \geq 1\}, \{b_n, n \geq 1\}$, такі, що\\
	$a_n = 1 + \dfrac{1}{2} + \dfrac{1}{3} + \dots + \dfrac{1}{n} - \ln (n+1)$\\
	$b_n = 1 + \dfrac{1}{2} + \dfrac{1}{3} + \dots + \dfrac{1}{n} - \ln n$\\
	Покажемо, що ці дві послідовності є взагалі збіжними\\
	1) Покажемо, що $\forall n \geq 1: a_n < b_n$. Справді,\\
	$a_n - b_n = - \ln (n+1) + \ln n = \ln \dfrac{n}{n+1} = \ln \left(1 - \dfrac{1}{n+1} \right) < 0$
	\bigline
	2) Покажемо, що $\forall n \geq 1: a_{n+1} > a_n$\\
	$a_n - a_{n+1} = - \ln(n+1) -\dfrac{1}{n+1} + \ln(n+2) = - \dfrac{1}{n+1} + \ln \left( 1 + \dfrac{1}{n+1} \right) \boxed{<}$
	Згадаємо нерівність $\dfrac{1}{n+1} < \ln \left(1 + \dfrac{1}{n} \right) < \dfrac{1}{n}$\\
	$\boxed{<} -\dfrac{1}{n+1} + \dfrac{1}{n+1} = 0$\\
	$\Rightarrow a_{n+1} > a_n$
	\bigline
	3) Показується аналогічним чином як в 1), що $\forall n \geq 1: b_{n+1} < b_n$\\
	Тоді ми маємо:\\
	$\forall n \geq 1: 1 - \ln 2 = a_1 < a_2 < \dots < a_n < b_n < \dots < b_2 < b_1 = 1$\\
	Остаточно: $\{b_n, n \geq 1\}$ - монотонна та обмежена послідовність. Тоді\\
	$\exists \huge \lim_{n \to \infty} \left(1 + \dfrac{1}{2} + \dfrac{1}{3} + \dots + \dfrac{1}{n} - \ln n \right) = \gamma \approx 0.57$ - \textbf{стала Ейлера-Маскероні}\\
	Туди ж прямує послідовність $\{a_n, n \geq 1\}$
	\bigline
	
	\subsection{*Теорема Штольца}
	\th{2.9.1. Теорема Штольца}\\
	Задані дві послідовності $\{a_n, n \geq 1\}, \{b_n, n \geq 1\}$, які мають наступні властивості\\
	1) $\{b_n\}$ - н.в. та монотонно строго зростає (можливо, з якогось номера)\\
	2) $\exists \huge \lim_{n \to \infty} \dfrac{a_{n+1} - a_n}{b_{n+1} - b_n} = L$\\
	Тоді $\exists \huge \lim_{n \to \infty} \dfrac{a_n}{b_n} = L$\\
	\proof
	Розглянемо випадок, коли $L < \infty$\\
	$\huge \lim_{n \to \infty} \dfrac{a_{n+1} - a_n}{b_{n+1} - b_n} = L \Rightarrow \exists N: \forall n \geq N: \abs{\dfrac{a_{n+1}-a_n}{b_{n+1}-b_n} - L} < \varepsilon$\\
	$\Rightarrow L - \varepsilon < \dfrac{a_{n+1} - a_n}{b_{n+1} - b_n} < L + \varepsilon$\\
	Оскільки $\{b_n\}$ - монотонно зростає, то ми домножимо на $b_{n+1} - b_n$, отримаємо:\\
	$\Rightarrow (L - \varepsilon)(b_{n+1} - b_n) < a_{n+1} - a_n < (L + \varepsilon)(b_{n+1} - b_n)$\\
	Зафіксуємо $k > N$ та просумуємо це до $k$\\
	Матимемо, що $(b_{N+1} - b_N) + (b_{N+2} - b_{N+1}) + \dots + (b_{k+1} - b_{k}) = b_{k+1} - b_N$\\
	Аналогічно $(a_{N+1} - a_N) + (a_{N+2} - a_{N+1}) + \dots + (a_{k+1} - a_{k}) = a_{k+1} - a_N$\\
	$\Rightarrow(L - \varepsilon)(b_{k+1} - b_N) < a_{k+1} - a_N < (L + \varepsilon)(b_{k+1} - b_N)$\\
	Поділимо на $b_{k+1} > 0$, оскільки вона строго монотонно зростає\\
	$\Rightarrow (L - \varepsilon) \left(1 - \dfrac{b_N}{b_{k+1}} \right) < \dfrac{a_{k+1}}{b_{k+1}} - \dfrac{a_N}{b_{k+1}} < (L + \varepsilon) \left(1 - \dfrac{b_N}{b_{k+1}} \right)$\\
	$\Rightarrow (L - \varepsilon) \left(1 - \dfrac{b_N}{b_{k+1}} \right) + \dfrac{a_N}{b_{k+1}}  < \dfrac{a_{k+1}}{b_{k+1}} < (L + \varepsilon) \left(1 - \dfrac{b_N}{b_{k+1}} \right) + \dfrac{a_N}{b_{k+1}}$\\
	Спрямуємо $k \to \infty$, тоді за теоремою про нерівність та з урахуванням тим, що $\{b_n\}$ - н.в. величина, маємо\\
	$(L-\varepsilon) < \huge \lim_{k \to \infty} \dfrac{a_{k+1}}{b_{k+1}} < (L + \varepsilon)$\\
	$\Rightarrow \abs{\huge \lim_{k \to \infty} \dfrac{a_{k+1}}{b_{k+1}} - L} < \varepsilon, \forall \varepsilon > 0$\\
	Остаточно отримаємо:\\
	$\huge \lim_{k \to \infty} \dfrac{a_{k+1}}{b_{k+1}} = L$
	\bigline
	\\
	А тепер $\huge \lim_{n \to \infty} \dfrac{a_{n+1} - a_n}{b_{n+1} - b_n} = + \infty$\\
	$\Rightarrow E = 1: \exists N: \forall n \geq N: a_{n+1} - a_n > b_{n+1} - b_n \Rightarrow a_{n+1} > a_n$\\
	Тоді $\huge \lim_{n \to \infty} \dfrac{b_{n+1} - b_n}{a_{n+1} - a_n} = 0$\\
	Щойно зверху довели, що звідси $\huge \lim_{n \to \infty} \dfrac{b_n}{a_n} = 0 \Rightarrow \lim_{n \to \infty} \dfrac{a_n}{b_n} = + \infty$\\
	Для $-\infty$ треба розглянути послідонвість $\{-a_n\}$ \qed
	\bigline
	\th{2.9.2. Теорема Штольца 2}\\
	Задані дві послідовності $\{a_n, n \geq 1\}, \{b_n, n \geq 1\}$, які мають наступні властивості\\
	1) $\{a_n\}, \{b_n\}$ - н.м. та монотонно строго спадають (можливо, з якогось номера)\\
	2) $\exists \huge \lim_{n \to \infty} \dfrac{a_{n+1} - a_n}{b_{n+1} - b_n} = L$\\
	Тоді $\exists \huge \lim_{n \to \infty} \dfrac{a_n}{b_n} = L$\\
	\textit{Доведення є аналогічним}
	\bigline
	\ex{2.9.3.} Знайдемо границю $\huge \lim_{n \to \infty} \dfrac{1^k + 2^k + \dots + n^k}{n^{k+1}}, k \in \mathbb{N}$\\
	Маємо зверху послідовність $\{a_n = 1^k + 2^k + \dots + n^k, n \geq 1 \}$, що строго монотонно зростає, а також є н.в. Знайдемо одну границю як в теоремі Штольца:\\
	Перед цим ми маємо ще послідовність $\{b_n = n^{k+1}, n \geq 1\}$\\
	$\huge \lim_{n \to \infty} \dfrac{a_{n+1} - a_n}{b_{n+1} - b_n}  =\lim_{n \to \infty} \dfrac{(n+1)^k}{(n+1)^{k+1} - n^{k+1}} \boxed{=}$\\
	Скористаємось тотожністю: \\ $x^n - y^n = (x-y)(x^{n-1} + x^{n-2}y + \dots + xy^{n-2} + y^{n-1})$\\
	$\boxed{=} \huge \lim_{n \to \infty} \dfrac{(n+1)^k}{(n+1-n)((n+1)^k + (n+1)^{k-1}n + \dots + (n+1) n^{k-1} + n^k)} = \\ = \lim_{n \to \infty} \dfrac{1}{1 + \dfrac{n}{(n+1)} + \dots + \dfrac{n^{k-1}}{(n+1)^{k-1}} + \dfrac{n^k}{(n+1)^k}} = \dfrac{1}{1+ \underset{\textrm{k разів}}{1 + \dots + 1}} = \dfrac{1}{k+1}$\\
	Тоді за теоремою Штольца, $\huge \lim_{n \to \infty} \dfrac{1^k + 2^k + \dots + n^k}{n^{k+1}} = \dfrac{1}{k+1}, k \in \mathbb{N}$
	\bigline
	\th{2.9.4. Теореми Чезаро}\\
	1. Якщо $\exists \huge \lim_{n \to \infty} a_n = L$, то $\exists \huge \lim_{n \to \infty} \dfrac{a_1 + \dots + a_n}{n} = L$\\
	2. Якщо $\exists \huge \lim_{n \to \infty} a_n = L$, то $\exists \huge \lim_{n \to \infty} \sqrt[n]{a_1 \dots a_n} = L$. Причому $\forall n \geq 1: a_n > 0$\\
	3. Якщо $\exists \huge \lim_{n \to \infty} \dfrac{a_{n+1}}{a_n} = L$, то $\exists \huge \lim_{n \to \infty} \sqrt[n]{a_n} = L$. Причому $\forall n \geq 1: a_n > 0$\\
	\proof
	1. Зафіксуємо послідовність $\{S_n = a_1 + \dots + a_n, n \geq 1\}$\\
	Тоді маємо $\huge \lim_{n \to \infty} \dfrac{S_n}{n} \overset{\textrm{\textbf{Th. Штольца}}}{=} \lim_{n \to \infty} \dfrac{S_{n+1} - S_n}{(n+1) - n} = \lim_{n \to \infty} a_{n+1} = L$
	\bigline
	2. $\exists \huge \lim_{n \to \infty} \dfrac{a_1+\dots+a_n}{n} = L \Rightarrow \forall \varepsilon > 0: \exists N: \forall n: \abs{\dfrac{a_1+\dots+a_n}{n} - L} < \varepsilon$\\
	Тоді $\abs{\sqrt[n]{a_1 \dots a_n} - L} \leq \abs{\dfrac{a_1+\dots+a_n}{n} - L} < \varepsilon$\\
	$\Rightarrow \exists \huge \lim_{n \to \infty} \sqrt[n]{a_1 \dots a_n} = n$
	\bigline
	3. Зафіксуємо послідовність $\{b_1 = a_1, b_n = \dfrac{a_{n}}{a_{n-1}}, n \geq 2\}$. Тоді маємо:\\
	$\huge \lim_{n \to \infty} \dfrac{a_{n}}{a_{n-1}} = \lim_{n \to \infty} b_n = \lim_{n \to \infty} \sqrt[n]{b_1 \dots b_n} = \lim_{n \to \infty} \sqrt[n]{a_1 \dfrac{a_2}{a_1} \dots \dfrac{a_n}{a_{n-1}}} = \lim_{n \to \infty} \sqrt[n]{a_n} = L$
	\qed
	\bigline
	\ex{2.9.5.} Знайти границю $\huge \lim_{n \to \infty} \dfrac{n}{\sqrt[n]{n!}}$\\
	$\huge \lim_{n \to \infty} \dfrac{n}{\sqrt[n]{n!}} = \lim_{n \to \infty} \sqrt[n]{\dfrac{n^n}{n!}}$\\
	За третім пунктом \textbf{Th. 2.9.4.}, спробуємо обчислити границю:\\
	$\huge \lim_{n \to \infty} \dfrac{\dfrac{(n+1)^{n+1}}{(n+1)!}}{\dfrac{n^n}{n!}} = \lim_{n \to \infty} \dfrac{(n+1)^n}{n^n} = \lim_{n \to \infty} \left(1 + \dfrac{1}{n} \right)^n = e$\\
	Тоді $\huge \lim_{n \to \infty} \dfrac{n}{\sqrt[n]{n!}} = \lim_{n \to \infty} \sqrt[n]{\dfrac{n^n}{n!}} = e$
	\bigline
	\subsection{*Ірраціональність числа $e$}
	Спочатку треба довести інше означення числа $e$
	\bigline
	\prp{2.10.1.} $\huge \lim_{n \to \infty} \left(1 + \dfrac{1}{1!} + \dfrac{1}{2!} + \dots + \dfrac{1}{n!} \right) = e$\\
	\proof
	Згадаємо, що $e = \huge \lim_{n \to \infty} \left(1 + \dfrac{1}{n} \right)^n$. Розкриємо тепер за біномом Ньютона дужку:\\
	$\left(1 + \dfrac{1}{n} \right)^n = 1 + C_{n}^1 \dfrac{1}{n} + C_{n}^2 \dfrac{1}{n^2} + \dots + C_n^n \dfrac{1}{n^n} = \\ = 1 + \dfrac{n}{1! n} + \dfrac{n(n-1)}{2! n^2} + \dfrac{n(n-1)(n-2)}{3! n^3} + \dots + \dfrac{n(n-1)\dots2\cdot 1}{n! n^n}\\
	\Rightarrow \huge \lim_{n \to \infty} \left( 1 + \dfrac{1}{n} \right)^n = \lim_{n \to \infty} \left(1 + \dfrac{n}{1!n} + \dfrac{n(n-1)}{2! n^2} + \dots + \dfrac{n(n-1)\cdots 2 \cdot 1}{n! n^n} \right) = \\ = \lim_{n \to \infty} \left(1 + \dfrac{1}{1!} + \dfrac{1}{2!} + \dots + \dfrac{1}{n!} \right) = e$ \qed
	\bigline
	А тепер час припустити, що $e \in \mathbb{Q}$, тобто це - раціональне, отже\\
	$e = \dfrac{m}{k}, m \in \mathbb{Z}, k \in \mathbb{N}$\\
	Водночас ми знаємо, що\\
	$e = 1 + \dfrac{1}{1!} + \dfrac{1}{2!} + \dots + \dfrac{1}{k!} + \dfrac{1}{(k+1)!} + \dots$\\
	Помножимо обидві частини на $k!$, отримаємо:\\
	$e k! = k! + \dfrac{k!}{1!} + \dfrac{k!}{2!} + \dots + 1 + \dfrac{1}{k+1} + \dfrac{1}{(k+1)(k+2)} + \dots$\\
	Ліва частина - ціле число, в правій частині всі доданки до дробу - цілі числа, але\\
	$\dfrac{1}{(k+1)(k+2)} + \dfrac{1}{(k+1)(k+2)(k+3)} + \dots = \\
	\\ = \huge \lim_{n \to \infty} \left( \dfrac{1}{(k+1)(k+2)} + \dfrac{1}{(k+1)(k+2)(k+3)} + \dots + \dfrac{1}{(k+1)(k+2)\dots(k+n)} \right) < \\
	< \huge \lim_{n \to \infty} \left( \dfrac{1}{(k+1)(k+2)} + \dfrac{1}{(k+2)(k+3)} + \dots + \dfrac{1}{(k+(n-1))(k+n)} \right) = \\ \huge \lim_{n \to \infty} \left( \dfrac{1}{k+1} - \dfrac{1}{k+2} + \dfrac{1}{k+2} - \dfrac{1}{k+3} + \dots + \dfrac{1}{k-(n-1)} - \dfrac{1}{k+n} \right) = \dfrac{1}{k+1}$\\
	Тобто $ek! < M + \dfrac{2}{k+1}$\\
	Число, яке ми оцінювали, виявляється, менше за 1, тому що вона менше за такий вираз, який також менше за 1. Отримаємо суперечність! \qed
	
	\subsection{*Звідки виник окіл в нескінченності}
	Розглянемо таку картину - коло Рімана. Нижній дотик кола буде відповідати границі $0$, а верхня точка - границя $\infty$.
	\begin{figure}[H]
	\centering
	\begin{tikzpicture}
	\draw (0,2) circle (2) node at (0,4.5) {$\infty$};
	\filldraw (0,4) circle (1pt);
	\filldraw (0,0) circle (1pt);
	\draw[thick, ->] (-6,0)--(6,0);
	\draw node at (0,-0.5) {$0$};
	\draw[shorten >= -1cm, shorten <= -0cm] (0,4)--(2-0.4,0);
	\draw[shorten >= -1cm, shorten <= -0cm] (0,4)--(2+0.4,0);
	\draw[red] (2-0.4,0)--(2+0.4,0);
	\filldraw (1.6,0.8) circle (1pt) node[anchor = north] {$a$};
	\node[red] at (2-0.4,0) {$($};
	\node[red] at (2+0.4,0) {$)$};
	\node[red, scale = 0.7] at (2-0.5,-0.5) {$a-\varepsilon$};
	\node[red, scale = 0.7] at (2+0.5,-0.5) {$a+\varepsilon$};
	\end{tikzpicture}
	\end{figure}
	Проведемо промінь так, щоб вона перетнула вісь. Кожна точка кола ставить у відповідність точку на вісі - отже, й окіл теж. На цьому малюнку окіл т. $a$ кола ставить у відповідність звичний окіл т. $a$, тобто $U_{\varepsilon}(a)$\\
	Візьмемо тепер окіл в нескінченності. Я розглядатиму праву частину півкола, де $+\infty$, для іншої аналогічно. Відступимо від $+\infty$ трошки праворуч. Нарешті, проведемо між двома точками пряму\\
	\begin{figure}[H]
	\centering
	\resizebox{1.1\textwidth}{!} {
	\begin{tikzpicture}
	\pgfmathsetmacro{\a}{0.8}
	\draw (0,2) circle (2) node at (0,4.5) {$\infty$};
	\filldraw (0,4) circle (1pt);
	\filldraw (0,0) circle (1pt);
	\draw[thick, ->] (-3,0)--(20,0);
	\draw node at (0,-0.5) {$0$};
	\filldraw (\a, {2+sqrt(4-\a*\a)}) circle(1pt);
	\draw [shorten >= -20cm, shorten <= -0cm] (0,4)--(\a, {2+sqrt(4-\a*\a)});
	\node[red] at (19.1651, 0) {$($};
	\node[red] at (19.1651, -0.5) {$E$};
	\draw[red] (19.1651, 0)--(20,0);
	\end{tikzpicture}
	}
	\end{figure}
	Тоді якщо подивитись на малюнок, околом $U_{E}(+\infty) = \{x \in \mathbb{R}: x > E\}$\\
	Аналогічними міркуваннями $U_{E}(-\infty) = \{x \in \mathbb{R}: x < -E\}$\\
	Узагальнення: $U_{E}(\infty) = \{x \in \mathbb{R}: |x| > E\}$\\
	\newpage
	
	
	\section{Границі функції}
	\subsection{Основні поняття про функції}
	\textbf{Definition 3.1.1} Задані дві множини $X,Y$.\\
	\textbf{Відображенням} $f$ із множини $X$ в множину $Y$ називають правило, в якому кожному елементу з $X$ ставиться у відповідність елемент з $Y$\\
	Позначення: $f: X \to Y$\\
	Якщо $X$ та $Y$ є числовими множинами, то відображенням називають \textbf{функцією}\\
	\begin{figure}[H]
\centering {
\begin{tikzpicture}
\fill[red!40] (0,0) ellipse (1cm and 2cm);
\fill[blue!40] (4,0) ellipse (1cm and 2cm);
\node (A1) at (0.7,1) [circle,fill,inner sep=1.5pt]{};
\node (A2) at (-0.5,-0.7) [circle,fill,inner sep=1.5pt]{};
\node (A3) at (0.5,-0.1) [circle,fill,inner sep=1.5pt]{};

\node (B1) at (4+0,1) [circle,fill,inner sep=1.5pt]{};
\node (B2) at (4-0.8,0.2) [circle,fill,inner sep=1.5pt]{};
\node (B3) at (4+0.5,-0.9) [circle,fill,inner sep=1.5pt]{};
\node[anchor = south east] at (0,0) {$X$};
\node[anchor = north west] at (4,0) {$Y$};

\draw[thick, ->] (A1)--(B1);
\draw[thick, ->] (A2)--(B3); \draw[thick, ->] (A3)--(B1);
\end{tikzpicture}
}
\end{figure}
	\textbf{Example 3.1.2.(1)} Задані дві множини:\\
	$X = \{0; 1; 2; 3 \}$ \\ $Y = \{-1; \sqrt{2}, 17, \sqrt{101}, 124, 1111\}$\\ Можна побудувати таке відображення $X \to Y$:\\
	$\begin{matrix}
	X & Y \\
	0 & \sqrt{2} \\
	1 & 1111 \\
	2 & \sqrt{2} \\
	3 & \sqrt{101} \\
	\end{matrix}$
	\\
	\\
	\textbf{Example 3.1.2.(2)} Задане таке відображення: $f: [-4; 5] \to \mathbb{R}$\\
	$f(x) = 2^x$\\
	Це вже можна називати функцією
	\\
	\\
	\textbf{Definition 3.1.3.} Задані два відображення: $f: X \to Y$, $g: Y \to Z$\\
	\textbf{Композицією відображень} $f$ та $g$ називають відображення \\ $h: X \to Z$ таке, що:
	\begin{align*}
	\forall x: h(x) = g(f(x)) \textrm{, або } h(x) = (g \circ f) (x)
	\end{align*}
	\begin{figure}[H]
\centering {
\begin{tikzpicture}
\fill[red!40] (0,0) ellipse (1cm and 2cm);
\fill[blue!40] (4,0) ellipse (1cm and 2cm);
\fill[green!40] (8,0) ellipse (1cm and 2cm);

\node[anchor = south east] at (0,1) {$X$};
\node[anchor = south east] at (4,1) {$Y$};
\node[anchor = south east] at (8,1) {$Z$};

\draw[thick, ->](0.5,0) .. controls (2,1) .. (4,0) node at (2,1.2) {$f$};
\draw[thick, ->](4.5,0) .. controls (6,1) .. (8,0) node at (6,1.2) {$g$};
\draw[thick, ->](0.5,-0.1) .. controls (4,-1) .. (8,-0.1) node at (6,-1) {$h = g \circ f$};;

\end{tikzpicture}
}
\end{figure}
	\textbf{Example 3.1.4.} $f,g: \mathbb{R} \to \mathbb{R}$, $f(x) = x^2$, $g(x) = \sin x$\\
	Тоді $h: \mathbb{R} \to \mathbb{R}$, $h(x) = g(f(x)) = \sin x^2$\\
	\\
	\textbf{Proposition 3.1.5. Асоціативність композиції}\\
	Задані $f: W \to X$, $g: X \to Y$, $h: Y \to Z$\\
	Тоді $h \circ (g \circ f) = (h \circ g) \circ f$\\
	\textbf{Proof.}\\
	$(h \circ (g \circ f)) (x) = h((g \circ f) (x)) = h(g(f(x)))$\\
	$((h \circ g) \circ f) (x) = (h \circ g) (f(x)) = h(g(f(x)))$ \qed
	\begin{figure}[H]
\centering {
\begin{tikzpicture}
\fill[yellow!40] (0,0) ellipse (1cm and 2cm);
\fill[red!40] (4,0) ellipse (1cm and 2cm);
\fill[blue!40] (8,0) ellipse (1cm and 2cm);
\fill[green!40] (12,0) ellipse (1cm and 2cm);

\node[anchor = south east] at (0,1) {$W$};
\node[anchor = south east] at (4,1) {$X$};
\node[anchor = south east] at (8,1) {$Y$};
\node[anchor = south east] at (12,1) {$Z$};

\draw[thick, ->](0.5,0) .. controls (2,1) .. (4,0) node at (2,1.2) {$f$};
\draw[thick, ->](4.5,0) .. controls (6,1) .. (8,0) node at (6,1.2) {$g$};
\draw[thick, ->](4+4.5,0) .. controls (4+6,1) .. (4+8,0) node at (4+6,1.2) {$h$};
\draw[thick, ->](0.5,-0.1) .. controls (4,-1) .. (8,-0.1) node at (6,0) {$g \circ f$};
\draw[thick, ->](4+0.5,-0.1-1) .. controls (4+4,-1-1) .. (4+8,-0.1-1) node at (4+6,-1-1) {$h \circ g$};
\draw[thick, ->] (0,1.5) .. controls (6,3) .. (12,1.5) node at(6,3.2) {$h \circ (g \circ f) = (h \circ g) \circ f$};
\end{tikzpicture}
}
\end{figure}
	\textbf{Definition 3.1.6.} Відображення $f: X \to Y$ називається: \\
	- \textbf{ін'єкцією}, якщо $\forall x_1, x_2 \in X: x_1 \neq x_2 \Rightarrow f(x_1) \ne f(x_2)$\\
	- \textbf{сюр'єкцією}, якщо $\forall y \in Y: \exists x: f(x) = y$\\
	- \textbf{бієкцією}, якщо $\forall y \in Y: \exists! x: f(x) = y$
	\bigline
	\textbf{Example 3.1.7.}\\
	1) $f: \mathbb{R} \to [-1,1]$: $f(x) = \cos x$ - сюр'єкція\\
	2) $f: \mathbb{R} \to \mathbb{R}$: $f(x)=3^x$ - ін'єкція\\
	3) $f: \mathbb{R} \to \mathbb{R}$: $f(x) = x^3$ - бієкція\\
	4) $f: \mathbb{R} \to \mathbb{R}$: $f(x) = |x|$ - жодна з означень\\
\begin{figure}[H]
\centering 
\resizebox{1\textwidth}{!} {
\begin{tikzpicture}
\fill[red!40] (0,0) ellipse (1cm and 2cm);
\fill[blue!40] (3,0) ellipse (1cm and 2cm);
\node (A1) at (0,0.8) [circle,fill,inner sep=1.5pt]{};
\node (A2) at (0,0) [circle,fill,inner sep=1.5pt]{};
%\node (A3) at (0,-0.8) [circle,fill,inner sep=1.5pt]{};

\node (B1) at (3+0,0.8) [circle,fill,inner sep=1.5pt]{};
\node (B2) at (3+0,0) [circle,fill,inner sep=1.5pt]{};
\node (B3) at (3+0,-0.8) [circle,fill,inner sep=1.5pt]{};
\node[anchor = south east] at (0,1) {$X$};
\node[anchor = south east] at (3,1) {$Y$};

\draw[thick, ->] (A1)--(B1);
\draw[thick, ->] (A2)--(B3);
\node at (1.5,-3) {$\textrm{ін'єкція}$};
\end{tikzpicture}
\qquad

\begin{tikzpicture}
\fill[red!40] (0,0) ellipse (1cm and 2cm);
\fill[blue!40] (3,0) ellipse (1cm and 2cm);
\node (A1) at (0,0.8) [circle,fill,inner sep=1.5pt]{};
\node (A2) at (0,0) [circle,fill,inner sep=1.5pt]{};
\node (A3) at (0,-0.8) [circle,fill,inner sep=1.5pt]{};

\node (B1) at (3+0,0.8) [circle,fill,inner sep=1.5pt]{};
%\node (B2) at (4+0,0) [circle,fill,inner sep=1.5pt]{};
\node (B3) at (3+0,-0.8) [circle,fill,inner sep=1.5pt]{};
\node[anchor = south east] at (0,1) {$X$};
\node[anchor = south east] at (3,1) {$Y$};

\draw[thick, ->] (A1)--(B1);
\draw[thick, ->] (A2)--(B3); \draw[thick, ->] (A3)--(B3);
\node at (1.5,-3) {$\textrm{сюр'єкція}$};
\end{tikzpicture}

\qquad
\begin{tikzpicture}
\fill[red!40] (0,0) ellipse (1cm and 2cm);
\fill[blue!40] (3,0) ellipse (1cm and 2cm);
\node (A1) at (0,0.8) [circle,fill,inner sep=1.5pt]{};
\node (A2) at (0,0) [circle,fill,inner sep=1.5pt]{};
\node (A3) at (0,-0.8) [circle,fill,inner sep=1.5pt]{};

\node (B1) at (3+0,0.8) [circle,fill,inner sep=1.5pt]{};
\node (B2) at (3+0,0) [circle,fill,inner sep=1.5pt]{};
\node (B3) at (3+0,-0.8) [circle,fill,inner sep=1.5pt]{};
\node[anchor = south east] at (0,1) {$X$};
\node[anchor = south east] at (3,1) {$Y$};

\draw[thick, ->] (A1)--(B1);
\draw[thick, ->] (A2)--(B2); \draw[thick, ->] (A3)--(B3);
\node at (1.5,-3) {$\textrm{бієкція}$};
\end{tikzpicture}
}
\end{figure}
	\textbf{Proposition 3.1.8.} Відображення є бієкцією $\iff$ є одначно сюр'єкцією та ін'єкцією\\
	\textbf{Proof.} \\
	$f: X \to Y$\\
	$\boxed{\Rightarrow}$ Дано: $f$ - бієкція, тобто $\forall y \in Y: \exists! x: f(x) = y \Rightarrow $\\
	1) $\forall y \in Y: \exists x: f(x) = y$ - сюр'єкція\\
	2) Якщо при $x_1 \neq x_2$ вважати, що $f(x_2) = f(x_1) = y$, то це суперечить умові бієкції. Тому $f(x_1) \neq f(x_2)$ - ін'єкція\\
	Отже, $f$ - одночасно сюр'єкція та ін'єкція\\
	\\
	$\boxed{\Leftarrow}$ Дано: $f$ - одночасно сюр'єкцієя та ін'єкція\\
	Візьмемо довільне $y_0 \in Y$. Тоді $\exists x_1, x_2: f(x_1) = f(x_2) = y_0$. Суперечить ін'єкції. Тоді $\exists! x: f(x) = y$ - бієкція \qed
	\\
	\\
	\textbf{Definition 3.1.9.} Відображення $f: X \to Y$, $g: Y \to X$ називаються \textbf{взаємно оберненою}, якщо:
	\begin{align*}
	\forall x \in X: g(f(x)) = x \\
	\forall y \in Y: f(g(y)) = y
	\end{align*}
	Позначення: $g = f^{-1}$\\
\begin{figure}[H]
\centering {
\begin{tikzpicture}
\fill[red!40] (0,0) ellipse (1cm and 2cm);
\fill[blue!40] (4,0) ellipse (1cm and 2cm);

\node[anchor = south east] at (0,1) {$X$};
\node[anchor = south east] at (4,1) {$Y$};

\draw[thick, ->](0.5,0) .. controls (2,1) .. (4,0) node at (2,1.2) {$f$};
\draw[thick, ->](3.5,0) .. controls (2,-1) .. (0,0) node at (2,-1.2) {$f^{-1}$};

\end{tikzpicture}
}
\end{figure}
	\textbf{Example 3.1.10.}\\
	1) $f: \mathbb{R} \to \mathbb{R}$:\\
	$f(x) = x^3, g(x) = \sqrt[3]{x}$\\ \\
	2) $f: \mathbb{R} \to \mathbb{R}$:\\
	$f(x) = x^2$, але $g(x) = \sqrt{x}$ бути не може\\
	\\
	\textbf{Proposition 3.1.11.} 
	Функції $f,g$ - взаємно обернені $\iff$ вони є бієкціями\\
	\textbf{Proof.}\\
	$\boxed{\Rightarrow}$ Дано: $f(g(y)) = y, \forall y$\\
	Тобто $\forall y \in Y: \exists x = g(y): f(x) = f(g(y)) = y$. Отже, $f$ - сюр'єкція\\
	Дано $x_1 \neq x_2$ і нехай $f(x_1) = f(x_2) = y_0$.\\
	Тоді $g(y_0) = g(f(x_1)) = x_1$ та $g(y_0) = g(f(x_2)) = x_2$ і вони рівні. Отже, суперечення\\
	Тоді $x_1 \neq x_2 \Rightarrow f(x_1) \neq f(x_2)$. Отже, $f$ - ін'єкція\\
	З $g$ все аналогічно
	\bigline
	$\boxed{\Leftarrow}$ Дано: $f,g$ - бієкції\\
	Розглянемо функцію $f: X \to Y$. Визначимо функцію $g: Y \to X$ так, що $x = g(y)$ \qed
	\\
	\\
	\textbf{Definition 3.1.12.} Задано відображення $f: X \to Y$\\
	\textbf{Образом} множини $X_0 \subset X$ називається множина
	\begin{align*}
	f(X_0) = \{f(x) \in Y: x \in X_0 \}
	\end{align*}
	\textbf{Повним прообразом} множини $Y_0 \subset Y$ називається множина
	\begin{align*}
	f^{-1}(Y_0) = \{x \in X:  f(x) \in Y_0 \}
	\end{align*}
	\textbf{Example 3.1.13.} $f: \mathbb{R} \to \mathbb{R}$: $f(x) = x^2$\\
	$A = [-5, 4)$\\
	$\Rightarrow f(A) = \{f(x) = x^2: x \in [-5, 4) \} = [0, 25]$\\
	$\Rightarrow f^{-1}(A) = \{x: f(x) = x^2 \in [-5, 4) \} \overset{x^2 < 4}{=} (-2, 2)$\\
	\\
	\textbf{Proposition 3.1.14. Властивості повних прообразів}\\
	$1) f^{-1}(A \cup B) = f^{-1}(A) \cup f^{-1}(B)$\\
	$2) f^{-1}(A \cap B) = f^{-1}(A) \cap f^{-1}(B)$\\
	$3) f^{-1}(\overline{A}) = \overline{f^{-1}(A)}$\\
	\textit{Випливає з теорії множин}\bigline
	\textbf{Remark 3.1.14.} Властивість образів не часто співпадають:\\
	$f(A \cap B) \neq f(A) \cap f(B)$\\
	
	
	\subsection{Границі функції}
	\defin{3.2.1.} Задана функція $f: A \to \mathbb{R}$ та $x_0 \in A$ - гранична точка\\
	Число $b$ називається \textbf{границею функції в т.} $x_0$, якщо:
	\begin{align*}
	\forall \varepsilon > 0: \exists \delta(\varepsilon) > 0: \forall x \in A: x \neq x_0: |x-x_0|<\delta \Rightarrow |f(x)-b|<\varepsilon \textrm{ - def. Коші}
	\end{align*}
	\begin{align*}
	\forall \{x_n, n \geq 1\}\subset A: x_n \neq x_0: \forall n \geq 1: \lim_{n \to \infty} x_n = x_0 \Rightarrow \lim_{n \to \infty} f(x_n) = b \textrm{ - def. Гейне}
	\end{align*}
	Позначення: $\huge \lim_{x \to x_0} f(x) = b$ \bigline
	\th{3.2.2.} Означення Коші $\iff$ Означення Гейне\\
	\proof
	$\boxed{\Rightarrow}$ Дано: означення Коші, тобто\\
	$\forall \varepsilon > 0: \exists \delta > 0: \forall x \in A: x \neq x_0: |x-x_0|<\delta \Rightarrow |f(x)-b|<\varepsilon$\\
	Зафіксуємо послідовність $\{x_n, n \geq 1\} \subset A$ таку, що:\\
	$\forall n \geq 1: x_n \neq x_0: \huge \lim_{n \to \infty} x_n = x_0$\\
	На це ми мали права, оскільки $x_0$ - гранична точка $A$\\
	Тоді для нашого заданого $\delta: \exists N: \forall n \geq N: |x_n - x_0| < \delta$\\
	$\Rightarrow \forall \varepsilon > 0: \exists N: \forall n \geq N: |f(x_n) - b| < \varepsilon$\\
	Таким чином, $\huge \lim_{n \to \infty} f(x_n) = b$ - означення Гейне
	\bigline
	$\boxed{\Leftarrow}$ Дано: означення Гейне, тобто\\
	$\huge \forall \{x_n, n \geq 1\}\subset A: x_n \neq x_0: \forall n \geq 1: \lim_{n \to \infty} x_n = x_0 \Rightarrow \lim_{n \to \infty} f(x_n) = b$\\
	!Припустимо, що означення Коші не виконується, тобто\\
	$\exists \varepsilon^*>0: \forall \delta > 0: \exists x_{\delta} \in A: x_{\delta} \neq x_{0}: |x_{\delta} - x_0| < \delta \Rightarrow |f(x_{\delta}) - b| \geq \varepsilon^*$\\
	Зафіксуємо $\delta = \huge \frac{1}{n}$. Тоді побудуємо послідовність $\{x_n, n \geq 1\}$ таким чином, що $x_n \in A: |x_n-x_0| < \huge \frac{1}{n} \Rightarrow \exists \lim_{x \to \infty} x_n = x_0$ за теоремою про поліцаї, але водночас $|f(x_n) - b| \geq \varepsilon^*$\\
	Отже, суперечність! \qed
	\bigline
	\rm{3.2.1.} Границя функції має єдине значення\\
	\textit{Випливає з означення Гейне, оскільки границя числової послідовності є єдиною}
	\bigline
	\ex{3.2.3.(1)} Довести, що $\huge \lim_{x \to 2} x^2 = 4$\\
	За означенням Коші,\\
	$\forall \varepsilon > 0: \exists \delta > 0: \forall x: |x-2|<\delta \Rightarrow |x^2-4|<\varepsilon$\\
	$|x^2-4| = |x-2||x+2| \boxed{<}$\\
	Нехай $|x-2| < 1$. Тоді $-1<x-2<1 \Rightarrow |x+2|<5$\\
	$\boxed{<} 5|x-2|<\varepsilon$\\
	Якщо вказати $\delta = \huge \min \left\{1, \frac{\varepsilon}{5} \right\}$, то тоді наше означення Коші буде виконаним \qed
\begin{figure} [H]
\centering
\resizebox{0.4\textwidth}{!}
{
\begin{tikzpicture}

\draw[thick, ->] (-1,0)--(4,0) node[anchor = north] {$x$};
\draw[thick, ->] (0,-1)--(0,7.25) node[anchor = east] {$y$};

\draw[thick, domain=-0.5:2.5, variable=\x] plot({\x}, {\x*\x}) node[anchor = west, scale = 0.7] {$f(x) = x^2$};
\draw (2 cm, 1pt) -- (2 cm, -1pt) node[anchor = north] {$2$};
\draw (1 pt, 4cm) -- (-1 pt, 4cm) node[anchor = east] {$4$};
\draw[thick, blue, dashed] (2.2,2.2*2.2)--(2.2,0) node[anchor = north west, scale=0.9] {$2+\delta$};
\draw[thick, blue, dashed] (1.8,1.8*1.8)--(1.8,0) node[anchor = north east, scale=0.9] {$2-\delta$};
\draw[thick, red, dashed] (1.8,1.8*1.8)--(0,1.8*1.8) node[anchor = east, scale=0.9] {$4-\varepsilon$};
\draw[thick, red, dashed] (2.2,2.2*2.2)--(0,2.2*2.2) node[anchor = east, scale=0.9] {$4+\varepsilon$};
\draw[thick, dashed] (2,0)--(2,4);
\draw[thick, dashed] (2,4)--(0,4);
\node[white] at (2,4) [circle,fill,inner sep=1.5pt, draw = black]{};

\end{tikzpicture}

}
\captionsetup{justification=centering}
\caption*{Схематично це виглядає ось так}
\end{figure}
	\ex{3.2.3.(2)} Довести, що не існує границі $\huge \lim_{x \to 0} \arctg \frac{1}{x}$\\
	За означенням Гейне, зафіксуємо наступну послідовність:\\
	$\huge \left\{x_n = \frac{(-1)^n}{n}, n \geq 1\right\}$, де $\huge \lim_{n \to \infty} x_n = 0$\\
	Але $\huge \lim_{n \to \infty} \arctan \frac{1}{x_n} = \left[ \begin{gathered} \frac{\pi}{2}, n = 2k \\ -\frac{\pi}{2}, n = 2k-1 \end{gathered} \right.$ - не збіжна\\
	Таким чином, прийшли до висновку: границі не існує \qed
	\bigline
	\th{3.2.4. Властивості границь функції}\\
	1) Задана функція $f: A \to \mathbb{R}$, що містить границю навколо т. $x_0$. Тоді вона є обмеженою в околі т. $x_0$\\
	\proof
	$\exists \huge \lim_{x \to x_0} f(x) = b \Rightarrow \forall \varepsilon > 0: \exists \delta: \forall x \in A: |x-x_0|<\delta \Rightarrow |f(x)-b|<\varepsilon$\\
	Зафіксуємо $\varepsilon = 1$, тоді $|f(x) -b| < 1$\\
	$|f(x)| = |f(x) - b + b| \leq |f(x) - b| + |b| < 1 + |b|$\\
	Покладемо $c = \max\{1 + |b|, f(x_0) \}$. А тому отримаємо:\\
	$\forall x \in A: |x-x_0| < \delta \Rightarrow |f(x)| < c$. Отже, обмежена \qed
	\bigline
	Задані функції $f,g: A \to \mathbb{R}$, такі, що $\exists \huge \lim_{x \to x_0} f(x) = b_1$, $\exists \huge \lim_{x \to x_0} g(x) = b_2$. Тоді:\\
	$2.1) \forall c \in \mathbb{R}: \exists \huge \lim_{x \to x_0} cf(x) = c b_1$\\
	$2.2) \exists \huge \lim_{x \to x_0} (f(x)+g(x)) = b_1 + b_2$\\
	$2.3) \exists \huge \lim_{x \to x_0} f(x)g(x) = b_1 b_2$\\
	$2.4) \exists \huge \lim \frac{f(x)}{g(x)} = \frac{b_1}{b_2}$ при $b_2, g(x) \neq 0$\\
	\textit{Випливають з властивостей границь числової послідовності, якщо доводити за Гейне. Доведу лише перший підпункт}\\
	\proof
	$\forall \{x_n, n \geq 1\} \subset A: x_n \neq x_0: \forall n \geq 1: \huge \lim_{n \to \infty} x_n = x_0 \Rightarrow \lim_{n \to \infty} f(x_n) = b$\\
	Тоді $\huge \forall c \in \mathbb{R}: \lim_{n \to \infty} c f(x_n) = c \lim_{n \to \infty} f(x_n) = c b_1$\\
	Таким чином, $\exists \huge \lim_{x \to x_0} cf(x) = c b_1$ \qed
	\bigline
	
	\ex{3.2.5.} Обчислити границю: $\huge \lim_{x \to 0} \frac{x^2-1}{2x^2-2x-1}$\\
	$\huge \lim_{x \to 0} \frac{x^2-1}{2x^2-x-1} = \frac{\huge \lim_{x \to 0} (x^2-1)}{\huge \lim_{x \to 0}(2x^2-x-1)} = \frac{\huge \lim_{x \to 0}x^2 - \lim_{x \to 0}1}{\huge 2\lim_{x \to 0}x^2 - \lim_{x \to 0}x - \lim_{x \to 0}1} = \frac{0-1}{0-0-1} = \\ = 1$ \qed
	\bigline
	\defin{3.2.6.} Задана функція $f: A \to \mathbb{R}$ та $x_0 \in A$ - гранична точка\\
	Функція \textbf{прямує до нескінченності в т.} $x_0$, якщо:
	\begin{align*}
	\forall E > 0: \exists \delta(E) > 0: \forall x \in A: x \neq x_0: |x-x_0|<\delta \Rightarrow |f(x)|>E \textrm{ - def. Коші}
	\end{align*}
	\begin{align*}
	\forall \{x_n, n \geq 1\}\subset A: x_n \neq x_0: \forall n \geq 1: \lim_{n \to \infty} x_n = x_0 \Rightarrow \lim_{n \to \infty} f(x_n) = \infty \textrm{ - def. Гейне}
	\end{align*}
	Позначення: $\huge \lim_{x \to x_0} f(x) = \infty$\bigline
	\defin{3.2.7.} Задана функція $f: A \to \mathbb{R}$ та $x_0 \in A$ - гранична точка \\
	Якщо $\huge \lim_{x \to x_0} f(x) = \infty$, то функцію $f(x)$ називають \textbf{нескінченно великою в т.} $x_0$, або \textbf{н.в.}\\
	Якщо $\huge \lim_{x \to x_0} f(x) = 0$, то функцію $f(x)$ називають \textbf{нескінченно малою в т.} $x_0$, або \textbf{н.м.}\\
	\bigline
	\th{3.2.8. Арифметичні властивості н.м. та н.в. великих функцій}\\
	Задані функції $f,g,h: A \to \mathbb{R}$ - відповідно н.м., н.в., обмежена, та $x_0 \in A$ - гранична точка. Тоді:\\
	1) $f(x) \cdot h(x)$ - н.м.\\
	2) $\huge \frac{1}{f(x)}$ - н.в.\\
	3) $\huge \frac{1}{g(x)}$ - н.м.\\
	\proof
	Зафіксуємо $\{x_n, n \geq 1\}$, таку, що $\huge \lim_{n \to \infty} x_n = x_0$. Тоді за Гейне, \\ $\huge \lim_{n \to \infty} f(x_n) = 0$, $\huge \lim_{n \to \infty} g(x_n) = \infty$, отже\\
	$\{f(x_n), n \geq 1\}$ - н.м.\\
	$\{g(x_n), n \geq 1\}$ - н.в.\\
	$\{h(x_n), n \geq 1\}$ - досі обмежена\\
	За властивостями границь числової послідовності, $\left\{f(x_n) \cdot h(x_n) \right\}$ - н.м., $\left\{ \huge \frac{1}{f(x_n)} \right\}$ - н.в., $\left\{ \huge \frac{1}{g(x_n)} \right\}$ - н.м.\\
	Ну а тому, існують відповідні границі: $\huge \lim_{n \to \infty} f(x_n) h(x_n) = 0$, \\ $\huge \lim_{n \to \infty} \frac{1}{f(x_n)} = \infty$, $\huge \lim_{n \to \infty} \frac{1}{g(x_n)} = 0$\\
	За гейне, отримаємо бажане \qed
	\bigline

	\ex{3.2.9.} Знайти границю $\huge \lim_{x \to \infty} \frac{(x-1)(x-2)(x-3)}{(4x-5)^3}$\\
	Завдяки щойно доведеної теореми, ми отримаємо наступне:\\
	$\huge \lim_{x \to \infty} \frac{(x-1)(x-2)(x-3)}{(4x-5)^3} = \lim_{x \to \infty} \frac{(1-\frac{1}{x})(1-\frac{2}{x})(1-\frac{3}{x})}{(4-\frac{5}{x})^3} = \frac{1}{64}$ \qed
	\bigline
	
	\defin{3.2.10.} Задана функція $f: \mathbb{R} \to \mathbb{R}$ \\
	Число $b$ називається \textbf{границею функції} при $x \to \infty$, якщо:
	\begin{align*}
	\forall \varepsilon > 0: \exists \Delta(\varepsilon) > 0: \forall x \in \mathbb{R}: |x|>\Delta \Rightarrow |f(x)-b|<\varepsilon \textrm{ - def. Коші}
	\end{align*}
	\begin{align*}
	\forall \{x_n, n \geq 1\} \subset \mathbb{R}: \forall n \geq 1: \lim_{n \to \infty} x_n = \infty \Rightarrow \lim_{n \to \infty} f(x_n) = b \textrm{ - def. Гейне}
	\end{align*}
	Позначення: $\huge \lim_{x \to \infty} f(x) = b$ \bigline
	\rm{3.2.10.(1)} Можна спробувати самостійно записати def. Коші та def. Гейне для випадку $\huge \lim_{x \to \infty} f(x) = \infty$\\
	\rm{3.2.10.(2)} Для інших варіації границь функції, еквівалентність двох означень залишається в силі
	\bigline
	\th{3.2.11.} Задана функція $f: \mathbb{A} \to \mathbb{R}$ та $x_0 \in A$ - гранична точка\\
Відомо, що в околі т. $x_0$ функція $f(x) < c$ та $\exists \huge \lim_{x \to x_0} f(x) = b$. Тоді $b \leq c$\\
\proof
За Гейне, $\huge \forall \{x_n, n \geq 1\} \subset A:  \lim_{n \to \infty} x_n = x_0 \Rightarrow \lim_{n \to \infty} f(x_n) = b$. За властивостями границь числової послідовності, $b \leq c$ \qed
\bigline
\crl{3.2.11.} Задані функції $f,g: A \to \mathbb{R}$ такі, що в околі т. $x_0$ справедлива $f(x) \leq g(x)$. Також $\exists \huge \lim_{x \to x_0} f(x) = b_1$, $\exists \huge \lim_{x \to x_0} g(x) = b_2$. Тоді $b_1 \leq b_2$\\
\textit{Вказівка: розглянути функцію} $h(x) = f(x) - g(x)$
\bigline
\th{3.2.12. Теорема про 3 функції}\\
Задані функції $f,g,h: A \to \mathbb{R}$ та $x_0 \in A$ - гранична точка\\
Відомо, що в околі т. $x_0$: $f(x) \leq g(x) \leq h(x)$ та \\ $\exists \huge \lim_{x \to x_0} f(x) = \lim_{x \to x_0} h(x) = a$\\
Тоді $\exists \huge \lim_{x \to x_0} g(x) = a$\\
\textit{Випливає з теореми про поліцаїв в числової послідовності}
\bigline
\th{3.2.13. Критерій Коші}\\
Задана функція $f: A \to \mathbb{R}$ та $x_0 \in A$ - гранична точка\\
$\exists \huge \lim_{x \to x_0} f(x) \iff \forall \varepsilon > 0: \exists \delta(\varepsilon): \forall x_1,x_2 \in A: x_1,x_2 \neq x_0:$\\
$\begin{cases} |x_1-x_0|<\delta \\ |x_2-x_0|<\delta \end{cases} \Rightarrow |f(x_1)-f(x_2)|<\varepsilon
$\\
\proof
$\boxed{\Rightarrow}$ Дано: $\exists \huge \lim_{x \to x_0} f(x) = b$, тобто за def. Коші,\\
$\forall \varepsilon > 0: \exists \delta: \forall x \in A: x \neq x_0: |x-x_0|<\delta \Rightarrow |f(x)-b|< \huge \frac{\varepsilon}{2}$\\
Тоді $\forall x_1, x_2 \in A: |x_1 - x_0| < \delta$ і одночачно $|x_2 - x_0| < \delta \Rightarrow$\\
$|f(x_1)-f(x_2)| = |f(x_1)-b + b - f(x_2)| \leq |f(x_1) - b| + |f(x_2)-b| < \varepsilon$\\
Отримали праву частину критерія
\bigline
$\boxed{\Leftarrow}$ Дано: $\forall \varepsilon > 0: \exists \delta(\varepsilon): \forall x_1,x_2 \in A:, x_1,x_2 \neq x_0:$
$\begin{cases} |x_1-x_0|<\delta \\ |x_2-x_0|<\delta \end{cases} \Rightarrow |f(x_1)-f(x_2)|<\varepsilon$\\
Розглянемо послідовність $\{t_n, n \geq 1\}$, таку, що $\huge \lim_{n \to \infty} t_n = x_0$\\
Тоді за означенням, $\exists N: \forall n,m \geq N: \begin{cases} |t_n-x_0|<\delta \\ |t_m-x_0|<\delta \end{cases} \\ \Rightarrow |f(t_n)-f(t_m)|<\varepsilon$\\
Отримаємо, що $\{f(t_n),n \geq 1\}$ - фундаментальна послідовність, а тому є збіжною, тобто\\
$\exists \huge \lim_{n \to \infty} f(t_n) = b$\\
Розглянемо послідовність $\{s_n, n \geq 1\}$, таку, що $\huge \lim_{n \to \infty} s_n = x_0$\\
Тоді за аналогічними міркуваннями, $\exists \huge \lim_{n \to \infty} f(s_n) = a$\\
Оскільки критерій Коші в границях числової послідовності лише визнає збіжність, то ми не знаємо, куди вона прямує чисельно
\bigline
І нарешті, побудуємо послідовність $\{p_n, n \geq 1\}$ таким чином, що $p_{2k} = t_k$, $p_{2k-1} = s_k$. Тобто $\{s_1, t_1, s_2, t_2, \dots \}$\\
Тут $\exists \huge \lim_{n \to \infty} p_n = x_0$. Тоді знову за аналогічними міркуваннями, $\exists \huge \lim_{n \to \infty} f(p_n)$, але чому буде дорівнювати, зараз побачимо\\
Оскільки $\exists \huge \lim_{n \to \infty} f(p_n)$, то одночасно $\exists \huge \lim_{k \to \infty} f(p_{2k}) = b$, $\exists \huge \lim_{k \to \infty} f(p_{2k-1}) = a$\\
У збіжної послідовності є лише одна часткова послідовність, тому $a = b$\\
Це означає, що результат не залежить від вибору послідовності\\
Тому за Гейне, отримаємо, що $\exists \huge \lim_{x \to x_0} f(x) = b$ \qed
\bigline
\th{3.2.14. Границя від композиції функції}\\
Задані функції $f: A \to B$, $g: B \to \mathbb{R}$ та композиція $h = g(f(x))$. Більш того, $x_0 \in A$ - гранична точка, $\exists \huge \lim_{x \to x_0} f(x) = y_0$ та $\exists \huge \lim_{y \to y_0} g(y) = b$\\
Тоді $\exists \huge \lim_{x \to x_0} h(x) = b$\\
\proof
$\exists \huge \lim_{y \to y_0} g(y) = b \overset{\textrm{def.}}{\Rightarrow} \forall \varepsilon > 0: \exists \delta: \forall y \in B: |y-y_0|<\delta \Rightarrow |g(y)-b|<\varepsilon$\\
$\exists \huge \lim_{x \to x_0} f(x) = y_0 \overset{\textrm{def.}}{\Rightarrow} \forall \delta > 0: \exists \tilde{\delta}: \forall x \in A: |x-x_0|<\tilde{\delta} \Rightarrow |f(x)-y_0|<\delta$\\
Таким чином, можемо отримати:\\
$\forall \varepsilon > 0: \exists \delta > 0 \Rightarrow \exists \tilde{\delta}: \forall x \in A: |x-x_0| < \tilde{\delta} \Rightarrow \\ |f(x)-y_0| = |y-y_0|<\delta \Rightarrow |g(y)-b|=|g(f(x))-b| = |h(x)-b|<\varepsilon$\\
Отже, $\exists \huge \lim_{x \to x_0} h(x) = b$ \qed
\bigline

\subsection{Перша чудова границя}
Розглянемо наступний геометричний малюнок:\\
\begin{figure}[H]
\centering
\resizebox{0.5\textwidth}{!} {
\begin{tikzpicture}
\fill[thick, fill = blue!40] (1,0) arc (0:45:1cm) -- (0,0) -- (1,0) -- cycle node[anchor = south east] {$\alpha$};
\draw[thick, ->] (-1.5*3cm,0)--(1.5*3cm,0) node[anchor = west] {$x$};
\draw[thick, ->] (0,-1.5*3cm)--(0,1.5*3cm) node[anchor = east] {$y$};
\draw[thick] (0,0) circle (1*3cm);
\draw[thick] (0,0)--(1*3,1*3) node[anchor = south west] {$K$};
\draw[thick] (1*3,1*3)--(1*3,0) node[anchor = south west] {$C$};
\draw[thick] ({cos(45)*3},{sin(45)*3})--({cos(45)*3},0) node[anchor = north] {$B$};
\node[anchor = north west] at (0,0) {$0$};
\node[anchor = south] at ({cos(45)*3},{sin(45)*3}) {$A$};
\draw (1pt, 1*3 cm) -- (-1pt, 1*3 cm) node[anchor = south east] {$1$};
\draw (1*3 cm, 1pt) -- (1*3 cm, -1pt) node[anchor = north west] {$1$};
\end{tikzpicture}
}
\captionsetup{justification=centering}
\caption*{Коло радіусом $1$}
\end{figure}
	Виділимо з малюнку наступні дані:\\
	$|AB|=\sin \alpha$\\
	$|AC|=\alpha$\\
	$|KC|=\tg \alpha$\\
	Зрозуміло, що $|AB|<|AC|<|KC| \Rightarrow \\ \sin \alpha < \alpha < \tg \alpha$\\
	Розглянемо обидва сторони:\\
	$\displaystyle \sin \alpha < \alpha \Rightarrow \frac{\sin \alpha}{\alpha} < 1$\\
	$\displaystyle \alpha < \tg \alpha = \frac{\sin \alpha}{\cos \alpha} \Rightarrow \frac{\sin \alpha}{\alpha} > \cos \alpha = 1-2 \sin^2 \frac{\alpha}{2} > 1 - 2 \frac{\alpha^2}{4} = 1 - \frac{\alpha ^2}{2}$\\
	$\displaystyle 1- \frac{\alpha^2}{2}<\frac{\sin \alpha}{\alpha} < 1$\\
Можна розширити інтервал до $\huge \left(-\frac{\pi}{2},\frac{\pi}{2} \right)$\\
Тому за теоремою про 3 функції, маємо наступне:\\
\th{3.3.1. Перша чудова границя} \\ $\huge \lim_{x \to 0} \frac{\sin x}{x} = 1$
\bigline
\crl{3.3.1.(1)} $\huge \lim_{x \to 0} \frac{\tg x}{x} = 1$\\
\crl{3.3.1.(2)} $\huge \lim_{x \to 0} \frac{\arcsin x}{x} = 1$\\
\proof
$\huge \lim_{x \to 0} \frac{\arcsin x}{x} \boxed{=} $\\
Проведемо заміну: $\arcsin x = t$, тобто $x = \sin t$. Оскільки $x \to 0$, то $t \to 0$. Тоді за теоремою про границю композиції, рівність буде справедливою\\
$\boxed{=} \huge \lim_{t \to 0} \frac{t}{\sin t} = 1$ \qed \\
\crl{3.3.1.(3)} $\huge \lim_{x \to 0} \frac{\arctg x}{x} = 1$\bigline

\subsection{Друга чудова границя}
Відомо, що $\forall x \in \mathbb{R}$ справедлива нерівність: $[x] \leq x < [x]+1$\\
Тоді можна дійти до цієї нерівності:\\
$\huge \left(1 + \frac{1}{[x]+1} \right)^{[x]} < \left(1 + \frac{1}{x} \right)^x < \left(1 + \frac{1}{[x]} \right)^{[x]+1}$\\
Вважаємо, що $x \to +\infty$, тоді відповідно $[x] \to + \infty$ та $[x]+1 \to + \infty$\\
Також $[x] \in \mathbb{N}$, тому за визначенням числа Ейлера,\\
$\huge \lim_{[x] \to +\infty} \left(1 + \frac{1}{[x]} \right)^{[x]} = e$\\
Скористаємось цим фактом в нашої нерівності:\\
$\huge \left(1 + \frac{1}{[x]+1} \right)^{[x]} = \frac{\huge \left(1 + \frac{1}{[x]+1} \right)^{[x]+1}}{\huge 1 + \frac{1}{[x]+1}} \to \frac{e}{1} = e$\\
$\huge \left(1 + \frac{1}{[x]} \right)^{[x]+1} = \left(1 + \frac{1}{[x]} \right)^{[x]} \left(1 + \frac{1}{[x]} \right) \to e \cdot 1 = e$\\
І це все при $x \to +\infty$. Тоді за теоремою про поліцаїв, отримаємо так звану ще одну чудову границю\\
\th{3.4.1. Друга чудова границя} \\ $\huge \lim_{x \to +\infty} \left(1 +\frac{1}{x} \right)^x = e$
\bigline
\crl{3.4.1.(1)} $\huge \lim_{x \to -\infty} \left(1 +\frac{1}{x} \right)^x = e$\\
\crl{3.4.1.(2)} $\huge \lim_{x \to 0} \left(1 +x \right)^{\textstyle \frac{1}{x}} = e$\\
\crl{3.4.1.(3)} $\huge \lim_{x \to 0} \frac{\ln(1+x)}{x} = 1$\\
\crl{3.4.1.(4)} $\huge \lim_{x \to 0} \frac{e^x - 1}{x} = 1$\\
\crl{3.4.1.(5)} $\huge \lim_{x \to 0} \frac{(1+x)^\alpha - 1}{x} = \alpha$\\
\textit{Вказівка: $1 + x = e^t$}\\

\subsection{Односторонні границі та границі монотонних функцій}
\defin{3.5.1.} Задана функція $f: A \to \mathbb{R}$, та $x_0 \in A$ - гранична точка\\
Числом $b$ називають \textbf{границею справа}, якщо
\begin{align*}
\forall \varepsilon > 0: \exists \delta(\varepsilon)>0: \forall x \in A: x \neq x_0: 0<x-x_0<\delta \Rightarrow |f(x)-b|<\varepsilon \textrm{ - def. Коші}\\
\forall \{x_n,n\geq 1\} \subset A: \forall n \geq 1: x_n > x_0: \lim_{n \to \infty}x_n = x_0 \Rightarrow \lim_{n \to \infty}f(x_n) = b \textrm{ - def. Гейне}
\end{align*}
Позначення: $\huge \lim_{x \to x_0^+} f(x) = b$\\
Числом $\tilde{b}$ називають \textbf{границею зліва}, якщо
\begin{align*}
\forall \varepsilon > 0: \exists \delta(\varepsilon)>0: \forall x \in A: x \neq x_0: 0<x_0-x<\delta \Rightarrow |f(x)-\tilde{b}|<\varepsilon \textrm{ - def. Коші}\\
\forall \{x_n,n\geq 1\} \subset A: \forall n \geq 1: x_n < x_0: \lim_{n \to \infty}x_n = x_0 \Rightarrow \lim_{n \to \infty}f(x_n) = \tilde{b} \textrm{ - def. Гейне}
\end{align*}
Позначення: $\huge \lim_{x \to x_0^-} f(x) = \tilde{b}$
\bigline
\th{3.5.2.} Задана функція $f: A \to \mathbb{R}$, та $x_0 \in A$ - гранична точка\\
$\exists \huge \lim_{x \to x_0} f(x) = b \iff \exists \begin{cases} \huge \lim_{x \to x_0^+} f(x) = b \\ \huge \lim_{x \to x_0^-} f(x) = b \end{cases}$\\
\proof
$\exists \huge \lim_{x \to x_0} f(x) = b \iff$
$\forall \varepsilon > 0: \exists \delta: \forall x \in A: |x-x_0|<\delta \Rightarrow |f(x)-b|<\varepsilon$\\
$\iff \forall \varepsilon > 0: \exists \delta: \forall x \in A: |x-x_0|<\delta \Rightarrow \begin{cases} x-x_0<\delta \\ x_0-x<\delta \end{cases} \Rightarrow |f(x)-b|<\varepsilon$ \\ $\iff \exists \begin{cases} \huge \lim_{x \to x_0^+} f(x) = b \\ \huge \lim_{x \to x_0^-} f(x) = b \end{cases}$ \qed \bigline
\defin{3.5.3.} Задана функція $f: (a,b) \to \mathbb{R}$\\
Її називають \textbf{монотонно}:\\
\textbf{- строго зростаючою}, якщо $\forall x_1,x_2 \in (a,b): x_1 > x_2 \Rightarrow f(x_1)>f(x_2)$\\
\textbf{- не спадною}, якщо $\forall x_1,x_2 \in (a,b): x_1 > x_2 \Rightarrow f(x_1) \geq f(x_2)$\\
\textbf{- строго спадною}, якщо $\forall x_1,x_2 \in (a,b): x_1 > x_2 \Rightarrow f(x_1) < f(x_2)$\\
\textbf{- не зростаючою}, якщо $\forall x_1,x_2 \in (a,b): x_1 > x_2 \Rightarrow f(x_1) \leq f(x_2)$
\bigline
Її називають \textbf{обмеженою}, якщо $\exists M>0: \forall x \in (a,b): |f(x)| \leq M$
\bigline
\th{3.5.4.} Задана функція $f: (a,b) \to \mathbb{R}$ - монотонна та обмежена\\
Тоді $\huge \exists \lim_{x \to b^-} f(x) = d$ або $\huge \exists \lim_{x \to a^+} f(x) = c$\\
\proof
Доведу лише першу границю і буду вважати, що функція строго спадна. Для решти аналогічно\\
\\
Отже, $f$ - строго спадає, тобто $\forall x_1,x_2 \in (a,b): x_1>x_2 \Rightarrow f(x_1)<f(x_2)$\\
Більш того, $f$ - обмежена, тому $\exists \huge \inf_{x \in (a,b)} f(x) = d$\\
Доведемо, що вона є границею ліворуч. За критерієм $\inf$:\\
1) $\forall x \in (a,b): f(x) \geq d$\\
2) $\forall \varepsilon > 0: \exists x_{\varepsilon} \in (a,b): f(x_{\varepsilon})<c + \varepsilon$\\
Оберемо $\delta = b - x_{\varepsilon} > 0$. Тоді $\forall x \in (a,b): b-x<\delta \Rightarrow \\ x > b - (b-x_{\varepsilon}) = x_{\varepsilon} \Rightarrow f(x) < f(x_\varepsilon)$\\
Звідси справедлива наступна нерівність:\\
$d - \varepsilon < d \leq f(x) < f(x_\varepsilon) < d + \varepsilon \Rightarrow |f(x)-d| < \varepsilon$\\
Остаточно, за def. Коші, $\exists \huge \lim_{x \to b^-} f(x) =  d$ \qed
\bigline
\subsection{Порівняння функцій, відношення О-велике, о-маленьке та еквівалентності}
\defin{3.6.1.} Задані функції $f,g: A \to \mathbb{R}$ та $x_0 \in A$ - гранична точка\\
Функція $f$ називається \textbf{порівнянною} з функцією $g$, якщо
\begin{align*}
\exists L>0: \exists \delta > 0: \forall x \in A: x \neq x_0: |x-x_0| < \delta \Rightarrow |f(x)| \leq L |g(x)|
\end{align*}
Позначення: $f(x) = O(g(x)), x \to x_0$\\
Інакше називають, що $f$ - \textbf{обмежена відносно} $g$ при $x \to x_0$\\
Чоловіча мова: "функція $f(x)$ наближається до т. $x_0$ не швидше за $g(x)$"
\bigline
\th{3.6.2. Властивості}\\
1) $f(x) = O(g(x)), x \to x_0 \iff \huge \frac{f(x)}{g(x)}$ - обмежена в околі т. $x_0$\\
\\
2) Якщо $\exists \huge \lim_{x \to x_0} \frac{f(x)}{g(x)} = c$, то $f(x) = O(g(x)), x \to x_0$\\
\\
3) Нехай $f_1(x) = O(g(x)), f_2(x) = O(g(x))$. Тоді:\\
a) $f_1(x) + f_2(x) = O(g(x))$\\
b) $\forall \alpha \in \mathbb{R}: \alpha f_1(x) = O(g(x))$\\
c) $\forall \alpha \neq 0: f_1(x) = O(\alpha g(x))$\\
Всюди $x \to x_0$\\
\\
4) Нехай $f(x) = O(g(x))$, $g(x) = O(h(x))$. Тоді $f(x) = O(h(x)), x \to x_0$\\
\\
\proof
Доведу лише 3 а). Інші очевидно\\
$f_1(x) = O(g(x)) \Rightarrow \exists L_1: \exists \delta_1: \forall x : |x-x_0| < \delta_1 \Rightarrow |f_1(x)| \leq L_1 |g(x)|$
$f_2(x) = O(g(x)) \Rightarrow \exists L_2: \exists \delta_2: \forall x : |x-x_0| < \delta_2 \Rightarrow |f_2(x)| \leq L_2 |g(x)|$\\
Тоді $\exists \delta = \min\{\delta_1, \delta_2 \}: \forall x: |x-x_0|<\delta \Rightarrow$\\
$|f(x_1)+f(x_2)| \leq |f(x_1)|+|f(x_2)| \leq (L_1+L_2)|g(x)|$\\
А тому $f_1(x) + f_2(x) = O(g(x))$ \qed
\bigline
\ex{3.6.3.} Довести, що $x+x^2 = O(x), x \to 0$\\
Знайдемо наступну границю:\\
$\huge \lim_{x \to 0} \frac{x+x^2}{x} = \lim_{x \to 0}(1+x) = 1$\\
Отже, $x+x^2 = O(x), x \to 0$ \qed
\bigline
\defin{3.6.4.} Задані функції $f,g: A \to \mathbb{R}$ та $x_0 \in A$ - гранична точка\\
Функція $f$ називається \textbf{знехтувально малою} відносно $g$, якщо
\begin{align*}
\forall \varepsilon>0: \exists \delta > 0: \forall x \in A: x \neq x_0: |x-x_0| < \delta \Rightarrow |f(x)| < \varepsilon |g(x)|
\end{align*}
Позначення: $f(x) = o(g(x)), x \to x_0$\\
Інакше кажуть, що $f$ - \textbf{нескінченно малой порівняльно з} $g$ при $x \to x_0$\\
Чоловіча мова: "функція $f(x)$ приймає одне значення в околі т. $x_0$ довше, ніж $g(x)$"
\bigline
\th{3.6.5. Властивості}\\
1) $f(x) = o(g(x)), x \to x_0 \iff \huge \exists \lim_{x \to x_0} \frac{f(x)}{g(x)} = 0$\\
\\
2) Нехай $f_1(x) = o(g(x)), f_2(x) = o(g(x))$. Тоді:\\
a) $f_1(x) + f_2(x) = o(g(x))$\\
b) $\forall \alpha \in \mathbb{R}: \alpha f_1(x) = o(g(x))$\\
c) $\forall \alpha \neq 0: f_1(x) = o(\alpha g(x))$\\
Всюди $x \to x_0$\\
\\
3) Нехай $f(x) = o(g(x))$, $g(x) = o(h(x))$. Тоді $f(x) = o(h(x)), x \to x_0$\\
\\
\proof
Доведу лише 1. Інші очевидно\\
$f(x) = o(g(x)),x \to x_0 \iff \forall \varepsilon>0: \exists \delta: \forall x \in A: |x-x_0| < \delta \Rightarrow |f(x)| < \varepsilon |g(x)| \iff \huge \abs{\frac{f(x)}{g(x)} - 0} < \varepsilon \iff \exists \lim_{x \to x_0} \frac{f(x)}{g(x)} = 0$ \qed
\bigline
\ex{3.6.6.} Довести, що $x^3 - x^2 - x + 1 = o(x-1), x \to 1$\\
Знайдемо наступну границю:\\
$\huge \lim_{x \to 1} \dfrac{x^3-x^2-x+1}{x-1} = \lim_{x \to 1} \dfrac{x^2(x-1)-(x-1)}{x-1} = \lim_{x \to 1} (x^2-1) = 0$\\
Отже, $x^3 - x^2 - x + 1 = o(x-1), x \to 1$ \qed
\begin{figure} [H]
\centering
{
\begin{tikzpicture}
\draw[thick, ->] (-1.2,0)--(4,0) node[anchor = north] {$x$};
\draw[thick, ->] (0,-2)--(0,5) node[anchor = east] {$y$};


\draw[thick, domain=-1.2:2.2, variable=\x, samples = 1000] plot({\x}, {(\x)^3 - (\x)^2 - \x + 1}) node[anchor = south west, scale = 0.7] {$f(x) = x^3 - x^2 - x + 1$};
\draw[thick, domain=-1:2.2, variable=\x, samples = 1000] plot({\x}, {\x - 1}) node[anchor = south west, scale = 0.7] {$g(x) = x - 1$};
\end{tikzpicture}
\caption*{Тут $x-1$ миттєво стала нулем і миттєво пішла далі. А $x^3-x^2-x+1$ набагато довше була близька в нулі}
}
\end{figure}

\th{3.6.7. Інші властивості}\\
1.1) Нехай $f(x) = o(g(x))$ та $g(x) = O(h(x))$. Тоді $f(x) = o(h(x)), x \to x_0$\\
1.2) Нехай $f(x) = O(g(x))$ та $g(x) = o(h(x))$. Тоді $f(x) = o(h(x)), x \to x_0$\\
\\
2) Нехай $f(x) = o(g(x))$. Тоді $f(x) = O(g(x)), x \to x_0$\\
\proof
1) для обох випадків\\
$\huge \lim_{x \to x_0} \frac{f(x)}{h(x)} = \lim_{x \to x_0} \frac{f(x)}{g(x)} \frac{g(x)}{h(x)} =$ (обм *н.м.) $= 0 \Rightarrow f(x) = o(h(x)), x \to x_0$
\bigline
2) \textit{Випливає з властивості 2 О-великого} \qed
\bigline
\defin{3.6.8.} Задані функції $f,g: A \to \mathbb{R}$ та $x_0 \in A$ - гранична точка\\
Функція $f$ називається \textbf{еквівалентною} $g$, якщо
\begin{align*}
f(x) - g(x) = o(g(x)), x \to x_0
\end{align*}
Позначення: $f(x) \sim g(x), x \to x_0$\\
Чоловіча мова: "функції $f(x)$ та $g(x)$ в околі т. $x_0$ дуже близькі між собою"
\bigline
\th{3.6.9.} $f(x) \sim g(x) \iff \huge \exists \lim_{x \to x_0} \frac{f(x)}{g(x)} = 1$\\
\proof
$f(x) \sim g(x), x \to x_0 \iff f(x) - g(x) = o(g(x)), x \to x_0 \iff \huge \exists \lim_{x \to x_0} \frac{f(x)-g(x)}{g(x)} = 0 \iff \lim_{x \to x_0} \frac{f(x)}{g(x)} = 1$
\bigline
\th{3.6.10. Граничний перехід}\\
Нехай $f_1(x) \sim g_1(x)$ та $f_2(x) \sim g_2(x)$, $x \to x_0$. Тоді:\\
1) $\huge \lim_{x \to x_0} f_1(x) f_2(x) = \lim_{x \to x_0} g_1(x) g_2(x)$\\
2) $\huge \lim_{x \to x_0} \frac{f_1(x)}{f_2(x)} = \lim_{x \to x_0} \frac{g_1(x)}{g_2(x)}$\\
\proof
З початкових умов, отримаємо за \textbf{Th. 3.6.9.}, що:\\
$\huge \exists \lim_{x \to x_0} \frac{f_1(x)}{g_1(x)} = 1$, $\huge \exists \lim_{x \to x_0} \frac{f_2(x)}{g_2(x)} = 1$. Тоді\\
1) $\huge \lim_{x \to x_0} f_1(x) f_2(x) = \lim_{x \to x_0} \frac{f_1(x) f_2(x) g_1(x) g_2(x)}{g_1(x) g_2(x)} = \lim_{x \to x_0} \frac{f_1(x) f_2(x)}{g_1(x) g_2(x)} \lim_{x \to x_0} g_1(x) g_2(x) = \\ = \lim_{x \to x_0} g_1(x) g_2(x)$\\
2) $\huge \lim_{x \to x_0} \frac{f_1(x)}{f_2(x)} = \lim_{x \to x_0} \frac{f_1(x)g_1(x)g_2(x)}{f_2(x)g_1(x)g_2(x)} = \lim_{x \to x_0} \frac{f_1(x)g_2(x)}{f_2(x)g_1(x)} \lim_{x \to x_0} \frac{g_1(x)}{g_2(x)} = \lim_{x \to x_0} \frac{g_1(x)}{g_2(x)}$ \qed
\bigline
\rm{3.6.10.} Еквівалентні функції задають відношення еквівалентності - рефлексивність, симетричність, транзитивність
\bigline
Використовуючи всі наслідки від чудових границь, ми можемо отримати наступні еквівалентні функції:\\
$x \to 0$
\begin{center}
\begin{tabular}{ c c }
 $\sin x \sim x$ & $\ln(1+x) \sim x$ \\ 
 $\tg x \sim x$ & $e^x - 1 \sim x$ \\
 $\arcsin x \sim x$ & $(1+x)^\alpha - 1 \sim \alpha x$ \\
 $\arctg x \sim x$ & $a^x - 1 \sim x \ln a$ \\ 
\end{tabular}
\end{center}

\ex{3.6.11.} Обчислити границю $\huge \lim_{x \to 0} \frac{\arcsin x \cdot (e^x - 1)}{1 - \cos x}$\\
Маємо, з таблиці еквівалентності:\\
$\huge \lim_{x \to 0} \frac{\arcsin x \cdot (e^x - 1)}{1 - \cos x} = \lim_{x \to 0} \frac{x \cdot x}{2 \sin^2 \frac{x}{2}} = \lim_{x \to 0} \frac{x \cdot x}{2 \frac{x^2}{4}} = 2$ \qed
\bigline
\rm{3.6.12.} Узагальнене зауваження:\\
$f(x) = O(1), x \to x_0 \iff f(x)$ - обмежена в околі т. $x_0$\\
$f(x) = o(1), x \to x_0 \iff f(x)$ - н.м. функція
\begin{figure} [H]
\centering
{
\begin{tikzpicture}
\draw[thick, ->] (-4,0)--(4.2,0) node[anchor = north] {$x$};
\draw[thick, ->] (0,-4)--(0,4.2) node[anchor = east] {$y$};


\draw[thick, domain=-4:4, variable=\x, samples = 1000] plot({\x}, {sin(deg(\x))}) node[anchor = north west, scale = 0.7] {$f(x) = \sin x$};
\draw[thick, domain=-4:4, variable=\x, samples = 1000] plot({\x}, {\x}) node[anchor = south west, scale = 0.7] {$g(x) = x$};
\end{tikzpicture}
\caption*{В околі т. $x_0 = 0$ функція $\sin x$ дуже схожа на $x$}
}
\end{figure}
\newpage
\section{Неперервність функції}
\subsection{Неперервність в точці}
\defin{4.1.1.} Задана функція $f: A \to \mathbb{R}$ та $x_0 \in A$ - гранична точка\\
Функція $f(x)$ називається \textbf{неперервною в т.} $x_0$, якщо
\begin{align*}
\exists \lim_{x \to x_0} f(x) = f(x_0)
\end{align*}
Якщо $\huge \exists \lim_{x \to x_0^+} f(x) = f(x_0)$, то \textbf{неперервна праворуч в т.} $x_0$\\
Якщо $\huge \exists \lim_{x \to x_0^-} f(x) = f(x_0)$, то \textbf{неперервна ліворуч в т.} $x_0$\\
Якщо в т. $x_0$ вона не є неперервною, то її називають \textbf{точкою розриву}
\bigline
\th{4.1.2.} Задана функція $f: A \to \mathbb{R}$ та $x_0 \in A$ - гранична точка\\
Функція $f$ - неперервна в т. $x_0 \iff$ Функція $f$ - неперервна ліворуч та праворуч\\
\textit{Випливає з} \textbf{Th. 3.5.2.}
\bigline

\textbf{Класифікації точок розриву}\\
\textbf{I роду:}\\
- \textbf{усувна}, якщо $\exists \huge \lim_{x \to x_0} f(x) \neq f(x_0)$\\
- \textbf{стрибок}, якщо $\exists \huge \lim_{x \to x_0^+} f(x)$, $\exists \huge \lim_{x \to x_0^-} f(x)$, але вони не рівні
\bigline
\textbf{II роду:}\\
якщо виконується один з 4 випадків:\\
1) $\huge \lim_{x \to x_0^-} f(x) = \infty$\\
2) $\huge \lim_{x \to x_0^+} f(x) = \infty$\\
3) $\huge \not\exists \lim_{x \to x_0^-} f(x)$\\
4) $\huge \not\exists \lim_{x \to x_0^+} f(x)$\\
\bigline
\ex{4.1.1.(1)} $f(x) = \begin{cases} \huge \frac{\sin x}{x}, x \neq 0 \\ 1, x = 0 \end{cases}$\\
В т. $x_0$ функція $f(x)$ є неперервною, оскільки
\bigline
$\huge \lim_{x \to 0} \frac{\sin x}{x} \overset{\textrm{I чудова границя}}{=} 1 = f(0)$\\
\ex{4.1.1.(2)} $f(x) = \begin{cases} \huge \frac{\sin x}{x}, x \neq 0 \\ 0, x = 0 \end{cases}$\\
А в цьому випадку в т. $x_0$ буде усувною, оскільки\\
$\huge \lim_{x \to 0} \frac{\sin x}{x} \overset{\textrm{I чудова границя}}{=} 1 \neq f(0)$\\
В цьому випадку у нас $f(0) = 0$\\
\begin{figure} [H]
\centering
\resizebox{1\textwidth}{!}
{
\begin{tikzpicture}

\draw[thick, ->] (-7,0)--(7.5,0) node[anchor = north] {$x$};
\draw[thick, ->] (0,-1)--(0,2) node[anchor = east] {$y$};

\draw[thick, domain=-7:7, variable=\x, samples = 1000] plot({\x}, {sin(deg(\x))/\x}) node[anchor = south east, scale = 0.8] {$f(x) = \dfrac{\sin x}{x}$};
\node[white] at (0,1) [circle,fill,inner sep=1.5pt, draw = black]{};
\node[black] at (0,0) [circle,fill,inner sep=1.5pt, draw = black]{};
\end{tikzpicture}
}
\end{figure}

\ex{4.1.2.} $f(x) = 2x - \dfrac{x-2}{|x-2|}$\\
Тут проблема виникає в т. $x_0 = 2$. Розглянемо границі в різні сторони:\\
$\huge \lim_{x \to 2^-} \left(2x - \dfrac{x-2}{2-x}\right) = \lim_{x \to 2^-} (2x-1) = 3$\\
$\huge \lim_{x \to 2^+} \left(2x - \dfrac{x-2}{x-2}\right) = \lim_{x \to 2^+} (2x+1) = 5$\\
\begin{figure}[H]
\centering
{
\begin{tikzpicture}

\draw[thick, ->] (-2,0)--(4,0) node[anchor = north] {$x$};
\draw[thick, ->] (0,-3)--(0,6) node[anchor = east] {$y$};

\draw[thick, domain=-2:1.99, variable=\x, samples = 1000] plot({\x}, {2*\x - (\x-2)/(abs(\x-2))});
\draw[thick, domain= (2.01:3.5, variable=\x, samples = 1000] plot({\x}, {2*\x - (\x-2)/(abs(\x-2))}) node[anchor = south, scale = 0.8] {$f(x) = 2x - \dfrac{x-2}{|x-2|}$};
\node[white] at (2,5) [circle,fill,inner sep=1.5pt, draw = black]{};
\node[white] at (2,3) [circle,fill,inner sep=1.5pt, draw = black]{};
\draw[thick, dashed] (2,5)--(2,0) node[anchor = north] {$2$};
\draw (1 pt, 5cm) -- (-1 pt, 5cm) node[anchor = east] {$5$};
\draw (1 pt, 3cm) -- (-1 pt, 3cm) node[anchor = east] {$3$};
\end{tikzpicture}
}
\end{figure}

Обидва ліміти не рівні, а отже, $x_0 = 2$ - стрибок\\
\\
\ex{4.1.3.} $f(x) = \dfrac{1}{x+1}$\\
Проблема в т. $x_0 = -1$. Але принаймні по одну сторону, наприклад $\huge \lim_{x \to -1^+0} =\dfrac{1}{x+1} = +\infty$, матимемо нескінченність\\
Тому одразу т. $x_0 = -1$ - розрив 2 роду
\bigline
\th{4.1.3. Арифметичні властивості неперервних функцій}\\
Задані функції $f,g: A \to \mathbb{R}$ та $x_0 \in A$ - гранична точка\\
$f,g$ - неперервні в т. $x_0$. Тоді:\\
1) $\forall c \in \mathbb{R}: (cf)(x)$ - неперервна в т. $x_0$\\
2) $(f+g)(x)$ - неперервна в т. $x_0$\\
3) $(fg)(x)$ - неперервна в т. $x_0$\\
4) $\dfrac{f}{g}(x)$ - неперервна в т. $x_0$ при $g(x_0) \neq 0$\\
\proof
1), 2), 3), 4) Всі вони випливають із означення\\ \\
Але в 4) Тут більш детально розпишу одну деталь\\
Переконаємось, що все буде коректно визначено:\\
$g$ - неперервна в $x_0$, тобто $\forall \varepsilon > 0: \exists \delta: \forall x \in A: |x-x_0|<\delta \Rightarrow |g(x)-g(x_0)|<\varepsilon$\\
Оберемо $\varepsilon = \dfrac{|g(x_0)|}{2}$\\
Тоді $g(x_0)-\varepsilon <g(x) <g(x_0)+\varepsilon$\\
Якщо $g(x_0) > 0$, то $\varepsilon = \dfrac{g(x_0)}{2} \Rightarrow 0 < g(x) < \dfrac{3}{2}g(x_0)$\\
Якщо $g(x_0) < 0$, то $\varepsilon = -\dfrac{g(x_0)}{2} \Rightarrow \dfrac{3}{2}g(x_0) < g(x) < \dfrac{1}{2}g(x_0) < 0$\\
Тобто $\exists \delta: \forall x \in A: |x-x_0|<\delta \Rightarrow g(x) \neq 0$\\
Отже, наше означення є коректним \qed
\bigline

\th{4.1.4. Неперервність композиції}\\
Задані функції $f: A \to B, g: B \to \mathbb{R}$ та $h = g \circ f$\\
Відомо, що $x_0 \in A$ - гранична т. $A$, де $f$ неперервна; та $f(x_0) = y_0$ - гранична т. $B$, де $g$ неперервна.\\
Тоді $h$ - неперервна в т. $x_0$\\
\textit{Випливає з означення та властивості композиції}
\bigline
\defin{4.1.5.} Функція $f: A \to \mathbb{R}$ називається \textbf{неперервною на множині} $A$, якщо вона є неперервною $\forall x \in A$\\
Позначення: $C(A)$ - множина неперервних функцій в $A$\\
Тобто з означення, $f \in C(A)$
\bigline
\subsection{Неперервність елементарних функцій}
0) $f(x) = x$ - неперервна на $\mathbb{R}$\\
\proof
$\forall \varepsilon > 0: \exists \delta = \varepsilon: \forall x: |x-x_0|<\delta \Rightarrow |f(x)-f(x_0)| = |x-x_0|<\delta = \varepsilon$ \qed
\bigline
1) $f(x) = a_0 + a_1 x + \dots + x_n x^n$ неперервна на $\mathbb{R}$\\
\proof
Оскільки $g(x) = x \in C(\mathbb{R})$, то \\ $h(x)=x^n = x \cdot \dots \cdot x \in C(\mathbb{R})$ як добуток функцій $\forall n \geq 1$\\
Отже,\\
$f(x) = a_0 + a_1 x + \dots + x_n x^n \in C(\mathbb{R})$ як сума неперервних функцій, множений на константу \qed
\bigline
2) $f(x) = \sin x$ - неперервна на $\mathbb{R}$\\
\proof
Вже відома давно нерівність:\\
$1 - \dfrac{x^2}{2} < \dfrac{\sin x}{x} < 1 \Rightarrow x - \dfrac{x^3}{2} < \sin x < x$\\
Якщо $x \to 0$, то за теоремою про 2 поліцая, $\huge \lim_{x \to 0} \sin x = 0 = \sin 0$\\
Отже, $\sin x$ - неперервна лише в т. $0$\\
Перевіримо неперервність в т. $a \in \mathbb{R}$:\\
$\huge \lim_{x \to a} (\sin x - \sin a) = \lim_{x \to a} 2 \sin \dfrac{x-a}{2} \cos \dfrac{x+a}{2} = $\\
Проведемо заміну: $\dfrac{x-a}{2} = t$. Тоді $t \to 0$\\
$= \huge \lim_{t \to 0} 2 \sin t \cos (t+a) =$ (н.м * обм) $= 0$\\
$\Rightarrow \huge \lim_{x \to a} \sin x = \sin a$\\
Остаточно, $f(x) = \sin x \in C(\mathbb{R})$ \qed
\bigline
3) $f(x) = \cos x$ - неперервна на $\mathbb{R}$\\
\proof
$f \in C(\mathbb{R})$ як композиція, бо $\cos x = \sin\left(\dfrac{\pi}{2} -x \right)$ \qed
\bigline
4.1) $f(x) = \tg x$ - неперервна всюди, окрім $x = \dfrac{\pi}{2} + \pi k, k \in \mathbb{Z}$\\
4.2) $f(x) = \ctg x$ - неперервна всюди, окрім $x = \pi k, k \in \mathbb{Z}$\\
\proof
1.$f \in C$ як частка, бо $\tg x = \dfrac{\sin x}{\cos x}$\\
2.$f \in C$ за аналогічними міркуваннями \qed
\bigline
5) $f(x) = e^x$ - неперервна на $\mathbb{R}$\\
\proof
Спочатку побудуємо цю функцію:\\
Ми навчились зводити в натуральну, цілу та навіть в раціональну степіні. Покажемо зведення в дійсну степінь\\
Визначення: $\forall x \in \mathbb{R} \setminus \mathbb{Q}: e^x = \sup\{e^y | y \in \mathbb{Q}, y < x\}$
\bigline
\ex{} Для $x = \pi$:\\
$\huge e^{\textstyle \frac{3}{1}}, e^{\textstyle \frac{31}{10}}, e^{\textstyle \frac{314}{100}}, \dots$
\bigline
Хочемо виконання наступних умов:\\
1) $e^{x_1}e^{x_2} = e^{x_1+x_2}$\\
2) $(e^{x_1})^{x_2} = e^{x_1 x_2}$\\
3) $x_1 < x_2 \Rightarrow e^{x_1} < e^{x_2}$\\
4) $e^0 = 1$\\
\\
Зафіксуємо два числа:\\
$e^{x_1} = \sup\{e^{y_1} | y_1 \in \mathbb{Q}, y_1 < x_1\}$\\
$e^{x_2} = \sup\{e^{y_2} | y_2 \in \mathbb{Q}, y_2 < x_2\}$\\
\\
1) $e^{x_1+x_2} = \sup\{e^{y} | y \in \mathbb{Q}, y < x_1+x_2\} \overset{y=y_1+y_2}{=} \\ = \sup\{e^{y_1+y_2} | y_1,y_2 \in \mathbb{Q}, y_1+y_2 < x_1+x_2\} = \\ = \sup\{e^{y_1+y_2} | y_1,y_2 \in \mathbb{Q}, y_1<x_1, y_2<x_2\} = \\ = \sup\{e^{y_1} | y_1\in \mathbb{Q}, y_1<x_1\} \cdot \sup\{e^{y_2} | y_2 \in \mathbb{Q}, y_2 < x_2\} = e^{x_1} e^{x_2}$\\
\textit{Краще читати в зворотньому напрямку}
\bigline
2) $(e^{x_1})^{x_2} = \sup\{(e^{x_1})^{y_2}, y_2 \in \mathbb{Q}: y_2 < x_2 \} = \sup\{(e^{y_1})^{y_2}, y_1, y_2 \in \mathbb{Q}: y_1 < x_1, y_2 < x_2 \} = \sup\{e^{y}, y \in \mathbb{Q}: y<x_1 x_2 \} = e^{x_1 x_2}$
\bigline
3) $x_1 < x_2 \Rightarrow \{y_1 \in \mathbb{Q}, y_1 < x_1\} \subset \{y_2 \in \mathbb{Q}: y_2 < x_2\} \Rightarrow \\ \{e^{y_1} \in \mathbb{Q}, y_1 < x_1\} \subset \{e^{y_2} \in \mathbb{Q}: y_2 < x_2\}$\\
За властивостями $\sup$, $\sup\{e^{y_1} \in \mathbb{Q}, y_1 < x_1\} \leq \sup\{e^{y_2} \in \mathbb{Q}: y_2 < x_2\} \Rightarrow e^{x_1} < e^{x_2}$
\bigline
4) $\huge \lim_{x \to 0} (e^x - 1) = \huge \lim_{x \to 0} \frac{e^x-1}{x}\cdot x = 0$\\
$\Rightarrow \huge \lim_{x \to 0} e^x = 1 = e^0$, тобто неперервна в т. $0$\\
Тоді\\
$\huge \lim_{x \to a} (e^x - e^a) = \lim_{x \to a} e^a(e^{x-a}-1) \overset{x-a=t}{=} \lim_{t \to 0} e^a(e^t-1) = 0$\\
$\Rightarrow \huge \lim_{x \to a} e^x = e^a$\\
Отже, $f(x) = e^x \in C(\mathbb{R})$ \qed
\\
\subsection{Неперервність функції на відрізку}
\th{4.5.1. Теорема Вейєрштрасса 1}\\
Задана функція $f \in C([a,b])$. Тоді вона є обмеженою на $[a,b]$\\
\contra
Припустимо, що $f$ не є обмежено, тобто\\
$\forall n \geq 1: \exists x_n \in [a,b]: |f(x_n)| > n$\\
Отримаємо послідовність $\{x_n,n \geq 1\}$\\
Є два випадки, тому виділимо 2 підпослідовності:\\
1) $\{x_{n_k}, k \geq 1\}: f(x_{n_k})>n_k$\\
2) $\{x_{n_m}, m \geq 1\}: f(x_{n_m})<-n_m$\\
Розглянемо другу. Вона є обмеженою, оскільки $\{x_{n_m}, m \geq 1\} \subset [a,b]$\\
Тоді за Вейєрштраса, для підпослідовності $\{x_{n_{m_p}}, p \geq 1\}: \\ \exists \huge \lim_{n \to \infty} x_{n_{m_p}} =x_*$\\
Тому за означенням Гейне і за неперервністю, $\exists \huge \lim_{p \to \infty} f(x_{n_{m_p}}) = f(x_*)$\\
Але в той же час ми маємо, що функція не є обмеженою знизу, тобто $\exists \huge \lim_{p \to \infty} f(x_{n_{m_p}}) = -\infty$. Суперечність!\\
Для першого пункту все аналогічно і теж є суперечність\\
Отже, $f$ - все ж таки обмежена на $[a,b]$ \qed
\bigline
\th{4.5.2. Теорема Вейєрштрасса 2}\\
Задана функція $f \in C([a,b])$. Тоді:\\
- $\huge \exists x_* \in [a,b]: f(x_*) = \inf_{x \ in [a,b]} f(x)$\\
- $\huge \exists x^* \in [a,b]: f(x^*) = \sup_{x \ in [a,b]} f(x)$\\
\proof
Доведемо перший випадок, другий є аналогічним\\
Нехай $\huge \inf_{x \in [a,b]} f(x) = c$. За означенням:\\
1) $\forall x \in [a,b]: f(x) \geq c$\\
2) $\forall \varepsilon > 0: \exists x_{\varepsilon} \in [a,b]: f(x_{\varepsilon}) < c + \varepsilon$\\
Зафіксуємо $\varepsilon = \dfrac{1}{n}$\\
Тоді $\exists x_n \in [a,b]: c \leq f(x_n) < c + \dfrac{1}{n}$\\
Ми також маємо обмежену послідовність $\{x_n, n \geq 1\} \subset [a,b]$\\
Тому за Вейєрштрасом, для $\{x_{n_k},k \geq 1\}: \exists \huge \lim_{n \to \infty} x_{n_k} = x_*$\\
Отже, за Гейне і за неперервністю, $\huge \exists \lim_{k \to \infty} f(x_{n_k}) = f(x_*)$\\
Але в той самий час, $\exists x_{n_k} \in [a,b]: c \leq f(x_{n_k}) < c + \dfrac{1}{n_k}$\\
Коли $k \to \infty$, то за теоремою про поліцаїв, $\exists \huge \lim_{k \to \infty} f(x_{n_k}) = c$\\
Таким чином отримали, що $c = f(x_*) = \huge \inf_{x \in [a,b]} f(x) = \min_{x \in [a,b]} f(x)$ \qed
\bigline
\th{4.5.3. Теорема Коші про нульове значення}\\
Задана функція $f \in C([a,b])$, причому $f(a) \cdot f(b) < 0$\\
Тоді $\exists x_0 \in (a,b): f(x_0) = 0$\\
\proof
Розглянемо випадок, коли $f(a) < 0$, $f(b) > 0$\\
Розглянемо множину $M= \{x \in [a,b], f(x) < 0\}$\\
Оскільки $f$ - неперервна, то $\huge \exists \lim_{x \to a} f(x) = f(a)$\\
$\Rightarrow$ для $\varepsilon = -\dfrac{f(a)}{2}: \exists \delta: \forall x: |x-a|<\delta \Rightarrow |f(x)-f(a)|< -\dfrac{f(a)}{2}$\\
$\Rightarrow \dfrac{3f(a)}{2} < f(x) < \dfrac{f(a)}{2} \Rightarrow \forall x: |x-a|<\delta \Rightarrow f(x) < 0$\\
Отже, $M \neq \emptyset$\\
З іншого боку, ми маємо $\huge \lim_{x \to b} f(x) = f(b)$\\
$\Rightarrow$ для $\tilde{\varepsilon} = \dfrac{f(b)}{2}: \exists \tilde{\delta}: \forall x: |x-b| < \tilde{\delta} \Rightarrow |f(x)-f(b)|<\dfrac{f(b)}{2}$\\
$\Rightarrow \dfrac{f(b)}{2} < f(x) < \dfrac{3f(b)}{2} \Rightarrow \forall x: |x-b| < \tilde{\delta} \Rightarrow f(x) > 0$\\
Жодна з цих значень аргументів не потрапляє в нашу множину $M$\\
А оскільки $M \subset [a,b]$, то вона є обмеженою\\
\\
З двох міркувань випливає, що $\exists \sup M \overset{\textrm{позн.}}{=} x_0$\\
А тепер перевіримо, що дійсно $f(x_0) = 0$\\
За критерієм $\sup$:\\
$\forall x \in M: x \leq x_0$\\
Для $\varepsilon = \dfrac{1}{n}: \exists x_n \in M: x_n > x_0 - \dfrac{1}{n}$\\
Тобто $\forall n \geq 1: x_0 - \dfrac{1}{n} < x_n \leq x_0$\\
Розглянемо послідовність $\{x_n, n \geq 1\} \subset M: \exists \huge \lim_{n \to \infty} x_n = x_0$\\
Отже, за Гейне та неперервністю, $\huge \exists \lim_{n \to \infty} f(x_n) = f(x_0) \leq 0$
\\
Оскільки ми маємо $\sup M = x_0$, то тоді $\forall n \geq 1: x_0 + \dfrac{1}{n} \not\in M$\\
Тому розглянемо послідовність $\{\tilde{x_n} = x_0 + \dfrac{1}{n}, n \geq 1\}$\\
Тут $\huge \lim_{n \to \infty} \tilde{x_n} = x_0 \Rightarrow \exists \lim_{n \to \infty} f(\tilde{x_n}) = f(x_0) > 0$\\
Остаточно, $f(x_0) = 0$ \qed
\bigline
\crl{4.5.3. Теорема Коші про проміжкове значення}\\
Задана функція $f \in C([a,b])$\\
Тоді $\forall L \in \underset{(f(b),f(a))}{(f(a),f(b))}: \exists x_L \in (a,b): f(x_L) = L$\\
\textit{Вказівка: розглянути функцію $g(x) = f(x) - L$}
\bigline
\th{4.2.1. Про існування оберненої функції}\\
Задана функція $f: (a,b) \to (c,d)$ - строго монотонна і неперервна\\
Відомо, що $\huge \lim_{x \to a^+} f(x) = c$, $\huge \lim_{x \to b^-} f(x) = d$\\
Тоді існує функція $g: (c,d) \to (a,b)$ - строго монотонна (як і $f$) і неперервна, яка є оберненою до $f$\\
\proof
Розглянемо випадок монотонно зростаючої функції $f$. Для спадної аналогічно\\
Тоді $c < d$\\
За теоремою про проміжкове значення, $\forall y \in (c,d): \exists x \in (a,b): y = f(x)$\\
Покажемо, що $\forall y \in (c,d): \exists ! x \in (a,b): y = f(x)$\\
!Припустимо, не єдиний $x$ існує, тобто $\exists x_1, x_2: f(x_1) = y, f(x_2) = y$, але при цьому $x_1 \neq x_2$\\
Тоді якщо $x_1 < x_2$, то через монотонно зростаючу функцію $f(x_1) < f(x_2)$\\
Тоді якщо $x_1 > x_2$, то через монотонно зростаючу функцію $f(x_1) > f(x_2)$\\
Суперечність!\\
Таким чином, $\exists ! x \in (a,b): y = f(x)$ - бієкція\\
Ба більше, $\forall x \in (a,b): f(x) \in (c,d)$\\
Тоді створімо функцію $g: (c,d) \to (a,b)$, що є оберненою до $f$\\
1. Покажемо, що $g(x)$ - монотонно зростає\\
$\forall y_1,y_2: y_1 > y_2$\\
$x_1 = g(y_1), x_2 = g(y_2)$\\
$y_1 \neq y_2 \iff x_1 \neq x_2$\\
Якщо $x_1 < x_2$, то тоді $y_1 = f(x_1) < f(x_2) = y_2$, що не є можливим\\
Отже, $x_1 > x_2 \implies g(y_1) > g(y_2)$\\
Це й є ознака строгого зростання
\bigline
2. Покажемо, що $g \in C((c,d))$\\
!Припустимо, що це не так, тобто $\exists y_0: g(y)$ - не є неперервною в т. $y_0$\\
Зафіксуємо дві послідовності, що збігаються до т. $y_0$\\
$\exists \{y_n^1, n \geq 1\}, \{y_n^2, n \geq 1\}: \huge \lim_{n \to \infty} y_n^1 = y_0, \huge \lim_{n \to \infty} y_n^2 = y_0$\\
Але водночас $\huge \lim_{n \to \infty} g(y_n^1) \neq g(y_0), \huge \lim_{n \to \infty} g(y_n^1) \neq g(y_0)$\\
А це означає, що $\huge \lim_{n \to \infty} g(y_n^1) \neq \lim_{n \to \infty} g(y_n^2)$\\
Позначимо $\{x_n^1 = g(y_n^1), n \geq 1 \}$, $\{x_n^2 = g(y_n^2), n \geq 1 \}$\\
Тоді $\huge \lim_{n \to \infty} x_n^1 \neq \lim_{n \to \infty} x_n^2$\\
Позначимо $\huge \lim_{n \to \infty} x_n^1 = u_1$, $\huge \lim_{n \to \infty} x_n^2 = u_2$\\
Тоді з неперервності $f(x)$ отримаємо, що:\\
$f(u_1) = \huge \lim_{n \to \infty} f(x_n^1) = \lim_{n \to \infty} f(g(y_n^1)) = \lim_{n \to \infty} y_n^1 = y_0 = \lim_{n \to \infty} y_n^2 = \lim_{n \to \infty} f(g(y_n^2)) = \lim_{n \to \infty} f(x_n^2) = f(u_2)$\\
Тобто $f(u_1) = f(u_2)$. Суперечність! Оскільки $f$ - СТРОГО монотонно зростаюча функція\\
Отже, наше припущення - невірне. Тоді $g \in C((c,d))$
\bigline
Фінальний висновок: $g \in C((c,d))$ та строго монотонно зростаюча на $(c,d)$ \qed
\bigline
\subsection{Рівномірна неперервність}
\defin{4.6.1.} Функція $f$ називається \textbf{рівномірно неперервною на множині} $A$, якщо
\begin{align*}
\forall \varepsilon > 0: \exists \delta(\varepsilon) > 0: \forall x_1,x_2 \in A: |x_1-x_2|<\delta \Rightarrow |f(x_1) - f(x_2)| < \varepsilon
\end{align*}
\prp{4.6.2.} Якщо функція $f$ - рівномірно неперервна на $A$, то тоді вона є (просто) неперервною на $A$\\
\textit{Випливає з def.}
\bigline
\th{4.6.3. Теорема Кантора}\\
Якщо $f \in C([a,b])$, то вона - рівномірно неперервна на $[a,b]$\\
\contra
Припустимо, що вона не є рівномірно неперервною, тобто\\
$\exists \varepsilon^* > 0: \forall \delta: \exists x_{1 \delta}, x_{2 \delta} \in [a,b]: |x_{1 \delta} - x_{2 \delta}| < \delta \Rightarrow |f(x_{1 \delta}) - f(x_{2 \delta})| \geq \varepsilon^*$\\
Розглянемо $\delta = \dfrac{1}{n}$. Тоді $x_{1 \delta}, x_{2 \delta} = x_{1n}, x_{2n}$\\
Створимо послідовність $\{x_{1n}, n \geq 1\}$ - обмежена, бо всі в відрізку $[a,b]$, тому \\ 
для $\{x_{{1n}_k}, k \geq 1\}: \exists \huge \lim_{k \to \infty} x_{{1n}_k} = x_0$\\
Оскільки $|x_{1n} - x_{2n}| < \dfrac{1}{n}$, то маємо, що $|x_{1n_k} - x_{2n_k}| < \dfrac{1}{n_k}$\\
Тоді $x_{1n_k} - \dfrac{1}{n_k} < x_{2n_k} < x_{1n_k} + \dfrac{1}{n_k}$\\
Якщо $k \to \infty$, то за теоремою про поліцаї, $\exists \huge \lim_{k \to \infty} x_{2n_k} = x_0$\\
За умовою неперервності, отримаємо, що $\huge\lim_{k \to \infty} f(x_{1n_k}) = \lim_{k \to \infty} f(x_{2n_k}) = f(x_0)$\\
Але $\varepsilon \leq |f(x_{1n_k}) - f(x_{2n_k})| \to 0$, коли $k \to \infty$. Суперечність! \qed
\newpage


\subsection{*Неперервність функції Діріхле та Рімана}
Розглянемо функцію Діріхле
\begin{align*}
\mathfrak{D}(x) = \begin{cases} 1, x \in \mathbb{Q} \\ 0, x \in \mathbb{R} \setminus \mathbb{Q} \end{cases}
\end{align*}
Графік такої функції уявити навіть складно\\
Треба довести, що вона всюди розривна\\
Зафіксую спочатку $x_0 \in \mathbb{Q}$. І нехай в цій т. вона є неперервною, тобто\\
$\forall \varepsilon > 0: \exists \delta: \forall x: |x-x_0|<\delta \Rightarrow |\mathfrak{D}(x)-\mathfrak{D}(x_0)| < \varepsilon$\\
Встановлю $\varepsilon = \dfrac{1}{2}$. Тоді $x_0 - \delta < x < x_0 + \delta \Rightarrow \dfrac{1}{2} < \mathfrak{D}(x) < \dfrac{3}{2}$\\
За принципом Дедекінда, можна знайти в $\delta$-окілу таку точку $x \in \mathbb{Q} \setminus \mathbb{R}$, а тоді наступна нерівність не виконується
\bigline
Зафіксую тепер т. $x_0 \in \mathbb{R} \setminus \mathbb{Q}$. І знову вважаємо, що вона - неперевна\\
Тут знову $\varepsilon = \dfrac{1}{2}$, але тепер в $\delta$-околі в будь-якому інтервалі ми можемо завжди знайти раціональне число. Тож знову фейл\\
Таким чином, $\mathfrak{D}(x)$ - ніде не є неперервною
\bigline
Особливість функції Діріхле полягає в побудуванні функції, яка є неперервною лише в одній точці\\
Зокрема $f(x) = x \mathfrak{D}(x)$ - неперервна в т. $x_0 = 0$\\
Дійсно, $\forall \varepsilon > 0: \exists \delta = \varepsilon: \forall x: |x| <\delta \Rightarrow |x \mathfrak{D}(x)| \leq |x| < \varepsilon$
\bigline
\bigline
Розглянемо функцію Рімана
\begin{align*}
\mathfrak{R}(x) = \begin{cases} \dfrac{1}{n}, x = \dfrac{m}{n} \in \mathbb{Q} \\ 0, x \in \mathbb{R} \setminus \mathbb{Q} \end{cases}
\end{align*}
Зокрема $\mathfrak{R}(0) = 1$\\
Графік такої функції вже можна приблизно уявити\\
Треба довести, що вона неперервна в усіх ірраціональних числах\\
(INSERT INFO)
\bigline
\subsection{*Інші факти з неперервною функцією}
\th{4.6.1.} Задана функція $f: [a,b] \to \mathbb{R}$ - монотонна\\
Тоді вона може містити в $[a,b]$ лише точки розриву I роду, число їхніх точок не більше, ніж зліченна\\
(INSERT INFO)

\newpage
\section{Диференціювання}
\subsection{Основні означення}
\defin{5.1.1.} Задана функція $f: A \to \mathbb{R}$ та $x_0 \in A$ - гранична точка\\
Функцію $f$ називають \textbf{диференційованою} в т. $x_0$, якщо
\begin{align*}
\exists L \in \mathbb{R}: f(x) - f(x_0) = L(x-x_0)+o(x-x_0),x \to x_0
\end{align*}
\bigline
\prp{5.1.2.} Задана функція $f$ - диференційована в т. $x_0$. Тоді вона в т. $x_0$ неперервна\\
\proof
$\huge \lim_{x \to x_0} (f(x)-f(x_0)) = \lim_{x \to x_0}(L(x-x_0)+o(x-x_0)) = 0$ \qed
\bigline
\prp{5.1.3.} Функція $f$ - диференційована в т. $x_0 \iff \\ \iff \exists \huge \lim_{x \to x_0} \frac{f(x)-f(x_0}{x-x_0} = L = f'(x_0)$
\bigline
\defin{5.1.3.} Тут число $f'(x_0)$ називають \textbf{похідною} функції в т. $x_0$\\
\proof
$f$ - диференційована в т. $x_0 \overset{\textrm{def.}}{\iff} \\ \iff \exists L: \\ f(x)-f(x_0)=L(x-x_0)+o(x-x_0), x \to x_0 \iff \\ \iff \exists L: o(x-x_0) = f(x)-f(x_0)-L(x-x_0), x \to x_0 \iff \\ \iff \huge \lim_{x \to x_0} \frac{f(x)-f(x_0)-L(x-x_0)}{x-x_0} = 0 \iff \lim_{x \to x_0} \frac{f(x)-f(x_0)}{x-x_0} = L = f'(x_0)$ \qed
\bigline

\prp{5.1.4. Арифметичні властивості}\\
Задані функції $f,g$ - диференційовані в т. $x_0$, $f'(x_0),g'(x_0)$ - їхні похідні. Тоді:\\
1) $\forall c \in \mathbb{R}: cf$ - диференційована в т. $x_0$, а її похідна\\
$(cf)'(x_0) = cf'(x_0)$\\
2) $f \pm g$ - диференційована в т. $x_0$, а її похідна\\
$(f+g)'(x_0)=f'(x_0)+g'(x_0)$\\
3) $f \cdot g$ - диференційована в т. $x_0$, а її похідна\\
$(f \cdot g)(x_0) = f'(x_0)g(x_0)+f(x_0)g'(x_0)$\\
4) $\huge \frac{f}{g}$ - диференційована в т. $x_0$ при $g(x_0) \neq 0$, а її похідна\\
$\huge \left(\frac{f}{g}\right)'(x_0) = \frac{f'(x_0)g(x_0)-f(x_0)g'(x_0)}{(g(x_0))^2}$\\
\proof
Доведення буде проводитись за допомогою минуло доведеного твердження:\\
1) $(cf)'(x_0) = \huge \lim_{x \to x_0} \frac{cf(x)-cf(x_0)}{x-x_0} = c f'(x_0) \Rightarrow cf$ - диференційована в т. $x_0$
\bigline
2) $(f+g)'(x_0) = \huge \lim_{x \to x_0} \frac{f(x)+g(x) - f(x_0)-g(x_0)}{x-x_0} = \lim_{x \to x_0} \frac{f(x) - f(x_0)}{x-x_0} + \lim_{x \to x_0} \frac{g(x) - g(x_0)}{x-x_0} = f'(x_0) + g'(x_0) \Rightarrow f+g$ - диференційована в т. $x_0$
\bigline
3) $(f \cdot g)'(x_0) = \huge \lim_{x \to x_0} \frac{f(x)g(x) - f(x_0)g(x_0)}{x-x_0} = \\ = \lim_{x \to x_0} \frac{f(x)g(x) - f(x_0)g(x) + f(x_0)g(x) - f(x_0)g(x_0)}{x-x_0} = \\ = \lim_{x \to x_0} g(x) \frac{f(x)-f(x_0)}{x-x_0} + f(x_0) \lim_{x \to x_0} \frac{g(x)-g(x_0)}{x-x_0} = f'(x_0)g(x_0) + f(x_0)g'(x_0) \Rightarrow fg$ - диференційована в т. $x_0$
\bigline
4) $\huge \left(\frac{f}{g}\right)'(x_0) = \lim_{x \to x_0} \frac{\frac{f(x)}{g(x)} - \frac{f(x_0)}{g(x_0)}}{x-x_0} = \lim_{x \to x_0} \frac{f(x)g(x_0)-f(x_0)g(x)}{g(x)g(x_0)(x-x_0)} \overset{\textrm{так само як в 3)}}{=} \\ = \frac{1}{(g(x_0))^2} (f'(x_0)g(x_0) - f(x_0)g'(x_0)) \Rightarrow \frac{f}{g}$ - диференційована в т. $x_0$ \qed
\bigline
\prp{5.1.5. Похідна від композиції}\\
Задані функції $f,g$ та $h=g \circ f$. Відомо, що $f$ - диференційована в т. $x_0$, а $g$ - диференційована в т. $y_0 = f(x_0)$\\
Тоді функція $h$ - диференційована в т. $x_0$, а її похідна\\
$h'(x) = g'(f(x_0)) \cdot f'(x_0)$\\
\proof
$h'(x) = \huge \lim_{x \to x_0} \frac{h(x)-h(x_0)}{x-x_0} = \lim_{x \to x_0} \frac{g(f(x))-g(f(x_0))}{x-x_0} = \\ = \lim_{x \to x_0} \frac{g(f(x))-g(f(x_0))}{f(x)-f(x_0)} \frac{f(x)-f(x_0)}{x-x_0} =$\\
Розіб'ємо дві дроби на окремі границі. В першому дробі заміна: $y=f(x)$\\
Якщо $x \to x_0$, то в силу неперервності, $f(x) \to f(x_0)$ або $y \to y_0$\\
$= \huge \lim_{y \to y_0} \frac{g(y)-g(y_0)}{y-y_0} \lim_{x \to x_0} \frac{f(x)-f(x_0)}{x-x_0}= g'(y_0) f'(x_0)=g'(f(x_0))f'(x_0) \\ \Rightarrow h$ - диференційована в т. $x_0$ \qed
\bigline
\defin{5.1.6.} Функція $f$ є \textbf{диференційованою на множині} $A$, якщо $\forall x_0 \in A: f$ - диференційована
\bigline
\textbf{Таблиця похідних}
\begin{center}
\begin{tabular}{ c|c } 
 $f(x)$ & $f'(x)$ \\
 \hline 
 $const$ & $0$ \\ [2ex]
 \hline 
 $x^\alpha, \alpha \neq 0$ & $\alpha \cdot x^{\alpha-1}$ \\ [2ex]
 \hline 
 $e^x$ & $e^x$ \\ [2ex]
 \hline 
 $a^x$ & $a^x \cdot \ln a$ \\ [2ex]
 \hline 
 $\sin x$ & $\cos x$\\ [2ex]
 \hline 
 $\cos x$ & $-\sin x$\\ [2ex]
 \hline 
 $\tg x$ & $\huge \frac{1}{\cos^2 x}$\\ [2ex]
 \hline 
 $\ctg x$ & $-\huge \frac{1}{\sin^2 x}$\\ [2ex]
 \hline 
 $\ln x$ & $\huge \frac{1}{x}$\\ [2ex]
 \hline 
 $\log_a x$ & $\huge \frac{1}{x \cdot \ln a}$\\ [2ex]
 \hline 
 $\arcsin x$ & $\huge \frac{1}{\sqrt{1-x^2}}$\\ [2ex]
 \hline 
 $\arccos x$ & $\huge -\frac{1}{\sqrt{1-x^2}}$\\ [2ex]
 \hline 
 $\arctg x$ & $\huge \frac{1}{1+x^2}$\\ [2ex]
 \hline 
 $\arcctg x$ & $\huge -\frac{1}{1+x^2}$\\ [2ex]
 \hline 
 $\ln(x+\sqrt{1+x^2})$ & $\huge \frac{1}{\sqrt{1+x^2}}$\\ [2ex]
 \hline 
\end{tabular}
\end{center}
Для повного доведення таблиць похідних заведу останню теорему:\\
\th{5.1.7. Похідна від оберненої функції}\\
Задані функції $f,g$ - взаємно обернені. Відомо, що $f$ - диференційована в т. $x_0$. Тоді $g$ - диференційована в т. $y_0 = f(x_0)$, а її похідна\\
$g'(y_0) = \huge \frac{1}{f'(x_0)}$\\
\proof
$g'(y_0) = \huge \lim_{y \to y_0} \frac{g(y)-g(y_0)}{y-y_0} =$\\
Заміна: $y = f(x)$. Звідси через взаємну оберненість $g(y)=g(f(x))=x$. Якщо $y \to y_0$, то $g(y) \to g(y_0) \Rightarrow x \to x_0$\\
$= \huge \lim_{x \to x_0}\frac{x-x_0}{f(x)-f(x_0)} = \frac{1}{f'(x_0)} \Rightarrow g$ - диференційована в т. $y_0$ \qed
\bigline
Тепер почергово доведемо кожну похідну:\\
1. $f(x) = const$\\
$f'(x_0) = \huge \lim_{x \to x_0} \frac{c-c}{x-x_0} = \lim_{x \to x_0} 0 = 0$
\bigline
2. $f(x) = x^{\alpha}$\\
$f'(x_0) = \huge \lim_{x \to x_0} \frac{x^{\alpha} - x_0^{\alpha}}{x-x_0} \overset{x-x_0 = t \to 0}{=} \lim_{t \to 0} \frac{(t+x_0)^{\alpha} - x_0^{\alpha}}{t} = x_0^{\alpha-1} \lim_{t \to 0} \frac{\left(1 + \frac{t}{x_0}\right)^{\alpha} - 1}{\frac{t}{x_0}} = \alpha x_0^{\alpha-1}$
\bigline
3. $f(x) = e^x$\\
$f'(x_0) = \huge \lim_{x \to x_0} \frac{e^x-e^{x_0}}{x-x_0} = \lim_{x \to x_0} \frac{e^{x_0}(e^{x-x_0}-1)}{x-x_0} = e^{x_0}$
\bigline
4. $h(x) = a^x$\\
Перепишемо інакше: $h(x) = e^{x \cdot \ln a}$\\
Побачимо, що $y = f(x) = x \cdot \ln a$, а в той час $g(y) = e^y \Rightarrow h(x)=g(f(x))$\\
Тоді за композицією, \\ $h'(x_0) = g'(y_0) f'(x_0) = e^{y_0} \ln a = e^{x_0 \ln a} \ln a = a^{x_0} \ln a$
\bigline
5. $f(x) = \sin x$\\
$f'(x_0) = \huge \lim_{x \to x_0} \frac{\sin x - \sin x_0}{x-x_0} = \lim_{x \to x_0} \frac{2 \sin \frac{x-x_0}{2} \cos \frac{x-x_0}{2}}{x-x_0} = \lim_{x \to x_0} \frac{\sin \frac{x-x_0}{2}}{\frac{x-x_0}{2}} \cos \frac{x-x_0}{2} = \cos x_0$
\bigline
6. $h(x) = \huge \cos x = \sin \left(\frac{\pi}{2} - x \right)$\\
$f(x) = \huge \frac{\pi}{2} - x$, $g(y) = \sin y \Rightarrow h(x) = g(f(x))$\\
Отже, $h'(x_0) = g'(y_0)f'(x_0) = \cos y_0 (-1) = \huge -\cos \left(\frac{\pi}{2} - x \right) = -\sin x$
\bigline
7. $f(x) = \tg x$\\
Або $f(x) = \huge \frac{\sin x}{\cos x}$\\
Тоді $f'(x) = \huge \frac{(\sin x)' \cos x - \sin x (\cos x)'}{\cos^2 x} = \frac{\cos^2 x + \sin^2x}{\cos^2 x} = \frac{1}{\cos^2 x}$
\bigline
8. $f(x) = \ctg x$\\
\textit{За аналогічними міркуваннями до 7.}
\bigline
9. $g(y) = \ln y$\\
Маємо функцію $f(x) = e^x$, тоді $f,g$ - взаємно обернені\\
Тоді оскільки $f'(x_0) = e^{x_0}$, то $g'(y_0) = \huge \frac{1}{f'(x_0)} = \frac{1}{e^{x_0}} = \frac{1}{e^{\ln y_0}} = \frac{1}{y_0}$
\bigline
10. $f(x) = \log_a x$\\
Або $f(x) = \huge \frac{\ln x}{\ln a} \Rightarrow f'(x_0) = \frac{1}{\ln a} \frac{1}{x_0}$
\bigline
11. $g(y) = \arcsin y$\\
Маємо функцію $f(x) = \sin x$, тоді $f,g$ - взаємно обернені\\
Тоді оскільки $f'(x_0) = \cos x_0$, то $g'(y_0) = \huge \frac{1}{f'(x_0)} = \frac{1}{\cos x_0}= \frac{1}{\cos (\arcsin y_0)} = \\ = \frac{1}{\sqrt{1- \sin^2(\arcsin y_0)}} = \frac{1}{\sqrt{1-y_0^2}}$\\
Важливо, що тут функція $f: \huge \left[-\frac{\pi}{2},\frac{\pi}{2}\right] \to [-1,1]$
\bigline
12. $f(x) = \arccos x$\\
Або $f(x) = \huge \frac{\pi}{2} - \arcsin x \Rightarrow f'(x_0) = - \frac{1}{\sqrt{1-x_0^2}}$
\bigline
13. $g(y) = \arctg y$\\
\textit{За аналогічними міркуваннями до 11.}, але тут вже $f: \huge \left(-\frac{\pi}{2},\frac{\pi}{2}\right) \to \mathbb{R}$, $f(x) = \tg x$
\bigline
14. $f(x) = \arcctg x$\\
\textit{За аналогічними міркуваннями до 12.}, але $\arcctg x = \huge \frac{\pi}{2} - \arctg x$
\bigline
15. $f(x) = \ln(x + \sqrt{1+x^2})$\\
$f'(x_0) = \huge \frac{1}{x_0+ \sqrt{1+x_0^2}} \cdot (x_0+ \sqrt{1+x^2})'_{x = x_0} = \frac{1+ \frac{1}{2\sqrt{1+x_0^2}} \cdot (1+x^2)'_{x=x_0}}{x_0+ \sqrt{1+x_0^2}} = \\ = \frac{1+ \frac{x_0}{\sqrt{1+x_0^2}}}{x_0 + \sqrt{1+x_0^2}} = \frac{\sqrt{1+x_0^2}+x_0}{x_0 + \sqrt{1 + x_0^2}} \frac{1}{\sqrt{1+x_0^2}} = \frac{1}{\sqrt{1+x_0^2}}$
\bigline
\rm{} Тут треба більш детально про $f(x) = \ln(x+\sqrt{1+x^2})$ сказати:\\
Розглянемо рівняння $\sh x = y$\\
Розв'яжемо її відносно $x$\\
$\huge \frac{e^x-e^{-x}}{2} = y \Rightarrow e^x-e^{-x}=2y \Rightarrow e^{2x}- 2y e^x - 1 = 0$\\
$\Rightarrow e^x = y \pm \sqrt{1+y^2} \Rightarrow e^x = y + \sqrt{1+y^2}$\\
$\Rightarrow x = \ln(y + \sqrt{1+y^2})$\\
Таким чином, можна стверджувати, що $\ln(y+\sqrt{1+y^2}) = \textrm{arcsh } y$\\
Але найбільше застосування все ж таки виявляється згодом (коли підуть інтеграли)
\bigline
\subsection{Дотична та нормаль до графіку функції}
\defin{5.2.1.} Пряма $y = k (x-x_0) + f(x_0)$ називається \textbf{дотичною до графіку функції} $f(x)$ \textbf{в т.} $x_0$, якщо
\begin{align*}
f(x) - [k(x-x_0)+f(x_0)] = o(x-x_0), x\to x_0
\end{align*}
Чоловічою мовою: 'навколо т. $x_0$ функція $f(x)$ та пряма $k(x-x_0)+f(x_0)$ '
\bigline
\prp{5.2.2.} Функція $f$ має дотичну в т. $x_0 \iff f$ - диференційована в т. $x_0$. При цьому $k = f'(x_0)$\\
\proof
$f(x) - [k(x-x_0)+f(x_0)] = o(x-x_0), x \to x_0 \iff \\ \iff f(x)-f(x_0) = k(x-x_0) + o(x-x_0), x \to x_0 \overset{\textrm{def.}}{\iff}$ $f$ -диференційована в т. $x_0$, $k=f'(x_0)$ \qed \\
Таким чином, $y - f(x_0) = f'(x_0)(x-x_0)$ - \textbf{рівняння дотичної}
\bigline
Є ще інше пояснення дотичної\\
Нехай є фіксована точка $(x_0,f(x_0))$ та точка $(x^*,f(x^*))$. Через ці дві точки проведемо пряму - її ще називають \textbf{січною}. Маємо таке рівняння:\\
$\dfrac{x-x_0}{x^*-x_0} = \dfrac{y-f(x_0)}{f(x^*) - f(x_0)} \Rightarrow \dfrac{f(x^*)-f(x_0)}{x^*-x_0}(x-x_0) = y - f(x_0)$\\
Ну а далі спрямуємо $x^* \to x_0$. І якщо функція $f$ - диференційована в т. $x_0$, то одразу маємо\\
$f - f(x_0) = f'(x_0)(x-x_0)$\\
Що й хотіли
\bigline
\defin{5.2.3.} Пряма, яка проходить через т. дотику $(x_0, f(x_0))$ та перпендикулярна до дотичної, називається \textbf{нормаллю до графіку функції} $f(x)$ \textbf{в т.} $x_0$
\bigline
Знайдемо безпосередньо рівняння нормалі. Маємо рівняння дотичної:
$f'(x_0)(x-x_0) - (y-f(x_0)) = 0$\\
Нормальний вектор дотичної задається координатами $\vec{n} = (f'(x_0); -1)$\\
Тоді для рівняння нормалі даний вектор буде напрямленим. Нам також відомо, що нормаль проходить через т. $(x_0,f(x_0))$, а отже,\\
$\huge \frac{x-x_0}{f'(x_0)} = \frac{y-f(x_0)}{-1} \Rightarrow f'(x_0)(y-f(x_0)) = -(x-x_0)$\\
Таким чином $y-f(x_0) = \huge -\frac{1}{f'(x_0)}(x-x_0)$ - \textbf{рівняння нормалі}
\bigline
\begin{figure}
\centering
\begin{tikzpicture}
\draw[thick, ->] (-3,0)--(4,0) node[below = 2pt] {$x$};
\draw[thick, ->] (0,-2)--(0,4) node[left = 2pt] {$y$};
\draw[thick] plot [smooth, tension=1] coordinates { (-2,1) (-1,2) (1,1) (3,3.5) } node[align=center, below, left =4pt] {$f(x)$};
\draw[thick, red] (-1,1/3)--(4,2);
\draw[thick, blue] (0.5,2.5)--(2,-2);
\filldraw (1,1) circle (2pt) node[align=center, below = 4pt, right = 2pt] {$(x_0, f(x_0))$};
\end{tikzpicture}
	\captionsetup{justification=centering}
	\caption{Графік функції, до якої проведена дотична (червоний) та нормаль (синій)}
\end{figure}
\ex{5.2.4.} Знайти дотичну до графіку функції $f(x) = 2 \cos x + 5$ в т. $x_0 = \huge \frac{\pi}{2}$\\
$y = f'(x_0)(x-x_0)+f(x_0)$\\
$f(x_0) = \huge f(\frac{\pi}{2}) = 5$\\
$f'(x_0) = \huge f'(\frac{\pi}{2}) = -2 \sin x |_{x = \frac{\pi}{2}} = -2$\\
Отже, маємо:\\
$y = \huge -2(x-\frac{\pi}{2}) + 5 = -2x + (5 - \pi)$
\bigline
Але що таке дотична, більше до вподоби таке пояснення\\
На графіку функції $f$ задамо точку $(x_0,f(x_0))$, та точку $(x_1,f(x_1))$\\
Запишемо рівняння прямої, що проходить через ці дві точки\\
$\dfrac{x-x_0}{x_1-x_0} = \dfrac{y-f(x_0)}{f(x_1)-f(x_0)}$\\
Обережно виразимо $y$\\
$y = \dfrac{f(x_1)-f(x_0)}{x_1-x_0} (x-x_0) + f(x_0)$\\
А тепер $x_1 \to x_0$. Тоді отримаємо\\
$y = f'(x_0)(x-x_0) + f(x_0)$
\bigline
\subsection{Приблизне обчислення значень для диференційованих функцій}
Задана функція $f$ - диференційована в т. $x_0$\\
Тоді за твердженням, функція має дотичну $y = f'(x_0)(x-x_0)+f(x_0)$, для якого:\\
$f(x)-y = o(x-x_0), x \to x_0$\\
Тому коли $x$ 'близьке' до $x_0$, тобто $|x-x_0| <<1$, то маємо:\\
$f(x) -y \approx 0 \\ \Rightarrow f(x) \approx f'(x_0)(x-x_0)+f(x_0)$
\bigline
\ex{5.3.1.} Знайти значення $\sqrt{65}$\\
Перетворимо значення іншим чином:\\
$\sqrt{65} \huge = \sqrt{64 \cdot \frac{65}{64}} = 8 \sqrt{\frac{65}{64}} = 8 \sqrt{1 + \frac{1}{64}}$\\
А тепер розглянемо функцію $f(x) = 8\sqrt{x}$\\
Тут $x = \huge \frac{65}{64}$, в той час $x_0 = 1$\\
$|x-x_0| = \huge \abs{\frac{65}{64} - 1} = \frac{1}{64} <<1$\\
Знайдемо значення функції та похідну в т. $x_0$:\\
$f(x_0) = f(1) = 8$\\
$f'(x_0) = f'(1) = \huge 8\frac{1}{2 \sqrt{x}} |_{x = 1} = 4$\\
Таким чином, отримаємо:\\
$\sqrt{65} \approx \huge 4\left(\frac{65}{64}-1\right)+8 = \frac{1}{16} + 8 = 8.0625$
\subsection{Диференціал функції}
Задана функція $f$ - диференційована в т. $x_0$\\
Почнемо з позначення: $dx \overset{\textrm{позн.}}{=} x-x_0, x \to x_0$\\
\rm{} Якщо б $x \not\to x_0$, то це б було просто $\Delta x = x - x_0$. Ось і вся різниця між $dx$ та $\Delta x$
\bigline
\defin{5.4.1.} \textbf{Диференціалом} функції $f(x)$ в т. $x_0$ називають приріст дотичної, коли $x \to x_0$\\
Позначення: $df(x_0)$
\bigline
А тепер знайдемо, чому дорівнює це $df(x_0)$\\
Побудуємо дотичну в т. $x_0$\\
\begin{figure}[H]
\centering
\begin{tikzpicture}[spy using outlines={circle,yellow,magnification=5,size=6cm, connect spies}]
\draw[thick, ->] (-3,0)--(2.5,0) node[below = 2pt] {$x$};
\draw[thick, ->] (0,-0.5)--(0,4) node[left = 2pt] {$y$};
\draw[thick, domain=-2.5:2, variable=\x] plot({\x}, {exp(\x*ln(2))}) node[anchor = west, scale = 0.7] {$f(x) = 2^x$};
\draw[thick, domain=-0.5:1.5, red, variable=\x] plot({\x}, {sqrt(2)*ln(2)*(\x-0.5)+sqrt(2)});
\draw[dashed] (0.5, {sqrt(2)})--(0.5,0) node [anchor = north, scale = 0.7] {$x_0$};
\draw[dashed] (1, {exp(1*ln(2))})--(1,0) node [anchor = north, scale = 0.7] {$x$};
\draw[blue] (1, {2*ln(2)*(-0.1)+2})--(1, {sqrt(2)});
\draw[green] (1, {exp(1*ln(2))})--(1, {2*ln(2)*(-0.1)+2});
\draw[dashed] (1, {sqrt(2)})--(0.5, {sqrt(2)});
\draw (0.5+0.2, {sqrt(2)}) arc (0:{atan(2*ln(2))}:0.2) node [anchor = west,scale = 0.3] {$\alpha$};
\node at (0.8, {sqrt(2)}) [scale = 0.3,anchor = north] {$dx$};
\node at (1.15, {2*ln(2)*(-0.1)+2}) [scale = 0.2, anchor = north, blue] {$df(x_0)$};
\spy on (0.7, {exp(0.7*ln(2))}) in node[left] at (12,2);
\end{tikzpicture}
\caption*{Синій - це $df(x_0)$: приблизна різниця між функціями в двох точках. А синій+зелений - це $\Delta f(x_0)$: точна різниця між функціями в двох точках}
\end{figure}
Нагадування: $k = f'(x_0) = \tg \alpha$\\
Тоді з малюнка можна виділити:\\
$\tg \alpha = f'(x_0) = \huge \frac{df(x_0)}{dx}$\\
$\Rightarrow df(x_0) = f'(x_0) \,dx$
\bigline
Для більш простого випадку, якщо $y=f(x)$, то тоді:
\begin{align*}
dy = f'(x)\,dx
\end{align*}
\\
\subsection{Похідні по один бік}
\defin{5.5.1. Односторонню похідну} функції $f(x)$ в т. $x_0$ називають:\\
\textbf{- праворуч}: $\huge f'(x_0^+) = \lim_{x \to x_0^+} \frac{f(x)-f(x_0)}{x-x_0}$\\
\textbf{- ліворуч}: $\huge f'(x_0^-) = \lim_{x \to x_0^-} \frac{f(x)-f(x_0)}{x-x_0}$
\bigline
\th{5.5.2.} Функція $f$ - диференційована в т. $x_0$ $\iff$ вона містить похідну ліворуч та праворуч, а також $f'(x_0^+) = f'(x_0^-)$\\
\proof
$f$ - диференційована в т. $x_0$ $\iff$ $\exists f'(x_0)$, тобто $\exists$ границя $\iff$ $\exists$ та сама границя ліворуч та праворуч, які рівні $\iff$ вона містить похідну ліворуч та праворуч та $f'(x_0^+) = f'(x_0^-)$ \qed
\bigline
\ex{5.5.3.} Знайти похідну функції $f(x) = |x|$\\
Якщо $x>0$, то $f(x) = x \Rightarrow f'(x) = 1$\\
Якщо $x<0$, то $f(x) = -x \Rightarrow f'(x) = -1$\\
Перевіримо існування похідної в т. $x_0 = 0$\\
$f'(0^+) = \huge \lim_{x \to 0^+} \frac{|x|-|0|}{x-0} = 1$\\
$f'(0^-) = \huge \lim_{x \to 0^-} \frac{|x|-|0|}{x-0} = -1$\\
$\Rightarrow f'(0^+) \neq f'(0^-)$, отже $\not \exists f'(0)$\\
Взагалі-то кажучи, похідну функції можна переписати інакше:\\
$f'(x) = \huge \frac{|x|}{x}$
\bigline
\subsection*{Ліричний відступ}
Тут вже виникає необхідність поговорити про похідну функції, якщо вона раптом стане рівною нескінченність. І дійсно, ми можемо допускати такий випадок\\
$f'(x_0) = \huge \lim_{x \to x_)} \dfrac{f(x)-f(x_0)}{x-x_0} = \pm \infty$\\
Одразу зауважу, що просто $\infty$ границі бути не може, тобто це еквівалетно в неіснуванні
\bigline
\ex{} Нехай є функція $f(x) = \sqrt[3]{x^2}$. Знайдемо похідну цієї штуки в т. $x_0 = 0$ за означенням\\
$f'(0) = \huge \lim_{x \to 0} \dfrac{\sqrt[3]{x^2} - 0}{x} = \lim_{x \to 0} \dfrac{1}{\sqrt[3]{x}} = \infty$\\
Проте для існування похідної необхідно і достатньо існування похідних з різних боків, а тут\\
$f'(0^-) = \huge \lim_{x \to 0^-} \dfrac{\sqrt[3]{x^2} - 0}{x} = -\infty$\\
$f'(0^+) = \huge \lim_{x \to 0^+} \dfrac{\sqrt[3]{x^2} - 0}{x} = +\infty$\\
Зрозуміло, що жодним чином $f'(0^-) \neq f'(0^+)$, тож похідна в $\infty$ існувати точно не може
\bigline
А тепер повернімось до геометричних застосувань. Вже відомо, що $f'(x_0) = \tg \alpha$ для дотичних\\
Якщо $f'(x_0) \to \pm \infty$, тобто $\tg \alpha \to \pm \infty$, то тоді кут $\alpha \to \pm \dfrac{\pi}{2}$. Тобто це означає, що ми матимемо справу з дотичною, яка є вертикальною прямою в т. $x_0$, тобто\\
$x=x_0$
\bigline
\ex{} Нехай є функція $f(x) = \sqrt[3]{x}$. Знайдемо похідно цієї штуки в т. $x_0 = 0$ за означенням\\
$f'(0) = \huge \lim_{x \to 0} \dfrac{\sqrt[3]{x}-0}{x-0} = \lim_{x \to 0} \dfrac{1}{\sqrt[3]{x^2}} = + \infty$\\
Похідна існує. Це можна навіть перевірити, пошукавши похідну зліва та справа\\
Тоді дотичною графіка функції $f$ в т. $x_0 = 0$ буде вертикальна пряма\\
$x_0 = 0$\\
\begin{figure}[H]
\centering
{
\begin{tikzpicture}

\draw[thick, ->] (-3,0)--(3,0) node[anchor = north] {$x$};
\draw[thick, ->] (0,-2)--(0,2) node[anchor = east] {$y$};

\draw[thick, domain=0.001:3, variable=\x, samples = 1000] plot({\x}, {((\x)^(1/3)});
\draw[thick, domain=-3:-0.001, variable=\x, samples = 1000] plot({\x}, {(-(-\x)^(1/3)});
\end{tikzpicture}
}
\end{figure}

\subsection{Інваріантність форми першого диференціалу}
Задана функція $y = f(x)$, коли в той же час $x = x(t)$\\
Мета: знайти значення $dy$ - перший диференціал\\
З одного боку:\\
$dy \overset{\textrm{def.}}{=} f'(x)\,dx = f'(x(t))\,d(x(t)) = f'(x(t))\cdot x'(t)\,dt$\\
З іншого боку:\\
$dy = df(x(t)) = (f(x(t)))'\,dt \overset{\textrm{композиція}}{=} f'(x(t))\cdot x'(t)\,dt$\\
Помічаємо, що ми отримали один й той самий результат, що й свідчить про \textbf{інваріантність форми першого диференціалу}
\bigline
\subsection{Похідна від параметрично заданої функції}
Задана параметрично функція $y: \begin{cases} y = y(t) \\ x = x(t) \end{cases}$\\
Мета: знайти $y'_x$ - похідну функції за $x$\\
З точки зору диференціалу:\\
$\begin{cases} dx = x'_t\,dt \\ dy = y'_t\,dt \end{cases} \Rightarrow \huge \frac{dy}{dx} = \frac{y'_t}{x'_t} \Rightarrow$
\begin{align*}
y'_x(t) = \frac{y'_t(t)}{x'_t(t)}
\end{align*}
\ex{} Знайти похідну від функції: $y: \begin{cases} x = \ln t \\ y = t^3 \end{cases}$\\
$x'_t = \huge \frac{1}{t}$,    $y'_t = 3t^2$\\
$\Rightarrow y'_x = \huge \frac{3t^2}{\frac{1}{t}} = 3t^3$\\
Сюда ми ще повернемось \bigline
\subsection{Похідна вищих порядків}
\defin{5.8.1.(1)} Задана функція $f$, для якої $\exists f'(x)$\\
\textbf{Похідною $2$-го порядку від} $f(x)$ називають $f''(x) = (f'(x))'$, якщо вона існує
\bigline
\defin{5.8.1.(2)} Задана функція $f$, для якої $\exists f^{(n)}(x)$\\
\textbf{Похідною $(n+1)$-го порядку від} $f(x)$ називають $f^{(n+1)}(x) = (f^{(n)}(x))'$, якщо вона існує
\bigline
\ex{5.8.1.} Знайдемо похідну $n$-го порядку функції $f(x) = \cos x$\\
$g(x) = \cos x \Rightarrow g'(x) = -\sin x \Rightarrow g''(x) = -\cos x \Rightarrow g'''(x) = \sin x \Rightarrow g^{(4)}(x) = \cos x \Rightarrow \dots$\\
Продовжувати можна довго, але можемо помітити, що:\\
$\cos x = \cos x$\\
$- \sin x = \huge \cos \left(x + \frac{1\pi}{2} \right) = (\cos x)'$\\
$- \cos x = \huge \cos \left(x + \pi \right) = \cos \left(x + \frac{2 \pi}{2} \right) = (\cos x)''$\\
$ \sin x = \huge \cos \left(x + \frac{3\pi}{2} \right) = (\cos x)'''$\\
$\dots$\\
Спробуємо ствердити, що працює формула: $(\cos x)^{(n)} = \huge \cos \left(x + \frac{n\pi}{2} \right)$. Покажемо, що для $(n+1)$-го члену це теж виконується\\
$(\cos x)^{(n+1)} = \left((\cos x)^{(n)}\right)' = \huge \left( \cos \left(x + \frac{n\pi}{2} \right) \right)' = -\sin \left(x + \frac{n\pi}{2} \right) = \\ = \cos \left(x + \frac{n\pi}{2} + \frac{\pi}{2} \right) = \cos \left(x + \frac{(n+1)\pi}{2} \right)$\\
Остаточно отримаємо, що\\
$f(x) = \cos x$\\
$\forall n \geq 1: f^{(n)}(x) = \huge \cos \left(x + \frac{n\pi}{2} \right)$
\bigline

А тепер уявімо собі іншу проблему: задані функції $f,g$, для яких існують $n$ похідних\\
Спробуємо знайти $(fg)^{(n)}$\\
Будемо робити по черзі:\\
$(fg)' = f'g+fg'$\\
$(fg)''=((fg)')'=(f'g+fg')'=(f'g)'+(fg')'=(f''g+f'g')+(f'g'+fg'')= f''g + 2f'g' + fg''$\\
$(fg)''' = ((fg)'')' = (f''g + 2f'g' + fg'')' = \\ = f'''g+f''g+2f''g'+2f'g''+f'g''+fg''' = f'''g + 3f''g' + 3f'g'' + fg'''$\\
Це можна продовжувати до нескінченності, але можна зробити деякі зауваження, що форма виразу схожа дуже на формулу Бінома-Ньютона, якщо порядок похідної замінити УЯВНО на степінь\\
Тоді якщо посилатись на МІ, то доведемо таку формулу
\bigline
\th{5.8.2. Формула Лейбніца} \\ 
$(f(x)g(x))^{(n)} = \huge \sum_{k=0}^n C_n^k f^{(k)}(x) g^{(n-k)}(x)$
\bigline
\ex{5.8.2.} Знайти похідну $n$-го порядку функції $y = x^2 \cos x$\\
$f(x) = x^2 \Rightarrow f'(x) = 2x \Rightarrow f''(x) = 2 \Rightarrow f'''(x) = 0 \Rightarrow \dots$\\
Коротше, $\forall n \geq 3: f^{(n)}(x) = 0$\\
$g(x) = \cos x \overset{\textrm{минулий приклад}}{\Rightarrow} \forall n \geq 1: g^{(n)}(x) = \huge \cos \left(x + \frac{n\pi}{2} \right)$\\
Скористаємось ф-лою Лейбніца:\\
$y^{(n)} = (f(x) g(x))^{(n)} = \huge \sum_{k=0}^n C_n^k f^{(k)}(x)g^{(n-k)}(x) = \\
= C_n^0 f(x)g^{(n)}(x) + C_n^1 f'(x)g^{(n-1)}(x) + C_n^2 f''(x)g^{(n-2)}(x) + \\ + C_n^3 f'''(x)g^{(n-3)}(x) + \dots + C_n^n f^{(n)}(x)g(x) = \\
= f(x)g^{(n)}(x) + n f'(x)g^{(n-1)}(x) + \frac{n(n-1)}{2} f''(x)g^{(n-2)}(x) + 0 = \\
= x^2 \cos \left(x + \frac{n\pi}{2} \right) + 2nx \cos \left(x + \frac{(n-1)\pi}{2} \right) + n(n-1)\cos \left(x + \frac{(n-2)\pi}{2} \right) = \\
$
Тут зауважу, що \\ $\huge \cos \left(x + \frac{(n-1)\pi}{2} \right) = \cos \left(x + \frac{n\pi}{2} - \frac{\pi}{2} \right) = \sin \left(x + \frac{n\pi}{2} \right)$\\
$\huge \cos \left(x + \frac{(n-2)\pi}{2} \right) = \cos \left(x + \frac{n\pi}{2} - \pi \right) = - \cos \left(x + \frac{n\pi}{2} \right)$\\
$= \huge [x^2 - n(n-1)]\cos \left(x + \frac{n\pi}{2} \right) + 2nx \sin \left(x + \frac{n\pi}{2} \right)$\\
Остаточно,\\
$y^{(n)} = \huge [x^2 - n(n-1)]\cos \left(x + \frac{n\pi}{2} \right) + 2nx \sin \left(x + \frac{n\pi}{2} \right)$
\bigline
\bigline
Повертаємось до \textbf{п. 5.7.}\\
Нагадую, є функція $y: \begin{cases} y = y(t) \\ x = x(t) \end{cases}$\\
Вже з'ясували, що $\huge y'_x(t) = \frac{y'_t(t)}{x'_t(t)}$\\
Знайдемо другу похідну:\\
$\huge y''_{x^2}(t) = (y'_x(t))'_x = \frac{(y'_x(t))'_t}{x'_t(t)} = \frac{y''_{t^2}(t)x'_t(t)-x''_{t^2}(t)y'_t(t)}{(x'_t(t))^3}$
\bigline
Складно, тому краще повернемось до прикладу з того пункту\\
Маємо $y: \begin{cases} x = \ln t \\ y = t^3 \end{cases}$\\
$x'_t = \huge \frac{1}{t}$,    $y'_t = 3t^2$\\
$\Rightarrow y'_x = 3t^3$\\
Тоді отримаємо:\\
$y''_{x^2} = \huge \frac{(y'_x)'_t}{x'_t} = \frac{9t^2}{t^3} = \frac{9}{t}$
\bigline
\subsection{Основні теореми}
\th{5.9.1. 'Лема' Ферма}\\
Задана функція $f: (a,b) \to \mathbb{R}$ - диференційована в т. $x_0 \in (a,b)$. Більш того, в т. $x_0$ функція $f$ приймає найбільше (або найменше) значення\\
Тоді $f'(x_0)=0$\\
\proof
Розглянемо випадок $\max$. Для $\min$ аналогічно\\
В т. $x_0$ функція $f$ приймає найбільше значення, тобто \\ $\forall x \in (a,b): f(x_0) \geq f(x)$\\
Оскільки $\exists f'(x_0)$, то тоді $\exists f'(x_0^+), f'(x_0^-)$\\
$f'(x_0^+) \huge \overset{\textrm{def.}}{=} \lim_{x \to x_0^+} \frac{f(x)-f(x_0)}{x-x_0}$  $\left( \frac{\leq 0}{\geq 0} \right)$ $\leq 0$\\
$f'(x_0^-) \huge \overset{\textrm{def.}}{=} \lim_{x \to x_0^-} \frac{f(x)-f(x_0)}{x-x_0}$ $\left( \frac{\leq 0}{\leq 0} \right)$ $\geq 0$\\
Таким чином, $0 \leq f'(x_0^-) = f'(x_0^+) \leq 0 \Rightarrow f'(x_0^-) = f'(x_0^+) = 0\\ \Rightarrow f'(x_0) = 0$ \qed

\begin{figure}[H]
\centering
\begin{tikzpicture}
\draw[thick,->] (2,0)--(5,0) node[anchor = north west] {$x$};
\draw[thick, domain=3:4.7, variable=\x, samples = 100] plot({\x}, {-(4-\x)^2 + 2});
\draw (3,-1pt)--(3,1pt) node [anchor = north] {$a$};
\draw (4.7,-1pt)--(4.7,1pt) node [anchor = north] {$b$};
\draw [dashed] (4,2)--(4,0) node [anchor = north] {$x_0$};
\draw [thick, red] (3.5,2)--(4.5,2) node [anchor = south, black] {$f'(x_0) = 0$};
\end{tikzpicture}
\end{figure}

\th{5.9.2. Теорема Ролля}\\
Задана функція $f: [a,b] \to \mathbb{R}$, $f \in C([a,b])$ та диференційована на $(a,b)$\\
Більш того, $f(a) = f(b)$\\
Тоді $\exists \xi \in (a,b): f'(\xi) = 0$\\
\proof
Оскільки $f \in C([a,b])$, то за Th. Вейерштраса,\\
$\exists x_1 \in [a,b]: f(x_1) = \huge \min_{x \in [a,b]} f(x)$\\
$\exists x_2 \in [a,b]: f(x_2) = \huge \max_{x \in [a,b]} f(x)$\\
Розглянемо два випадки:\\
I. $f(x) = const \Rightarrow f'(x) = 0 \forall x \in (a,b)$, $\xi = x$\\
II. $f(x) \neq const \Rightarrow$ або є $x_1$, або є $x_2$, або навіть обидва\\
Якщо беремо $x_2$, то функція $f$ приймає найбільше значення, тому за лемою Ролля, $f'(x_2) = 0 \Rightarrow \xi = x_2$\\
Для $x_1$ - аналогічно \qed
\bigline
\th{5.9.3. Теорема Лагранжа}\\
Задана функція $f: [a,b] \to \mathbb{R}$, $f \in C([a,b])$ та диференційована на $(a,b)$\\
Тоді $\exists c \in (a,b): f'(c) = \huge \frac{f(b)-f(a)}{b-a}$\\
\proof
Розглянемо функцію $h(x) = \huge (f(x)-f(a))- \frac{f(b)-f(a)}{b-a}(x-a)$\\
За сумою та добутками, маємо, що $h \in C([a,b])$ і теж диференційована на $(a,b)$\\
$h'(x) = \huge f'(x) - \frac{f(b)-f(a)}{b-a}$\\
Зауважимо, що $h(a) = 0$ та $h(b) = 0 \Rightarrow h(a) = h(b)$\\
Тому за теоремою Ролля, $\exists \xi = c \in (a,b): f'(c) = 0 \\ \Rightarrow f'(c) = \huge \frac{f(b)-f(a)}{b-a}$ \qed

\begin{figure}[H]
\centering
\begin{tikzpicture}
\pgfmathsetmacro{\const}{(3*ln(6)-3*ln(2.3))/(6-2.3)};
\draw[thick,->] (2,0)--(6.5,0) node[anchor = north west] {$x$};
\draw[thick, domain=2.3:6, variable=\x, samples = 100] plot({\x}, {3*ln(\x)});
\draw (2.3,-1pt)--(2.3,1pt) node [anchor = north] {$a$};
\draw (6,-1pt)--(6,1pt) node [anchor = north] {$b$};
\draw (2.3, {3*ln(2.3)})--(6,{3*ln(6)});
\draw [dashed] ({(3)/(\const)},{3*ln((3)/(\const))})--({(3)/(\const)},0) node [anchor = north] {$c$};
\draw[thick, domain=3:5, variable=\x, samples = 100, red] plot({\x}, {\const*(\x-(3)/(\const)) + 3*ln((3)/(\const))});
\end{tikzpicture}
\caption*{Для $f$ в т. $c$ проведемо дотичну. І в цій точці відрізок, що сполучає початкову та кінцеву точку, буде паралельна дотичній}
\end{figure}

\crl{5.9.3. Всі наслідки з теореми Лагранжа}\\
1. Якщо $\forall x \in (a,b): f'(x) = 0$, то $f(x) = const$\\
2. Якщо $\forall x \in (a,b): f'(x) = k$, то $f(x) = kx + q$\\
3. Нехай $g$ - така ж за властивостями як і $f$ \\ Якщо $\forall x \in (a,b): f'(x) = g'(x)$, то $f(x) = g(x) + C$\\
4. Якщо $\exists M \in \mathbb{R}: \forall x \in (a,b): |f'(x)| \leq M$, то вона задовільняє умові Ліпшиця
\bigline
\rm{5.9.4.(4).} \textbf{Умова Ліпшиця} $f$ виконується, якщо:
\begin{align*}
\exists L \in \mathbb{R}: \forall x_1,x_2 \in [a,b]: |f(x_1)-f(x_2)| \leq L|x_1-x_2|
\end{align*}
\proof
1. $\exists c: f(b)-f(a) = f'(c)(b-a) \Rightarrow f(b) = f(a)$\\
Але взагалі-то кажучи $\exists c \in (x_1,x_2) \subset (a,b): f(x_1) = f(x_2)$\\
Коротше, $f(x) = const$\bigline
2. Розглянемо функцію $g(x) = f(x) - kx$, теж неперервна і диференційована на $(a,b)$\\
Тоді $g'(x) = f'(x) - k \Rightarrow g'(x) = 0 \overset{\textrm{насл 1.}}{\Rightarrow} g(x) = q$\\
Отже, $g(x) = kx + q$\bigline
3. Розглянемо функцію $h(x) = f(x) - g(x)$, теж неперервна і диференційована на $(a,b)$\\
Тоді $h'(x) = f'(x) - g'(x) = 0 \overset{\textrm{насл 1.}}{\Rightarrow} h(x) = C \Rightarrow f(x) = g(x) + C$\bigline
4. $\exists c \in (x_1,x_2) \subset (a,b): f(x_2)-f(x_1)=f'(c)(x_2-x_1)$\\
$\Rightarrow |f(x_2)-f(x_1)|=|f'(c)||x_2-x_1| \leq M|x_2-x_1|$. Тоді встановлюючи $L=M$, маємо умову Ліпшиця \qed
\bigline
\th{5.9.4. Теорема Коші}\\
Задані функції $f,g: [a,b] \to \mathbb{R}$, $f \in C([a,b])$ та диференційовані на $(a,b)$. При цьому $g(x) \not\equiv 0$\\
Тоді $\exists c \in (a,b): \huge \frac{f'(c)}{g'(c)}  =  \frac{f(b)-f(a)}{g(b)-g(a)}$\\
\proof
За теоремою Лагранжа, отримаємо, що $\exists c \in (a,b):$\\
$f'(c) = \huge \frac{f(b)-f(a)}{b-a}, g'(c) = \huge \frac{g(b)-g(a)}{b-a}$\\
$\Rightarrow \huge \frac{f'(c)}{g'(c)} = \frac{f(b)-f(a)}{g(b)-g(a)}$ \qed
\bigline

\ex{5.9.5.} Довести нерівність: $|\arctg a - \arctg b| \leq |a-b|$\\
Оскільки $\arctg x$ - неперервна та диференційована на $(a,b)$, то за теоремою Лагранжа,\\
$\exists c \in (a,b): (\arctg x)'_{x = c} = \dfrac{\arctg b - \arctg a}{b-a}$\\
Тобто $\dfrac{1}{1+c^2} =\dfrac{\arctg b - \arctg a}{b-a}$\\
Тоді $|\arctg a - \arctg b| = \abs{\dfrac{1}{1+c^2}} |a-b| \leq |a-b|$

\subsection{Дослідження функції на монотонність}
Означення монотонної функції можна побачити в розділу про границі функції. Тому приступимо безпосередньо до теорем\\
\th{5.10.1.} Задана функція $f: [a,b] \to \mathbb{R}$, $f \in C([a,b])$ та диференційована на $[a,b]$\\
Функція $f$ монотонно $\left[ \begin{gathered} \textrm{зростає} \\ \textrm{спадає} \end{gathered} \right. \iff \forall x \in [a,b]: \left[ \begin{gathered} f'(x) \geq 0 \\ f'(x) \leq 0 \end{gathered} \right.$\\
\proof
Розглянемо випадок зростаючої функції. Для спадної аналогічно\\
$\boxed{\Rightarrow}$ Дано: $f$ - зростає\\
Оскільки диференційована $\forall x_0 \in [a,b]$, то \\ 
$\exists f'(x_0^+) = \huge \lim_{x \to x_0^+} \frac{f(x)-f(x_0)}{x-x_0} \left( \frac{\geq 0}{\geq 0} \right) \geq 0$\\
$\exists f'(x_0^-) = \huge \lim_{x \to x_0^-} \frac{f(x)-f(x_0)}{x-x_0} \left( \frac{\leq 0}{\leq 0} \right) \geq 0$\\
Також $f'(x_0^+) = f'(x_0^-)$, а отже, $\forall x_0 \in [a,b]: f(x_0) \geq 0$
\bigline
$\boxed{\Leftarrow}$ Дано: $\forall x \in [a,b]: f'(x) \geq 0$\\
Зафіксуємо такі $x_1,x_2$, що $x_2 \geq x_1$\\
Знаємо, що $\forall x \in (x_1,x_2) \subset [a,b]: f$ - неперервна та диференційована. Тоді за Лагранжом,\\
$\exists c \in (x_1,x_2): f(x_2)-f(x_1) = f'(c)(x_2-x_1) \geq 0$\\
Остаточно, $f(x_2) \geq f(x_1)$, тобто монотонно зростає \qed
\bigline
\th{5.10.2. Критерій строгої монотонності}\\
Задана функція $f: [a,b] \to \mathbb{R}$, $f \in C([a,b])$ та диференційована на $(a,b)$\\
Функція $f$ строго монотонно $\left[ \begin{gathered} \textrm{зростає} \\ \textrm{спадає} \end{gathered} \right. \iff$\\
 1. $\forall x \in (a,b): \left[ \begin{gathered} f'(x) \geq 0 \\ f'(x) \leq 0 \end{gathered} \right.$\\
 2. $\not\exists (\alpha, \beta) \subset [a,b]: \forall x \in (\alpha, \beta): f'(x) = 0$\\
\proof
Розглянемо випадок зростаючої функції. Для спадної аналогічно\\
$\boxed{\Rightarrow}$ Дано: $f$ - монотонно строго зростає\\
Тоді за попередньою теоремою, $\forall x \in (a,b): f'(x) \geq 0$\\
А тепер припустимо, що $\exists (\alpha, \beta) \subset [a,b]: \forall x \in (\alpha, \beta): f'(x) = 0$\\
Тоді за наслідком Лагранжа, $f(x) = const$ на інтервалі $(\alpha, \beta)$, що суперечить умові строгої монотонності\\
Таким чином, отримали 2 пункти з теореми
\bigline
$\boxed{\Leftarrow}$ Дано:\\
1. $\forall x \in (a,b): f'(x) \geq 0$\\
2. $\not\exists (\alpha, \beta) \subset [a,b]: \forall x \in (\alpha, \beta): f'(x) = 0$\\
З першого пункту одразу випливає за попередньою теоремою, що $f$ - монотонно зростає\\
А тепер припустимо, що наша функція дійсно зростає нестрого, тобто\\
$\exists x_1^*, x_2^* \in (a,b): x_1^* < x_2^* \Rightarrow f(x_1^*) = f(x_2^*)$\\
Тоді $\forall x \in (x_1^*,x_2^*): f(x_1^*) \leq f(x) \leq f(x_2^*)$\\
Звідси $f(x) = const$ на інтервалі $(x_1^*, x_2^*) \subset [a,b]$, а отже, $f'(x) = 0$ - суперечність.\\
Таким чином, функція $f$ монотонно строго зростає \qed
\bigline
\ex{5.10.1.} Розглянемо функцію $f(x) = x^3$\\
Похідна $f'(x) = 3x^2 \geq 0$. Тобто дана функція монотонно зростає на всьому інтервалі
\bigline
\subsection{Екстремуми функції}
\subsubsection{Локальні}
\defin{5.11.1.} Задана функція $f: A \to \mathbb{R}, x_0 \in A$\\
Точку $x_0$ називають точкою \textbf{локального}\\
\textbf{максимуму}, якщо $\exists \varepsilon > 0: \forall x \in (x_0-\varepsilon, x_0 +\varepsilon) \cap A: f(x_0) \geq f(x)$\\
\textbf{мінімуму}, якщо $\exists \varepsilon > 0: \forall x \in (x_0-\varepsilon, x_0 +\varepsilon) \cap A: f(x_0) \leq f(x)$
\bigline
\defin{5.11.2.} Якщо в т. $x_0$ маємо $f'(x_0) = 0$ або $\not\exists f'(x_0)$, то таку точку називають \textbf{критичною}
\bigline
\th{5.11.3. Необхідна умова для екстремума}\\
Задана функція $f: A \to \mathbb{R}$ та т. $x_0 \in A$ - локальний екстремум\\
Тоді ця точка є критичною\\
\proof
Розглянемо випадок точки максимуму. Для мінімума аналогічно\\
$x_0$ - локальна точка максимуму - тобто, приймає в околі т. $x_0$ функція $f$ приймає найбільшого значення. Тоді за лемою Ферма, $f'(x_0) = 0$\\
При строгого локального максимуму $\not\exists f'(x_0)$ \qed
\bigline
\ex{5.11.3.} Головні приклади, чому ця умова не є достатньою, - функції $f(x) = x^3$, $g(x) = \huge \frac{1}{x}$\\
$f'(x) = 3x^2 \overset{f'(x) = 0}{\Rightarrow} x = 0$, але вона не є естремумом, оскільки минулого разу дізнались, що така функція зростає всюди\\
$f'(x) = \huge -\frac{1}{x^2}$, тобто $\not\exists f'(0)$, але не екстремум
\bigline
\th{5.11.4. Достатня умова для екстремума}\\
Задана функція $f: A \to \mathbb{R}$ та т. $x_0 \in A$ - критична точка\\
Відомо, що $f$ - диференційований на $(x_0-\varepsilon,x_0) \cup (x_0, x_0+\varepsilon)$\\
Також $\forall x \in \begin{cases} (x_0-\varepsilon,x_0): f'(x_0) < 0 \\ (x_0,x_0+\varepsilon): f'(x_0) > 0 \end{cases}$ (або нерівності навпаки)\\
Тоді $x_0$ - точка локального мінімуму (максимуму)\\
\proof
Розглянемо випадок, коли $\forall x \in \begin{cases} (x_0-\varepsilon,x_0): f'(x_0) < 0 \\ (x_0,x_0+\varepsilon): f'(x_0) > 0 \end{cases}$. Для нерівностей навпаки все аналогічно\\
Тоді звідси $f$ - спадає на $(x_0 - \varepsilon, x_0)$ і зростає на $(x_0, x_0 + \varepsilon)$\\
Або математично, \\ $\forall x \in (x_0-\varepsilon, x_0): f(x_0) < f(x)$ та $\forall x \in (x_0, x_0+\varepsilon): f(x_0) < f(x)$\\
За означенням, це й є точка локального мінімуму \qed
\bigline
\\
\textbf{Висновок:} щоб знайти локальний екстремум, треба спочатку знайти всі критичні точки, а потім дослідити, які значення вона приймає навколо
\bigline
\ex{5.11.4.} Задана функція $f(x) = x^3-3x^2-9x+2$. Знайдемо всі локальні екстремуми\\
Спочатку шукаємо критичні точки:\\
$f'(x) = 3x^2-6x-9 = 0$\\
$f'(x) = 0 \Rightarrow x^2-2x-3 = 0 \Rightarrow x = -1,x = 2$\\
Перевіримо екстремуми на інтервалі
\bigline
\begin{tikzpicture}
\draw[thick, ->] (-3,0)--(4,0) node[below = 2pt] {$x$};
\draw[thick] (-1,-0.25)--(-1,0.25) node[below = 6pt] {$-1$};
\draw[thick] (2,-0.25)--(2,0.25) node[below = 6pt] {$2$};
\draw[thick, ->] (-2,0.5)--(-1.5,1);
\draw[thick, ->] (0,1)--(0.5,0.5);
\draw[thick, ->] (2.5,0.5)--(3,1);
\end{tikzpicture}
\\
Стрілки вказують на зростання або на спадання функції на даному інтервалі. Тоді можемо зробити висновок, що $x=-1$ - локальний максимум, а $x=2$ - локальний мінімум
\bigline
\subsubsection{Глобальні}
\th{5.11.5.} Задана функція $f: [a,b] \to \mathbb{R}$, $f \in C([a,b])$\\
$x_1,x_2,\dots,x_n$ - критичні точки на $(a,b)$. Тоді\\
$\huge \max_{x \in [a,b]} f(x) = \max\{f(a),f(x_1),f(x_2),\dots,f(x_n),f(b)\}$\\
$\huge \min_{x \in [a,b]} f(x) = \min\{f(a),f(x_1),f(x_2),\dots,f(x_n),f(b)\}$\\
\proof
Нехай $\huge \max_{x \in [a,b]} f(x) = f(x^*)$, де $x^* \in (a,b)$\\
Оберемо $\varepsilon > 0: \varepsilon = \min\{b-x^*, x^*-a\}$\\
Тоді маємо, що $\forall x \in (x^*-\varepsilon, x^*+\varepsilon) \subset (a,b): f(x^*) \geq f(x)$\\
Тоді $x^*$ - локальний екстремум $\Rightarrow$ $x^*$ - критична точка, тобто $f'(x^*)=0$, або $\not\exists f'(x^*)$\\
Тому $x^* \in \{x_1,x_2,\dots,x_n\}$, тобто це є один з наборів критичних точок.\\
Таким чином, $\huge \max_{x \in [a,b]} f(x) = \max\{f(a),f(x_1),f(x_2),\dots,f(x_n),f(b)\}$\\
Випадок, коли $\huge \max_{x \in [a,b]} f(x) = f(a) \textrm{ або } f(b)$ автоматично доводить теорему\\
Випадок $\min$ аналогічний \qed
\bigline
\th{5.11.6.} Є три випадки для функції. $x_1,x_2,\dots,x_n$ - критичні точки на області визначення для кожного пункту. Розглянемо кожну окремо:\\
1. Задана фукнція $f: (a,+\infty) \to \mathbb{R}$, $f \in C((a,+\infty))$. Тоді\\
$\huge \sup_{x \in (a,+\infty)} f(x) = \max\{f(a^+),f(x_1),f(x_2),\dots,f(x_n),f(+\infty)\}$\\
$\huge \inf_{x \in (a,+\infty)} f(x) = \min\{f(a^+),f(x_1),f(x_2),\dots,f(x_n),f(+\infty)\}$\\
\bigline
2. Задана фукнція $f: (-\infty,b) \to \mathbb{R}$, $f \in C((-\infty,b))$. Тоді\\
$\huge \sup_{x \in (-\infty,b)} f(x) = \max\{f(-\infty),f(x_1),f(x_2),\dots,f(x_n),f(b^-)\}$\\
$\huge \inf_{x \in (-\infty,b)} f(x) = \min\{f(-\infty),f(x_1),f(x_2),\dots,f(x_n),f(b^-)\}$\\
\bigline
3. Задана фукнція $f: \mathbb{R} \to \mathbb{R}$, $f \in C(\mathbb{R})$. Тоді\\
$\huge \sup_{x \in \mathbb{R}} f(x) = \max\{f(-\infty),f(x_1),f(x_2),\dots,f(x_n),f(+\infty)\}$\\
$\huge \inf_{x \in \mathbb{R}} f(x) = \min\{f(-\infty),f(x_1),f(x_2),\dots,f(x_n),f(+\infty)\}$\\
\textit{Всі вони доводяться аналогічно до минулої теореми}
\bigline
\ex{5.11.5.} Задана функція $f(x) = |x^2-7x+10|$ на інтервалі $[-1,4]$. Знайдемо глобальні екстремуми\\
$f(x) = |(x-2)(x-5)| \Rightarrow f(x) = \begin{cases} x^2-7x+10, x \in [-1,2] \\ -x^2+7x-10, x \in (2,4] \end{cases}$\\
$f'(x) = \begin{cases} 2x-7, x \in [-1,2] \\ -2x+7, x \in (2,4] \end{cases}$\\
$f'(x) = 0 \Rightarrow \left[ \begin{gathered} x=3.5 \\ x=-3.5 \end{gathered} \right.$\\
Але точка $x=-3.5$ не в інтервалі, тому залишається $x=3.5$\\
Якщо перевірити на інтервалі, то ця точка буде локальним максимумом\\
Отже, ми маємо безліч кандидатів:\\
$x=-1$, $x=2$, $x=3.5$, $x=4$\\
$f(-1) = 18$, $f(2) = 0$, $f(3.5) = 2.25$, $f(4) = 2$\\
Остаточно маємо:\\
$\huge \max_{x \in [-1,4]} f(x) = 18$, $\huge \min_{x \in [-1,4]} f(x) = 0$
\bigline
\subsection{Формула Тейлора та правила Лопіталя}
\th{5.12.1. Теорема Тейлора (варіант 1)}\\
Задана функція $f \in C^{(n)}([a,b])$, т. $x_0 \in [a,b]$ та існує $f^{(n+1)}$ на $(a,b)$\\
Тоді $\exists \theta(x) \in (x_0,x):$\\
$f(x) = \huge \sum_{k=0}^n \frac{f^{(k)}(x_0)}{k!}(x-x_0)^k + \underset{\textrm{остатковий член у формі Лагранжа}}{\frac{f^{(n+1)}(\theta(x))}{(n+1)!}(x-x_0)^{n+1}}$\\
\proof
Розглянемо наступну функцію:\\
$g(t) = f(x) - \left(\huge \sum_{k=0}^n \frac{f^{(k)}(t)}{k!}(x-t)^k + \frac{L}{(n+1)!} (x-t)^{(n+1)} \right)$\\
За умовою теореми та властивостями неперервних функцій, $g \in C([a,b])$\\
Знайдемо її похідну:\\
$g'(t) = 0 - \huge \left(f(t) + \frac{f'(t)}{1!}(x-t) + \dots + \frac{f^{(n)}}{n!}(x-t)^n + \frac{L}{(n+1)!}(x-t)^{(n+1)} \right)' = \\
= -f'(t) - \frac{f''(t)}{1!}(x-t) - \frac{f'(t)}{1!}(-1) - \frac{f'''(t)}{2!}(x-t)^2 - \frac{f''(t)}{2!}(-2)(x-t) - \dots - \frac{f^{(n+1)}(t)}{n!}(x-t)^n - \frac{f^{(n)}(t)}{n!}(-n)(x-t)^{n-1} - \frac{L}{(n+1)!}(-1)(n-1)(x-t)^{n} =$\\
Якщо обережно придивитись, то із нашої суми залишуться лише два доданки, а решта скоротяться\\
$= \huge -\frac{f^{(n+1)}(t)}{n!}(x-t)^n + \frac{L}{n!}(x-t)^n$\\
Зрозуміло, що якщо $t=x$, то $g(x) = 0$\\
Ми уявно хочемо таке $L$, щоб $g(x_0) = 0$\\
Тоді уявно спрацьовує теорема Ролля на $[x_0,x]$, тобто $\exists c \in (x_0,x): g'(c) = 0$\\
Тоді:\\
$g'(c) = \huge \frac{L}{n!}(x-c)^n - \frac{f^{(n+1)}(c)}{n!}(x-c)^n = 0$\\
$\Rightarrow L = f^{(n+1)}(c)$\\
Підставимо отримане значення $L$ в наше рівняння $g(x_0) = 0$:\\
$g(x_0) = f(x) - \left(\huge \sum_{k=0}^n \frac{f^{(k)}(x_0)}{k!}(x-x_0)^k + \frac{f^{(n+1)}(c)}{(n+1)!} (x-x_0)^{(n+1)} \right) = 0$\\
Перепозначимо $c = \theta(x) \in (x_0,x)$ та отримаємо нашу бажану формулу:\\
$f(x) = \huge \sum_{k=0}^n \frac{f^{(k)}(x_0)}{k!}(x-x_0)^k + \frac{f^{(n+1)}(\theta(x))}{(n+1)!}(x-x_0)^{n+1}$\qed
\bigline
\th{5.12.1. Теорема Тейлора (варіант 2)}\\
Задана функція $f \in C^{(n)}([a,b])$, т. $x_0 \in [a,b]$ та існує $f^{(n+1)}$ на $(a,b)$\\
Тоді\\
$f(x) = \huge \sum_{k=0}^n \frac{f^{(k)}(x_0)}{k!}(x-x_0)^k + \underset{\textrm{остатковий член у формі Пеано}}{o((x-x_0)^n)}, x \to x_0$\\
\proof
Вже доведена попередня теорема. Тому достатньо довести, що:\\
$\huge \frac{f^{(n+1)}(\theta(x))}{(n+1)!}(x-x_0)^{(n+1)} = o((x-x_0)^n), x \to x_0$\\
Це теж саме, що:\\
$\huge \lim_{x \to x_0} \frac{\displaystyle \frac{f^{(n+1)}(\theta(x))}{(n+1)!}(x-x_0)^{(n+1)}}{(x-x_0)^n} = \lim_{x \to x_0} \frac{f^{(n+1)}(\theta(x))}{(n+1)!}(x-x_0)=$\\
У нас $f \in C^{(n+1)}([a,b])$, тому є обмеженою. $x-x_0 \to 0$ - н.м.. Отже, н.м*обм. = н.м\\
$= 0$ \qed
\bigline
\textbf{Основні розклади в Тейлора}\\
Всі вони розглядатимуться в т. $x_0 = 0$, всюди $x \to x_0$\\
I. $e^x \huge = 1 + \frac{x}{1!} + \frac{x^2}{2!} + \dots + \frac{x^n}{n!} + o(x^n)$\\
II. $\sin x \huge = \frac{x}{1!} - \frac{x^3}{3!} + \dots + \frac{(-1)^n}{(2n+1)!}x^{(2n+1)} + o(x^{2n+2})$\\
III. $\cos x \huge = 1 - \frac{x^2}{2!} + \frac{x^4}{4!} + \dots + \frac{(-1)^n}{(2n)!}x^{2n} + o(x^{2n+1})$\\
IV. $(1+x)^{\alpha} \huge = \frac{\alpha x}{1!} + \frac{\alpha (\alpha-1) x^2}{2!} + \dots + \frac{\alpha (\alpha-1)\dots(\alpha-(n-1)) x^n}{n!} + o(x^n)$
\bigline
\ex{5.12.2.} Обчислити границю функції $\huge \lim_{x \to 0} \frac{\huge e^x-1-\sin x - \frac{x^2}{2}}{x(1-\cos x)}$\\
Маємо, що:\\
$\huge \lim_{x \to 0} \frac{\huge e^x-1-\sin x - \frac{x^2}{2}}{x(1-\cos x)} = \lim_{x \to 0} \frac{\huge e^x - 1 - \sin x - \frac{x^2}{2}}{2x \sin^2 \frac{x}{2}} = 2\lim_{x \to 0} \frac{\huge e^x - 1 - \sin x - \frac{x^2}{2}}{x^3} = $\\
Розкладемо $e^x$ та $\sin x$ до степеня знаменника:\\
$= \huge 2\lim_{x \to 0} \frac{\displaystyle 1 + x + \frac{x^2}{2} + \frac{x^3}{6} + o(x^3) - 1 - x + \frac{x^3}{6} + o(x^4) - \frac{x^2}{2}}{x^3} = \\ = 2\lim_{x \to 0} \frac{\displaystyle \frac{x^3}{6} + o(x^3) + \frac{x^3}{6} + o(x^4)}{x^3} = 2 \lim_{x \to 0} \left(\frac{1}{3} + \frac{o(x^3)}{x^3} + \frac{o(x^4)}{x^3} \right) = \\ 2 \lim_{x \to 0} \left(\frac{1}{3} + \frac{x^4}{x^3} + \frac{x^5}{x^3} \right) = \frac{2}{3}$
\bigline
\th{5.12.3. I правило Лопіталя}\\
Задані функції $f,g \in C([a,b])$ - диференційовані на $(a,b)$ та $\forall x \in (a,b): g'(x) \neq 0$. Також відомо, що:\\
1. $\exists \huge \lim_{x \to b^-} f(x) = 0$, $\exists \huge \lim_{x \to b^-} g(x) = 0$\\
2. $\exists \huge \lim_{x \to b^-} \frac{f'(x)}{g'(x)} = L$\\
Тоді $\exists \huge \lim_{x \to b^-} \frac{f(x)}{g(x)} = L$\\
\textit{Тут можна замість $x \to b^-$ записати $x \to a^+$, доведення аналогічне}\\
\proof
$\huge \lim_{x \to b^-} \frac{f(x)}{g(x)} = \lim_{x \to b^-} \frac{f(x)-f(b)}{g(x)-g(b)} =$\\
За теоремою Коші, $\exists c \in (x, b): \huge \frac{f(x)-f(b)}{g(x)-g(b)} = \frac{f'(c)}{g'(c)}$\\
Тут $x < c < b$. Коли $x \to b^-$, $b \to b^-$. Отже, $c \to b^-$\\
$= \huge \lim_{c \to b^-} \frac{f'(c)}{g'(c)} = L$ \qed
\bigline

\th{5.12.4. II правило Лопіталя}\\
Задані функції $f,g \in C([a,b])$ - диференційовані на $(a,b)$ та $\forall x \in (a,b): g'(x) \neq 0$. Також відомо, що:\\
1. $\exists \huge \lim_{x \to b^-} g(x) = \infty$\\
2. $\exists \huge \lim_{x \to b^-} \frac{f'(x)}{g'(x)} = L$\\
Тоді $\exists \huge \lim_{x \to b^-} \frac{f(x)}{g(x)} = L$\\
\textit{Тут можна замість $x \to b^-$ записати $x \to a^+$, доведення аналогічне}\\
\proof
$\exists \huge \lim_{x \to b^-} \frac{f'(x)}{g'(x)} = L \overset{\textrm{def.}}{\iff} \forall \varepsilon > 0: \exists \delta: \forall x \in (b-\delta,b): \Rightarrow \abs{\frac{f'(x)}{g'(x)} - L} < \varepsilon \Rightarrow L-\varepsilon < \frac{f'(x)}{g'(x)} < L+\varepsilon$\\
Тоді за Коші, $\forall x,x_0 \in (b-\delta, b): \exists c \in (x_0,x): \huge \frac{f(x)-f(x_0)}{g(x)-g(x_0)} = \frac{f'(c)}{g'(c)}$\\
Для $\huge c \in (x_0,x) \Rightarrow c \in (b-\delta, b) \Rightarrow L-\varepsilon < \frac{f'(c)}{g'(c)} < L+\varepsilon$\\
$\Rightarrow \huge L-\varepsilon <\frac{f(x)-f(x_0)}{g(x)-g(x_0)} < L+\varepsilon \Rightarrow L-\varepsilon <\frac{\frac{f(x)}{g(x)}-\frac{f(x_0)}{g(x)}}{1 - \frac{g(x_0)}{g(x)}} < L+\varepsilon$\\
Фіксуємо $x_0 \in (b-\delta, b)$\\
Нам ще відомо з дано, що: \\ $\huge \huge \lim_{x \to b^-} g(x) = \infty \Rightarrow \huge \lim_{x \to b^-} \frac{f(x_0)}{g(x)} = 0$ та $\huge \huge \lim_{x \to b^-} \frac{g(x_0)}{g(x)} = 0 \overset{\textrm{def.}}{\iff} $\\
Для нашого $\varepsilon: \exists \delta_1, \delta_2: \forall x \in (b-\delta_1, b), \forall x \in (b-\delta_2, b) \\ \Rightarrow \huge \abs{\frac{f(x_0)}{g(x)}}<\varepsilon, \abs{\frac{g(x_0)}{g(x)}}<\varepsilon$\\
Розглянемо $\tilde{\delta} = \min\{\delta, \delta_1, \delta_2 \}$\\
$x_0 \in (b-\tilde{\delta},b)$\\
$\Rightarrow \huge -\varepsilon<\frac{f(x_0)}{g(x)}<\varepsilon, -\varepsilon<\frac{g(x_0)}{g(x)}<\varepsilon$\\
Скоро це сюди підставимо:\\
$\huge L-\varepsilon <\frac{\frac{f(x)}{g(x)}-\frac{f(x_0)}{g(x)}}{1 - \frac{g(x_0)}{g(x)}} < L+\varepsilon \iff \\ 
\iff (L-\varepsilon)\left(1 - \frac{g(x_0)}{g(x)}\right) < \frac{f(x)}{g(x)} -\frac{f(x_0)}{g(x)} < (L+\varepsilon)\left(1 - \frac{g(x_0)}{g(x)}\right) \iff
\iff (L-\varepsilon)(1-\varepsilon) < \frac{f(x)}{g(x)} - \frac{f(x_0)}{g(x)} < (L+\varepsilon)(1+\varepsilon) \iff \\
\iff (L-\varepsilon)(1-\varepsilon) + \frac{f(x_0)}{g(x)} < \frac{f(x)}{g(x)} < (L+\varepsilon)(1+\varepsilon) + \frac{f(x_0)}{g(x)} \iff \\
\iff L-2\varepsilon - L\varepsilon + \varepsilon^2 < + \frac{f(x)}{g(x)} < L + 2\varepsilon + L\varepsilon + \varepsilon^2 \iff \\
\iff -L\varepsilon-3\varepsilon<-L\varepsilon-2\varepsilon-\varepsilon^2<-L\varepsilon-2\varepsilon+\varepsilon^2 < + \frac{f(x)}{g(x)} - L < L\varepsilon + 2\varepsilon + \varepsilon^2 < L\varepsilon + 3\varepsilon
$
$\huge \Rightarrow \abs{\frac{f(x)}{g(x)} - L} < \varepsilon(L+3)$\\
Остаточно:\\
$\huge \forall \varepsilon>0: \exists \tilde{\delta}: \forall x \in (b-\tilde{\delta},b): \abs{\frac{f(x)}{g(x)}-L}<\varepsilon(L+3) \Rightarrow \exists \lim_{x \to b^-} \frac{f(x)}{g(x)} = L$ \qed
\bigline
\ex{5.12.5.} Обчислити границю $\huge \lim_{x \to 0^+} x^x$\\
$\huge \lim_{x \to 0^+} x^x = \lim_{x \to 0^+} e^{x \ln x} = e^{\displaystyle \lim_{x \to x_0} \frac{\ln x}{\frac{1}{x}}} \boxed{=}$\\
Перевіримо цю границю за Лопіталем:\\
$\huge \lim_{x \to 0^+} \frac{(\ln x)'}{\left(\frac{1}{x}\right)'} = \lim_{x \to 0^+} \frac{\frac{1}{x}}{-\frac{1}{x^2}} = 0$\\
Отже, можемо продовжувати наш ланцюг обчислення:\\
$\boxed{=} e^0 = 1$
\bigline
\rm{5.12.6.} Границю типа $\huge \lim_{x \to 0} \frac{\sin x}{x}$ в \underline{жодному (!)} випадку не можна рахувати за Лопіталем, хоча й результат буде таким самим. Все це тому, що $(\sin x)'$ ми отримали завдяки цієї границі, ми посилаємось на те, що ми знаємо цю границю \underline{вже}. Коротше, замнений круг відносно логічної послідовності виклада
\bigline

\subsection{Опуклі функції та точки перегину}
Розглянемо графік функції $f(x)$ на множині $A$. Із множини $A$ розглядаються дві точки $x_1,x_2$, так, що $x_1 > x_2$
\begin{figure}[H]
\centering
\begin{tikzpicture}
\draw[thick, ->] (-2,0)--(4,0) node[anchor = north] {$x$};
\draw[thick, ->] (0,-0.5)--(0,5) node[anchor = east] {$y$};

\draw[thick, domain=-1.5:4, variable=\x, samples = 1000] plot({\x}, {0.5*(\x-1)^2}) node at (2,1.2) {$f(x)$};
\node[black] at (-1,2) [circle,fill,inner sep=1pt, draw = black]{};
\node[black] at (3.5,3.125) [circle,fill,inner sep=1pt, draw = black]{};
\draw (-1,2)--(3.5,3.125) node[anchor = south east] {$l(x)$};
\draw[dashed] (-1,2)--(-1,0) node [anchor = north] {$x_1$};
\draw[dashed] (3.5,3.125)--(3.5,0) node [anchor = north] {$x_2$};

\end{tikzpicture}
\end{figure}
Це приклад так називаємої \textbf{опуклої функції донизу} (або просто опуклої), коли на множині $A$ справедлива нерівність
\begin{align*}
\forall x \in A: f(x) \leq l(x)
\end{align*}
Прийнято трошки інше означення, а це просто пояснення, звідки все це береться\\
Знайдемо рівняння прямої, що проходить через т. $(x_1,f(x_1)), (x_2,f(x_2))$\\
$\dfrac{x-x_1}{x_2-x_1} = \dfrac{l(x)-f(x_1)}{f(x_2)-f(x_1)} \Rightarrow l(x) = f(x_1) + \dfrac{f(x_2)-f(x_1)}{x_2-x_1}(x-x_1)$\\
Її підставити можна в нерівність, проте таке означення все рівно не є зручним\\
Зафіксуємо $\lambda \in [0,1]$ та розглянемо точку $x = \lambda x_1 + (1-\lambda) x_2$\\
Для довільних $\lambda$ точка $x \in [x_1,x_2]$. \\
Я якщо це рівняння розв'язти відносно $\lambda$, ми отримаємо, що\\
$\lambda = \dfrac{x_2-x}{x_2-x_1} \hspace{1cm} 1-\lambda = \dfrac{x-x_1}{x_2-x_1}$\\
Отримане $\lambda \in (0,1)$. Тоді\\
$x = \dfrac{x_2-x}{x_2-x_1} x_1 + \dfrac{x-x_1}{x_2-x_1} x_2$, це все $\forall x_1 < x < x_2$\\
Але поки що обмежимось першим виглядом\\
 Підставимо цю точку в рівняння прямої\\
$l(x) = l(\lambda x_1 + (1-\lambda) x_2) = f(x_1) + \dfrac{f(x_2)-f(x_1)}{x_2-x_1} (\lambda x_1 + (1-\lambda)x_2 - x_1) = \\
= f(x_1) + (f(x_2)-f(x_1))(1-\lambda) = \lambda f(x_1) + (1-\lambda)f(x_2)$\\
Таким чином, якщо повернутись до нерівності, то отримаємо наступне:
\begin{align*}
\forall \lambda \in [0,1]: f(\lambda x_1 + (1-\lambda)x_2)) \leq \lambda f(x_1) + (1-\lambda) f(x_2)
\end{align*}
А ось таке означення можна використовувати подалі для інших досліджень\\
Аналогічні міркування будуть для \textbf{опуклої функції догори} (або просто угнутої), але тут нерівність навпаки
\bigline
\defin{5.13.1.} Задана функція $f: A \to \mathbb{R}$\\
Цю функцію називають \textbf{опуклою $\underset{\textrm{догори}}{\textrm{донизу}}$}, якщо
\begin{align*}
\forall x_1,x_2 \in A: \forall \lambda \in [0,1]: f(\lambda x_1 + (1-\lambda)x_2) \underset{\geq}{\leq} \lambda f(x_1) + (1-\lambda)f(x_2)
\end{align*}
\rm{5.13.1.} Якщо $\lambda \in (0,1)$ нерівність строга
\bigline
\lm{5.13.2. Лема про 3 хорди}\\
Функція $f: A \to \mathbb{R}$ опукла донизу $\iff$ справедлива нерівність\\
$\dfrac{f(x)-f(x_1)}{x-x_1} \leq \dfrac{f(x_2)-f(x_1)}{x_2-x_1} \leq \dfrac{f(x_2)-f(x)}{x_2-x}$\\
де $x \in (x_1,x_2) \subset A$
\begin{figure}[H]
\centering
\begin{tikzpicture}
\draw[thick, ->] (-2,0)--(4,0) node[anchor = north] {$x$};
\draw[thick, ->] (0,-0.5)--(0,5) node[anchor = east] {$y$};

\draw[thick, domain=-1.5:4, variable=\x, samples = 1000] plot({\x}, {0.5*(\x-1)^2}) node at (3,4) {$f(x)$};
\node[black] at (-1,2) [circle,fill,inner sep=1pt, draw = black]{};
\node[black] at (3.5,3.125) [circle,fill,inner sep=1pt, draw = black]{};
\node[black] at (2,0.5) [circle,fill,inner sep=1pt, draw = black]{};

\node at (-1,2+0.5) {$P_1$};
\node at (3.5,3.125+0.5) {$P_2$};
\node at (2-0.1,0.5+0.4) {$P$};

\draw (-1,2)--(3.5,3.125);
\draw (-1,2)--(2,0.5);
\draw (2,0.5)--(3.5,3.125);
\draw[dashed] (-1,2)--(-1,0) node [anchor = north] {$x_1$};
\draw[dashed] (3.5,3.125)--(3.5,0) node [anchor = north] {$x_2$};
\draw[dashed] (2,0.5)--(2,0) node [anchor = north] {$x$};

\end{tikzpicture}
\end{figure}
Нерівність означає наступне: кутовий коефіцієнт $PP_1 \leq$ кутовий коефіцієнт $P_2P_1 \leq$ кутовий коефіцієнт $P_2P$
\bigline
\rm{5.13.2.(1)} Для опуклої догори нерівність навпаки\\
\rm{5.13.2.(2)} Для строгої опуклості нерінвість буде строгою
\bigline
\proof
Зафіксуємо точки $x_1,x_2 \in A$ та точку $x \in (x_1,x_2)$\\
$f$ - опукла донизу $\iff f(x) \leq \dfrac{x_2-x}{x_2-x_1}f(x_1) + \dfrac{x-x_1}{x_2-x_1}f(x_2)$\\
$\iff (x_2-x_1) f(x) \leq (x_2-x)f(x_1) + (x-x_1)f(x_1)$\\
$\iff (f(x)-f(x_1))(x_2-x_1) \leq (f(x_2)-f(x))(x-x_1)$\\
$\iff \dfrac{f(x)-f(x_1)}{x-x_1} \leq \dfrac{f(x_2)-f(x)}{x_2-x}$ \\
Середня нерівність мене поки що не цікавить, це я так, щоб було \qed
\bigline
\lm{5.13.3.} Задана функція $f: A \to \mathbb{R}$ - диференційована на $A$\\
$f$ - опукла $\underset{\textrm{догори}}{\textrm{донизу}}$ $\iff$ $f'$ $\underset{\textrm{не зростає}}{\textrm{не спадає}}$ на $A$\\
\proof
$\boxed{\Rightarrow}$ Дано: $f$ - опукла донизу\\
Розглянемо т. $x_1, x_2 \in A$, тоді\\
$\dfrac{f(x)-f(x_1)}{x-x_1} \leq \dfrac{f(x_2)-f(x)}{x_2-x}$\\
Ба більше, оскільки $f$ - диференційована, то $\exists f'(x_1), \exists f'(x_2)$\\
Тоді отримаємо ось що, використовуючи границі в нерівностях:\\
$f'(x_1) = \huge \lim_{x \to x_1^+} \dfrac{f(x)-f(x_1)}{x-x_1} \leq \dfrac{f(x_2)-f(x_1)}{x_2-x_1}$\\
$\dfrac{f(x_2)-f(x_1)}{x_2-x_1} \leq \huge \lim_{x \to x_2^-} \dfrac{f(x_2)-f(x)}{x_2-x} = f'(x_2)$\\
Отже, $\forall x_1,x_2 \in A: x_2 > x_1 \Rightarrow f'(x_2) \geq f'(x_1)$
\bigline
$\boxed{\Leftarrow}$ Дано: $f'$ - неспадна на $A$, тобто\\
$\forall x_1,x_2 \in A: x_1 < x_2 \Rightarrow f'(x_1) \leq f'(x_2)$\\
Оскільки $f$ - диференційована на $A$, то за теоремою Лагранжа,\\
$f'(x_1) = \dfrac{f(c) - f(c_1)}{c- c_1}$\\
$f'(x_2) = \dfrac{f(c_2) - f(c)}{c_2 - c}$\\
$\Rightarrow \dfrac{f(c) - f(c_1)}{c- c_1} \leq \dfrac{f(c_2) - f(c)}{c_2 - c}$\\
Тоді маємо, що $f$ - випукла донизу \qed
\bigline

\rm{5.14.3.} Майже аналогічно для строгої опуклості\\
Єдине, що в першій частині доведення треба застосувати теореми Лагранжа для точок $z_1 \in (x_1,x)$ та $z_2 \in (x,x_2)$
\bigline

\th{5.13.4.} Задана функція $f: (a,b) \to \mathbb{R}$ - $f \in C([a,b])$ та двічі диференційована на $(a,b)$\\
$f$ - опукла $\left[ \begin{gathered} \textrm{догори} \\ \textrm{донизу} \end{gathered} \right.$ $\iff$\\
1. $\forall x \in (a,b): \left[ \begin{gathered} f''(x) \leq 0 \\ f''(x) \geq 0 \end{gathered} \right.$\\
2. $\not\exists (\alpha, \beta) \subset (a,b): f''(x) = 0$\\
\proof
$f$ - опукла догори $\iff$ $f'$ - спадає $\iff$ \\
1. $\forall x \in (a,b): \left[ \begin{gathered} f''(x) \leq 0 \\ f''(x) \geq 0 \end{gathered} \right.$\\
2. $\not\exists (\alpha, \beta) \subset (a,b): f''(x) = 0$ \qed
\bigline
\ex{5.13.4.} Функція $f(x) = x^2$ буде опуклою донизу, оскільки\\
$f''(x) = 2 > 0$
\bigline

\th{5.13.5.} Задана функція $f: (a,b) \to \mathbb{R}$, $f \in C([a,b])$ та диференційована на $(a,b)$\\
$f$ - опукла донизу на $(a,b) \iff \forall x_0 \in (a,b):$ дотична в т. $x_0$ лежить ничже графіка функції $f$\\
\proof
$\boxed{\Rightarrow}$ Дано: $f$ - опукла донизу на $(a,b)$\\
Зафіксуємо т. $x_0 \in (a,b)$, тоді маємо рівняння дотичної\\
$\tau(x) = f(x_0) + f'(x_0)(x-x_0) \Rightarrow f(x) - \tau(x) = f(x) - f(x_0) - f'(x_0)(x-x_0)$\\
За теоремою Лагранжа, отримаємо: $\exists z \in (x,x_0):$ \\
$f(x) - f(x_0) = f'(z)(x-x_0) \Rightarrow f(x) - \tau(x) = (f'(z) - f'(x_0))(x-x_0)$\\
За дано, маємо: $x < x_0 \Rightarrow z \leq x_0 \Rightarrow f'(z) \leq f'(x_0)$\\
Тоді маємо, що $f(x) - \tau(x) \geq 0 \Rightarrow \tau(x) \leq f(x)$\\
Тобто дійсно, дотична знаходиться нижче за графіка функції\\
Для $(x_0,x)$ ситуація є аналогічною
\bigline
$\boxed{\Leftarrow}$ Дано: $\forall x_0 \in (a,b)$, дотична в т. $x_0$ нижче $f$, тобто\\
$\forall x \in [a,b]: \tau(x) = f(x_0) + f'(x_0)(x-x_0) \leq f(x)$\\
$\Rightarrow f(x) - \tau(x) = f(x) - f(x_0) - f'(x_0)(x-x_0) \geq 0$\\
Тоді отримаємо:\\
$\begin{cases}
\dfrac{f(x)-f(x_0)}{x-x_0} \geq f'(x_0), x > x_0 \\
\dfrac{f(x)-f(x_0)}{x-x_0} \leq f'(x_0), x < x_0 \\
\end{cases}
$\\
Візьмемо точки $x_1 < x < x_2$, тоді матимемо таку нерівність\\
$\dfrac{f(x_1)-f(x)}{x_1-x} \leq \dfrac{f(x_2)-f(x)}{x_2-x}$\\
Що й свідчить про випуклість функції $f$ донизу \qed
\bigline

\defin{5.13.6.} Задана функція $f: A \to \mathbb{R}$ - диференційована в т. $x_0 \in A$\\
Точку $x_0$ називають \textbf{точкою перегину}, якщо в лівому та правому околі т. $x_0$ вони мають протилежні напрямки опуклості\\
Варто уточнити, що може існувати $f'(x_0) = \pm \infty$
\bigline

\ex{5.13.5.} Маємо $f(x) = \huge \frac{(x-1)^3}{4} + 2$\\
$f''(x) = \huge \frac{3}{2}(x-1) = 0$\\
Тут буде т. $x_0 = 1$ - точка перегину\\
Якщо $x > 1$, то $f''(x) > 0$. А якщо $x < 1$, то $f''(x) < 0$\\
Отже, на $(-\infty,1)$ - випукла догори, а на $(1,+\infty)$ - випукла донизу
\bigline
\ex{5.13.5.} Маємо $f(x) = \sqrt[3]{x}$\\
$f''(x) = \dfrac{1}{3} \left( -\dfrac{2}{3} \right) x^{-\frac{5}{3}}$\\
Тут буде т. $x_0 = 0$ - точка перегину\\
Водночас $\exists y'(0^+) = \huge \lim_{x \to 0^+} \dfrac{\sqrt[3]{x} - 0}{x - 0} = +\infty \hspace{0.3cm} \exists y'(0^-) = \huge \lim_{x \to 0^-} \dfrac{\sqrt[3]{x} - 0}{x - 0} = +\infty$
\begin{figure}[H]
\centering
\begin{tikzpicture}
\draw[thick, ->] (-2,0)--(2,0) node[anchor = north] {$x$};
\draw[thick, ->] (0,-2)--(0,2) node[anchor = east] {$y$};

\draw[thick, domain=0.001:1.8, variable=\x, samples = 1000] plot({\x}, {(\x)^(1/3)}) node at (2,1.5) {$f(x)$};
\draw[thick, domain=-1.8:-0.001, variable=\x, samples = 1000] plot({\x}, {-(-\x)^(1/3)});
\end{tikzpicture}
\end{figure}

\ex{5.13.5.} Маємо $f(x) = \sqrt{|x|}$\\
Тут т. $x_0 = 0$ не може бути точкою перегину, оскільки $\not \exists f'(0)$
\bigline
\th{5.13.6. Необхідна умова для перегину}\\
Задана функція $f: A \to \mathbb{R}$ та т. $x_0 \in A$ - точка перегину\\
Тоді $f''(x_0) = 0$\\
\textit{Тут все зрозуміло}
\bigline
\th{5.13.7. Достатня умова для перегину}\\
Задана функція $f: A \to \mathbb{R}$, $f \in C(A)$ та диференційована в околі т. $x_0$ та має другу похідну. Якщо по обидва боки від точки $x_0$ маємо протилежні знаки, то тоді $x_0$ - точка перегину\\
\textit{Тут теж все зрозуміло}
\bigline
\th{5.13.8. Достатня умова 2 для перегину}\\
Задана функція $f: A \to \mathbb{R}$. Відомо, що в околі т. $x_0$\\
$f''(x_0) = 0, f'''(x_0) = 0, \dots, f^{(n-1)}(x_0) = 0$, але водночас $f^{(n)}(x_0) \neq 0$. Тоді\\
- $x_0$ - точка перегину, якщо $n = 2k-1$\\
- $x_0$ - НЕ точка перегину, якщо $n = 2k$\\
\proof
\textit{Доведемо згодом} \qed
\bigline

\th{5.13.6. Нерівність Єнсена}\\
Задана функція $f:(a,b) \to \mathbb{R}$ - опукла $\underset{\textrm{догори}}{\textrm{донизу}}$. Тоді\\
$\forall \alpha_1, \dots, \alpha_n \in (0,1): \huge \alpha_1 + \dots + \alpha_n = 1:$\\
$\huge f(\alpha_1 x_1 + \dots + \alpha_n x_n) \underset{>}{<} \alpha_1 f(x_1) + \dots + \alpha_n f(x_n)$\\
\proofMI
$n = 2$. Тоді $\forall \alpha_1, \alpha_2: \alpha_1 + \alpha_2 = 1 \Rightarrow \alpha_2 = 1- \alpha_1:$\\
$f(\alpha_1 x_2 + \alpha_2 x_2) = f(\alpha_1 x_2 + (1-\alpha_1)x_2) < \alpha_1 f(x_1) + (1-\alpha_1)f(x_2)$, оскільки наша функція опукла донизу\\
Припустимо, що для $n-1$ нерівність виконана. Доведемо для $n$:\\
$\forall \alpha_1,\dots,\alpha_n \in (0,1): \forall x \in (a,b):$\\
$f(\alpha_1 x_1 + \dots + \alpha_n x_n) = \huge f\left(\alpha_n x_n + (1-\alpha_n)\left(\frac{\alpha_1}{1-\alpha_n}x_1 + \dots + \frac{\alpha_{n-1}}{1-\alpha_{n-1}}x_{n-1} \right)\right) <$\\
Зауважу, що $\huge \frac{\alpha_1}{1-\alpha_n} + \dots + \frac{\alpha_{n-1}}{1-\alpha_{n-1}} = 1$ та всі доданки $>0$\\
$< \huge \alpha_n f(x_n) + (1-\alpha_n)\left(\frac{\alpha_1}{1-\alpha_n}x_1 + \dots + \frac{\alpha_{n-1}}{1-\alpha_{n-1}}x_{n-1} \right) = \\ = \alpha_1 f(x_1) + \dots + \alpha_n f(x_n)$ \qed
\bigline
\ex{5.13.7.} Розглянемо функцію $f(x) = \ln x$\\
Вона є опуклою догори, тому що $f''(x) = -\dfrac{1}{x^2} < 0$\\
Тоді за нерівністю Єнсена, отримаємо\\
$\ln(\alpha_1 x_1 + \dots + \alpha_n x_n) > \alpha_1 \ln x_1 + \dots + \alpha_n \ln x_n$\\
де $\alpha_1 + \dots + \alpha_n = 1$\\
Можемо встановити $\alpha_1 = \dots = \alpha_n = \dfrac{1}{n}$, сума буде також рівна одинички. Прийдемо до такої нерівності\\
$\ln \dfrac{x_1+\dots+x_n}{n} \geq \dfrac{1}{n} \left( \ln x_1 + \dots + \ln x_n \right)$
\bigline

\newpage
\section{Інтегрування}
\subsection{Первісна, невизначений інтеграл}
\defin{6.1.1. Первісною для функції} $f(x)$ називають функцію $F(x)$, для якої
\begin{align*}
F'(x) = f(x)
\end{align*}
\prp{6.1.2.} Якщо $F(x), \Phi(x)$ - первісні для $f(x)$, то \\ $\Phi(x) = F(x) + C$\\
\textit{Випливає з наслідків теореми Лагранжа}
\bigline
\defin{6.1.3.} Множину всіх первісних для функції $f(x)$ називають \textbf{невизначеним інтегралом функції} $f(x)$\\
Позначення: $\huge \int f(x) \,dx = \{F(x): F'(x) = f(x)\}$
\bigline
\rm{6.1.3.} Але надалі можна вважати, що
$\huge \int f(x) \,dx = F(x) + C$
\bigline
\prp{6.1.4. Властивості}\\
1) $\huge \int \alpha f(x)\,dx = \alpha \int f(x)\,dx$\\
2) $\huge \int f(x) + g(x) \,dx = \int f(x)\,dx + \int g(x)\,dx$\\
3) $\huge \int f'(x)\,dx = f(x) + C$\\
4) $\huge \left(\int f(x)\,dx \right)' = f(x)$\\
\proof
Покладемо $\huge \int f(x)\,dx = F(x) + C_1 \hspace{0.5cm} \int g(x)\,dx = G(x) + C_2$. Тоді\\
1) $\huge \int \alpha f(x)\,dx = \alpha F(x) + C = \alpha (F(x) + C) = \alpha \int f(x)\,dx$
\bigline
2) $\huge \int f(x) + g(x) \,dx = F(x) + G(x) + C = F(x) + C_1 + G(x) + C_2 = \int f(x)\,dx + \int g(x)\,dx$
\bigline
3), 4) \textit{випливають з означення} \qed
\bigline
\textbf{Таблиця первісних}
\begin{center}
\begin{tabular}{ c|c } 
 $f(x)$ & $F(x)$ \\
 \hline 
 $1$ & $x$ \\ [2ex]
 \hline 
 $x^\alpha$ & $\dfrac{x^{\alpha+1}}{\alpha+1}, \alpha \neq -1$ \\ [2ex]
 \hline
 $\dfrac{1}{x}$ & $\ln |x|$ \\ [2ex]
 \hline
 $\sin x$ & $-\cos x$\\ [2ex]
 \hline 
 $\cos x$ & $\sin x$\\ [2ex]
 \hline
 $\dfrac{1}{\cos^2 x}$ & $\tg x$\\ [2ex]
 \hline 
 $\dfrac{1}{\sin^2 x}$ & $-\ctg x$\\ [2ex]
 \hline
 $\dfrac{1}{\sqrt{1-x^2}}$ & $\arcsin x$\\ [2ex]
 \hline
 $\dfrac{1}{1+x^2}$ & $\arctg x$\\ [2ex]
 \hline
 $\dfrac{1}{\sqrt{1+x^2}}$ & $\ln(x+\sqrt{x^2+1})$\\ [2ex]
 \hline
 $e^x$ & $e^x$ \\ [2ex]
 \hline 
 $a^x$ & $\dfrac{a^x}{\ln a}$ \\ [2ex]
 \hline
 $\sh x$ & $\ch x$ \\ [2ex]
 \hline
 $\ch x$ & $\sh x$ \\ [2ex]
 \hline
 $\dfrac{1}{\ch^2 x}$ & $\th x$\\ [2ex]
 \hline
 $\dfrac{1}{\sh^2 x}$ & $-\cth x$\\
\end{tabular}
\end{center}

\subsection{Заміна змінної}
\th{6.2.1.} $\huge \int f(g(x)) g'(x)\,dx = F(g(x)) + C$\\
\proof
$\huge \int f(g(x))g'(x)\,dx =$\\
Тут заміна: $g(x) = t$\\
Тоді $g'(x)\,dx = dt$\\
$= \huge \int f(t)\,dt = F(t) + C = F(g(x)) + C$ \qed
\bigline
\ex{6.2.1.} Обчислити $\huge \int \dfrac{1}{x \ln x} \,dx$\\
$\huge \int \dfrac{1}{x \ln x} \,dx = $\\
Проведемо заміну: $\ln x = t$\\
Тоді $\dfrac{1}{x}\,dx = dt$\\
$= \huge \int \dfrac{1}{t}\,dt = \ln |t| + C = \ln |\ln x| + C$
\bigline

\subsection{Інтегрування за частинами}
Все починається з правила диференціювання добутку функції:\\
$(u(x)v(x))'=u'(x)v(x) + v(x)v'(x)$\\
Тоді отримаємо:\\
$u(x)v'(x) = (u(x)v(x))'-u'(x)v(x)$\\
$\huge \int u(x)v'(x) \,dx = \int (u(x)v(x))'-u'(x)v(x) \,dx$\\
$\huge \int u(x)v'(x) \,dx = \int (u(x)v(x))'\,dx - \int u'(x)v(x)\,dx$\\
Зауважимо, що: $v'(x)\,dx = dv(x) \hspace{0.5cm} u'(x)\,dx = du(x)$\\
Отримаємо таку формулу:\\
\th{6.3.1.} $\huge \int u(x)\,dv(x) = u(x)v(x) - \int v(x)\,du(x)$\\
Більш зручно записати таку формулу:\\
$\huge \int u\,dv = uv - \int v\,du$
\bigline
\ex{6.3.1.} Обчислити $\huge \int x^2 e^x \,dx = $\\
$u = x^2 \Rightarrow du = 2x\,dx$\\
$e^x\,dx = dv \Rightarrow v = e^x$\\
$= x^2 e^x - \huge \int 2x e^x\,dx =$\\
$u = 2x \Rightarrow du = 2\,dx$\\
$e^x\,dx = dv \Rightarrow v = e^x$\\
$= x^2 e^x - (2xe^x - \huge \int 2e^x \,dx) = x^2 e^x - 2xe^x + 2e^x + C$
\bigline

\subsection{Інтегрування дробово-раціональних функцій}
Розглянемо $\huge \int \dfrac{P(x)}{Q(x)}\,dx$\\
де $P(x), Q(x)$ - многочлени. Є два випадки:\\
I. $pow(P(x)) \geq pow(Q(x))$\\
Тоді можемо поділити їх з остачею:\\
$P(x) = S(x)Q(x) + R(x)$
Тоді $\huge \int \dfrac{P(x)}{Q(x)}\,dx = \int S(x) + \dfrac{R(x)}{Q(x)}\,dx$\\
,але тут $pow(R(x)) < pow(Q(x))$, зараз буде пункт, як такий інтегрувати\\
\bigline
II. $pow(P(x)) < pow(Q(x))$\\
Розкладемо $Q(x) = (x-a_1)^{k_1} \dots (x-a_m)^{k_m} (x^2+p_1x+q_1)^{l_1} (x^2+p_sx+q_s)^{l_s}$\\
Причому дискримінант квадратних трьохчленів - від'ємний\\
Тоді за теоремою десь із курсу ліналу,\\
$\dfrac{P(x)}{Q(x)} = S(x) + \dfrac{A_{11}}{x-a_1} + \dots + \dfrac{A_{1k_1}}{(x-a_1)^{k_1}}+ \dots + \dfrac{A_{m1}}{x-a_m} + \dots + \dfrac{A_{mk_m}}{(x-a_m)^{k_m}} + \\
+ \dfrac{B_{11}x + C_{11}}{x^2+p_1x+q_1} + \dots + \dfrac{B_{1l_1}x + C_{1l_1}}{(x^2+p_1x+q_1)^{l_1}} + \dots + \dfrac{B_{s1}x + C_{s1}}{x^2+p_sx+q_s} + \dots + \dfrac{B_{sl_s}x + C_{sl_s}}{(x^2+p_sx+q_s)^{l_s}}$\\
Коротше, залишається розглянути 4 вигляди інтегралу:
\bigline
1) $\huge \int \dfrac{1}{x-a}\,dx = ln|x-a| + C$
\bigline
2) $\huge \int \dfrac{1}{(x-a)^k}\,dx = \int (x-a)^{-k}\,dx = \dfrac{(x-a)^{-k+1}}{-k+1} + C = \dfrac{1}{(1-k)(x-a)^{k-1}} + C$
\bigline
3) $\huge \int \dfrac{Bx+C}{x^2+px+q}\,dx \boxed{=}$\\
$x^2 + px + q = \left(x + \dfrac{p}{2} \right)^2 + \dfrac{4q-p^2}{4}$\\
Зробимо заміну: $x + \dfrac{p}{2} = t \Rightarrow dx = dt$\\
Також $Bx+C = Bt - B\dfrac{p}{2} + C$\\
Перепозначення: $\dfrac{4q-p^2}{4} = a^2 > 0 \hspace{0.5cm} C - B \dfrac{p}{2} = M$\\
$\boxed{=} \huge \int \dfrac{Bt + M}{t^2 + a^2}\,dt = B \int \dfrac{t}{t^2+a^2}\,dt + M \int \dfrac{1}{t^2+a^2}\,dt \boxed{=}$\\
$\huge \int \dfrac{t}{t^2+a^2}\,dt = \dfrac{dt^2}{2(t^2+a^2)} = \dfrac{1}{2} \ln|t^2+a^2|$\\
$\huge \int \dfrac{1}{t^2+a^2}\,dt = \dfrac{1}{a^2} \int \dfrac{1}{1 + \left(\frac{t}{a}\right)^2}\,dt = \dfrac{1}{a} \int \dfrac{d \frac{t}{a}}{1 + \left(\frac{t}{a}\right)^2} = \dfrac{1}{a} \arctg \dfrac{t}{a}$\\
$\boxed{=} \huge \frac{B}{2} \ln|t^2+a^2| + \frac{M}{a} \arctg \frac{t}{a} + C$\\
Ну а далі робимо зворотню заміну - інтеграл розв'язан
\bigline
4) $\huge \int \dfrac{Bx+C}{(x^2+px+q)^l}\,dx =$\\
Тут робимо ті самі заміни, що в 3)\\
$= \huge \int \dfrac{Bt+M}{(t^2+a^2)^l}\,dt = B \int \dfrac{t}{(t^2+a^2)^l} \,dt + M \int \dfrac{1}{(t^2+a^2)^l}\,dt$\\
Ну і тут я ланцюг рівностей зупиню, якщо перший інтеграл - ще ок, то другий - це дупа\\
$\huge \int \dfrac{t}{(t^2+a^2)^l}\,dt = \int \dfrac{dt^2}{2(t^2+a^2)^l}\,dt = \dfrac{1}{2} \dfrac{1}{(1-l)s^{l-1}}$\bigline
$\huge \int \dfrac{1}{(t^2+a^2)^l}\,dt =$\\
$u = \dfrac{1}{(t^2+a^2)^l} \hspace{1cm} dv = dt$\\
$= \huge \dfrac{t}{(t^2+a^2)^l} + 2l \int \dfrac{t^2}{(t^2+a^2)^{l+1}}\,dt + \dfrac{t}{(t^2+a^2)^l} + 2l \left(\int \dfrac{dt}{(t^2+a^2)^l} - a^2 \dfrac{dt}{(t^2+a^2)^{l+1}} \right)$\\
Позначимо за $I_l = \huge \int \dfrac{t}{(t^2+a^2)^l}\,dt$\\
Тоді маємо таке рівняння:\\
$I_l = \dfrac{t}{(t^2+a^2)^l} + 2l \cdot I_l - 2la^2 \cdot I_{l+1}$\\
Залишилось виразити $I_{l+1}$ та розв'язати рівняння рекурсивно, причому $I_1$ ми вже рахували
\bigline
\ex{6.4.1.} Обчислити $\huge \int \dfrac{x^4}{1+x^3}\,dx$\\
Оскільки $pow(x^4) > pow(1+x^3)$, то ми поділимо многочлени. Отримаємо:\\
$\huge \int \dfrac{x^4}{1+x^3}\,dx = \int x - \dfrac{x}{x^3+1}\,dx = x^2 - \int \dfrac{x}{x^3+1}\,dx$\\
Обчислимо другий інтеграл:\\
$\dfrac{x}{x^3+1} = \dfrac{x}{(x+1)(x^2-x+1)} = \dfrac{A}{x+1} + \dfrac{Bx+C}{x^2-x+1} \boxed{=}$\\
$A(x^2-x+1) + (Bx+C)(x+1) = x$\\
$\Rightarrow \begin{cases}
A + B = 0 \\
-A + B + C = 1\\
A + C = 0
\end{cases} \Rightarrow A = -\dfrac{1}{3}, B = \dfrac{1}{3}, C = \dfrac{1}{3}$\\
$\boxed{=} -\dfrac{1}{3(x+1)} + \dfrac{1}{3} \dfrac{x+1}{x^2-x+1}$\\
$\Rightarrow \huge \int \dfrac{x}{x^3+1}\,dx = -\frac{1}{3} \int \frac{1}{x+1}\,dx + \frac{1}{3} \int \frac{x+1}{x^2-x+1}\,dx \boxed{=}$\\
І розглянемо другий інтеграл:\\
$\huge \int \frac{x+1}{x^2-x+1}\,dx = \int \frac{4x+4}{(2x-1)^2 + 3}\,dx = \int \frac{4x-2}{(2x-1)^2 +3}\,dx + \int \frac{6}{(2x-1)^2 +3}\,dx = \ln((2x-1)^2+3) + 6 \dfrac{1}{2\sqrt{3}} \arctg \dfrac{2x-1}{\sqrt{3}} = \ln(4x^2-4x+4) + \sqrt{3} \arctg \dfrac{2x-1}{\sqrt{3}}$\\
$\boxed{=} \huge -\dfrac{1}{3} \ln|x+1| + \dfrac{1}{3} \ln(4x^2-4x+4) + \dfrac{1}{\sqrt{3}} \arctg \dfrac{2x-1}{\sqrt{3}}$\\
Остаточно отримаємо:\\
$\huge \int \dfrac{x^4}{1+x^3}\,dx = x^2 + \dfrac{1}{3} \ln|x+1| - \dfrac{1}{3} \ln(4x^2-4x+4) - \dfrac{1}{\sqrt{3}} \arctg \dfrac{2x-1}{\sqrt{3}} + C$
\bigline
\subsection{Інтегрування тригонометричних функцій}
I. $\huge \int \sin^k x \cos^m x \,dx = \hspace{3cm} k,m \in \mathbb{Z}$\\
1) $k$ - непарне, тобто $k = 2l+1$ 
\\Тоді заміна: $\cos x = t$. Тоді\\
$-\sin x \,dx = dt$ і $\sin^2 x = 1 - \cos ^2x = 1 - t^2$\\
$= \huge \int \sin^{2l+1} x t^m \dfrac{dt}{-\sin x} = -\int t^m (1-t^2)^l\,dt$
\bigline
2) $m$ - непарне, тобто $m = 2l+1$
\\ Тоді заміна: $\sin x = t$. Тоді\\
$\cos x \,dx = dt$ і $\cos^2 x = 1 - \sin^2 x = 1 - t^2$\\
$= \huge \int t^k \cos^{2l+1}x \dfrac{dt}{\cos x} = \int t^k(1-t^2)^l \,dt$
\bigline
3) $k,m$ - парні, тобто $k=2l, m =2n$\\
Тоді $\sin^2 x = \dfrac{1-\cos 2x}{2} \hspace{0.5cm} \cos^2 x = \dfrac{1+\cos 2x}{2}$\\
$= \huge \int \left( \dfrac{1-\cos 2x}{2} \right)^l \left( \dfrac{1+\cos 2x}{2} \right)^n \,dx$
\bigline
Всі отримані інтеграли є випадком інтегрування дробово-раціональних виразів
\bigline
II. $\huge \int R(\sin x, \cos x)\,dx =$\\
де $R$ - дробово-раціональний вираз від $\sin x, \cos x$\\
Заміна: $t = \tg \dfrac{x}{2} \Rightarrow x = 2 \arctg t \Rightarrow dx = \dfrac{2}{1+t^2}\,dt$\\
$\sin x = \dfrac{2 \tg \frac{x}{2}}{1 + \tg^2 \frac{x}{2}} = \dfrac{2t}{1+t^2}$\\
$\cos x = \dfrac{1 - \tg^2 \frac{x}{2}}{1 + \tg^2 \frac{x}{2}} = \dfrac{1-t^2}{1+t^2}$
$= \huge \int R\left(\dfrac{2t}{1+t^2}, \dfrac{1-t^2}{1+t^2} \right) \cdot \dfrac{2}{1+t^2}\,dt$\\
Отримуємо випадок інтегрування дробово-раціональних виразів
\bigline
\ex{6.5.1.} Обчислити $\huge \int \cos^3 x \,dx$\\
Заміна: $t = \sin x$, випадок I.2)\\
Тоді: $dt = \cos x \,dx$\\
$\Rightarrow \huge \int \cos^3 x \,dx = \int (1-t^2) \,dt = t - \dfrac{t^3}{3} + C = \sin x -\dfrac{\sin^3 x}{3} + C$
\bigline
\ex{6.5.2.} Обчислити $\huge \int \dfrac{dx}{5-3\cos x}$\\
Заміна: $t = \tg \dfrac{x}{2}$, випадок II.
Тоді беремо решта замін звідси, з нашого пункту\\
$\Rightarrow \huge \int \dfrac{dx}{5-3\cos x} = \int \dfrac{1}{5-3 \frac{1-t^2}{1+t^2}} \dfrac{2}{1+t^2}\,dt = \int \dfrac{2\,dt}{5+5t^2-3+3t^2} = \int \dfrac{dt}{4t^2+1} = \dfrac{1}{2} \arctg 2t + C = \dfrac{1}{2} \arctg \left(2 \tg \dfrac{x}{2} \right) + C$
\bigline
\subsection{Інтегрування ірраціональних виразів}
I. $\huge \int R\left( \sqrt[k_1]{\dfrac{ax+b}{cx+d}}, \dots, \sqrt[k_n]{\dfrac{ax+b}{cx+d}} \right)\,dx \boxed{=}$\\
Нехай $m = \textrm{LCM} (k_1,\dots,k_n)$\\
Заміна: $\dfrac{ax+b}{cx+d} = t^m$\\
Виразимо $x$ з цього рівняння:\\
$ax+b =t^m cx + t^m d \Rightarrow x = \dfrac{t^md-b}{a-ct^m}$\\
Тоді $dx = \dfrac{dmt^{m-1}(a-ct^m) + (t^md-b)cmt^{m-1}}{(a-ct^m)^2}\,dt = \dfrac{mt^{m-1}(ad-bc)}{(a-ct^m)^2}\,dt$\\
$\huge \boxed{=} \int R(t^{m_1},\dots,t^{m_n}) \dfrac{mt^{m-1}(ad-bc)}{(a-ct^m)^2}\,dt$\\
де $m_1 = \dfrac{m}{k_1},\dots,m_n = \dfrac{m}{k_n} \in \mathbb{Z}$\\
Отримаємо інтеграл дробово-раціонального виразу
\bigline
II.1. $\huge \int R(x,\sqrt{a^2-x^2})\,dx \boxed{=}$\\
Заміна: $x = a\sin t \Rightarrow dx = a\cos t \,dt$\\
$\boxed{=} \huge \int R(a\sin t, a\cos t) \cdot a\cos t \,dt$
\bigline
II.2. $\huge \int R(x,\sqrt{a^2+x^2})\,dx \boxed{=}$\\
Заміна: $x = a\tg t \Rightarrow dx = \dfrac{a}{\cos^2 t} dt$\\
$\boxed{=} \huge \int R\left(a\tg t, \dfrac{a}{\cos t}\right) \cdot \dfrac{a}{cos^2 t} \,dt$
\bigline
Або інша заміна: $x = a \sh t \Rightarrow dx = a \ch t \,dt$\\
$\boxed{=} \huge \int R(a\sh t, a \ch t) \cdot a \ch t\,dt$
\bigline
II.3. $\huge \int R(x, \sqrt{x^2-a^2})\,dx \boxed{=}$\\
Заміна: $x = \dfrac{a}{\cos t} \Rightarrow dx = \dfrac{a}{\cos^2 t} \sin t\,dt$\\
$\boxed{=} \huge \int R \left(\dfrac{a}{\cos t}, a \tg t \right) \cdot \dfrac{a \sin t}{\cos ^2 t}\,dt$
\bigline
Або інша заміна: $x = a \ch t \Rightarrow dx = a \sh t \,dt$\\
$\boxed{=} \huge \int R(a \ch t, a \sh t) \cdot a \sh t \,dt$
\bigline
Усі отримані інтеграли II є інтегралами тригонометричних/гіперболічних функцій
\bigline
\ex{6.6.1.} Обчислити $\huge \int \dfrac{\sqrt{x+1}+2}{(x+1)^2 - \sqrt{x+1}}\,dx$\\
Заміна: $t^2 = x+1$, випадок I.\\
Тоді $x = t^2 -1 \Rightarrow dx = 2t \,dt$\\
$\Rightarrow \huge \int \dfrac{\sqrt{x+1}+2}{(x+1)^2 - \sqrt{x+1}}\,dx = \int \dfrac{t+2}{t^4-t} \cdot 2t\,dt = 2 \int \dfrac{t+2}{t^3-1}\,dt =$\\
обчислення цього інтегралу проводиться як в п. 4, тому я пропускаю цей момент\\
$= -\ln(t^2+t+1) - \dfrac{2}{\sqrt{3}} \arctg \dfrac{2t+1}{\sqrt{3}} + 2 \ln|t-1|+ C = \\
= -\ln(x+2+\sqrt{x+1}) - \dfrac{2}{\sqrt{3}} \arctg \dfrac{2\sqrt{x+1}+1}{\sqrt{3}} + 2 \ln|\sqrt{x+1}-1| + C$
\bigline
\ex{6.6.2.} Обчислити $\huge \int \sqrt{4-x^2}\,dx$\\
Заміна: $x = 2\sin t$, випадок II.1.\\
Тоді $dx = 2 \cos t \,dt$\\
$\Rightarrow \huge \int \sqrt{4-x^2}\,dx = \int 2 \cos t \cdot 2 \cos t \,dt = \int 2(1+\cos 2t)\,dt = 2t + \sin 2t + C
\\ = 2t + 2 \sin t \cos t + C = 2 \arcsin \frac{x}{2} + 2 \frac{x}{2} \sqrt{1-\frac{x^2}{4}}+C = \\ = 2 \arcsin \frac{x}{2} + \dfrac{x \sqrt{4-x^2}}{2} + C$
\bigline
\subsection{Диференціальний біном}
$\huge \int x^m (ax^n + b)^p\,dx = \hspace{3cm} m,n,p \in \mathbb{Q}$\\
1) $p \in \mathbb{Z}$, тоді маємо:\\
$m = \dfrac{p_1}{q_1}; n = \dfrac{p_2}{q_2}$\\
Нехай $q = \textrm{LCM}(q_1,q_2)$\\
Заміна: $x = t^q$
\bigline
2) $p \not \in \mathbb{Z}$, але $\dfrac{m+1}{n} \in \mathbb{Z}$, тоді маємо:\\
$p = \dfrac{j}{l}$\\
Заміна: $ax^n+b = t^l$
\bigline
3) $p \not \in \mathbb{Z}$, $\dfrac{m+1}{n} \not \in \mathbb{Z}$, але $p+ \dfrac{m+1}{n} \in \mathbb{Z}$, тоді маємо: \\ $p = \dfrac{j}{l}$\\
Заміна: $a+bx^{-n} = t^l$
\bigline
Заміни в 1), 2), 3) називають підстановками Чебишова, що призводять до інтегралу дробово-раціональних виразів\\
Якщо жодна з пунктів не спрацьовує, то інтеграл не може бути обчисленим через елементарні функції
\bigline
\ex{6.7.1.} Обчислити $\huge \int \sqrt[3]{x-x^3}\,dx = \int x^{\frac{1}{3}} (1-x^2)^\frac{1}{3}\,dx \boxed{=}$\\
Тут у нас $m = \dfrac{1}{3}$, $n = 2$, $p = \dfrac{1}{3}$\\
Спрацьовує п. 3, тому що $p + \dfrac{m+1}{n} = \dfrac{1}{3} + \dfrac{1+\frac{1}{3}}{2} = 1 \in \mathbb{Z}$\\
Заміна: $-1+x^{-2}=t^3$\\
$-2x^{-3}\,dx = 3t^2\,dt$\\
$\boxed{=} \huge \int (x^{-2}-1)^{\frac{1}{3}} x^{\frac{2}{3}} x^{\frac{1}{3}}\,dx = \int t \cdot x \cdot \frac{3t^2 x^3 \,dt}{-2} = \int \frac{3t^3\,dt}{-2(t^3+1)^2} = \\ = \frac{3}{-2} \left(\int \frac{dt}{t^3+1} - \int \frac{dt}{(t^3+1)^2} \right) =
$\\
обчислення цього інтегралу проводиться як в п. 4, тому я пропускаю цей момент\\
$= \huge -\frac{\ln|t+1|}{2} + \frac{\ln(t^2-t+1)}{4} - \frac{\sqrt{3}}{2} \arctg \frac{2x-1}{\sqrt{3}} + \frac{\ln |t+1|}{3} - \frac{\ln(t^2-t+1)}{6} + \frac{\sqrt{3}}{3} \arctg \frac{2x-1}{\sqrt{3}} + \frac{t}{2t^3+2} + C =\\
= - \frac{1}{6} \ln|t+1| + \frac{1}{12} \ln(t^2-t+1) - \frac{\sqrt{3}}{6} \arctg \frac{2x-1}{\sqrt{3}} + \frac{t}{2t^3+2} + C$\\
І підставляємо $t = \sqrt[3]{x^{-2}+1}$\\
\newpage

\section{Визначений інтеграл}
\subsection{Підхід Рімана}
\defin{7.1.1. Розбиттям} множини $[a,b]$ називають множину точок $\tau = \{x_0,x_1,\dots,x_{n-1},x_n\}$, для яких
\begin{align*}
a = x_0 < x_1 < \dots < x_{n-1} < x_{n} = b
\end{align*}
\defin{7.1.2.} Позначимо за $\Delta x_1 = x_1 - x_0, \dots, \Delta x_n = x_{n} - x_{n-1}$. Тоді числом
\begin{align*}
|\tau| = \max\{\Delta x_1,\dots, \Delta x_n\}
\end{align*}
називають \textbf{діаметром} розбиття $\tau$
\bigline
\defin{7.1.3.} Задані розбиття $\tau, \tau'$ відрізка $[a,b]$. Якщо $\tau \subset \tau'$, то $\tau'$ називають \textbf{підрозбиттям} розбиття $\tau$
\bigline
\prp{7.1.4.} Задано $\tau'$ - підрозбиття для $\tau$. Тоді $|\tau'| \leq |\tau|$\\
\proof
Дійсно, із розбиття ми можемо отримати підрозбиття шляхом додавання точок. Тоді деякі інтервали будуть ділитись на підінтервали через додавання точки. Відповідно діаметр зменшується  \qed
\bigline
\defin{7.1.5.} Задано $\tau = \{x_0,x_1,\dots,x_n\}$ - розбиття відрізка $[a,b]$\\
Елементи множини $\xi^{\tau} = \{\xi_1, \dots, \xi_n \}$ називають \textbf{відміченими точками}\\
Тут $\xi_1 \in [x_0,x_1), \xi_2 \in [x_1,x_2), \dots, \xi_n \in [x_{n-1}, x_n]$
\bigline
\defin{7.1.6.} Задана функція $f: [a,b] \to \mathbb{R}$, розбиття $\tau = \{x_0,x_1,\dots,x_n\}$ та відмічені точки $\xi^{\tau} = \{\xi_1, \dots, \xi_n \}$\\
\textbf{Інтегральною сумою Рімана} функції $f$ для нашого розбиття $\tau$ та відмічених точокк називають число
\begin{align*}
S_{\tau, \xi^{\tau}}(f) = \sum_{k=1}^n f(\xi_k) \Delta x_k
\end{align*}

\begin{tikzpicture}
\draw[thick] (1,-1pt)--(1,1pt) node[anchor = north] {$x_0$};
\draw[thick] (2,-1pt)--(2,1pt) node[anchor = north] {$x_1$};
\draw[thick] (4,-1pt)--(4,1pt) node[anchor = north east] {$x_{n-1}$};
\draw[thick] (5,-1pt)--(5,1pt) node[anchor = north] {$x_n$};


\draw[fill = black!30] (1,0) rectangle (2, {exp((1.5-2)/3)});
\draw[fill = black!30] (4,0) rectangle (5, {exp((4.5-2)/3)});

\draw[thick, dashed] (1.5,{exp((1.5-2)/3)})--(1.5,0) node[anchor = north, scale = 0.8] {$\xi_1$};
\draw[thick, dashed] (4.5,{exp((4.5-2)/3)})--(4.5,0) node[anchor = north, scale = 0.8] {$\xi_n$};

\draw[thick, ->] (-0.5,0)--(5.5,0) node[anchor = north] {$x$};
\draw[thick, ->] (0,-0.5)--(0,3) node[anchor = east] {$y$};

\draw[thick, domain=1:5, variable=\x, samples = 1000] plot({\x}, {exp((\x-2)/3)});
%node[anchor = south east, scale = 0.8] {$f(x) = \dfrac{\sin x}{x}$};
%\node[white] at (0,1) [circle,fill,inner sep=1.5pt, draw = black]{};
%\node[black] at (0,0) [circle,fill,inner sep=1.5pt, draw = black]{};
\end{tikzpicture}
\bigline
\defin{7.1.7.} Задана функція $f: [a,b] \to \mathbb{R}$\\
Функція $f$ називається \textbf{інтегрованою за Ріманом} на $[a,b]$, якщо існує таке число $I$, для якого виконана умова:
\begin{align*}
\forall \varepsilon > 0: \exists \tau_{\varepsilon}: \forall \tau \supset \tau_{\varepsilon}: \forall \xi^{\tau}: \abs{S_{\tau,\xi^{\tau}}(f) -I}<\varepsilon
\end{align*}
Число $I$ називають \textbf{інтегралом Рімана}\
\begin{align*}
I = \int_a^b f(x)\,dx
\end{align*}
Множина інтегрованих функцій за Ріманом: $R([a,b])$
\bigline
\rm{7.1.7.} Зауважимо, що:
\begin{align*}
\int_a^a f(x)\,dx = 0 \\
\int_b^a f(x)\,dx = -\int_a^b f(x)\,dx
\end{align*}
\ex{7.1.8.} Доведемо, що функція $f(x) = 1 \in R([a,b])$, а також:\\
$\huge \int_a^b 1\,dx = b-a$\\
Для початку зафіксуємо розбиття $\tau = \{x_0,x_1,\dots,x_n\}$ та відмітимо точки $\xi^\tau = \{\xi_1,\dots,\xi_n\}$. Це аби знайти інтегральну суму:\\
$S_{\tau, \xi^\tau} (f) = \huge \sum_{k=1}^n f(\xi_k) \Delta_k$\\
$\Rightarrow S_{\tau, \xi^\tau} (1) = \huge \sum_{k=1}^n \Delta_k = x_1 - x_0 + x_2 - x_1 + \dots + x_n - x_{n-1} = x_n - x_0 = b - a$\\
І ця інтегральна сума має це значенням при довільному розбитті\\
Якщо встановити $I = b -a$, то тоді:\\
$\forall \varepsilon > 0: \exists \tau_{\varepsilon}: \forall \tau \supset \tau_{\varepsilon}: |S_{\tau, \xi^{\tau}}(f) - I| = |b-a - (b-a)| = 0 < \varepsilon$\\
Отже, $f(x) = 1 \in R([a,b])$, а інтеграл:\\
$\huge \int_a^b 1\,dx = b-a$
\bigline
\ex{7.1.9.} Доведемо, що функція $f(x) = 1_{x^*} \in R([a,b])$, причому $x^* \in [a,b]$\\
Також покажемо, що $\huge \int_a^b 1_{x^*} (x)\,dx = 0$\\
Для початку зафіксуємо розбиття $\tau = \{x_0,x_1,\dots,x_n\}$ та відмітимо точки $\xi^\tau = \{\xi_1,\dots,\xi_n\}$. Знаходимо інтегральну суму:\\
Якщо виявиться, що $x^* \not \in \xi^\tau$, то $\forall k = 1,\dots, n: 1_{x^*}(\xi_k) = 0$. Отже, $S = 0$\\
А якщо $x^* \in \xi^\tau$, то існує єдина точка $\xi_m \in \tau$, що $x^* = \xi_m$. Тоді $\forall k \neq m: 1_{x^*}(\xi_k) = 0$, а тоді\\
$S = \Delta_m = x_m - x_{m-1}$\\
Для другого випадку якщо покласти $I = 0$, то:\\
$\forall \varepsilon > 0: \exists \tau_{\varepsilon}: |\tau_\varepsilon| < \varepsilon: \forall \tau \supset \tau_\varepsilon: |S_{\tau,\xi^\tau}(f) - I| = |x_m - x_{m-1}| < \varepsilon$
\bigline
\ex{7.1.10.} Доведемо, що функція $1_{\mathbb{Q}} \not \in R([a,b])$\\
Знайдемо інтегральні суми:\\
Якщо взяти всі точки $\xi_k \in \Delta_k \cap \mathbb{Q}$, то $\forall k = 1,\dots,n: 1_{\mathbb{Q}}(\xi_k) = 1$\\
$\Rightarrow S_{\tau, \xi^\tau}(1_{\mathbb{Q}}) = (x_1-x_0)+\dots+(x_n-x_{n-1}) = b-a$\\
Проте коли всі точки $\xi_k \in \Delta_k \setminus \mathbb{Q}$, то $\forall k = 1,\dots,n: 1_{\mathbb{Q}}(\xi_k) = 0$\\
$\Rightarrow S_{\tau, \xi^\tau}(1_{\mathbb{Q}}) = 0$\\
Отримані значення не залежать від розбиття, але залежить від відмічених точок. Тобто однозначно задати $I$ ми не можемо. Отже, $1_{\mathbb{Q}} \not \in R([a,b])$
\bigline
\subsection{Існування інтеграла}
\th{7.2.1.} Задана функція $f \in R([a,b])$. \\ Тоді значення інтеграла Рімана єдине\\
\proof
Припустимо, що $I_1,I_2$ - значення інтегралу Рімана для функції $f$\\
Тоді за означенням:\\
$\exists \tau_{\varepsilon 1}: \forall \tau \supset \tau_{\varepsilon 1}: \abs{S_{\tau, \xi} - I_1} < \dfrac{\varepsilon}{2}$\\
\vspace{0.2cm}\\
$\exists \tau_{\varepsilon 2}: \forall \tau \supset \tau_{\varepsilon 2}: \abs{S_{\tau, \xi} - I_2} < \dfrac{\varepsilon}{2}$\\
Зафіксуємо $\tau^* = \tau_{\varepsilon 1} \cup \tau_{\varepsilon 2}$. Причому $\tau^*$ - підрозбиття одночасно для $\tau_{\varepsilon 1}$ та $\tau_{\varepsilon 2}$\\
Тоді $\forall \tau \supset \tau^*:$\\
$|I_1 - I_2| = |I_1 - S_{\tau,\xi} + S_{\tau,\xi} - I_2| \leq |I_1 - S_{\tau,\xi}| + |S_{\tau,\xi} - I_2| < \varepsilon$\\
З цього випливає негайно, що $I_1 = I_2$. Суперечність! \qed
\bigline
\th{7.2.2.} Задана функція $f \in R([a,b])$. Тоді вона є обмеженою\\
\proof
$f \in R([a,b])$, тоді означення спрацьовує. Нехай $\varepsilon = 1$\\
Тоді $\exists \tau = \{x_0,x_1,\dots,x_n\}: |S_{\tau, \xi^{\tau}}(f) - I| < 1$\\
!Припустимо, що $f$ не є обмеженою. Тоді функція $f$ не буде обмеженою принаймні в одному інтервалі $\Delta_{k_0}$\\
Нехай вона не є обмеженою зверху. Тобто $\huge \exists \{\xi^n_{k_0} \in \Delta_{k_0}, n \geq 1\}: \\ \lim_{n \to \infty} \xi_{k_0}^n = \xi_{k_0} \Rightarrow \lim_{n \to \infty} f(\xi_{k_0}^n) = +\infty$\\
Розглянемо множину відмічених точок $\xi_n^{\tau} = \{\xi_1,\dots, \xi_{k_0}, \dots, \xi_n\}$\\
$S_n = S_{\tau, \xi_j^{\tau}}(f) = \huge \sum_{k=1}^n f(\xi_k) \Delta x_k$\\
Тоді маємо, що:\\
$\huge \lim_{n \to \infty} S_n = f(\xi_1)\Delta x_1 + \dots + \lim_{n \to \infty} f(\xi_{k_0}^n) \Delta x_{k_0} + \dots + f(\xi_n) \Delta x_n = +\infty$\\
Суперечність! Оскільки $|S_{\tau, \xi_n^{\tau}}(f) - I| < 1$\\
Аналогічно все для обмеженості знизу. Отже, $f$ - обмежена \qed (TODO)
\bigline
\defin{7.2.3.} Задана функція $f: [a,b] \to \mathbb{R}$ - обмежена. Визначмо такі значення для розбиття $\tau = \{x_0,x_1,\dots,x_n\}$:
\begin{align*}
m_k = \inf_{x \in \Delta_k} f(x) \hspace{2cm} M_k = \sup_{x \in \Delta_k} f(x) \hspace{0.5cm} k=1,\dots,n
\end{align*}
Визначимо такі функції $\underline{f}_{\tau}, \overline{f}_{\tau}: [a,b] \to \mathbb{R}:$
\begin{align*}
\underline{f}_{\tau} = \sum_{k=1}^n m_k 1_{\Delta _k} \hspace{2cm} \overline{f}_{\tau} = \sum_{k=1}^n M_k 1_{\Delta _k}
\end{align*}
\begin{tikzpicture}
\draw[thick] (1,-1pt)--(1,1pt) node[anchor = north] {$x_0$};
\draw[thick] (2,-1pt)--(2,1pt) node[anchor = north] {$x_1$};
\draw[thick] (5,-1pt)--(5,1pt) node[anchor = north] {$x_n$};

\foreach \i in {2,3,4,5} {
\draw[thick, dashed] (\i,{exp((\i-2)/3)})--(\i,0);
}

\draw[thick, ->] (-0.5,0)--(5.5,0) node[anchor = north] {$x$};
\draw[thick, ->] (0,-0.5)--(0,3) node[anchor = east] {$y$};

\draw[thick, domain=1:5, variable=\x, samples = 1000] plot({\x}, {exp((\x-2)/3)});

\draw[thick, red] (1,{exp((2-2)/3)})--(2,{exp((2-2)/3)}) node[anchor = south east] {$\overline{f}_{\tau}$};
\draw[thick, red] (2,{exp((3-2)/3)})--(3,{exp((3-2)/3)});
\draw[thick, red] (3,{exp((4-2)/3)})--(4,{exp((4-2)/3)});
\draw[thick, red] (4,{exp((5-2)/3)})--(5,{exp((5-2)/3)});
\end{tikzpicture}
\qquad
\begin{tikzpicture}
\draw[thick] (1,-1pt)--(1,1pt) node[anchor = north] {$x_0$};
\draw[thick] (2,-1pt)--(2,1pt) node[anchor = north] {$x_1$};
\draw[thick] (5,-1pt)--(5,1pt) node[anchor = north] {$x_n$};

\foreach \i in {2,3,4,5} {
\draw[thick, dashed] (\i,{exp((\i-1-2)/3)})--(\i,0);
}

\draw[thick, ->] (-0.5,0)--(5.5,0) node[anchor = north] {$x$};
\draw[thick, ->] (0,-0.5)--(0,3) node[anchor = east] {$y$};

\draw[thick, domain=1:5, variable=\x, samples = 1000] plot({\x}, {exp((\x-2)/3)});

\draw[thick, blue] (1,{exp((1-2)/3)})--(2,{exp((1-2)/3)}) node[anchor = south east] {$\underline{f}_{\tau}$};
\draw[thick, blue] (2,{exp((2-2)/3)})--(3,{exp((2-2)/3)});
\draw[thick, blue] (3,{exp((3-2)/3)})--(4,{exp((3-2)/3)});
\draw[thick, blue] (4,{exp((4-2)/3)})--(5,{exp((4-2)/3)});
\end{tikzpicture}
\bigline
\th{7.2.4.(1).} Функції $\underline{f}_{\tau}$, $\overline{f}_{\tau} \in R([a,b])$, а їхні інтеграли дорівнюють:\\
$\huge \int_a^b \underline{f}_{\tau}(x)\,dx = \sum_{k=1}^n m_k(x_k-x_{k-1})$ - нижня сума Дарбу\\
$\huge \int_a^b \underline{f}_{\tau}(x)\,dx = \sum_{k=1}^n M_k(x_k-x_{k-1})$ - верхня сума Дарбу\\
\proof
Фіксуємо $\varepsilon > 0$. І нехай $\tau_{\varepsilon} = \tau = \{x_0,x_1,\dots,x_n\}$\\
Тоді $\forall \tau' \supset \tau_\varepsilon$, де $\\ \tau' = \{x_0 = x_{0,0}, x_{0,1}, \dots, x_{0,s_0} = \\ = x_1 = x_{1,0}, x_{1,1}, \dots\, x_{1,s_1} = x_2 = x_{2,0}, \dots, \\ x_{n-1} = x_{n-1,0}, x_{n-1,1}, \dots, x_{n-1, s_{n-1}} = x_n\}$\\
$\forall \xi^{\tau'}: \underline{f}(\xi_{k,l}) = m_{k+1}, k=0,\dots,n-1$\\
Отримаємо:\\
$S_{\tau', \xi^{\tau'}}(f) = \huge \sum_{k=0}^{n-1} \sum_{l=1}^{s_k} f_{\tau}(\xi_{k,l})(x_{k,l}-x_{k,l-1}) = \sum_{k=0}^{n-1} \sum_{l=1}^{s_k} m_k(x_{k,l}-x_{k,l-1}) = \\ = \sum_{k=0}^{n-1} m_k \sum_{l=1}^{s_k} (x_{k,l}-x_{k,l-1}) = \sum_{k=0}^{n-1} m_k(x_{k+1}-x_k) = \sum_{k=1}^n m_k(x_k-x_{k-1})$\\
$\Rightarrow |S_{\tau', \xi'}(f) - I| = 0 < \varepsilon$\\
Для $\overline{f}_{\tau}$ аналогічно \qed
\bigline
\th{7.2.4.(2).} Задано $\tau' \supset \tau$ - підрозбиття розбиття $\tau$. Тоді $\forall x \in [a,b]:$\\
$\underline{f}_{\tau}(x) \leq \underline{f}_{\tau'}(x) \hspace{4cm} \overline{f}_{\tau}(x) \geq \overline{f}_{\tau'}(x)$\\
Більш того\\
$\huge \int_a^b \underline{f}_{\tau}(x) \leq \int_a^b \underline{f}_{\tau'}(x) \hspace{2cm} \int_a^b \overline{f}_{\tau}(x) \geq \int_a^b \overline{f}_{\tau'}(x)$\\
\proof
Зафіксуємо $x \in [a,b]$. Розглянемо підрозбиття $\tau'$ як в попередній теоремі, тобто $x \in \Delta_{k,l}$ для деяких $k,l$\\
Через те, що $\tau' \supset \tau$, то маємо: $\Delta_{k,l} \subset \Delta_{k+1}$\\
$\Rightarrow m_{k,l} = \huge \inf_{t \in \Delta_{k,l}} f(t) \geq \inf_{t \in \Delta_{k+1}} f(t) = m_k$\\
$\Rightarrow \underline{f}_{\tau}(x) \leq \underline{f}_{\tau'}(x)$
\\
Покажемо тепер нерівність з інтегралами\\
$\huge \int_a^b \underline{f}_{\tau}(x)\,dx = \sum_{k=1}^m m_k(x_k-x_{k-1}) = \sum_{k=1}^n \sum_{l=1}^{s_k} \boxed{m_k} (x_{k-1,l}-x_{k-1,l-1}) \leq \\ \leq \sum_{k=1}^n \sum_{l=1}^{s_k} \boxed{m_{k,l}} (x_{k-1,l}-x_{k-1,l-1}) = \int_a^b \underline{f}_{\tau'}(x)\,dx$\\
Для $\overline{f}_{\tau}$ аналогічно \qed
\begin{figure}[H]
\centering
\begin{tikzpicture}[scale = 0.7]
\foreach \i in {1,2,...,4}
	\draw[fill = black!30] (\i,0) rectangle (\i+1,0+\i);
\draw[thick, ->] (-1,0)--(6,0) node [anchor = north west] {$x$};
\draw[thick, ->] (0,-1)--(0,6) node [anchor = south east] {$y$};
\draw[thick] (1,1)--(5,5);
\end{tikzpicture}
\qquad
\begin{tikzpicture}[scale = 0.7]
\foreach \i in {1,1.5,...,4.5}
	\draw[fill = black!30] (\i,0) rectangle (\i+0.5,0+\i);
\draw[thick, ->] (-1,0)--(6,0) node [anchor = north west] {$x$};
\draw[thick, ->] (0,-1)--(0,6) node [anchor = south east] {$y$};
\draw[thick] (1,1)--(5,5);
\end{tikzpicture}
\end{figure}

\th{7.2.4.(3).} $\forall \tau', \tau''$ - розбиття відрізка $[a,b]: \forall x \in [a,b]:$\\
$\underline{f}_{\tau'}(x) \leq f(x) \leq \overline{f}_{\tau''}(x)$\\
\proof
Фіксуємо $\tau', \tau''$. Розглянемо розбиття $\tau = \tau' \cup \tau''$ - підрозбиття, до речі, обох розбить\\
Тому за попередньою теоремою,\\
$\underline{f}_{\tau'}(x) \leq \underline{f}_{\tau}(x) \leq f(x) \leq \overline{f}_{\tau}(x) \leq \overline{f}_{\tau''}(x)$ \qed
\bigline
\th{7.2.4.(4).} $\forall \tau$ - розбиття відрізка $[a,b]: \forall \xi^{\tau}:$\\
$\huge \int_a^b \underline{f}_{\tau}(x)\,dx \leq S_{\tau, \xi}(f) \leq \int_a^b \overline{f}_{\tau}(x)\,dx$\\
\proof
Фіксуємо $\tau, \xi^{\tau}$. Оскільки $\xi_k \in \Delta_k$, то\\
$\huge f(\xi_k) \geq \inf_{t \in \Delta_k} f(t) = m_k$\\
$\huge f(\xi_k) \leq \sup_{t \in \Delta_k} f(t) = M_k$\\
Обидві нерівності ми помножимо обидві частини на $(x_k - x_{k-1})$, а потім просумуємо по $k$. Отримаємо:\\
$\huge S_{\tau, \xi}(f) = \sum_{k=1}^n f(\xi_k)(x_k-x_{k-1}) \geq \sum_{k=1}^n m_k(x_k-x_{k-1}) = \int_a^b \underline{f}_{\tau}(x)\,dx$\\
$\huge S_{\tau, \xi}(f) = \sum_{k=1}^n f(\xi_k)(x_k-x_{k-1}) \leq \sum_{k=1}^n M_k(x_k-x_{k-1}) = \int_a^b \overline{f}_{\tau}(x)\,dx$ \qed
\bigline
Задана функція $f: [a,b] \to \mathbb{R}$ - обмежена. Покладемо наступні числа:\\
$\underline{I} = \huge \sup_{\tau} \int_a^b \underline{f}_{\tau}(x)\,dx \hspace{1cm} \overline{I} = \huge \inf_{\tau} \int_a^b \overline{f}_{\tau}(x)\,dx$\\
Тобто нас цікавить найбільша сума з прямокутників інфінума та найменша сума з прямокутників супремума\\
\th{7.2.5.(1).} $\underline{I} \leq \overline{I}$\\
\proof
Візьмемо два довільних розбиття $\tau', \tau''$. Зафіксуємо $\tau = \tau' \cup \tau''$. За \textbf{Th. 7.2.4.(2),(4)}, маємо:\\
$\huge \int_a^b \underline{f}_{\tau'}(x)\,dx \leq \int_a^b \underline{f}_{\tau}(x) \leq \int_a^b \overline{f}_{\tau}(x)\,dx \leq \int_a^b \overline{f}_{\tau''}(x)\,dx$\\
Візьмемо супремум по всім розбиттям $\tau'$, отримаємо:\\
$\huge \underline{I} \leq \int_a^b \overline{f}_{\tau''}(x)\,dx$\\
І це для довільного $\tau''$. Тому:\\
$\underline{I} \leq \overline{I}$ \qed
\bigline
\th{7.2.5.(2).} $f \in R([a,b]) \iff \underline{I} = \overline{I} = I$\\
\proof
$\boxed{\Leftarrow}$ Дано: $\underline{I} = \overline{I} = I$\\
Тобто за критерієм супремума та інфімума,  $\forall \varepsilon > 0: \exists \tau', \tau'':$\\
$0 \leq \underline{I} - \huge \int_a^b \underline{f}_{\tau'}(x)\,dx < \varepsilon (*)$\\
$0 \leq \overline{I} - \huge \int_a^b \overline{f}_{\tau''}(x)\,dx < \varepsilon (**)$\\
Покладемо $\tau_{\varepsilon} = \tau' \cup \tau''$. Тоді $\forall \tau \supset \tau_{\varepsilon} \Rightarrow \forall \tau \supset \tau', \tau \supset \tau'':$\\
$-\varepsilon < \huge \int_{a}^b \underline{f}_{\tau'} (x)\,dx - \underline{I} \leq \int_a^b \underline{f}_{\tau} (x)\,dx - \underline{I} \leq S_{\tau, \xi^{\tau}}(f) - I \leq \\ \leq \int_a^b \overline{f}_{\tau}(x)\,dx - \overline{I} \leq \int_a^b \overline{f}_{\tau''}(x)\,dx - \overline{I} <\varepsilon$\\
$\Rightarrow |S_{\tau, \xi^{\tau}}(f) - I| < \varepsilon$\\
Отже, $f \in R([a,b])$
\bigline
$\boxed{\Rightarrow}$ Дано: $f \in R([a,b])$\\
тобто $\forall \varepsilon > 0: \exists \tau_{\varepsilon}: \forall \tau \supset \tau_{\varepsilon}: \forall \xi: |S_{\tau, \xi} - I| < \varepsilon$\\
Оберемо такі точки $\xi_k'$, щоб $f(\xi_k')-m_k < \dfrac{\varepsilon}{b-a}$\\
Помножимо цю нерівність на $(x_k-x_{k-1})$ та просумуємо по $k$. Тоді:\\
$\huge \sum_{k=1}^n f(\xi_k')(x_k-x_{k-1}) - \sum_{k=1}^n m_k(x_k-x_{k-1}) < \frac{\varepsilon}{b-a} \sum_{k=1}^n (x_k-x_{k-1})$\\
$\Rightarrow 0 \leq S_{\tau_{\varepsilon}, \xi'}(f) - \huge \int_{a}^b \underline{f}_{\tau_{\varepsilon}}(x)\,dx < \varepsilon$\\
Аналогічно для точок $\xi_k''$, для яких\\
$M_k - f(\xi_k'') < \dfrac{\varepsilon}{b-a}$\\
можемо отримати, що:\\
$\Rightarrow 0 \leq \huge \int_a^b \overline{f}_{\tau_{\varepsilon}}(x)\,dx - S_{\tau_{\varepsilon},\xi''}(x) < \varepsilon$\\
Тоді отримаємо, що:\\
$0 \leq \overline{I} - \underline{I} \leq \huge \int_a^b \overline{f}_{\tau_{\varepsilon}}(x)\,dx - \int_a^b \underline{f}_{\tau_{\varepsilon}}(x)\,dx = \abs{\int_a^b \overline{f}_{\tau_{\varepsilon}}(x)\,dx - \int_a^b \underline{f}_{\tau_{\varepsilon}}(x)\,dx} = \\
= \abs{\int_a^b \overline{f}_{\tau_{\varepsilon}}(x)\,dx - S_{\tau,\xi''}(f) + S_{\tau,\xi'}(f) - \int_a^b \underline{f}_{\tau_{\varepsilon}}(x)\,dx + S_{\tau,\xi''}(f) - I + I - S_{\tau_{\varepsilon},\xi'}(f)} \leq \abs{\int_a^b \overline{f}_{\tau_{\varepsilon}}(x)\,dx - S_{\tau,\xi''}(f)} + \abs{S_{\tau,\xi'}(f) - \int_a^b \underline{f}_{\tau_{\varepsilon}}(x)\,dx} + \abs{S_{\tau,\xi''}(f) - I} + \abs{I - S_{\tau_{\varepsilon},\xi'}(f)} < \varepsilon + \varepsilon + \varepsilon + \varepsilon = 4\varepsilon$\\
Оскільки це $\forall \varepsilon > 0$, то отримаємо, що $\overline{I} = \underline{I} = 0$ \qed
\bigline
\th{7.2.6.} Визначимо наступні величини
\begin{align*}
\omega_{\Delta_k} (f) = M_k - m_k \hspace{2cm} \omega_{\tau}(f) = \int_a^b \overline{f}(x)\,dx - \int_a^b \underline{f}(x)\,dx
\end{align*}
$f \in R([a,b]) \iff $\\
1) $f$ - обмежена на $[a,b]$\\
2) $\forall \varepsilon > 0: \exists \tau_\varepsilon: \omega_{\tau_{\varepsilon}}(f) < \varepsilon$\\
\proof
$\boxed{\Leftarrow}$ Дано: дві умови\\
Доведемо, що $\overline{I} = \underline{I}$\\
Задамо $\varepsilon > 0$, тоді $\exists \tau_\varepsilon: \omega_{\tau_{\varepsilon}}(f) < \varepsilon$\\
$\Rightarrow \overline{I} - \underline{I} = \huge \inf_{\tau'} \int_a^b \overline{f}_{\tau'}(x)\,dx - \sup_{\tau'} \int_a^b \underline{f}_{\tau'}(x)\,dx \leq \int_{a}^b \overline{f}_{\tau_\varepsilon}(x)\,dx - \int_{a}^b \underline{f}_{\tau_\varepsilon}(x)\,dx < \varepsilon$\\
$\Rightarrow \overline{I} = \underline{I} \Rightarrow f \in R([a,b])$
\bigline
$\boxed{\Rightarrow}$ Дано: $f \in R([a,b])$\\
Тоді вона є автоматично обмеженою, а тому завдяки \th{7.2.5.(2).}, маємо, що $\overline{I} = \underline{I}$\\

\newpage

\section{Визначені інтеграли (спроба 2)}
\subsection{Визначений інтеграл Рімана}
\defin{7.1.1.} \textbf{Розбиттям} відрізка $[a,b]$ називають множину точок
\begin{align*}
P: a = x_0 < x_1 < \dots < x_n = b
\end{align*}
Визначимо деякі величини
\begin{align*}
\Delta x_i = x_i - x_{i-1}, i = \overline{1,n} \\
M_i = \sup_{x \in [x_{i-1}, x_i]} f(x) \hspace{0.5cm} m_i = \inf_{x \in [x_{i-1},x_i]} f(x)
\end{align*}
\defin{7.1.2. Сумою Дарбу} називають такі значення
\begin{align*}
U(P,f) = \sum_{i=1}^n M_i \Delta x_i \hspace{0.5cm} L(P,f) = \sum_{j=1}^n m_i \Delta x_i
\end{align*}
\defin{7.1.3. Верхнім та нижнім} інтегралами називають такі значення
\begin{align*}
\overline{I}(f) = \inf_{P} U(P,f) \hspace{0.5cm} \underline{I}(f) = \sup_{P} U(P,f)
\end{align*}
\begin{figure}[H]
\centering
\begin{tikzpicture}
\draw[thick] (1,-1pt)--(1,1pt) node[anchor = north] {$x_0$};
\draw[thick] (2,-1pt)--(2,1pt) node[anchor = north] {$x_1$};
\draw[thick] (5,-1pt)--(5,1pt) node[anchor = north] {$x_n$};

\foreach \i in {2,3,4,5} {
\draw[thick, dashed] (\i,{exp((\i-2)/3)})--(\i,0);
}

\draw[thick, ->] (-0.5,0)--(5.5,0) node[anchor = north] {$x$};
\draw[thick, ->] (0,-0.5)--(0,3) node[anchor = east] {$y$};

\draw[thick, domain=1:5, variable=\x, samples = 1000] plot({\x}, {exp((\x-2)/3)});

\draw[thick, red] (1,{exp((2-2)/3)})--(2,{exp((2-2)/3)}) node[anchor = south east] {$U(P,f)$};
\draw[thick, red] (2,{exp((3-2)/3)})--(3,{exp((3-2)/3)});
\draw[thick, red] (3,{exp((4-2)/3)})--(4,{exp((4-2)/3)});
\draw[thick, red] (4,{exp((5-2)/3)})--(5,{exp((5-2)/3)});
\end{tikzpicture}
\qquad
\begin{tikzpicture}
\draw[thick] (1,-1pt)--(1,1pt) node[anchor = north] {$x_0$};
\draw[thick] (2,-1pt)--(2,1pt) node[anchor = north] {$x_1$};
\draw[thick] (5,-1pt)--(5,1pt) node[anchor = north] {$x_n$};

\foreach \i in {2,3,4,5} {
\draw[thick, dashed] (\i,{exp((\i-1-2)/3)})--(\i,0);
}

\draw[thick, ->] (-0.5,0)--(5.5,0) node[anchor = north] {$x$};
\draw[thick, ->] (0,-0.5)--(0,3) node[anchor = east] {$y$};

\draw[thick, domain=1:5, variable=\x, samples = 1000] plot({\x}, {exp((\x-2)/3)});

\draw[thick, blue] (1,{exp((1-2)/3)})--(2,{exp((1-2)/3)}) node[anchor = south east] {$L(P,f)$};
\draw[thick, blue] (2,{exp((2-2)/3)})--(3,{exp((2-2)/3)});
\draw[thick, blue] (3,{exp((3-2)/3)})--(4,{exp((3-2)/3)});
\draw[thick, blue] (4,{exp((4-2)/3)})--(5,{exp((4-2)/3)});
\end{tikzpicture}
\end{figure}
\defin{7.1.4.} Функція $f(x)$ називається \textbf{інтегрованою за Ріманом} на $[a.b]$, якщо $\overline{I}(f) = \underline{I}(f)$\\
Позначення: $f \in R([a,b])$\\
В цьому випадку \textbf{визначеним інтегралом Рімана} від $f(x)$ на $[a,b]$ називають значення
\begin{align*}
\int\displaylimits_a^b f(x)\,dx = \overline{I}(f) = \underline{I}(f)
\end{align*}
\defin{7.1.5. Підрозбиттям} розбиття $P$ називають розбиття \\ $P^* \supset P$
\bigline
\lm{7.1.6.} Задано розбиття $P$ та підрозбиття $P^*$. Тоді $\begin{gathered} L(P^*,f) \geq L(P,f) \\ U(P^*,f) \leq U(P,f) \end{gathered}$\\
\proof
Розглянемо першу нерівність\\
Нехай $P^* = P \cup \{x^*\}$, де $x^* \in (x_{k-1}, x_k)$ - фіксований інтервал\\
Тоді $U(P^*, f) = \huge\sum_{j=1}^{n+1} M_j^* \Delta x_j^* = \\ = \huge \sum_{i=1}^{k-1} M_i \Delta x_i + \sum_{i=k+1}^{n} M_i \Delta x_i + M^*_k (x^* - x_{k-1}) + M^*_{k+1} (x_k - x^*) \leq$\\
$M_k^* \leq M_k \hspace{0.5cm} M_{k+1}^* \leq M_k$\\
$\leq \huge \sum_{i=1}^{k-1} M_i \Delta x_i + \sum_{i=k+1}^n M_i \Delta x_i + M_k (x_k - x_{k-1}) = U(P,f)$\\
І так ми можем додавати по точці скільки нам буде потрібно\\
Для другої нерівності аналогічно \qed
\begin{figure}[H]
\centering
\begin{tikzpicture}
\draw[thick] (1,-1pt)--(1,1pt) node[anchor = north] {$x_0$};
\draw[thick] (2,-1pt)--(2,1pt) node[anchor = north] {$x_1$};
\draw[thick] (3,-1pt)--(3,1pt) node[anchor = north] {$x_{k-1}$};
\draw[thick] (4,-1pt)--(4,1pt) node[anchor = north] {$x_k$};
\draw[thick] (5,-1pt)--(5,1pt) node[anchor = north] {$x_n$};

\foreach \i in {2,3,4,5} {
\draw[thick, dashed] (\i,{exp((\i-2)/3)})--(\i,0);
}

\draw[thick, ->] (-0.5,0)--(5.5,0) node[anchor = north] {$x$};
\draw[thick, ->] (0,-0.5)--(0,3) node[anchor = east] {$y$};

\draw[thick, domain=1:5, variable=\x, samples = 1000] plot({\x}, {exp((\x-2)/3)});

\draw[thick, red] (1,{exp((2-2)/3)})--(2,{exp((2-2)/3)}) node[anchor = south east, scale = 0.9] {$U(P,f)$};
\draw[thick, red] (2,{exp((3-2)/3)})--(3,{exp((3-2)/3)});
\draw[thick, red] (3,{exp((4-2)/3)})--(4,{exp((4-2)/3)});
\draw[thick, red] (4,{exp((5-2)/3)})--(5,{exp((5-2)/3)});
\end{tikzpicture}
\qquad
\begin{tikzpicture}
\draw[thick] (1,-1pt)--(1,1pt) node[anchor = north] {$x_0$};
\draw[thick] (2,-1pt)--(2,1pt) node[anchor = north] {$x_1$};
\draw[thick] (3,-1pt)--(3,1pt) node[anchor = north] {$x_{k-1}$};
\draw[thick] (3.6,-1pt)--(3.6,1pt) node[anchor = south, scale = 0.8] {$x^*$};
\draw[thick] (4,-1pt)--(4,1pt) node[anchor = north] {$x_k$};
\draw[thick] (5,-1pt)--(5,1pt) node[anchor = north] {$x_n$};

\foreach \i in {2,3,4,5} {
\draw[thick, dashed] (\i,{exp((\i-2)/3)})--(\i,0);
}
\draw[thick, dashed] (3.6,{exp((3.6-2)/3)})--(3.6,0);

\draw[thick, ->] (-0.5,0)--(5.5,0) node[anchor = north] {$x$};
\draw[thick, ->] (0,-0.5)--(0,3) node[anchor = east] {$y$};

\draw[thick, domain=1:5, variable=\x, samples = 1000] plot({\x}, {exp((\x-2)/3)});

\draw[thick, red] (1,{exp((2-2)/3)})--(2,{exp((2-2)/3)}) node[anchor = south east, scale = 0.9] {$U(P^*,f)$};
\draw[thick, red] (2,{exp((3-2)/3)})--(3,{exp((3-2)/3)});
\draw[thick, red] (3,{exp((3.6-2)/3)})--(3.6,{exp((3.6-2)/3)});
\draw[thick, red] (3.6,{exp((4-2)/3)})--(4,{exp((4-2)/3)});
\draw[thick, red] (4,{exp((5-2)/3)})--(5,{exp((5-2)/3)});
\end{tikzpicture}

\caption{Праворуч сума площ прямокутників менша за площ прямокутників ліворуч, що й свідчить наша нерівність}
\end{figure}

\crl{7.1.6.} Для довільної функції $f$ справедлива нерівність: \\ $\underline{I}(f) \leq \overline{I}(f)$
\bigline
\th{7.1.7. Критерій інтегрованості}\\
$f \in R([a,b]) \iff \forall \varepsilon > 0: \exists P: U(P,f) - L(P,f) < \varepsilon$\\
\proof
$\boxed{\Rightarrow}$ Дано $f \in R([a,b])$, тобто\\
$\huge \int\displaylimits_a^b f(x)\,dx = \overline{I}(f) = \underline{I}(f)$\\
Зафіксуємо $\varepsilon > 0$, тоді за критерієм $\sup, \inf$:\\
$\exists P_1: \huge\int_a^b f(x)\,dx = \inf_P U(P, f) > U(P_1,f) - \dfrac{\varepsilon}{2} > U(P^*,f) - \dfrac{\varepsilon}{2}$\\
$\exists P_2: \huge\int_a^b f(x)\,dx = \sup_P U(P, f) < L(P_2,f) + \dfrac{\varepsilon}{2} < L(P^*,f) + \dfrac{\varepsilon}{2}$\\
Де $P^* = P_1 \cup P_2$\\
Другу нерівність домножимо на $(-1)$, а потім додамо ці нерівності - отримаємо\\
$U(P^*,f) - L(P^*,f) < \varepsilon$
\bigline
$\boxed{\Leftarrow}$ Дано: $\forall \varepsilon > 0: \exists P: U(P,f) - L(P,f) < \varepsilon$\\
Відомо, що $L(P,f) \leq \underline{I}(f) \leq \overline{I}(f) \leq U(P,f)$\\
Звідси $0 \leq \overline{I}(f) - \underline{I}(f) \leq U(P,f) - L(P,f) < \varepsilon$\\
Оскільки виконується $\forall \varepsilon > 0$, то маємо, що $\overline{I}(f) = \underline{I}(f)$, а отже, $f \in R([a,b])$ \qed
\bigline

\th{7.1.8.} Задана функція $f \in C([a,b])$. Тоді $f \in R([a,b])$\\
\proof
$f \in C([a,b]) \Rightarrow f \in C_{unif}([a,b]) \Rightarrow \\ \forall \varepsilon > 0: \exists \delta > 0: \forall x_1,x_2: |x_1 - x_2| < \delta \Rightarrow |f(x_1)-f(x_2)| < \varepsilon$\\
Зафіксуємо таке розбиття $P: a = x_0 < x_1 < \dots < x_n = b$ таким чином, що $\huge\max_{i = \overline{1,n}} \Delta x_i < \delta$
\bigline
\rm{7.1.8.} $\huge\max_{i = \overline{1,n}} \Delta x_i = \mu(P)$ - діаметр розбиття
\bigline
Візьмемо відрізок $[x_{k-1},x_k]$. Тоді $\forall x^{(1)}, x^{(2)} \in [x_{k-1},x_k]: \\ |x^{(1)} - x^{(2)}| < \mu(P) < \delta \Rightarrow |f(x^{(1)}) - f(x^{(2)})| < \varepsilon$\\
Зокрема для таких точок, де $f(x^{(1)}) = M_k$, $- f(x^{(2)}) = m_k$ матимемо:\\
$M_k - m_k < \varepsilon$\\
$\Rightarrow U(P,f) - L(P,f) = \huge\sum_{k=1}^n (M_k-m_k) \Delta x_k < \varepsilon \sum_{k=1}^n \Delta x_k = \varepsilon(b-a)$\\
Остаточно, $f \in R([a,b])$ \qed
\bigline
\th{7.1.9.} Задана функція $f$ - монотонна на $[a,b]$. Тоді $f \in R([a,b])$\\
\proof
Нехай $f$ зростає, тоді $\forall P: \begin{gathered} M_k = f(x_k) \\ m_k = f(x_{k-1}) \end{gathered}$\\
Візьмемо таке розбиття $P$, щоб $x_k = a + \dfrac{b-a}{n}k$, $k = \overline{0,n}$\\
Тобто $\Delta x_k = \dfrac{b-a}{n}, k = \overline{1,n}$\\
Тоді $U(P,f) - L(P,f) = \huge\sum_{k=1}^n (f(x_k) - f(x_{k-1})) \dfrac{b-a}{n} = (M-m) \dfrac{b-a}{n} <$\\
Ми знаємо, що $\forall \varepsilon > 0: \exists N: \forall n \geq N: \dfrac{1}{n} < \varepsilon$\\
$< (M-m)(b-a)\varepsilon$\\
Остаточно $f \in R([a,b])$ \qed
\bigline

\th{7.1.10. Властивості}\\
1. Задано $f_1, f_2 \in R([a,b])$. Тоді $\begin{gathered} f_1 + f_2 \in R \left( \left[ a,b \right] \right) \\ \forall c \in \mathbb{R}: cf_1 \in R \left( \left[ a,b \right] \right) \end{gathered}$, причому\\
$\huge\int\displaylimits_a^b f_1(x)+f_2(x)\,dx = \int\displaylimits_a^b f_1(x)\,dx + \int\displaylimits_a^b f_2(x)\,dx$\\
$\huge\int\displaylimits_a^b c f_1(x) \,dx = c \int\displaylimits_a^b f_1(x)\,dx$
\bigline
2. Задано $f_1, f_2 \in R([a,b])$ таким чином, що $\forall x \in [a,b]: f_1(x) \leq f_2(x)$\\
Тоді $\huge\int\displaylimits_a^b f_1(x)\,dx \leq \int\displaylimits_a^b f_2(x)\,dx$
\bigline
3. Функція $f \in R([a,b]) \iff \forall x \in (a,b): \begin{cases} f \in R([a,c]) \\ f \in R([c,b]) \end{cases}$, причому\\
$\huge \int\displaylimits_a^b f(x)\,dx = \int\displaylimits_a^c f(x)\,dx + \int\displaylimits_c^b f(x)\,dx$
\bigline
4. Задано $f$ - обмежена. Функція $f \in R([a,b]) \iff \abs{\huge\int\displaylimits_a^b f(x)\,dx} \leq M(b-a)$\\
\proof
1. $\forall \varepsilon > 0: \begin{cases}
\exists P_1: U(P_1,f_1) - L(P_1,f_1) < \varepsilon \\
\exists P_2: U(P_2,f_2) - L(P_2,f_2) < \varepsilon
 \end{cases}$\\
Зафіксуємо $P = P_1 \cup P_2$. Тоді ці нерівності виконуються одночасно\\
Розглянемо таку суму Дарбу\\
$U(P,f_1+f_2) = \huge\sum_{k=1}^n M_k(f_1 + f_2) \Delta x_k \boxed{\leq}$\\
$M_k(f_1+f_2) = \huge\sup_{x \in [x_{k-1},x_k]} (f_1(x) + f_2(x)) \leq \sup_{x \in [x_{k-1},x_k]} f_1(x) + \sup_{x \in [x_{k-1},x_k]} f_2(x) = M_k(f_1) + M_k(f_2)$\\
$\boxed{\leq} \huge\sum_{k=1}^n M_k(f_1) \Delta x_k + \huge\sum_{k=1}^n M_k(f_2) \Delta x_k = U(P,f_1) + U(P,f_2)$\\
Аналогічно з іншою сумою Дарбу\\
$L(P, f_1+f_2) \geq L(P,f_1) + L(P,f_2)$\\
$\Rightarrow U(P,f_1+f_2) - L(P,f_1+f_2) \leq (U(P,f_1) - L(P,f_1)) + (U(P,f_2) - L(P,f_2)) < 2 \varepsilon$\\
Отже, $f \in R([a,b])$\\
А тепер доведемо рівність. Ми знаємо, що\\
$L(P,f_1) \leq \huge\int\displaylimits_a^b f_1(x)\,dx \leq U(P,f_1)$\\
$L(P,f_2) \leq \huge\int\displaylimits_a^b f_2(x)\,dx \leq U(P,f_2)$\\
Якщо складемо два нерівності, отримаємо\\
$0 \leq \abs{\huge \int\displaylimits_a^b f_1(x)+f_2(x)\,dx - \left( \int\displaylimits_a^b f_1(x)\,dx + \int\displaylimits_a^b f_2(x)\,dx \right) } < 2 \varepsilon$\\
А оскільки $\forall \varepsilon > 0$ виконано, то тоді\\
$\huge\int\displaylimits_a^b f_1(x)+f_2(x)\,dx = \int\displaylimits_a^b f_1(x)\,dx + \int\displaylimits_a^b f_2(x)\,dx$
\bigline
Довести $c f_1 \in R([a,b])$ нескладно. Єдине треба зауважити, що\\
$\forall c > 0: \huge\sup cf(x) = c \sup f(x)$\\
$\forall c < 0: \huge\sup cf(x) = c \inf f(x)$
\bigline
2. Оскільки $f_1(x) \leq f_2(x)$, то тоді $f_2(x) - f_1(x) \geq 0, \forall x \in [a,b]$\\
$M_k(f_2-f_1) = \huge \sup_{x \in [x_{k-1}, x_k]} (f_2(x) - f_1(x)) \geq 0$\\
$m_k(f_2-f_1) = \huge \inf_{x \in [x_{k-1}, x_k]} (f_2(x) - f_1(x)) \geq 0$\\
$\Rightarrow \huge \int\displaylimits_a^b f_2(x) - f_1(x)\,dx \geq 0 \Rightarrow \int\displaylimits_a^b f_1(x) \,dx \leq \int\displaylimits_a^b f_2(x) \,dx$
\bigline
3. Нехай є розбиття $P: a = x_0 < x_1 < \dots < x_n = b$\\
Вважаємо точку $c = x_k$. Тоді маємо два розбиття:\\
$P_1: a =x_0 < x_1 < \dots < x_k = c$\\
$P_2: c = x_k < x_{k+1} < \dots < x_n = b$\\
Тоді $U(P,f) = U(P_1,f) + U(P_2,f)$\\
Аналогічно $L(P,f) = L(P_1,f) + L(P_2,f)$\\
Нам відомо, що $f \in R([a,b]) \Rightarrow \forall \varepsilon > 0: \exists P: U(P,f) - L(P,f) < \varepsilon$\\
$\Rightarrow \begin{cases} U(P_1,f) - L(P_1,f) < \varepsilon \\ U(P_2,f) - L(P_2,f) < \varepsilon \end{cases} \Rightarrow \begin{cases} f \in R([a,c]) \\ f \in R([c,b]) \end{cases}$\\
Більш того, \\
$\displaystyle \abs{\int\displaylimits_a^b f(x)\,dx - \left(\int\displaylimits_a^c f(x)\,dx + \int\displaylimits_c^b f(x)\,dx \right)} < \varepsilon$ (перевірити самостійно), це виконується $\forall \varepsilon > 0$\\
Отже, $\huge \int\displaylimits_a^b f(x)\,dx = \int\displaylimits_a^c f(x)\,dx + \int\displaylimits_c^b f(x)\,dx$
\bigline
4. $|U(P,f)| =\huge \abs{ \sum_{k=1}^n M_k \Delta x_k} \leq \sum_{k=1}^n |M_k| \Delta x_k \leq M \sum_{k=1}^n \Delta x_k = M (b-a)$\\
Тут $M = \huge \max_{k = \overline{1,n}} M_k$ \qed
\bigline
\th{7.1.11. Інтегрованість композиції}\\
Задано $g \in R([a,b])$. Відомо, що $\forall x \in [a,b]: m \leq g(x) \leq M$\\
а також $f \in C([m,M])$. Тоді $h = f \circ g \in R([a,b])$\\
\proof
$f \in C([m,M]) \Rightarrow f \in C_{unif}([m,M]) \Rightarrow \forall \varepsilon > 0: \exists \delta > 0: \forall y_1,y_2 \in [m,M]: |y_1-y_2| < \delta \Rightarrow |f(y_1)-f(y_2)| < \varepsilon$\\
$g \in R([a,b]) \Rightarrow \exists P: U(P,g) - L(P,g) < \delta^2$\\
Розглянемо $h(x) = f(g(x))$ та його суми Дарбу на цьому розбитті\\
$U(P,h) - L(P,h) = \huge \sum_{k=1}^n (M_k(h) - m_k(h)) \Delta x_k$
\newpage
\section{Невласні інтеграли}
(TODO)
\newpage

\section{Ряди}
\defin{9. Рядами} називають формальну нескінченну суму нескінченної послідовності чисел $\{a_n, n \geq 1\}$
\begin{align*}
a_1 + a_2 + \dots + a_n + \dots + \huge \sum_{n=1}^{\infty} a_n
\end{align*}
\textbf{Частковою сумою} даного ряда називають суму перших $k$ членів
\begin{align*}
S_k = \sum_{n=1}^k a_n = a_1 + a_2 + \dots + a_k
\end{align*}
В такому випадку в нас виникає послідовність часткових сум $\{S_k, k \geq 1\}$\\
Якщо така послідовність часткових сум є збіжною, то ряд $\huge \sum_{n=1}^{\infty} a_n$ називають \textbf{збіжним} та значення цього ряду дорівнює
\begin{align*}
\sum_{n=1}^{\infty} a_n = \lim_{k \to \infty} S_k = S
\end{align*}
Інакше - \textbf{розбіжним}
\bigline
\ex{9.} Знайдемо суму: $1 + q + q^2 + \dots$\\
Розглянемо часткову суму $S_k = 1 + q + \dots + q^k = \dfrac{1 - q^k}{1 - q}$ - сума геом. прогресії\\
$\huge \lim_{k \to \infty} S_k = \lim_{k \to \infty} \dfrac{1-q^k}{1 - q} = \left[ \begin{gathered} \dfrac{1}{1-q}, |q|<1 \\ \infty, |q|>1 \end{gathered} \right.$
\\
При $q = 1$ маємо: $1 + 1 + 1 + \dots$\\
$S_k = k \Rightarrow \huge \lim_{k \to \infty} S_k = \infty$\\
Підсумуємо:\\
- сума є збіжною при $|q| < 1$ та дорівнює\\
$1 + q + q^2 + \dots = \dfrac{1}{1-q}$\\
- сума є розбіжнрю при $|q| \geq 1$
\bigline

\subsection{Первинний аналіз збіжності та арифметика рядів}
\prp{9.1.1. Необхідна ознака збіжності ряду}\\
Задано $\huge \sum_{n=1}^{\infty} a_n$ - збіжний. Тоді $\huge \lim_{n \to \infty} a_n = 0$\\
\proof
Зафіксуємо часткові суми:\\
$\huge S_{k+1} = \sum_{n=1}^{k+1} a_n \hspace{0.5cm} S_{k} = \sum_{n=1}^{k} a_n$\\
Оскільки ряд є збіжним, то $\huge \lim_{k \to \infty} S_{k+1} = \lim_{k \to \infty} S_k = S$\\
Тоді $\huge \lim_{k \to \infty} a_{k+1} = \lim_{k \to \infty} (S_{k+1} - S_k) = S - S = 0$ \qed
\bigline
\rm{9.1.1.} Якщо виникне, що $\lim_{n \to \infty} a_n \neq 0$, або її взагалі не існує, то $\huge \sum_{n=1}^{\infty} a_n$ - розбіжний
\bigline
\rm{9.1.1.(2)} Це лише - необхідна ознака, в жодному випадку не достатня. Якщо границя буде нулевою, то це не означає, що ряд збігається, потрібну інші дослідження
\bigline
\ex{9.1.2.} Розглянемо ряд $\huge \sum_{n=1}^{\infty} (-1)^n = -1 + 1 - 1 + \dots$\\
Оскільки $\not\exists \huge \lim_{n \to \infty} (-1)^n$, то за \textbf{Rm. 9.1.1.} маємо, що ряд - розбіжний
\bigline
\th{9.1.3. Критерій Коші}\\
Задано $\huge \sum_{n=1}^{\infty} a_n$\\
Ряд - збіжний $\iff \forall \varepsilon > 0: \exists K: \forall k \geq K: \forall p \geq 1: \abs{\huge \sum_{n=k+1}^{k+p} a_n} < \varepsilon$\\
\proof
$\huge \sum_{n=1}^{\infty} a_n$ - збіжний $\iff \exists \huge \lim_{k \to \infty} S_k$ - збіжна границя $\overset{\textrm{критерій Коші}}{\iff} \\ \iff \forall \varepsilon > 0: \exists K: \forall k \geq K: \forall p \geq 1: |S_{k+p} - S_k| =\huge \abs{\sum_{n=k+1}^{k+p} a_n} < \varepsilon$ \qed
\bigline
\prp{9.1.4.} Задані $\huge \sum_{n=1}^{\infty} a_n \hspace{0.3cm} \sum_{n=1}^{\infty} b_n$ - збіжні. Тоді збіжними будуть й наступні ряди\\
1) $\forall \alpha \in \mathbb{R}: \huge \sum_{n=1}^{\infty} \alpha a_n = \alpha \sum_{n=1}^\infty a_n$\\
2) $\huge \sum_{n=1}^{\infty} (a_n+b_n) = \sum_{n=1}^{\infty} a_n + \sum_{n=1}^{\infty} b_n$\\
\proof
Доведу друге. Зафіксуємо часткові суми\\
2) $\huge S_k(a) = \sum_{n=1}^k a_n \hspace{0.5cm}, S_k(b) = \sum_{n=1}^k b_n$\\
Тоді $S_k(a) + S_k(b) = \huge \sum_{n=1}^k (a_n+b_n) = \sum_{n=1}^k a_n + \sum_{n=1}^k b_n$\\
Оскільки $\huge \sum_{n=1}^{\infty} a_n \hspace{0.3cm} \sum_{n=1}^{\infty} b_n$ - збіжні, то $\huge \lim_{k \to \infty} S_k(a) = S(a), \hspace{0.3cm} \huge \lim_{k \to \infty} S_k(b) = S(b)$
$\huge \sum_{n=1}^{\infty} (a_n+b_n) = \lim_{k \to \infty} (S_k(a) + S_k(b)) = S(a) + S(b) = \sum_{n=1}^{\infty} a_n + \sum_{n=1}^{\infty} b_n$\\
Перший пункт аналогічно \qed
\bigline
\defin{9.1.5. Хвостом} ряда $\huge \sum_{n=1}^\infty a_n$ називають ряд $\huge \sum_{n=m}^{\infty} a_n$, де $m \in \mathbb{N}$\\
Тобто ми відкидуємо перші $m-1$ доданків та сумуємо, починаючи з $m$
\bigline
\prp{9.1.6.} $\huge \sum_{n=1}^\infty a_n$ - збіжний $\iff$ $\huge \sum_{n=m}^\infty a_n$ - збіжний\\
\proof
$\huge \sum_{n=1}^\infty a_n$ - збіжний $\overset{\textrm{критерій Коші}}{\iff} \forall \varepsilon > 0: \exists K: \forall k \geq K: \forall p \geq 1: \\ \abs{\huge \sum_{n=k+1}^{k+p} a_n} < \varepsilon \iff \exists K' = \max\{K,m\}: \forall k \geq K': \forall p \geq 1: \\ \abs{\huge \sum_{n=k+1}^{k+p} a_n} < \varepsilon \iff \huge \sum_{n=m}^\infty a_n$ - збіжний \qed
\bigline
\subsection{Знакододатні ряди}
Тобто розглядаємо зараз лише ряди $\huge \sum_{n=1}^{\infty} a_n$, такі, що $\forall n \geq 1: a_n \geq 0$\\
\prp{9.2.1.} $\{S_k, k \geq 1\}$ - мононтонно неспадна послідовність\\
\proof
$\forall k \geq 1: S_{k-1} - S_k = a_{k+1} \geq 0 \Rightarrow S_{k} \leq S_{k+1}$ \qed
\bigline
\prp{9.2.2.} Якщо $\{S_k, k \geq 1\}$ - обмежена, то тоді $\huge \sum_{n=1}^{\infty} a_n$ - збіжний\\
\proof
Щойно дізнались що послідовність часткових сум монотонна. До того ж, вона є обмеженою за умовою. Отже, $\exists \huge \lim_{k \to \infty} S_k = S$, тобто  $\huge \sum_{n=1}^{\infty} a_n$ - збіжний \qed
\bigline
\th{9.2.3. Ознака порівняння в нерівностях}\\
Задані $\huge \sum_{n=1}^{\infty} a_n \hspace{0.3cm} \sum_{n=1}^{\infty} b_n$ таким чином, що $\exists N: \forall n \geq N: a_n \leq b_n$. Тоді:\\
1) якщо $\huge \sum_{n=1}^{\infty} b_n$ - збіжний, то $\huge \sum_{n=1}^{\infty} a_n$ - збіжний теж\\
2) якщо $\huge \sum_{n=1}^{\infty} a_n$ - розбіжний, то $\huge \sum_{n=1}^{\infty} b_n$ - розбіжний теж\\
\proof
Оскільки скінченна кількість чисел членів ряду жодним чином не впливає на збіжність, то можна вважати, що $\forall n \geq 1: a_n \leq b_n$\\
Тоді $\huge \sum_{n=1}^k a_n \leq \sum_{n=1}^k b_n$\\
1) Якщо $\huge \sum_{n=1}^\infty b_n$ - збіжний ряд, то збіжною буде часткова сума, яка буде обмеженою\\
Тоді $\{S_k(a), k \geq 1\}$ - обмежена послідовність та монотонна. Отже, існує границя, а тому $\huge \sum_{n=1}^k a_n$ - збіжний\\
2) Це є оберненим твердженням до 1) (TODO) \qed
\bigline
\th{9.2.4. Ознака порівняння в границях}\\
Задані $\huge \sum_{n=1}^{\infty} a_n \hspace{0.3cm} \sum_{n=1}^{\infty} b_n$, тут члени строго додатні\\
Нехай $\exists \huge \lim_{n \to \infty} \dfrac{a_n}{b_n} = l$. Тоді\\
1) Якщо $l \neq 0$ та $l \neq \infty$, то $\huge \sum_{n=1}^{\infty} a_n, \sum_{n=1}^{\infty} b_n$ збіжні або розбіжні одночасно\\
2) Якщо $l = 0$, то із збіжності $\huge \sum_{n=1}^{\infty} b_n$ випливає збіжність $\huge \sum_{n=1}^{\infty} a_n$
\bigline
\rm{9.2.4.} До речі, $l \geq 0$, оскільки всі члени - додатні\\
\proof
1) $\exists \huge \lim_{n \to \infty} \dfrac{a_n}{b_n} = l \neq 0$, тобто\\
$\forall \varepsilon > 0: \exists N: \forall n \geq N: \abs{\dfrac{a_n}{b_n}-l} < \varepsilon$\\
Оберемо $\varepsilon = \dfrac{l}{2}$, тоді\\
$\dfrac{l}{2} < \dfrac{a_n}{b_n} < \dfrac{3l}{2} \Rightarrow \dfrac{l}{2}b_n < a_n < \dfrac{3l}{2} b_n$, $\forall n \geq N$\\
Припустимо, що $\huge \sum_{n=1}^{\infty} b_n$ - збіжний, тоді збіжним буде $\huge \sum_{n=1}^{\infty} \dfrac{3l}{2} b_n$, а отже, за попередньою теоремою, $\huge \sum_{n=1}^{\infty} a_n$ - збіжний\\
Припустимо, що тепер $\huge \sum_{n=1}^{\infty} b_n$ - розбіжний, тоді розбіжним буде $\huge \sum_{n=1}^{\infty} \dfrac{l}{2}b_n$, а отже, за попередньою теоремою, $\huge \sum_{n=1}^{\infty} a_n$ - розбіжний\\
Остаточно маємо, що при $l \neq 0$ властивість збіжності обох рядів однакова
\bigline
2) $\exists \huge \lim_{n \to \infty} \dfrac{a_n}{b_n} = l = 0$, тобто\\
$\forall \varepsilon > 0: \exists N: \forall n \geq N: \abs{\dfrac{a_n}{b_n}} < \varepsilon$\\
Оберемо $\varepsilon = 1$, тоді\\
$\forall n \geq N: a_n < b_n$\\
Тоді виконується попередня теорема, один з двох пунктів \qed
\bigline
\ex{9.2.5!} Розглянемо $\huge \sum_{n=1}^{\infty} \dfrac{1}{n}$ - \textbf{гармонічний ряд}\\
Доведемо, що даний ряд - розбіжний, використовуючи критерій Коші, тобто\\
$\exists \varepsilon > 0: \forall K: \exists k_1,k_2 \geq K: \huge \abs{\sum_{n=k_1}^{k_2} \dfrac{1}{n}} \geq \varepsilon$\\
Дійсно, якщо $\varepsilon = 0.5$, $k_1 = K, k_2 = 2K$, то отримаємо:\\
$\huge \abs{\sum_{n=K}^{2K} \dfrac{1}{n}} = \dfrac{1}{K} + \dfrac{1}{K+1} + \dots + \dfrac{1}{2K} > K \dfrac{1}{2K} = 0.5$\\
Отже, цей ряд - розбіжний
\bigline
\ex{9.2.6!} Розглянемо далі $\huge \sum_{n=1}^{\infty} \dfrac{1}{n^\alpha}$ \textbf{ряд Діріхле} \\
Нехай $\alpha < 1$, тоді $\forall n \geq 1: \dfrac{1}{n} < \dfrac{1}{n^{\alpha}}$\\
За ознакою порівняння та минулим прикладом, отримаємо, що $\huge \sum_{n=1}^{\infty} \dfrac{1}{n^\alpha}$ - розбіжний
\bigline
Нехай $\alpha > 1$, тоді розглянемо часткову суму\\
$\huge \sum_{n=1}^\infty \dfrac{1}{n^{\alpha}} = 1 + \left(\dfrac{1}{2^{\alpha}} + \dfrac{1}{3^{\alpha}} \right) + \left( \dfrac{1}{4^{\alpha}} + \dfrac{1}{5^{\alpha}} + \dfrac{1}{6^{\alpha}} + \dfrac{1}{7^{\alpha}} \right) + \dots \leq \\
\leq 1 + \left(\dfrac{1}{2^{\alpha}} + \dfrac{1}{2^{\alpha}} \right) + \left( \dfrac{1}{4^{\alpha}} + \dfrac{1}{4^{\alpha}} + \dfrac{1}{4^{\alpha}} + \dfrac{1}{4^{\alpha}} \right) + \dots = \\ = 1 + \dfrac{1}{2^{\alpha-1}} + \dfrac{1}{4^{\alpha-1}} + \dfrac{1}{8^{\alpha-1}} + \dots = 1 + \dfrac{1}{2^{\alpha-1}} + \left( \dfrac{1}{2^{\alpha-1}} \right)^2 + \dots = \dfrac{1}{1 - \dfrac{1}{2^{\alpha-1}}}$\\
Наш ряд - обмежений, а послідовність часткових сум - монотонна. Тоді - збіжний\\
Підсумуємо:\\
$\huge \sum_{n=1}^{\infty} \dfrac{1}{n^{\alpha}}$ - $\left[ \begin{gathered} \textrm{розбіжний}, \alpha \leq 1 \\ \textrm{збіжний}, \alpha > 1   \end{gathered} \right.$
\bigline
\th{9.2.7. Ознака Даламбера}\\
Задано $\huge \sum_{n=1}^{\infty} a_n$ - строго додатній\\
Нехай $\exists \huge \uplim_{n \to \infty} \dfrac{a_{n+1}}{a_n} = q$. Тоді:\\
1) Якщо $q<1$, то ряд - збіжний\\
2) Якщо $q>1$, то ряд - розбіжний\\
3) Якщо $q=1$, то відповіді нема\\
\proof
1) $\exists \huge \lim_{n \to \infty} \dfrac{a_{n+1}}{a_n} = q <1$, тобто\\
$\forall \varepsilon > 0: \exists N: \forall n \geq N: \abs{\dfrac{a_{n+1}}{a_n} - q} < \varepsilon$\\
Встановимо $\varepsilon = \dfrac{1-q}{2}$, тоді\\
$\dfrac{a_{n+1}}{a_n} < q + \varepsilon = \dfrac{1+q}{2}$\\
$\forall n \geq N: a_{n+1} < \dfrac{1+q}{2}a_n$\\
$\Rightarrow a_{N+1} < \dfrac{1+q}{2}a_N$\\
$\Rightarrow a_{N+2} < \dfrac{1+q}{2}a_{N+1} < \left( \dfrac{1+q}{2} \right)^2 a_N$\\
$\dots$\\
$\Rightarrow \forall k \geq 1: a_{N+k} < \left( \dfrac{1+q}{2} \right)^k a_N$\\
Розглянемо ряд $\huge \sum_{k=1}^{\infty} \left( \dfrac{1+q}{2} \right)^k a_N = a_N \sum_{k=1}^{\infty} \left( \dfrac{1+q}{2} \right)^k$\\
Вираз під сумою буде менше за $1$, цей ряд - геом. прогресія, збіжний\\
Тоді $\huge \sum_{k=1}^{\infty} a_{N+k} = \sum_{n = N+1}^{\infty} a_{n}$ - збіжний, отже, $\huge \sum_{n = 1}^{\infty} a_n$ - збіжний
\bigline
2) Якщо встановити $\varepsilon = \dfrac{q-1}{2}$, то отримаємо, що \\ $\dfrac{a_{n+1}}{a_n} > q - \varepsilon = \dfrac{q+1}{2}$\\
$\forall n \geq N: a_{n+1} > \dfrac{q+1}{2}a_n$\\
Аналогічними міркуваннями, отримаємо\\
$\forall k \geq 1: a_{N+k} > \left( \dfrac{q+1}{2} \right)^k a_N$\\
Розглянемо ряд $\huge \sum_{k=1}^{\infty} \left( \dfrac{q+1}{2} \right)^k a_N = a_N \sum_{k=1}^{\infty} \left( \dfrac{q+1}{2} \right)^k$\\
А тут геом. прогресія при виразі, що більше одиниці - розбіжний\\
Тоді $\huge \sum_{k=1}^{\infty} a_{N+k} = \sum_{n = N+1}^{\infty} a_{n}$ - розбіжний, отже, $\huge \sum_{n = 1}^{\infty} a_n$ - розбіжний
\bigline
А тепер в чому проблема при $q = 1$\\
Розглянемо обидва ряди: $\huge \sum_{n=1}^\infty \dfrac{1}{n}, \hspace{0.5cm} \sum_{n=1}^\infty \dfrac{1}{n^2}$\\
Використаємо для обох ознаку Даламбера:\\
$\huge \lim_{n \to \infty} \dfrac{1}{n+1} \cdot n = 1 \hspace{0.5cm} \lim_{n \to \infty} \dfrac{1}{(n+1)^2} \cdot n^2 = 1$\\
Результат - однаковий, проте один ряд - розбіжний, а інший - збіжний\\
Тож $q = 1$ не дає відповіді, шукаємо інші методи \qed
\bigline
\th{9.2.8. Радикальна ознака Коші}\\
Задано $\huge \sum_{n=1}^{\infty} a_n$ - знакододатній\\
Нехай $\exists \huge \uplim_{n \to \infty} \sqrt[n]{a_n} = q$. Тоді:\\
1) Якщо $q<1$, то ряд - збіжний\\
2) Якщо $q>1$, то ряд - розбіжний\\
3) Якщо $q=1$, то відповіді нема\\
\proof
1) $\exists \huge \uplim_{n \to \infty} \sqrt[n]{a_n} = q < 1$, тобто\\
$\forall \varepsilon > 0: \exists N: \forall n \geq N: \sqrt[n]{a_n} < q + \varepsilon$\\
$\Rightarrow a_n < (q + \varepsilon)^n$\\
Оберемо $\varepsilon = \dfrac{1-q}{2}$. Тоді маємо:\\
$a_n < \left( \dfrac{1+q}{2} \right)^n$\\
Розглянемо ряд $\huge \sum_{n = N}^{\infty} \left( \dfrac{1+q}{2} \right)^n$ - геом. прогресія, вираз в сумі менше за одиниці - збіжний\\
Отже, $\huge \sum_{n = 1}^{\infty} \left( \dfrac{1+q}{2} \right)^n$ - збіжний, а тому $\huge \sum_{n=1}^{\infty} a_n$ - збіжний
\bigline
2) $\exists \huge \uplim_{n \to \infty} \sqrt[n]{a_n} = q < 1$, тобто\\
$\exists \{\sqrt[n(p)]{a_{n(p)}}, p \geq 1 \}: \huge \lim_{p \to \infty} \sqrt[n(p)]{a_{n(p)}} = q$ - така підпослідовність, що містить цю границю\\
$\Rightarrow \forall \varepsilon > 0: \exists P: \forall p \geq P: \abs{\sqrt[n(p)]{a_{n(p)}} - q} < \varepsilon$\\
Оберемо $\varepsilon = \dfrac{q-1}{2}$, тоді\\
$a_{n(p)} > \left( \dfrac{q+1}{2} \right)^{n(p)}$\\
Тоді $\huge \lim_{p \to \infty} a_{n(p)} \geq \lim_{p \to \infty} \left( \dfrac{q+1}{2} \right)^{n(p)} = \infty$\\
Отже, $\huge \lim_{n \to \infty} a_n \neq 0$. Це означає, що необхідна умова збіжності не виконується - розбіжний
\bigline
3) Щоб з'ясувати випадок $q=1$, розгляньте такі самі ряди як при доведенні ознаки Даламбера \qed
\bigline
\rm{9.2.8.(1)} Тепер питання, чому саме верхня границя\\
Якщо, насправді, порахувати просто границю, то автоматично існує й верхня границя\\
Але виникають такі ряди, де стандартно границю не порахуєш. Тому треба розбивати на підпослідовності та шукати верхню границю, що й дасть відповідь на збіжність
\bigline
\rm{9.2.8.(2)} Ознака Коші сильніша за ознаку Даламбера\\
Нехай є ряд $\huge \sum_{n=1}^\infty 3^{-n-(-1)^n}$
\th{9.2.9. Радикальна ознака Коші}\\
Задано $\huge \sum_{n=1}^{\infty} a_n$ - знакододатній, такий, що:\\
1) $\exists f: [1,+\infty) \to \mathbb{R}: \forall n \geq 1: a_n = f(x)$\\
2) $f(x)$ спадає на $[1.+\infty)$\\
Тоді $\huge \sum_{n = 1}^\infty a_n$ та $\huge \int_1^{+\infty} f(x)\,dx$ збіжні або розбіжні одночасно\\
\proof
Оскільки $f(x)$ спадає, то $\forall k \geq 1: \forall x \in [k, k+1]:$\\
$a_k \geq f(x) \geq a_{k+1}$\\
$\huge a_k = \int_k^{k+1} a_k \,dx \geq \int_k^{k+1} f(x)\,dx \geq \int_k^{k+1} a_{k+1}\,dx = a_{k+1}$\\
Просумуємо ці нерівності від $k = 1$ до $k = M$, отримаємо:\\
$\huge \sum_{k=1}^M a_k \geq \int_1^{M} f(x)\,dx \geq \sum_{k=1}^M a_{k+1}$\\
Якщо $M \to \infty$, то за теоремою про поліцаїв отримаємо:\\
$\huge \lim_{M \to \infty}  \sum_{k=1}^M a_{k} = \sum_{k=1}^{\infty} a_k$ та $\huge \lim_{M \to \infty} \int_1^M f(x)\,dx = \int_1^\infty f(x)\,dx$\\
Із збіжності ряду випливає збіжність інтегралу і навпаки \qed
\bigline
\subsection{Знакозмінні ряди}
\defin{9.3.1.} Ряд $\huge \sum_{n=1}^\infty a_n$ називається \textbf{абсолютно збіжним}, якщо збігається ряд $\huge \sum_{n=1}^\infty \abs{a_n}$
\bigline
\defin{9.3.2.} Ряд $\huge \sum_{n=1}^\infty a_n$ називається \textbf{умовно збіжним}, якщо $\huge \sum_{n=1}^\infty a_n$ - збіжний, але $\huge \sum_{n=1}^\infty \abs{a_n}$ - не збіжний
\bigline
\prp{9.3.3.} $\huge \sum_{n=1}^\infty a_n$ - абсолютно збіжний $\iff$ $\huge \sum_{n=1}^\infty a_n$ - збіжний\\
\proof
$\huge \sum_{n=1}^\infty a_n$ - абсолютно збіжний $\iff$ $\huge \sum_{n=1}^\infty \abs{a_n}$ - збіжний $\iff \\ \iff \forall \varepsilon > 0: \exists K: \forall k \geq K: \forall p \geq 1: \huge \abs{\sum_{n=k}^{k+p} \abs{a_n}} < \varepsilon \iff \\
\iff \abs{\sum_{n=k}^{k+p} a_n} \leq \abs{\sum_{n=k}^{k+p} \abs{a_n}} < \varepsilon \iff  \sum_{n=1}^\infty a_n$ - збіжний \qed
\bigline
\th{9.3.4. Ознака Лейбніца}\\
Задано  $\huge \sum_{n=1}^\infty (-1)^{n+1}a_n$. Відомо, що\\
1) $\forall n \geq 1: a_n \geq 0$\\
2) $\{a_n, n \geq 1 \}$ - монотонно спадає\\
3) $\huge \lim_{n \to \infty} a_n = 0$\\
Тоді заданий ряд - збіжний\\
\proof
Розглянемо послідовність часткових сум $\{S_{2k}, k \geq 1 \}$. Отримаємо наступне:\\
$S_{2k} = \underset{\geq 0}{(a_1 - a_2)} + \underset{\geq 0}{(a_3 - a_4)} + \dots + \underset{\geq 0}{(a_{2k-1} - a_{2k})} \geq 0$\\
$S_{2k} = a_1 - \underset{\geq 0}{(a_2 - a_3)} - \underset{\geq 0}{(a_4 - a_5)} - \dots - \underset{\geq 0}{(a_{2k-2} - a_{2k-1})} - a_{2k} \leq a_1$\\
Тобто $0 \leq S_{2k} \leq a_1$ - обмежена послідовність\\
Також $S_{2(k+1)} = S_{2k} + (a_{2k+1}-a_{2k+2}) \geq S_{2k}$ - монотонна\\
Таким чином, $\exists \huge \lim_{k \to \infty} S_{2k} = S$\\
Розглянемо ще одну послідовність часткових сум $\{S_{2k+1}, k \geq 1\}$. Зрозуміло, що\\
$S_{2k+1} = S_{2k} + a_{2k+1}$\\
$\Rightarrow \huge \lim_{k \to \infty} S_{2k+1} = \lim_{k \to \infty} S_{2k} + \lim_{k \to \infty} a_{2k+1} = S + 0 = S$\\
Остаточно, маємо, що послідовність $\{S_m, m \geq 1\}$ - збіжна, тоді\\
$\huge \sum_{n=1}^\infty (-1)^{n+1}a_n$ - збіжний \qed
\bigline
\crl{9.3.4.} $\forall k \geq 1: |S-S_k| \leq a_{k+1}$\\
\proof
Розглянемо хвіст ряду $S-S_k = \huge \sum_{n=k+1}^{\infty} (-1)^{n+1}a_n$\\
А також $\tilde{S_m} = \huge \sum_{n=k+1}^{m} (-1)^{n+1}a_n$. Тоді\\
$\tilde{S_m} = S_m - S_k = (-1)^{k+1} \left(a_{k+1}-(a_{k+2}-a_{k+3})-(a_{k+1}-a_{k+5}) - \dots - \right. \\ \left. - \left[ \begin{gathered} (a_{m-1}-a_m), k \not \vdots 2 \\ a_m, k \vdots 2 \end{gathered} \right. \right)$\\
$\Rightarrow |\tilde{S_m}| = \left|a_{k+1}-(a_{k+2}-a_{k+3})-(a_{k+1}-a_{k+5}) - \dots - \left[ \begin{gathered} (a_{m-1}-a_m), k \not \vdots 2 \\ a_m, k \vdots 2 \end{gathered} \right. \right| = \\
= a_{k+1}-(a_{k+2}-a_{k+3})-(a_{k+1}-a_{k+5}) - \dots - \left[ \begin{gathered} (a_{m-1}-a_m), k \not \vdots 2 \\ a_m, k \vdots 2 \end{gathered} \right. \leq a_{k+1}$\\
$\Rightarrow |S - S_k| = \huge \lim_{m \to \infty} |\tilde{S_m}| \leq a_{k+1}$ \qed
\bigline
\th{9.3.5. Ознака Абеля-Діріхле}\\
Задано $\huge \sum_{n=1}^{\infty} a_n b_n$\\
Нехай виконано один з двох блок умов:\\
\begin{tabular}{c | c}
$\huge \sum_{n=1}^{\infty} a_n$ - збіжний & $\exists M > 0: \forall k \geq 1: \huge \abs{\sum_{n=1}^k a_n} \leq M$ \\
$\{b_n, n \geq 1\}$ - монотонна та обмежена & $\{b_n, n \geq 1\}$ - спадна та $\huge \lim_{n \to \infty} b_n = 0$ \\
\textit{ознаки Абеля} & \textit{ознаки Діріхле}
\end{tabular}
Тоді $\huge \sum_{n=1}^{\infty} a_n b_n$ - збіжний
\bigline
Але спочатку треба сформулювати одну лему\\
\lm{9.3.5. Тотожність Абеля}\\
Задані $a_n$, $b_n$ - довільні числа, $S_k = C + \huge \sum_{n=1}^k a_n$, де $C$ - якась константа\\
Тоді $\forall C: \huge \sum_{n = k}^{k+p} a_n b_n = \huge \sum_{n=k}^{k+p-1} S_n(b_n-b_{n+1}) + S_{k+p}b_{k+p} - S_{k-1}b_k$\\
\proof
Базується на тотожності $a_k = S_k - S_{k-1}$, що виконана $\forall C$. Тоді:\\
$\huge \sum_{n=k}^{k+p} a_n b_n = \sum_{n=k}^{k+p} (S_n - S_{n-1}) b_n = \sum_{n=k}^{k+p} S_n b_n - \sum_{n=k}^{k+p} S_{n-1} b_n = \sum_{n=k}^{k+p} S_n b_n - \sum_{n=k-1}^{k+p-1} S_n b_{n+1} =S_{k+p}b_{k+p} + \sum_{n=k}^{k+p-1} S_n b_n - S_{k-1}b_k - \sum_{n=k}^{k+p-1} S_n b_{n+1} = \\ = \sum_{n=k}^{k+p-1} S_n(b_n-b_{n+1}) + S_{k+p}b_{k+p} - S_{k-1}b_k$ \qed
\bigline
Трохи ліричного відступу, нафіга така тотожність:\\
Подивіться на цю формулу:\\
$\huge \int_a^b f'(x) g(x)\,dx = f(b)g(b) - f(a)g(a) - \int_a^b f(x)g'(x)\,dx$\\
А тепер перепишемо тотожність Абеля:\\
$\huge \sum_{n=k}^{k+p} (S_n - S_{n-1})b_n = S_{k+p}b_{k+p} - S_{k-1}b_k - \sum_{n=k}^{k+p-1} S_n(b_{n+1}-b_n)$\\
По суті, це - дискретна формула інтегрування за частинами, якщо робити криву паралель: сума - інтеграл, різниця - аналог похідної
\bigline
\proof
Доведемо ознаку Діріхле\\
Відомо, що $\huge \lim_{n \to \infty} b_n = 0 \Rightarrow \forall \varepsilon > 0: \exists N: \forall n \geq N: |b_n| < \varepsilon$\\
Також $\forall n \geq 1: b_{n+1} \leq b_n$\\
Більш того, $\exists M > 0: \forall k \geq 1: \huge \abs{\sum_{n=1}^k a_n} \leq M$
За критерієм Коші:\\
$\forall k \geq N: \forall p \geq 1: \huge \abs{\sum_{n=k}^{k+p} a_n b_n} = \abs{\sum_{n=k}^{k+p-1} S_n(b_n-b_{n+1}) + S_{k+p}b_{k+p} - S_{k-1}b_k} \leq \\
\leq \abs{\sum_{n=k}^{k+p-1} S_n(b_n-b_{n+1})} + \abs{S_{k+p}b_{k+p} - S_{k-1}b_k} \leq \\ \leq \sum_{n=k}^{k+p-1} \abs{S_n} \abs{b_n - b_{n+1}} + |b_k| |S_{k+p} - S_{k-1}| \leq M \sum_{n=k}^{k+p-1}(b_n-b_{n+1}) = \\ = M(b_k - b_{k+p}) < 2M \varepsilon$\\
Отже, отримали, що $\huge \sum_{n=1}^{\infty} a_n b_n$ - збіжний\\
\textit{Там не обов'язково, щоб послідовність спадала, можна й для зростаючої, буде все ок теж}
\bigline
Доведемо тепер ознаку Абеля\\
Оскільки $\{b_n, n \geq 1\}$ - монотонна та обмежена, то $\exists \huge \lim_{n \to \infty} b_n = b$\\
$\Rightarrow \huge \lim_{n \to \infty} \beta_n = 0$, де $\beta_n = b_n - b$\\
Теж буде монотонною\\
Оскільки $\huge \sum_{n=1}^{\infty} a_n$ - збіжний, то збіжною буде послідовності часткових сум, а звідси вони є обмеженими, отже,\\
$\exists M > 0: \forall k \geq 1: \huge \abs{\sum_{n=1}^k a_n} \leq M$\\
Тоді за ознакою Діріхле, $\huge \sum_{n=1}^{\infty} a_n \beta_n$ - збіжна, а тому як сума, $\huge \sum_{n=1}^{\infty} a_n b_n$ - збіжна \qed \bigline
\th{9.3.6. Теорема Рімана}\\
Задано $\huge \sum_{n=1}^\infty a_n$ - умовно збіжний\\
Тоді для довільного $M$ буде існувати перестановка членів ряду, після якої новий ряд із переставленими членами буде збіжним до числа $M$\\
\textit{Поки без доведення}
\bigline
\th{9.3.7. Теорема Діріхле}\\
Задано $\huge \sum_{n=1}^\infty a_n$ - абсолютно збіжний\\
Тоді будь-яка перестановка членів ряду не змінить суму\\
\textit{Поки без доведення}
\newpage
\section{Функціональні ряди}
\subsection{Функціональні послідовності}
\defin{10.1.1. Функціональною послідовністю} назвемо послідовність $\{f_n(x), n \geq 1 \}$, всі функції задані на одній множині $A$
\bigline
\defin{10.1.2.} Функція $f(x)$, що задана теж на множині $A$, називається \textbf{точковою границею} функціональної послідовності $\{f_n(x), n \geq 1\}$, якщо
\begin{align*}
\forall x \in A: \lim_{n \to \infty} f_n(x) = f(x)
\end{align*}
\rm{10.1.2.(1)} Якщо всі функції парні/непарні, то точкова границя теж парна/непарна
\bigline
\rm{10.1.2.(2)} Якщо всі функції монотонні, то точкова границя буде також монотонною\\
\textit{Випливає з нерівностей границь}
\bigline
\defin{10.1.3.} Функція $f(x)$ називається \textbf{рівномірною границею} функціональної послідовності $\{f_n(x), n \geq 1 \}$ на множині $A$, якщо
\begin{align*}
\sup_{x \in A} |f_n(x) - f(x)| \to 0, n \to \infty
\end{align*}
Позначення: $f_n(x)^\rightarrow_\rightarrow f(x)$, $n \to \infty$
\bigline
\prp{10.1.4.} Задано $\{f_n(x), n \geq 1\}$ - послідовність\\
Якщо $f_n(x)^\rightarrow_\rightarrow f(x)$, $n \to \infty$ на множині $A$, то \\ $\forall x \in A: f_n(x) \to f(x), n \to \infty$\\
\proof
За умовою, $\huge \sup_{x \in A} |f_n(x) - f(x)| \to 0, n \to \infty$, тобто\\
$\forall \varepsilon > 0: \exists N: \forall n \geq N: \huge \sup_{x \in A} |f_n(x) - f(x)| < \varepsilon$\\
$\Rightarrow \forall x \in A: |f_n(x) - f(x)| < \varepsilon \Rightarrow f_n(x) \to f(x), n \to \infty$ \qed
\bigline
Позначимо множину $Fun(A)$ - множина всіх функції, що задані на множині $A$
\bigline
\defin{10.1.5.} \textbf{Нормою} функції $f(x)$ назвемо число
\begin{align*}
||f|| = \sup_{x \in A} |f(x)|
\end{align*}
Властивості:\\
$\forall f(x) \in Fun(A):$\\
1) $||f|| \geq 0$\\
2) $||f|| = 0 \iff f(x) = 0, \forall x \in A$\\
3) $||\lambda f|| = |\lambda| \cdot ||f||, \forall \lambda \in \mathbb{R}$\\
4) $\forall f,g \in Fun(A): ||f+g|| \leq ||f|| + ||g||$\\
Наслідок) $\abs{||f||-||g||} \leq ||f-g||$\\
\proof
1), 3) зрозуміло \\
2) $||f|| = 0 \Rightarrow \huge \sup_{x \in A}|f(x)| = 0 \Rightarrow 0 \leq |f(x)| \leq 0 \Rightarrow f(x) \equiv 0$
\bigline
4) $||f+g|| = \huge \sup_{x \in A}|f(x)+g(x)| \leq \sup_{x \in A}(|f(x)| + |g(x)|) \leq  \\ \leq\sup_{x \in A}|f(x)| +\sup_{x \in A}|g(x)| = ||f|| + ||g||$
\bigline
Наслідок) \textit{Вказівка:} $||f|| \leq ||f-g||$ та $||g|| \leq ||g-f||$ \qed
\bigline

\th{10.1.6.} Задана $\{f_n(x), n \geq 1\}$ - послідовність та $f_n(x)^\rightarrow_\rightarrow f(x)$, $n \to \infty$ на $A$\\
Відомо, що $\forall n \geq 1: f_n(x) \in C(A)$. Тоді $f(x) \in C(A)$\\
\proof
Зафіксуємо т. $x_0 \in A$\\
За умовою, $||f_n-f|| \to 0, n \to \infty$\\
$\Rightarrow \forall \varepsilon > 0: \exists N: \forall n \geq N: \forall x \in A: |f_n(x)-f(x)| < \dfrac{\varepsilon}{3}$\\
$f_n(x) \in C(A) \Rightarrow \exists \delta(\varepsilon) > 0: \forall x_1: |x_1 - x_0| < \delta \Rightarrow |f_n(x_1) - f(x_0)| < \dfrac{\varepsilon}{3}$\\
$\Rightarrow |f(x_1) - f(x_0)| = |(f(x_1)-f_n(x_1)) + (f_n(x_1)-f_n(x_0)) + (f_n(x_0)-f(x_0))| \leq |f(x_1)-f_n(x_1)| + |f_n(x_1) - f_n(x_0)| + |f_n(x_0) - f(x_0)| < \dfrac{\varepsilon}{3} + \dfrac{\varepsilon}{3} + \dfrac{\varepsilon}{3} = \varepsilon$\\
$\Rightarrow f(x)$ - неперервна в т. $x_0$, яка є довільною. Отже, $f(x) \in C(A)$ \qed
\bigline
\th{10.1.7.} Задана $\{f_n(x), n \geq 1\}$ - послідовність та $f_n(x)^\rightarrow_\rightarrow f(x)$, $n \to \infty$ на $A$\\
Відомо, що $\forall n \geq 1: f_n(x) \in R(A)$. Тоді $f(x) \in R(A)$\\
\textit{Без доведення}
\bigline
\th{10.1.8. Критерій Коші}\\
$f_n(x)^\rightarrow_\rightarrow f(x)$, $n \to \infty$ на $A$ $\iff \forall \varepsilon > 0: \exists N: \forall n,m \geq N: ||f_n - f_m|| < \varepsilon$\\
\proof
$\boxed{\Rightarrow}$ Дано: $f_n(x)^\rightarrow_\rightarrow f(x)$, $n \to \infty$ на $A$\\
Тоді $||f_n - f|| \to 0, n \to \infty \Rightarrow \forall \varepsilon > 0: \exists N: \forall n,m \geq N: \begin{gathered} ||f_n-f|| < \dfrac{\varepsilon}{2} \\ ||f_m-f|| < \dfrac{\varepsilon}{2} \end{gathered}$\\
$\Rightarrow ||f_n - f_m|| = ||f_n - f + f - f_m|| \leq ||f_n - f|| + ||f_m - f|| < \varepsilon$
\bigline
$\boxed{\Leftarrow}$ Дано: $\forall \varepsilon > 0: \exists N: \forall n,m \geq N: ||f_n - f_m|| < \varepsilon$\\
$\Rightarrow \forall x \in A: |f_n(x) - f_m(x)| < \varepsilon$\\
Якщо зафіксувати точку $x_0 \in A$, то отримаємо фундаментальну послідовність $\{f_n(x_0), n \geq 1\} \Rightarrow \exists \huge \lim_{n \to \infty} f_n(x_0) = f(x_0)$\\
Якщо $m \to \infty$, то маємо, що $|f_n(x_0) - f(x_0)| < \varepsilon$\\
Оскільки це може бути $\forall x_0 \in A$, то тоді $||f_n - f|| < \varepsilon \Rightarrow$ $f_n(x)^\rightarrow_\rightarrow f(x)$, $n \to \infty$ на $A$ \qed
\bigline
\subsection{Функціональні ряди}
\defin{10.2.1. Функціональним рядом} називають суму членів функціональної послідовності $\{a_n(x), n \geq 1\}$
\begin{align*}
a_1(x) + a_2(x) + \dots + a_n(x) + \dots = \huge \sum_{n=1}^\infty a_n(x)
\end{align*}
\textbf{Частковою сумою} даного ряда називають суму перших $k$ функцій
\begin{align*}
S_k(x) = \sum_{n=1}^k a_n(x) = a_1(x) + a_2(x) + \dots + a_k(x)
\end{align*}
В такому випадку в нас виникає функціональна послідовність часткових сум $\{S_k(x), k \geq 1\}$\\
Якщо така послідовність збігається в т. $x_0$, то ряд є \textbf{збіжним} в т. $x_0$ та значення цього ряду дорівнює
\begin{align*}
\sum_{n=1}^\infty a_n(x_0) = \lim_{k \to \infty} S_k(x_0) = S(x_0)
\end{align*}
Якщо ряд збігається $\forall x \in B$, то $B$ називають \textbf{областю збіжності}
\bigline
Якщо ряд абсолютно збігається $\forall x \in B$, то $B$ називають \textbf{областю абсолютної збіжності}
\bigline
Якщо ряд умовно збігається $\forall x \in B$, то $B$ називають \textbf{областю умовної збіжності}
\bigline
\defin{10.2.2.} Якщо послідовність часткових сум $\{S_k(x), k \geq 1\}$ збігається рівномірно на множині $A$, то ряд $\huge \sum_{n=1}^\infty a_n(x)$ називають \textbf{рівномірно збіжним} на $A$
\bigline
\th{10.2.3. Критерій Коші}\\
$\huge \sum_{n=1}^\infty a_n(x)$ - рівномірно збіжний на множині $A$ $\iff \\ \iff \forall \varepsilon > 0: \exists N: \forall k,m \geq M: ||S_k - S_m|| < \varepsilon$ або $\huge \sup_{x \in A} \abs{\sum_{n=k+1}^m a_n(x)} < \varepsilon$\\
\textit{Випливає з критерію Коші рівновірної збіжності функціональних послідовностей}
\bigline
\th{10.2.4. Мажорантна ознака Вейерштраса}\\
Задано $\huge \sum_{n=1}^\infty a_n(x)$ - ряд на множині $A$. Відомо, що\\
1) $\exists \{c_n, n \geq 1\}: \forall n \geq 1: \forall x \in A: |a_n(x)| \leq c_n$\\
2) $\huge \sum_{n=1}^\infty c_n$ - збіжний. Його ще називають мажорантним рядом\\
Тоді $\huge \sum_{n=1}^\infty a_n(x)$ збігається рівномірно та абсолютно на множині $A$\\
\proof
За критерієм Коші,$\huge \sum_{n=1}^\infty c_n$ - збіжний $\iff \forall \varepsilon > 0: \exists N: \forall k,m \geq M: \huge \abs{\sum_{n=k+1}^m c_n} < \varepsilon$. Тоді\\
$\huge \left| \left| \sum_{n=k+1}^m a_n(x) \right| \right| = \sup_{x \in A} \abs{ \sum_{n=k+1}^m a_n(x)} \leq \abs{ \sup_{x \in A} \abs{ \sum_{n=k+1}^m a_n(x)} } \leq \sum_{n=k+1}^m \sup_{x \in A} |a_n(x)| \leq \sum_{n=k+1}^m c_n < \varepsilon$\\
Тому за критерієм Коші, $\huge \sum_{n=1}^\infty a_n(x)$ - рівномірно та абсолютно збіжний на множині $A$ \qed
\bigline
\th{10.2.5. Ознака Абеля-Діріхле}\\
Задано  $\huge \sum_{n=1}^\infty a_n(x) b_n(x)$ - ряд на множині $A$\\
Нехай виконано один з двох блок умов:\\
\begin{tabular}{c | c}
$\huge \sum_{n=1}^{\infty} a_n(x)$ - збіжний рівномірно на $A$ & $\exists M > 0: \forall k \geq 1: \huge \abs{\abs{\sum_{n=1}^k a_n(x)}} \leq M$ \\
$\{b_n(x), n \geq 1\}$ - рівномірно обмежена & $\{b_n(x), n \geq 1\}$ - спадна та $b_n(x)^\rightarrow_\rightarrow 0$ \\
та монотонна & \\
\textit{ознаки Абеля} & \textit{ознаки Діріхле}
\end{tabular}
Тоді $\huge \sum_{n=1}^{\infty} a_n(x) b_n(x)$ - збіжний рівномірно на множині $A$
\bigline
\th{10.2.6.} Задано $S(x) = \huge \sum_{n=1}^\infty a_n(x)$ - рівномірно збіжний на $A$\\
Відомо, що $\forall n \geq 1: a_n(x) \in C(A)$. Тоді $S(x) \in C(A)$\\
\proof
З умови теореми випливає, що $\forall k \geq 1: S_k(x) = \huge \sum_{n=1}^k a_n(x) \in C(A)$ як сума неперервних функцій\\
Оскільки ряд - рівномірно збіжний, то тоді $\{S_k(x), k \geq 1\}$ - рівномірно збіжна. Тоді за \textbf{Th. 10.1.6.}, $S(x) \in C(A)$ \qed
\bigline
\th{10.2.7.} Задано $S(x) = \huge \sum_{n=1}^\infty a_n(x)$ - рівномірно збіжний на $[a,b]$\\
Відомо, що $\forall n \geq 1: a_n(x) \in R([a,b])$. Тоді $S(x) \in R([a,b])$, а також\\
$\huge \int_a^b \left( \sum_{n=1}^\infty a_n(x) \right) \,dx = \sum_{n=1}^\infty \left( \int_a^b a_n(x)\,dx \right)$\\
\proof
З умови теореми випливає, що $\forall k \geq 1: S_k(x) = \huge \sum_{n=1}^k a_n(x) \in R([a,b])$ як сума інтегрованих функцій\\
Оскільки ряд - рівномірно збіжний, то тоді $\{S_k(x), k \geq 1\}$ - рівномірно збіжна. Тоді за \textbf{Th. 10.1.7.}, $S(x) \in R([a,b])$\\
Доведемо тепер тотожність:\\
$\huge \int_a^b \left( \sum_{n=1}^\infty a_n(x) \right) \,dx = \int_a^b \left( \lim_{k \to \infty} \sum_{n=1}^k a_n(x) \right) \,dx = \lim_{k \to \infty} \int_a^b \left( \sum_{n=1}^k a_n(x) \right) \,dx = \\ = \lim_{k \to \infty} \sum_{n=1}^k \left( \int_a^b a_n(x)\,dx \right) = \sum_{n=1}^\infty \left( \int_a^b a_n(x)\,dx \right)$ \qed
\bigline
\th{10.2.8.} Задано $S(x) = \huge \sum_{n=1}^\infty a_n(x)$. Відомо, що:\\
1) $\exists x_0 \in [a,b]: \huge \sum_{n=1}^\infty a_n(x_0)$ - збіжний\\
2) $\forall n \geq 1: a_n(x) \in C'([a,b])$\\
3) $\huge \sum_{n=1}^\infty a_n'(x)$ - рівномірно збіжний на $[a,b]$\\
Тоді $S(x)$ - збіжний рівномірно та $S(x) \in C'([a,b])$, а також\\
$\huge \left(  \sum_{n=1}^\infty a_n(x) \right)' = \sum_{n=1}^\infty a_n'(x)$\\
\proof
Розглянемо ряд $\tilde{S}(x) = \huge \sum_{n=1}^\infty a_n'(x)$\\
За минулою теоремою, можемо отримати, що\\
$\forall x \in [a,b]: \huge \int_{x_0}^x \left( \sum_{n=1}^\infty a_n'(t) \right) \,dt = \sum_{n=1}^\infty \left( \int_{x_0}^x a_n'(t) \,dt \right) = \sum_{n=1}^\infty \left( a_n(x) - a_n(x_0) \right)$ - збіжний рівномірно ряд\\
$\Rightarrow \huge \sum_{n=1}^\infty a_n(x) = \sum_{n=1}^\infty \left(a_n(x) - a_n(x_0) + a_n(x_0) \right) = \\ = \sum_{n=1}^\infty (a_n(x) - a_n(x_0)) + \sum_{n=1}^\infty a_n(x_0)$ - рівномірно збіжний\\
Доведемо тотожність\\
$ \huge \left( \sum_{n=1}^\infty a_n(x) \right)' = \left( \sum_{n=1}^\infty (a_n(x) - a_n(x_0)) \right)' + \left( \sum_{n=1}^\infty a_n(x_0) \right)' = \sum_{n=1}^\infty a_n'(x)$ \qed
\bigline
\subsection{Степеневі ряди}
\defin{10.3.1. Степеневим рядом} називаємо ми такий ряд
\begin{align*}
\sum_{n=0}^\infty a_n(x-x_0)^n
\end{align*}
де $\{a_n, n \geq 1\}$ - числова послідовність
\bigline
\th{10.3.2. Теорема Коші-Адамара}\\
Задано $\huge \sum_{n=0}^\infty a_n(x-x_0)^n$ - степеневий ряд\\
Нехай $\dfrac{1}{\huge \uplim_{n \to \infty} \sqrt[n]{|a_n|}} = R$ - \textbf{радіус збіжності}. Тоді ряд:\\
при $|x-x_0|<R$ - збіжний абсолютно\\
при $|x-x_0|>R$ - розбіжний\\
при $|x-x_0|=R$ - відповіді нема\\
\proof
Скористаємось радикальною ознакою Коші для нашого ряду:\\
$\huge \uplim_{n \to \infty} \sqrt[n]{|a_n(x-x_0)|^n} = |x-x_0| \uplim_{n \to \infty} \sqrt[n]{|a_n|} = q$\\
Тоді:\\
При $q < 1$, тобто $|x-x_0| < \dfrac{1}{\huge \uplim_{n \to \infty} \sqrt[n]{|a_n|}} = R$ - збіжний абсолютно\\
Аналогічно для решти \qed
\bigline
\crl{10.3.2. Наслідок із ознаки Даламбера}\\
Задано $\huge \sum_{n=0}^\infty a_n(x-x_0)^n$ - степеневий ряд\\
Нехай $\huge \lim_{n \to \infty} \abs{\dfrac{a_n}{a_{n+1}}} = R$ - \textbf{радіус збіжності}. Тоді ряд:\\
при $|x-x_0|<R$ - збіжний абсолютно\\
при $|x-x_0|>R$ - розбіжний\\
при $|x-x_0|=R$ - відповіді нема\\
\proof
Скористаємось ознакою Даламбера для нашого ряду:\\
$\huge \lim_{n \to \infty} \abs{\dfrac{a_{n+1}(x-x_0)^{n+1}}{a_n(x-x_0)^n}} = |x-x_0| \lim_{n \to \infty} \abs{\dfrac{a_{n+1}}{a_n}} = q$\\
Тоді:\\
При $q < 1$, тобто $|x-x_0| < \huge \lim_{n \to \infty} \abs{\dfrac{a_n}{a_{n+1}}} = R$ - збіжний абсолютно\\
Аналогічно для решти \qed
\bigline
\th{10.3.3. Теорема Абеля}\\
Задано $\huge \sum_{n=0}^\infty a_n(x-x_0)^n$ - степеневий ряд\\
Тоді ряд - рівномірно збіжний на будь-якому відрізку із області збіжності\\
\proof
Зафіксуємо довільний відрізок $[a,b]$\\
1. $[a,b] \subset (x_0-R,x_0+R)$\\
Зафіксуємо число $M = \max\{|x_0-a|,|x_0-b|\}$\\
Звідси $\forall x \in [a,b]: |x-x_0| < M < R$, а тому\\
$|a_n(x-x_0)^n| < |a_n| M^n$\\
Розглянемо ряд $\huge \sum_{n=0}^\infty a_n M^n$\\
$\huge \lim_{n \to \infty} \sqrt[n]{|a_n| M^n} = M \lim_{n \to \infty} \sqrt[n]{|a_n|} < R \lim_{n \to \infty} \sqrt[n]{|a_n|} = 1$\\
Отже, цей ряд - збіжний\\
Тоді за ознакою Вейерштраса, степеневий ряд - збіжний рівномірно на $[a,b]$
\bigline
2. $[a,b] \subset [x_0,x_0+R]$\\
$\huge \sum_{n=0}^\infty a_n(x-x_0)^n = \sum_{n=0}^\infty a_n R^n \left( \dfrac{x-x_0}{R} \right)^n$\\
Розглянемо випадок, коли ряд $\huge \sum_{n=0}^\infty a_n R^n$ - збіжний\\
Тоді дослідимо ряд $\huge \sum_{n=0}^\infty a_n R^n (x-x_0)^n$ за ознакою Абеля:\\
$f_n(x) = a_n R^n$\\
$g_n(x) = \left( \dfrac{x-x_0}{R} \right)^n$\\
Домовились, що $\huge \sum_{n=0}^\infty f_n(x)$ - збіжний, причому рівномірно, оскільки не залежить від $x$\\
Послідовність $\left\{ g_n(x) = \left( \dfrac{x-x_0}{R} \right)^n, n \geq 1 \right\}$ - рівномірно обмежена, оскільки\\
$\forall x \in [a,b] \subset [x_0, x_0+R]: |x - x_0| \leq R \Rightarrow \forall n \geq 1: \abs{\dfrac{x-x_0}{R}}^n \leq 1$\\
А також послідовність є монотонною, тому що $\dfrac{x-x_0}{R} < 1$\\
Отже, за Абелем-Діріхле, ряд - рівномірно збіжний на $[a,b]$
\bigline
Аналогічно, коли $[a,b] \subset [x_0-R, x_0]$ та $\huge \sum_{n=0}^\infty a_n (-R)^n$ - збіжний \qed
\bigline
\th{10.3.4.} Степеневий ряд $\huge S(x) = \sum_{n=0}^\infty a_n(x-x_0)^n \\ \in C([x_0-R,x_0+R])$\\
\proof
Візьмемо якусь точку $x_*$ з області збіжності\\
Нехай відрізок $[a,b] \ni x_*$. Якщо $x_* \neq x_0 - R, x_* \neq x_0 + R$, то беремо відрізок $(a,b) \ni x_*$\\
На відрізку $[a,b]$ ряд - збіжний рівномірно за теоремою Абеля, члени ряду - неперервні функції. Отже, за \textbf{Th. 10.2.6.}, $S(x) \in C([a,b]) \Rightarrow S(x) \in C(\{x_*\})$\\
Оскільки т. $x_*$ була довільною, то одразу $S(x) \in C([x_0-R, x_0+R])$ \qed
\bigline
\th{10.3.5.} Степеневий ряд $\huge S(x) = \sum_{n=0}^\infty a_n(x-x_0)^n \\ \in R([x_0-R,x_0+R])$\\
$\huge \int_a^b \sum_{n=0}^\infty a_n(x-x_0)^n \,dx = \sum_{n=0}^\infty \int_a^b a_n(x-x_0)^n \,dx$\\
\proof
За теоремою Абеля, на $[a,b]$ із області збіжності ряд - рівномірно збіжний, а тоді за \textbf{Th. 10.2.7.}, $S(x) \in R([a,b])$\\
$\huge \int_a^b \sum_{n=0}^\infty a_n(x-x_0)^n \,dx = \sum_{n=0}^\infty \int_a^b a_n(x-x_0)^n \,dx$ \qed
\bigline
\th{10.3.6.} Степеневий ряд $\huge S(x) = \sum_{n=0}^\infty a_n(x-x_0)^n \\ $ диференційований на $[x_0-R,x_0+R]$\\
$\huge \left( \sum_{n=0}^\infty a_n(x-x_0)^n \right)' = \sum_{n=1}^\infty a_n\cdot n(x-x_0)^{n-1}$\\
\proof
Розглянемо ряд $\huge \sum_{n=1}^\infty a_n\cdot n(x-x_0)^{n-1}$\\
Радіус збіжності збігається, оскільки\\
$ \dfrac{1}{\huge \uplim_{n \to \infty} \sqrt[n]{n |a_n|}} = \dfrac{1}{\huge \uplim_{n \to \infty} \sqrt[n]{|a_n|}} = R$\\
Візьмемо якусь точку $x_*$ з області збіжності\\
Нехай відрізок $[a,b] \ni x_*$\\
На відрізку $[a,b]$ ряд - збіжний рівномірно за теоремою Абеля. Використаємо далі \textbf{Th. 10.2.8.}\\
1) $\huge \sum_{n=0}^\infty a_n(x-x_0)^n$ - збіжний принаймні в одній точці\\
2) Всі члени ряду - неперервно-диференційовані функції\\
3) $\huge \sum_{n=1}^\infty a_n \cdot n(x-x_0)^{n-1}$ - рівномірно збіжний на $[a,b]$\\
Отже, $S(x)$ - диференційований на $[a,b]$, зокрема і в т. $x_*$\\
Оскільки т. $x_*$ була довільною, то одразу $S(x)$ - диференційований в $[x_0-R,x_0+R]$\\
Тому дійсно, $S'(x) = \huge \sum_{n=1}^\infty n(x-x_0)^{n-1}$ \qed
\bigline
\subsection{Зв'язок з Тейлором}
\th{10.4.1. Теорема Тейлора}
Задана функція $f$, така, що:\\
1) $f(x) \in C^{(\infty)}((x_0-R,x_0+R)), x_0 \in \mathbb{R}$\\
2) $\exists M \in \mathbb{R}: \forall n \geq 1: \forall x \in (x_0-R, x_0+R): \abs{f^{(n)}(x)} \leq M^n$\\
Тоді $\forall x \in (x_0-R, x_0+R)$ функція розкладується в ряд Тейлора\\
$f(x) = \huge \sum_{n=0}^\infty \dfrac{f^{(n)}(x_0)}{n!}(x-x_0)^n$\\
Якщо $\left[ \begin{gathered} R < \infty \\ R = \infty \end{gathered} \right.$ то ряд рівномірно збігається на $\left[ \begin{gathered} (x_0-R,x_0+R) \\ [x_0-R_0,x_0+R_0] \end{gathered} \right.$, причому $\forall R_0 \in \mathbb{R}$\\
\proof
Розкладемо функцію в ряд Тейлора за остачею Лагранжа:\\
$f(x) = \huge \sum_{n=0}^k \dfrac{f^{(n)}(x_0)}{n!}(x-x_0)^n + \dfrac{f^{(k+1)}(c)}{(k+1)!}(x-x_0)^{k+1}$\\
Тоді маємо, що:\\
$\huge \abs{f(x) - \sum_{n=0}^k \dfrac{f^{(n)}(x_0)}{n!}(x-x_0)^n} = \abs{\dfrac{f^{(k+1)}(c)}{(k+1)!}(x-x_0)^{k+1}} \leq \dfrac{M^{k+1}}{(k+1)!}r^{k+1}$\\
Розглянемо тепер ряд $\huge \sum_{k=0}^\infty \dfrac{M^{k+1}}{(k+1)!} r^{k+1}$\\
За ознакою Даламбера, $\huge \lim_{k \to \infty} \dfrac{a_{k+1}}{a_k} = \huge \lim_{k \to \infty} \dfrac{Mr}{k+2} = 0 < 1$\\
Цей ряд є збіжним. Отже, $\huge \lim_{k \to \infty} a_k = 0$\\
Звідси випливає, що\\
$\huge \sup_{x \in (x_0-R,x_0+R)} \abs{f(x) - \sum_{n=0}^k \dfrac{f^{(n)}(x_0)}{n!}(x-x_0)^n} \leq \dfrac{M^{k+1}}{(k+1)!}r^{k+1} \to 0$, $k \to \infty$\\
Отримали: $\huge \sum_{n=0}^k \dfrac{f^{(n)}(x_0)}{n!}(x-x_0)^n \rightarrow^\rightarrow f$, $k \to \infty$\\
Таким чином, $f(x) = \huge \sum_{n=0}^\infty \dfrac{f^{(n)}(x_0)}{n!}(x-x_0)^n$ - збіжний ріномірно на \\ $(x_0-R,x_0+R)$ \qed
\bigline
\th{10.4.2.} Степеневий ряд задається єдиним чином\\
\proof
Інакше кажучи, доведемо, що якщо $\huge \sum_{n=0}^\infty a_n(x-x_0)^n, \sum_{n=0}^\infty b_n(x-x_0)^n$ мають одне значення на $(x_0-\varepsilon, x_0+\varepsilon)$, то $\forall n \geq 0: a_n = b_n$\\
$S(x_0) = a_0 = b_0$\\
$S'(x_0) = \huge \sum_{n=1}^\infty a_n \cdot n(x-x_0)^{n-1} = \sum_{n=1}^\infty b_n \cdot n(x-x_0)^{n-1}$\\
$\Rightarrow S'(x_0) = a_1 = b_1$\\
$\dots$\\
Таким чином, $\forall n \geq 0: a_n = b_n$ \qed
\bigline
\crl{10.4.2.} Ряд Тейлора для суми степеневого ряду співпадають с самим степеневим рядом на області збіжності

\newpage
\section*{Цитати означень по-своєму}
$c' = \sup A \iff \begin{cases} c \in UpA \\ \forall \varepsilon > 0: \exists a_{\varepsilon} \in A: a_{\varepsilon} > c' - \varepsilon \end{cases}$\\
Другий пункт каже ось що: якщо ми візьмемо супремум та трохи зменшимо, то ми знайдемо такий елемент, що буде явно більше за 'зменшеного супремуму' - а отже, цей 'зменший супремум' не буде супремумом. І так для кожного зменшеного\\
\\
'набір цілих чисел, що менше за задану дріб'

\newpage

\section{Premilinaries}


\newpage
\section*{Проблематичні задачі}
\ex{} Знайти точну верхню та нижню грань множини\\
$A = \left\{ \dfrac{mn}{4m^2 + n^2} | m \in \mathbb{Z}, n \in \mathbb{N} \right\}$
\bigline
Запишемо дріб таким чином:\\
$\dfrac{mn}{4m^2 + n^2} = \dfrac{1}{4 \dfrac{m}{n} + \dfrac{n}{m}} = \dfrac{1}{4q + \dfrac{1}{q}}$\\
Де $q \in \mathbb{Q} \setminus \{0\}$\\
Бачимо, що $4q + \dfrac{1}{q} \geq 2 \huge \sqrt{4q \dfrac{1}{q}} = 4$, виконано для $q > 0$\\
Тому $\dfrac{1}{4q + \dfrac{1}{q}} \leq \dfrac{1}{4}$\\
Потенційно, може бути $\sup A = \dfrac{1}{4}$. Перевіримо:\\
1) $\forall q \in \mathbb{Q} \setminus \{0\}: \dfrac{1}{4q + \dfrac{1}{q}} \leq \dfrac{1}{4}$ - тут доведено та зрозуміло\\
2) $\forall \varepsilon > 0: \exists q_{\varepsilon} = \dfrac{1}{2}: \dfrac{1}{4q_{\varepsilon} + \dfrac{1}{q_{\varepsilon}}} > \dfrac{1}{4} - \varepsilon$\\
Отже, $\sup A = \dfrac{1}{4}$\\
Пошук $\inf A = -\dfrac{1}{4}$ є аналогічним, якщо $q = -u$ для $q < 0$
\bigline
\ex{} Задана функція $f: \mathbb{Z}^2 \to \mathbb{Z}^2$, така, що:\\
$\mathbb{Z}^2 \ni (m,n) \mapsto (m,0) \in \mathbb{Z}^2$\\
Знайти $f^{-1}(D)$, якщо $D = \{(m,n)| m \in \mathbb{Z}, n \in \{-1,0,1,2\}\}$
\bigline
$f^{-1}(D) = \{(m,n) \in \mathbb{Z}^2: (m,0) \in D \} = \{(m,n): m \in \mathbb{Z}, n \in \mathbb{N}\}$\\
Для $n=-1,1,2$ прообраз має порожню множину. Також ми можемо множину розбити на об'єднання, тому все легітимно тут
\bigline
\ex{} Довести, що $\forall a,b \in \mathbb{R}:$\\
$\min\{a,b\} = \dfrac{1}{2}(a+b - |a-b|)$\\
$\max\{a,b\} = \dfrac{1}{2}(a+b + |a-b|)$\\
\textit{Просто перевірити, коли $a>b$, а коли $a<b$. Тут все зрозуміло}
\bigline
\ex{} Довести, що $\forall n \geq 1: \dfrac{1}{2} \dfrac{3}{4} \dots \dfrac{2n-1}{2n} \leq \dfrac{1}{\sqrt{3n+1}}$
\bigline
База: $n = 1$. Зрозуміло\\
Крок: нехай для $n$ це виконується. Перевіримо для $n+1$\\
$\dfrac{1}{2} \dfrac{3}{4} \dots \dfrac{2n-1}{2n} \dfrac{2n+1}{2n+2} \leq \dfrac{1}{\sqrt{3n+1}} \dfrac{2n+1}{2n+2} < $
\bigline
\ex{} Довести, що $\huge \sup_{x \in A} (f(x) + g(x)) \leq \sup_{x \in A}f(x) + \sup_{x \in A} g(x)$
\bigline
Маємо, що $\huge \forall x \in A: \begin{cases} f(x) \huge \leq \sup_{x \in A} f(x) \\\huge g(x) \leq \sup_{x \in A} g(x) \end{cases}$\\
$\Rightarrow f(x) + g(x) \leq \huge \sup_{x \in A}f(x) + \sup_{x \in A}g(x)$\\
Для $f(x) + g(x)$ ми маємо, що сума праворуч - верхня межа, але вона не є точною, тобто є межі, що менше за цю\\
Маємо $\forall x \in A: f(x) + g(x) \leq \huge \sup_{x \in A} (f(x) + g(x))$\\
$\forall \varepsilon > 0: \exists x_{\varepsilon} \in A: f(x_{\varepsilon}) + g(x_{\varepsilon}) < \huge \sup_{x \in A} (f(x) + g(x)) - \varepsilon$\\
А тому здвіси й випливає, що $\huge \sup_{x \in A} (f(x) + g(x)) \leq \sup_{x \in A}f(x) + \sup_{x \in A} g(x)$
\end{document}